% header -----------------------------------------------------------------------
% Template created by texnew (author: Alex Rutar); info can be found at 'https://github.com/alexrutar/texnew'.
% version (1.13)


% doctype ----------------------------------------------------------------------
\documentclass[11pt, a4paper]{memoir}
\usepackage[utf8]{inputenc}
\usepackage[left=3cm,right=3cm,top=3cm,bottom=4cm]{geometry}
\usepackage[protrusion=true,expansion=true]{microtype}


% packages ---------------------------------------------------------------------
\usepackage{amsmath,amssymb,amsfonts}
\usepackage{graphicx}
\usepackage{etoolbox}
\usepackage{braket}

% Set enimitem
\usepackage{enumitem}
\SetEnumitemKey{nl}{nolistsep}
\SetEnumitemKey{r}{label=(\roman*)}
\SetEnumitemKey{a}{label=(\alph*)}

% Set tikz
\usepackage{tikz, pgfplots}
\pgfplotsset{compat=1.15}
\usetikzlibrary{intersections,positioning,cd}
\usetikzlibrary{arrows,arrows.meta}
\tikzcdset{arrow style=tikz,diagrams={>=stealth}}

% Set hyperref
\usepackage[hidelinks]{hyperref}
\usepackage{xcolor}
\newcommand\myshade{85}
\colorlet{mylinkcolor}{violet}
\colorlet{mycitecolor}{orange!50!yellow}
\colorlet{myurlcolor}{green!50!blue}

\hypersetup{
  linkcolor  = mylinkcolor!\myshade!black,
  citecolor  = mycitecolor!\myshade!black,
  urlcolor   = myurlcolor!\myshade!black,
  colorlinks = true,
}


% macros -----------------------------------------------------------------------
\DeclareMathOperator{\N}{{\mathbb{N}}}
\DeclareMathOperator{\Q}{{\mathbb{Q}}}
\DeclareMathOperator{\Z}{{\mathbb{Z}}}
\DeclareMathOperator{\R}{{\mathbb{R}}}
\DeclareMathOperator{\C}{{\mathbb{C}}}
\DeclareMathOperator{\F}{{\mathbb{F}}}

% Boldface includes math
\newcommand{\mbf}[1]{{\boldmath\bfseries #1}}

% proof implications
\newcommand{\imp}[2]{($#1\Rightarrow#2$)\hspace{0.2cm}}
\newcommand{\impe}[2]{($#1\Leftrightarrow#2$)\hspace{0.2cm}}
\newcommand{\bij}[2]{($#1\leftrightarrow#2$)\hspace{0.2cm}}
\newcommand{\impr}{{($\Longrightarrow$)\hspace{0.2cm}}}
\newcommand{\impl}{{($\Longleftarrow$)\hspace{0.2cm}}}

% align macros
\newcommand{\agspace}{\ensuremath{\phantom{--}}}
\newcommand{\agvdots}{\ensuremath{\hspace{0.16cm}\vdots}}

% convenient brackets
\newcommand{\brac}[1]{\ensuremath{\left\langle #1 \right\rangle}}
\newcommand{\norm}[1]{\ensuremath{\left\lVert#1\right\rVert}}
\newcommand{\abs}[1]{\ensuremath{\left\lvert#1\right\rvert}}

% arrows
\newcommand{\lto}[0]{\ensuremath{\longrightarrow}}
\newcommand{\fto}[1]{\ensuremath{\xrightarrow{\scriptstyle{#1}}}}
\newcommand{\hto}[0]{\ensuremath{\hookrightarrow}}
\newcommand{\mapsfrom}[0]{\mathrel{\reflectbox{\ensuremath{\mapsto}}}}
 
% Divides, Not Divides
\renewcommand{\div}{\bigm|}
\newcommand{\ndiv}{%
    \mathrel{\mkern.5mu % small adjustment
        % superimpose \nmid to \big|
        \ooalign{\hidewidth$\big|$\hidewidth\cr$/$\cr}%
    }%
}

% Convenient overline
\newcommand{\ol}[1]{\ensuremath{\overline{#1}}}

% Big \cdot
\makeatletter
\newcommand*\bigcdot{\mathpalette\bigcdot@{.5}}
\newcommand*\bigcdot@[2]{\mathbin{\vcenter{\hbox{\scalebox{#2}{$\m@th#1\bullet$}}}}}
\makeatother

% Big and small Disjoint union
\makeatletter
\providecommand*{\cupdot}{%
  \mathbin{%
    \mathpalette\@cupdot{}%
  }%
}
\newcommand*{\@cupdot}[2]{%
  \ooalign{%
    $\m@th#1\cup$\cr
    \sbox0{$#1\cup$}%
    \dimen@=\ht0 %
    \sbox0{$\m@th#1\cdot$}%
    \advance\dimen@ by -\ht0 %
    \dimen@=.5\dimen@
    \hidewidth\raise\dimen@\box0\hidewidth
  }%
}

\providecommand*{\bigcupdot}{%
  \mathop{%
    \vphantom{\bigcup}%
    \mathpalette\@bigcupdot{}%
  }%
}
\newcommand*{\@bigcupdot}[2]{%
  \ooalign{%
    $\m@th#1\bigcup$\cr
    \sbox0{$#1\bigcup$}%
    \dimen@=\ht0 %
    \advance\dimen@ by -\dp0 %
    \sbox0{\scalebox{2}{$\m@th#1\cdot$}}%
    \advance\dimen@ by -\ht0 %
    \dimen@=.5\dimen@
    \hidewidth\raise\dimen@\box0\hidewidth
  }%
}
\makeatother


% macros (theorem) -------------------------------------------------------------
\usepackage[thmmarks,amsmath,hyperref]{ntheorem}
\usepackage[capitalise,nameinlink]{cleveref}

% Numbered Statements
\theoremstyle{change}
\theoremindent\parindent
\theorembodyfont{\itshape}
\theoremheaderfont{\bfseries\boldmath}
\newtheorem{theorem}{Theorem.}[section]
\newtheorem{lemma}[theorem]{Lemma.}
\newtheorem{corollary}[theorem]{Corollary.}
\newtheorem{proposition}[theorem]{Proposition.}

% Claim environment
\theoremstyle{plain}
\theorempreskip{0.2cm}
\theorempostskip{0.2cm}
\theoremheaderfont{\scshape}
\newtheorem{claim}{Claim}
\renewcommand\theclaim{\Roman{claim}}
\AtBeginEnvironment{theorem}{\setcounter{claim}{0}}
\newtheorem{ppart}{Part}
\renewcommand\theppart{\Roman{ppart}}
\AtBeginEnvironment{theorem}{\setcounter{ppart}{0}}

% Un-numbered Statements
\theorempreskip{0.1cm}
\theorempostskip{0.1cm}
\theoremindent0.0cm
\theoremstyle{nonumberplain}
\theorembodyfont{\upshape}
\theoremheaderfont{\bfseries\itshape}
\newtheorem{definition}{Definition.}
\theoremheaderfont{\itshape}
\newtheorem{example}{Example.}
\newtheorem{remark}{Remark.}

% Proof / solution environments
\theoremseparator{}
\theoremheaderfont{\hspace*{\parindent}\scshape}
\theoremsymbol{$//$}
\newtheorem{solution}{Sol'n}
\theoremsymbol{$\blacksquare$}
\theorempostskip{0.4cm}
\newtheorem{proof}{Proof}
\theoremsymbol{}
\newtheorem{nmproof}{Proof}

% Format references
\crefformat{equation}{(#2#1#3)}
\Crefformat{theorem}{#2Thm. #1#3}
\Crefformat{lemma}{#2Lem. #1#3}
\Crefformat{proposition}{#2Prop. #1#3}
\Crefformat{corollary}{#2Cor. #1#3}
\crefformat{theorem}{#2Theorem #1#3}
\crefformat{lemma}{#2Lemma #1#3}
\crefformat{proposition}{#2Proposition #1#3}
\crefformat{corollary}{#2Corollary #1#3}


% macros (algebra) -------------------------------------------------------------
\DeclareMathOperator{\Ann}{Ann}
\DeclareMathOperator{\Aut}{Aut}
\DeclareMathOperator{\chr}{char}
\DeclareMathOperator{\coker}{coker}
\DeclareMathOperator{\disc}{disc}
\DeclareMathOperator{\End}{End}
\DeclareMathOperator{\Fix}{Fix}
\DeclareMathOperator{\Frac}{Frac}
\DeclareMathOperator{\Gal}{Gal}
\DeclareMathOperator{\GL}{GL}
\DeclareMathOperator{\SL}{SL}
\DeclareMathOperator{\Hom}{Hom}
\DeclareMathOperator{\id}{id}
\DeclareMathOperator{\im}{im}
\DeclareMathOperator{\lcm}{lcm}
\DeclareMathOperator{\Nil}{Nil}
\DeclareMathOperator{\rank}{rank}
\DeclareMathOperator{\Res}{Res}
\DeclareMathOperator{\Spec}{Spec}
\DeclareMathOperator{\spn}{span}
\DeclareMathOperator{\Stab}{Stab}
\DeclareMathOperator{\Tor}{Tor}

% Lagrange symbol
\newcommand{\lgs}[2]{\ensuremath{\left(\frac{#1}{#2}\right)}}

% Quotient (larger in display mode)
\newcommand{\quot}[2]{\mathchoice{\left.\raisebox{0.14em}{$#1$}\middle/\raisebox{-0.14em}{$#2$}\right.}
                                 {\left.\raisebox{0.08em}{$#1$}\middle/\raisebox{-0.08em}{$#2$}\right.}
                                 {\left.\raisebox{0.03em}{$#1$}\middle/\raisebox{-0.03em}{$#2$}\right.}
                                 {\left.\raisebox{0em}{$#1$}\middle/\raisebox{0em}{$#2$}\right.}}


% macros (analysis) ------------------------------------------------------------
\DeclareMathOperator{\M}{{\mathcal{M}}}
\DeclareMathOperator{\B}{{\mathcal{B}}}
\DeclareMathOperator{\ps}{{\mathcal{P}}}
\DeclareMathOperator{\pr}{{\mathbb{P}}}
\DeclareMathOperator{\E}{{\mathbb{E}}}
\DeclareMathOperator{\supp}{supp}
\DeclareMathOperator{\sgn}{sgn}

\renewcommand{\Re}{\ensuremath{\operatorname{Re}}}
\renewcommand{\Im}{\ensuremath{\operatorname{Im}}}
\renewcommand{\d}[1]{\ensuremath{\operatorname{d}\!{#1}}}


% file-specific preamble -------------------------------------------------------
\newcommand{\defname}[1]{{\textit{(#1)}:}}
\newcommand{\exname}[1]{{\textit{#1}:}}
\newcommand{\defn}[1]{{\boldmath\bfseries #1}}
% \usepackage{therefore}
\newcommand{\TODO}[1]{[\textit{\textbf{TODO: #1}}]}
\newcommand{\NOTE}[1]{[\textit{\textbf{NOTE: #1}}]}
\DeclareMathOperator*{\esssup}{ess\,sup}
\DeclareMathOperator{\ext}{ext}
\DeclareMathOperator{\conv}{conv}
\DeclareMathOperator{\dist}{dist}
\newcommand{\mae}{\ensuremath{\,\operatorname{a.e.}}}
\DeclareMathOperator{\Pol}{Pol}
\DeclareMathOperator{\SO}{SO}
\newcommand{\cwx}{\ensuremath{\overline{\operatorname{conv}}^{w^*}\,}}
\newcommand{\idc}[1]{\ensuremath{\mathbf{1}_{#1}}}
\newcommand{\FA}{\ensuremath{\operatorname{F}\!\operatorname{A}}}
\newcommand{\cw}{\ensuremath{\overline{\operatorname{conv}}\,}}

% Tons of notation:
% \newcommand{\Lip}[1]{\ensuremath{\operatorname{Lip}_{\F}(#1)}}
\newcommand{\Lipspace}{\ensuremath{\operatorname{Lip}_{\F}(X,d)}}


\newcommand{\lp}[1]{\ensuremath{\ell^{#1}}}
\newcommand{\csn}{\ensuremath{\mathbf{c}}}
\newcommand{\csz}{\ensuremath{\mathbf{c}_0}}
\newcommand{\lpspace}[1]{\ensuremath{\ell^{#1}_{\F}}}
\newcommand{\Lp}[1]{\ensuremath{L^{#1}_{\F}}}
% \newcommand{\Lpm}{\ensuremath{L^{#1}_{\F}(X,\mathcal{M},\mu)}}
\DeclareMathOperator{\Lip}{Lip}
\newcommand{\lbr}[1]{\ensuremath{\left[#1\right]}}
\newcommand{\inr}[1]{\ensuremath{\left(#1\right)}}


% constants --------------------------------------------------------------------
\newcommand{\subject}{Harmonic Analysis}
\newcommand{\semester}{Winter 2020}


% formatting -------------------------------------------------------------------
% Fonts
\usepackage{kpfonts}
\usepackage{dsfont}

% Adjust numbering
\numberwithin{equation}{section}
\counterwithin{figure}{section}
\counterwithout{section}{chapter}
\counterwithin*{chapter}{part}

% Footnote
\setfootins{0.5cm}{0.5cm} % footer space above
\renewcommand*{\thefootnote}{\fnsymbol{footnote}} % footnote symbol

% Table of Contents
\renewcommand{\thechapter}{\Roman{chapter}}
\renewcommand*{\cftchaptername}{Chapter } % Place 'Chapter' before roman
\setlength\cftchapternumwidth{4em} % Add space before chapter name
\cftpagenumbersoff{chapter} % Turn off page numbers for chapter
\maxtocdepth{subsection} % table of contents up to section

% Section / Subsection headers
\setsecnumdepth{subsection} % numbering up to and including "subsection"
\newcommand*{\shortcenter}[1]{%
    \sethangfrom{\noindent ##1}%
    \Large\boldmath\scshape\bfseries
    \centering
\parbox{5in}{\centering #1}\par}
\setsecheadstyle{\shortcenter}
\setsubsecheadstyle{\large\scshape\boldmath\bfseries\raggedright}

% Chapter Headers
\chapterstyle{verville}

% Page Headers / Footers
\copypagestyle{myruled}{ruled} % Draw formatting from existing 'ruled' style
\makeoddhead{myruled}{}{}{\scshape\subject}
\makeevenfoot{myruled}{}{\thepage}{}
\makeoddfoot{myruled}{}{\thepage}{}
\pagestyle{myruled}
\setfootins{0.5cm}{0.5cm}
\renewcommand*{\thefootnote}{\fnsymbol{footnote}}

% Titlepage
\title{\subject}
\author{Alex Rutar\thanks{\itshape arutar@uwaterloo.ca}\\ University of Waterloo}
\date{\semester\thanks{Last updated: \today}}

\begin{document}
\pagenumbering{gobble}
\hypersetup{pageanchor=false}
\maketitle
\newpage
\frontmatter
\hypersetup{pageanchor=true}
\tableofcontents*
\newpage
\mainmatter


% main document ----------------------------------------------------------------
\chapter{Harmonic Analysis}
\section{Locally Compact Groups}
\begin{definition}
    Let $G$ be a group.
    A topology $\tau$ on $G$ is a \defn{group topology} provided that
    \begin{itemize}[nl]
        \item $x\mapsto x^{-1}:G\to G$ is continuous, and
        \item $(x,y)\mapsto xy:G\times G\to G$ is continuous.
    \end{itemize}
    We call $(G,\tau)$ a \defn{topological group} where we omit $\tau$ when it is clear from context.
\end{definition}
Equivalently, we may assert that $(x,y)\mapsto xy^{-1}$ is $\tau\times\tau-\tau-$continuous.
Write $L_g(x)=gx$ and $R_g(x)=xg$ to denote the left and right multiplication maps; then it is easy to see that $L_g$ and $R_g$ are homeomorphisms.
Similarly, $x\mapsto x^{-1}$ is a homeomorphism.
\begin{definition}
    We say that a subset $A\subset G$ is \defn{symmetric} if $A^{-1}=A$.
\end{definition}
We have the following basic properties:
\begin{proposition}\label{p:tgrp}
    Let $(G,\tau)$ be a topological group.
    \begin{enumerate}[nl,r]
        \item If $\emptyset\neq A\subseteq G$ and $U$ is open, then $AU=\{ay:a\in A,y\in U\}$ and likewise $UA$ are open.
        \item Given $U\in\tau$ and $x\in U$, then there is a symmetric $V\in\tau$ with $e\in V$ such that $VxV\subseteq U$.
            In particular, if $e\in U$, then we can find symmetric $V$ so that $V^2\subseteq U$.
        \item If $H$ is a subgroup of $G$, then $\overline{H}$ is also a subgroup.
        \item An open subgroup is automatically closed.
        \item If $K,L\subseteq G$ are compact, then $KL$ is compact.
        \item If $K$ is compact and $C$ is closed in $G$, then $KC$ is closed.
    \end{enumerate}
\end{proposition}
In $(\R,+)$, then $\Z+\sqrt{2}\Z$ is not closed, so it is necessary to assume compactness in (vi).
\begin{proof}
    \begin{enumerate}[nl,r]
        \item $AU=\bigcup_{a\in A}L_a(U)$ is a union of open sets.
        \item Consider the continuous map $(y,z)\mapsto yxz$.
            Since $exe=x\in U$, there is a $\tau\times\tau-$neighbourhood of $(e,e)$ which maps into $U$ have a basic neighbourhood $V_1\times V_2$.
            Let $V=V_1\cap V_2$.
            Moreover, we may replace $V$ by $V^{-1}\cap V$. to attain symmetry.
        \item Let $x,y\in\ol{H}$, $U\in\tau$ with $xy\in U$.
            Then (ii) provides $V$ with $VxyV\subseteq U$.
            But $Vx\cap H\neq\emptyset$ and $\neq yV\cap H$ so there are $_1\in Vx\cap H$, $h_2\in yV\cap H$, and $h_1h_2\in VxyV\subseteq U$.
            Thus $U\cap H\neq\emptyset$.
            Thus $xy\in\ol{H}$.

            To use nets for inverses, if $x\in\ol{H}$, then $x=\lim_\alpha x_\alpha$ where $(x_\alpha)_{\alpha\in A}\subset H$ is a net.
            Then $x^{-1}=\lim_\alpha x_\alpha^{-1}\in\ol{H}$ as each $x_\alpha^{-1}\in H$.
        \item If $H$ is an open subgroup, then $H=G\setminus\bigcup_{x\in G\setminus H}xH$ is closed.
        \item $K\times L$ is compact, and hence so is its image under multiplication.
        \item If $x\in\ol{KC}$, then $x=\lim_\alpha k_\alpha x_\alpha$ where $k_\alpha\in H$ and $x_\alpha\in C$.
            Since $K$ is compact, we may assume (passing to a subnet if necessary) $k=\lim_\alpha k_\alpha$ exists in $K$.
            Then
            \begin{equation*}
                k^{-1}x=\lim_\alpha k_\alpha^{-1}\cdot\lim_\alpha k_\alpha x_\alpha=\lim_\alpha k_\alpha^{-1}k_\alpha x_\alpha=\lim x_\alpha\in C
            \end{equation*}
            so $x=kk^{-1}x\in KC$.
    \end{enumerate}
\end{proof}
\subsection{Homogenous Spaces}
Let $(G,\tau)$ be a topological group, $H$ a subgroup of $G$, and $\quot{G}{H}=\{xH;x\in G\}$.
Let $\pi:G\to \quot{G}{H}$ be given by $\pi(x)=xH$ be the projection map.
The \defn{quotient topology} on $\quot{G}{H}$ is $\tau_{\quot{G}{H}}=\{W\in \quot{G}{H}:\pi^{-1}(W)\in\tau\}$.
Notice that if $U\in\tau\setminus\{\emptyset\}$, then $UH=\pi^{-1}(\pi(U))$ is open, so $\pi:G\to \quot{G}{H}$ is an open map.
\begin{proposition}
    Let $(G,\tau)$, $H$ be as above.
    \begin{enumerate}[nl,r]
        \item The map $(x,yH)\mapsto xyH:G\times\quot{G}{H}\to\quot{G}{H}$ is $\tau\times\tau_{\quot{G}{H}}-\tau_{\quot{G}{H}}$ continuous and open.
        \item If $H$ is normal, then $(\quot{G}{H},\tau_{\quot{G}{H}})$ is a topological group.
        \item If $H$ is closed, then $\tau_{\quot{G}{H}}$ is Hausdorff.
    \end{enumerate}
\end{proposition}
\begin{proof}
    \begin{enumerate}[nl,r]
        \item Let $x,y\in G$, $W\in\tau_{\quot{G}{H}}$ satisfy $xyH=\pi(xy)\in W$.
            Then $xy\in\pi^{-1}(W)$ and by \cref{p:tgrp} we have $V\in\tau$ with $e\in V$ such that $VxyV\subseteq \pi^{-1}(W)$.
            But then $(x,\pi(y))\in Vx\times\pi(yV)\in \tau\times\tau_{\quot{G}{H}}$ and the latter set maps into $\pi(VxyV)\subseteq W$.

            Also, if $U\in\tau\times\tau_{\quot{G}{H}}$, then $U=\bigcup_{(x,yH)\in U}V_x\times W_{yH}$ and
            \begin{equation*}
                \pi(U)=\bigcup_{(x,yH)\in U}\pi(V_x\pi^{-1}(W_{yH}))
            \end{equation*}
            since $\pi$ is open.
        \item Recall that $(xH)(yH)=xyH$ is our multiplication operation on $\quot{G}{H}$ and $\pi$ is a group homomorphism.
            Then the following diagram commutes:
        % \begin{tikzcd}
            % M'\arrow[r,"f"]\arrow[rd,swap,"u"] & M\arrow[r,"g"]\arrow[d,"\alpha"] & M''\\
            % & M'\oplus M'' \arrow[ru,swap,"u"]
        % \end{tikzcd}
            % \begin{center}
                % \begin{tikzcd}
                    % G\times\quot{G}{H}\arrow[rd,"(x,yH)\mapsto xyH"]\arrow[d,"\pi\times\id"]\\
                    % \quot{G}{H}\times\quot{G}{H}\arrow[r,"(xH,yH)\mapsto xyH"] & \quot{G}{H}
                % \end{tikzcd}
            % \end{center}
            We have that $\pi\times\id$ is open and $(x,yH)\mapsto xyH$ is open from (i), so the multiplication from $\quot{G}{H}\times\quot{G}{H}\to\quot{G}{H}$ must be open and continuous.
        \item If $x,y\in G$ with $\pi(x)\neq\pi(y)$, then $e\notin xHy^{-1}$.
            Now $xHy^{-1}=L_x(R_{y^{-1}}(H))$ so $xHy^{-1}$ is closed.
            Hence by the last proposition, there is a symmetric open $V$ with $e\in V$ so $V^2\subseteq G\setminus(xHy^{-1})$.
            But then $e\notin (VxH)(VyH)^{-1}=VxHy^{-1}V$: if we had $e=vxhy^{-1}u$ with $v,u\in V$ and $h\in H$, then $v^{-1}u^{-1}=xhy^{-1}\in V^2\cap(xHy^{-1})=\emptyset$, a contradiction.
            Hence $VxH\cap VyH=\emptyset$ so $\pi(Vx)$, $\pi(Vy)$ is a pair of separating neighbourhoods of $\pi(x)$, $\pi(y)$.
    \end{enumerate}
\end{proof}
\begin{corollary}
    $G$ is Hausdorff if and only if there exists $x\in G$ so that $\{x\}$ is closed.
\end{corollary}
\begin{proof}
    In a Hausdorff space, points are closed.
    Conversely, if $\{x\}$ is closed, $\{e\}=L_{x^{-1}}(\{x\})$ is closed and a normal subgroup.
    Then $G\cong\quot{G}{\{e\}}$ is Hausdorff.
\end{proof}
If $(G,\tau)$ is not Hausdorff, then $\{e\}\subsetneq\overline\{e\}$ is the smallest closed subgrup in $G$.
Thus $\overline{\{e\}}\subseteq\bigcap_{x\in G}x\overline{\{e\}}x^{-1}\subseteq\overline{\{x\}}$ so $\overline{\{e\}}$ is normal.
In particular, $\quot{G}{\overline{\{e\}}}$ is Hausdorff.
\begin{definition}
    A \defn{locally compact group} is a Hausdorff topological group $(G,\tau)$ which is locally compact.
\end{definition}
\begin{enumerate}[nl,r]
    \item If there is any $U\in\tau\setminus\{\emptyset\}$ such that $\overline{U}$ is compact, then for any $x\in U$, we have $e\in x^{-1}U\subseteq L_{x^{-1}}(\overline{U})$ so $\overline{x^{-1}U}$ is compact.
        If $V\in\tau$ with $e\in V$ and $\overline{V}$ compact, then for any $x\in H$, $x\in xV$ and $\overline{xV}\subseteq L_x(\overline{V})$ and $\overline{xV}$ is compact.
        In particular, $(G,\tau)$ is locally compact if and only if there is some $U\in\tau\setminus\{\emptyset\}$ with $\overline{U}$ compact.
    \item If $(G,\tau)$ is locally cmpact and $N$ is a closed normal subgroup, then $(\quot{G}{N},\tau_{\quot{G}{N}})$ is locally compact.
        Indeed, $U\in\tau\setminus\{e\}$ with $\overline{U}$ compact, then $\overline{\pi(U)}\subseteq\pi(\overline{U})$ is compact.
\end{enumerate}
\begin{example}
    \begin{enumerate}[nl,r]
        \item If $G$ is any group and $\tau$ is the discrete topology, then $(G,\tau_d)$ is locally compact.
        \item If $((\R,+),\tau_{\norm{\cdot}})$ is locally compact.
        \item If $\{G_i\}_{i\in I}$ is a family of locally compact groups, then $\prod_{i\in I}G_i$ is a locally compact group if and only if all but finitely many $(G_i,\tau_i)$ are compact.
        \item $((\R^n,+),\tau_{\norm{\cdot}})$ is a locally compact group
        \item Suppose $\{F_i\}_{i\in I}$ is an infinite family of finite groups (with discrete topologies), then $G=\prod_{i\in I}F_i$ is a compact group.
            If $F\subset I$ is finite, then $N_F=\{(x_i)_{i\in I}\in G:x_i=e\text{ for }i\in F\}$ is open and a normal subgroup.
            $\{N_F:F\subset I\text{ finite}\}$ is a base for the topology at $e$.
        \item Let $(\mathfrak{k},\tau)$ be a locally compact field..
            Then $\det^{-1}(\mathfrak{k}\setminus\{0\})=\GL_n(\mathfrak{k})\subseteq M_n(\mathfrak{k})\cong\mathfrak{k}^{n^2}$ is an open subset and multiplicative subgroup, and hence locally compact.
            Notice that multiplication is governed by linear equations, and hence continuous, while inversion is continuous thanks to Cramer's rule.

            Here are some common closed subgroups:
            \begin{align*}
                \det^{-1}(\{1\})&=\SL_n(\mathfrak{k})\\
                O_n(\mathfrak{k})=\{x\in\GL_n(\mathfrak{k}):x^{-1}=X^T\}
            \end{align*}
            As a special case, consider $U_n=\{x\in\GL_n(\C):x^*=x^{-1}\}$ is governed by continuous equations, and thus closed in $M_n(\C)$.
            Furthermore, $U_n$ is bounded in $M_n(\C)$, and hence compact.
    \end{enumerate}
\end{example}
\subsection{\texorpdfstring{$p$}{p}-adic Numbers}
Let $p$ be a prime in $\N$.
We will establish product structures and topologies on
\begin{align*}
    \mathbb{O}_p&=\left\{\sum_{k=0}^\infty a_kp^k:a_k\in\{0,1,\ldots,p-1\}\right\}\cong\{0,1,\ldots,p-1\}^{\N}\\
    \mathbb{Q}_p&=\left\{\sum_{k=N}^\infty a_kp^k:N\in\Z, a_k\in\{0,1,\ldots,p-1\}\right\}
\end{align*}
are topological rings and a topological field respectively.
Let $R_p=\prod_{n=0}^\infty\quot{\Z}{p^{n+1}\Z}$ which is a ring under pointwise operations.
\begin{lemma}
    The map $\rho:R_p\times R_p\to[0,1]$ given by
    \begin{align*}
        \rho(x,y)&=\sum_{n\in\N_0}\frac{\rho_n(x_n,y_n)}{p^n} & &\rho_n(x_n,y_n)=\begin{cases}1 &:x_n=y_n\\ 0 &:x_n\neq y_n\end{cases}
    \end{align*}
    is a metric on $R_p$ which satisfies
    \begin{itemize}[nl]
        \item \defname{additively invariant}
            $\rho(x+z,y+z)=\rho(x,y)$ for $x,y,z\in R_p$
        \item $\tau_\rho$ is the product topology
    \end{itemize}
\end{lemma}
\begin{proof}
    Additive invariance is routine.
    Notice that if $\frac{1}{p^m}\geq\epsilon>\frac{1}{p^{m+1}}$, then the open $\epsilon-$ball around a point $x$ is $\{x_0\}\times\cdots\{x_m\}\times\prod_{n=m+1}^\infty\quot{\Z}{p^{n+1}\Z}$ is product-open.
    Conversely, any product-open set is a finite union of such $\epsilon-$balls.
\end{proof}
\begin{corollary}
    The function $\norm{x}_p=\rho(x,0)$ in $R_p$ satisfies
    \begin{itemize}[nl]
        \item $\norm{x}_p=0$ if and only if $x=0$
        \item $\norm{x+y}_p\leq\norm{x}_p+\norm{y}_p$
        \item $\norm{xy}_p\leq\norm{x}_p\norm{y}_p$
        \item $\norm{-x}_p=\norm{x}_p$
    \end{itemize}
    Hence $(R_p,\tau_\rho)$ is a compact topological ring.
\end{corollary}
\begin{proof}
    The properties follow directly using additive invariance.
    To see that $R_p$ is a topological ring, if $(x_\alpha),(y_\alpha)$ have $x=\lim x_\alpha$ and $y=\lim y_\alpha$, then, for example,
    \begin{align*}
        \norm{xy-x_\alpha y_\alpha}_p&\leq\norm{xy-x_\alpha y}_p+\norm{x_\alpha y-x_\alpha y_\alpha}_p\\
                                     &\leq\norm{x-x_\alpha}_p+\norm{y-y_\alpha}_p
    \end{align*}
    as $\norm{y}_p$, $\norm{x_\alpha}_p\leq 1$.
\end{proof}
We now view $\mathbb{O}_p$ as a closed subring of $R_p$.
Define $\alpha:\mathcal{O}_p\to R_p$ be given on $a=\sum_{k=0}^\infty a_kp^k$ by
\begin{equation*}
    \alpha(a) = \left(\sum_{k=0}^n a_kp^k+p^{n+1}\Z\right)_{n=0}^\infty.
\end{equation*}
This map is an injection with range $\alpha(\mathcal{O}_p)=\{(x_n)_{n=0}^\infty\in R_p:x_n=\pi_n(x_{n+1})\text{ for all }n\}$ where $\pi_n:\quot{\Z}{p^{n+2}\Z}\to\quot{\Z}{p^{n+1}\Z}$ is the canonical quotient map.
In fact, this is called an inductive limit with respect to the maps $\pi_n$.
Hence it is routine to show that
\begin{itemize}[nl]
    \item $\alpha(\mathbb{O}_p)$ is a subring of $R_p$, and
    \item $\alpha(\mathbb{O}_p)$ is closed in $R_p$ (just check net limits in product topology)
\end{itemize}
If $a,b\in\mathbb{O}_p$, define $a+b=\alpha^{-1}(\alpha(a)+\alpha(b))$.
\begin{remark}
    \begin{enumerate}[nl,r]
        \item $1+\sum_{k=1}^\infty 0\cdot p^k$ is the multiplicative identity in $\mathbb{O}_p$.
            Then $-1=\sum_{k=0}^\infty(p-1)p^k$.
        \item If $n\in\N$, we can uniquely write $n=\sum_{k=0}^{m(n)} a_kp^k$ with $a_k\in\{0,\ldots,p-1\}$.
            Then $n\cdot 1=\sum_{k=0}^{m(n)}a_kp^k\in\mathbb{O}_p$.
            In particular, $n\mapsto n\cdot 1:\N\to\mathbb{O}_p$ is an additive semigroup homomorphism with dense ring.
            Hence $n\mapsto n\cdot 1:\Z\to \mathbb{Q}_p$ has dense range.
            We call $\mathbb{O}_p$ the \defn{$p-$adic integers}.
    \end{enumerate}
\end{remark}

Let $a=\sum_{k=0}^\infty a_kp^k$ in $\mathbb{O}_p$.
Let
\begin{align*}
    \nu_p(a) &= \min\{k\in\N_0:a_k\neq 0\},\min\emptyset=-\infty\\
    |a|_p=p^{-\nu_p(a)},p^{-\infty}=0
\end{align*}
and notice that $|a|_p=\norm{\alpha(a)}_p$.
However, $|a|_p$ has even nicer properties:
\begin{enumerate}[nl,r]
    \item $|a|_p=0$ if and only if $a=0$
    \item $\nu_p(ab)=\nu_p(a)+\nu_p(b)$.
        Thus $|ab|_p=|a|_p|b|_p$
    \item $\nu_p(a+b)\geq\min\{\nu_p(a),\nu_p(b)\}$.
        Thus $|a+b|_p\leq\max\{|a|_p,|b|_p\}\leq|a|_p+|b|_p$
\end{enumerate}
Notice that (i) and (ii) imply that $\mathbb{O}_p$ is an integral domain.
\begin{proposition}
    The multiplicative unit group of $\mathbb{O}_p$ is $\mathbb{O}_p\setminus p\mathbb{O}_p=\{a\in\mathbb{O}_p:|a|_p=1\}$.
    Hence $\mathbb{O}_p^\times$ is open and a topological group.
\end{proposition}
\begin{proof}
    The second equality is trivial.
    If $a\in\mathbb{O}_p^\times$, then $|a|_p$, $|a^{-1}|_p\leq 1$ and $1=|1|_p=|aa^{-1}|_p=|a|_p|a^{=1}|_p$, so $|a|_p=1$.
    If $|a|_p=1$, let
    \begin{equation*}
        x=\alpha(a)=\left(\sum_{k=0}^n a_kp^k+p^{n+1}\Z\right)_{n=0}^\infty\in R_p.
    \end{equation*}
    Then $x_n\in(\Z/p^{n+1}\Z)^\times$ since $p\nmid\sum_{k=0}^n a_kp^k$ in $\Z$.
    Hence there is $y_n\in(\Z/p^{n+1}\Z)^\times$ so $x_ny_n=1+p^{n+1}\Z$ and thus
    \begin{equation*}
        1+p^n\Z=\pi_N(1+p^{n+2}\Z)=\pi_n(x_{n+1}y_{n+1})=\pi(x_{n+1})\pi(y_{n+1})=x_n\pi_n(y_{n+1})
    \end{equation*}
    so that $\pi_n(y_{n+1})=y_n$.
    Thus if $y\in\alpha(\mathbb{O}_p)$, i.e. $y=\alpha(b)$ with $ab=\alpha^{-1}(\alpha(a)\alpha(b))=\alpha^{-1}((1+p^{n+1}\Z)_{n=0}^\infty)=1$ and $a\in\mathbb{O}_p^\times$.

    Since $p\mathbb{O}_p$ is closed, we see that $\mathbb{O}_p^\times$ is open in $\mathbb{O}_p$.
    Continuity of multiplication follows (ii).
    Finally, if $a,b\in\mathbb{O}_p$,
    \begin{equation*}
        |a^{-1}-b^{-1}|_p=|a|_p|a^{-1}-b^{-1}|_p|b|_p=|b-a|_p
    \end{equation*}
\end{proof}
\begin{corollary}
    Each ideal in $\mathbb{O}_p$ is of the form $p^k\mathbb{O}_p$ for $k\in\N_0$.
\end{corollary}
\begin{proof}
    If $I$ is an ideal in $\mathbb{O}_p$, then let $k(I)=\min\{k\in\N_0:\nu_p(a)=k\text{ for some }a\in I\}$.
    Then there is $a\in I$ with $\nu_p(a)=k(I)$, so $p^{-k(I)}\in a\mathbb{O}_p^\times\subseteq a\mathbb{O}_p\subseteq I$.
    Thus $p^{-k(I)}\mathbb{O}_p\subseteq I$.
    Clearly $I\subseteq p^{-k(I)}\mathbb{O}_p$ as well.
\end{proof}
We now extend the valuation and norm to $\Q_p$.
If $a=\sum_{k\in\Z}a_kp^k\in\Q_p$, let $\nu_p(a)=\min\{k\in\Z:a_k\neq 0\}$ and $|a|_p=p^{-\nu_p(a)}$.
Then for $a\in\Q_p\setminus\{0\}$ admits (formal) factorization
\begin{equation*}
    a=\sum_{k=\nu_p(a)}^\infty a_kp^k=p^{\nu_p(a)}\sum_{k=\nu_p(a)}^\infty a_kp^{k-\nu_p(a)}=p^{\nu_p(a)}\underbrace{\sum_{k=0}^\infty a_{k+\nu_p(a)}p^k}_{:=a'\in\mathbb{O}_p^\times}
\end{equation*}
Thus, if $a,b\in\Q_p\setminus\{0\}$, we define multiplication and addition by $ab=p^{\nu_p(a)+\nu_p(b)}a'b'$ and $a^{-1}=p^{-\nu_p(a)}(a')^{-1}$.
Furthermore, assuming $\nu_p(a)\leq\nu_p(b)$, we define addition by
\begin{equation*}
    a+b=p^{\nu_p(a)}(a'+p^{\nu_p(b)-\nu_p(a)}b')
\end{equation*}
and $0+a=a$, $0a=0$.
Notice that $|ab|_p=|a|_p|b|_p$, $|1/a|_p=1/|a|_p$ and if $\nu_p(a)\leq \nu_p(b)$, $|a+b|_p=p^{-\nu_p(a)}|a'+p^{\nu_p(b)-\nu_p(a)}b'|_p\leq|a|_p$ so, generally, $|a+b|_p\leq\max\{|a|_p,|b|_p\}$.
Also, if $|a|_p=0$, then $|a|=0$.
Thus $(\Q_p,|\cdot|_p)$ is a ``normed field'', and hence a topological field.

Note that
\begin{equation*}
    \mathbb{O}_p=\{a\in\Q_p:|a|_p\leq 1\}=\{a\in\Q_p:|a|_p<p\}
\end{equation*}
is a compact open neighbourhood of $0$, so $\Q_p$ is locally compact.
Moreover, each $p^k\Q_p=\{a\in\Q_p:|a|_p<p^{k-1}\}$ is a closed and open ball about $0$.
\subsection{Haar Integral and Haar Measure}
Let $G$ be a locally compact group.
Define for $f:G\to\C$, $x\in G$, $f\cdot x=f\circ L_x$, and $x\cdot f=f\circ R_x$.
Notice that $(f,x)\mapsto f\cdot x$, as an adjoint action, is a right (group) action of $G$ on functions.
Likewise, $(x,f)\mapsto x\cdot f$ is a left action.
\begin{proposition}
    If $f\in C_c(G)$, then $\lim_{x\to e}\norm{f\cdot x-f}_\infty=0=\lim_{x\to e}\norm{x\cdot f-f}_\infty$.
\end{proposition}
\begin{proof}
    Let $\epsilon>0$, $W$ a symmetric relatively compact neighbourhood of $e$, and let $K=\overline{W}\supp f$.
    Given $y\in K$, $x\mapsto|f(xy)-f(y)|$ is continuous, so there is a neighbourhood $U_y$ of $e$ so $|f(xy)-f(y)|<\epsilon/2$ whenever $x\in U_y$.
    Let $V_y$ be a symmetric neighbourhood of $e$ so that $V_y^2\subseteq U_y$.
    Then $K\subseteq\bigcup_{y\in K}V_yy$ so by compactness get some finite subcover indexed by $\{y_1,\ldots,y_n\}$.
    Set $V=\left(\bigcap_{j=1}^n V_{y_j}\right)\cap W$ and note that $V$ is a symmetric relatively compact neighbourhood of $e$.

    If $x\in V$, then for $y\in K$ we have $y\in V_{y_j}y_j$ for some $j$, i.e. $yy_j^{-1}\in V_{y_j}$, and hence
    \begin{equation*}
        xy=xyy_j^{-1}y_j\in VV_{y_j}y_j\subseteq V_{y_j}^2y_j\subseteq U_{y_j}y_j.
    \end{equation*}
    Note that $xy=x'y_j$ for some $x'\in U_{y_j}$.
    Similarly, since $y_jy^{-1}\in U_{y_j}$, we $y_j=x''y$ for some $x''\in U_{y_j}$.
    Thus by definition of $U_{y_j}$, we have
    \begin{equation*}
        |f(xy)-f(y)|\leq|f(x'y_j)-f(y_j)|+|f(x''y)-f(y)|<\epsilon.
    \end{equation*}

    Otherwise, if $y\notin K$, then $Wy\cap\supp(f)=\emptyset$: by contrapositive, if $x\in\supp(f)\cap Wy$, then $yx^{-1}\in W$ so $y\in\overline{W}\supp f=K$.
    Thus for $x\in V\subseteq W$, we have $f(xy)=0=f(y)$, and the desired result follows.
\end{proof}
\begin{theorem}[Existence of Haar Integral]
    There exists a linear functional $I:C_c(G)\to\C$ satisfying
    \begin{itemize}[nl]
        \item \defname{positivity} $I(f)>0$ if $f\in C_c^+(G)=\{g\in C_c(G)\setminus\{0\}:g\geq 0\}$.
        \item \defname{left invariance} $I(f\cdot x)=I(f)$ for $f\in C_c(G)$, $x\in G$.
    \end{itemize}
\end{theorem}
Let for $f,\phi\in C_c^+(G)$
\begin{equation*}
    (f:\phi)=\inf\left\{\sum_{j=1}^n c_j:\text{there are }x_1,\ldots,x_n\in G,c_i>0,n\in\N \text{ s.t. }f\leq\sum_{j=1}^n c_j\phi\cdot x_j\right\}
\end{equation*}
Notive that $0<\frac{\norm{f}_\infty}{\norm{\phi}_\infty}\leq(f:\phi)$.
Also, if $U=\{x\in G:\phi(x)>\frac{1}{2}\norm{\phi}_\infty\}$, then $\supp f$ is covered by finitely many $x^{-1}U$, $x\in G$, and thus $(f:\phi)<\infty$.
\begin{claim}
    For $f,g$ in $C_c^+(G)$, we have
    \begin{enumerate}[nl,r]
        \item $(f\cdot x:\phi)=(f:\phi)$ for $x$ in $G$
        \item $(cf:\phi)=c(f:\phi)=(f:\frac{1}{c}\phi)$ for $c>0$
        \item $(f+g,\phi)\leq (f:\phi)+(g:\phi)$.
        \item $(f:\phi)\leq (f:g)(g:\phi)$
    \end{enumerate}
\end{claim}
\begin{nmproof}
    Note that (i) and (ii) are straightforward.
    To see (iii) and (iv), consider
    \begin{align*}
        f&\leq\sum_{j=1}^n c_j\phi\cdot x_j & g&\leq\sum_{j=n+1}^Nc_j\phi\cdot x & f&\leq\sum_{k=1}^m b_kg\cdot y_k
    \end{align*}
    so that $f+g\leq\sum_{j=1}^Nc_j\phi\cdot x_k$ and $(f+g:\phi)\leq\sum_{j=1}^n c_j+\sum_{j=n+1}^N c_j$, giving (iii).
    To get (iv), note $f\leq\sum_{k=1}^m b_k\sum_{j=n+1}^N c_j\phi\cdot(x_jy_k)$ so $(f:\phi)\leq\sum_{k=1}^m b_k\sum_{j=k+1}^Nc_j$, giving (iv).
\end{nmproof}
We wish to ``homogonize'' the effect of $\phi$.
Fix $\psi_0\in C_c^+(G)$ and let $I_\phi(f)=\frac{(f:\phi)}{(\psi_0:\phi)}$.
Then $I_\phi:C_c^+(G)\to\R_{\geq 0}$ is
\begin{enumerate}[nl]
    \item[(i')] left translation invariant
    \item[(ii')] $\R_{\geq 0}-$homogenous
    \item[(iii')] subadditive.
\end{enumerate}
by using the claim above directly.
Thus by (iv), $(\psi_0:\phi)\leq(\psi_0:f)(f:\phi)$ and $(f:\phi)\leq(f:\psi_0)(\psi_0:\phi)$, giving
\begin{enumerate}[nl]
    \item[iv'] $0<\frac{1}{(\psi_0:f)}\leq I_\phi(f)\leq (f:\psi_0)$.
\end{enumerate}
\begin{claim}\label{cl:1a}
    If $f,g\in C_c^+(G)$, $\epsilon>0$, there is a neighbourhood $V$ of $e$ such that
    \begin{equation*}
        I_\phi(f)+I_\phi(g)<I_\phi(f+g)+\epsilon
    \end{equation*}
    whenever $\phi\in C_c^+(G)$ with $\supp(\phi)\subseteq V$.
\end{claim}
\begin{nmproof}
    Let $k\in C_c^+(G)$ be so $k|_{\supp(f+g)}=1$ and let $\delta>0$.
    Take $h=f+g+\delta k$ and $f'=\frac{f}{h}$, $g'=\frac{g}{h}\in C_c^+(G)$.
    Uniform continuity of $f',g'$ provides a neighbourhood $v$ of $e$ such that $|f'(x)-f'(y)|<\delta$, $|g'(x)-g'(y)|<\delta$ if $y^{-1}x\in V$.
    If $\phi\in C_c^+(G)$, $\supp(\phi)\subseteq V$, and $x_1,\ldots,x_n$ in $G$, $c_1,\ldots,c_n>0$ satisfy
    \begin{equation*}
        h\leq\sum_{j=1}^n c_j\phi_j\cdot x_j^{-1}
    \end{equation*}
    then for $x$ in $G$
    \begin{align*}
        f(x) = f'(x)h(x) &\leq\sum_{j=1}^n f'(x)c_j\phi(x_j^{-1}x)\\
                         &\leq\sum_{j=1}^n[f'(x_j)+\delta]c_j\phi(x_j^{-1}x)
    \end{align*}
    by properties of $f',g'$ proven above and $\supp(\phi)\subseteq C$.
    Likewise,
    \begin{equation*}
        g\leq\sum_{j=1}^n[g'(x_j)+\delta]c_j\phi\cdot x_j^{-1}.
    \end{equation*}
    Now $f'+g'=(f+g)/h=\frac{f+g}{f+g+\delta k}\leq 1$ so
    \begin{align*}
        (f:\phi)+(g:\phi)&\leq\sum_{j=1}^n[f'(x_j)+\delta]c_j+\sum_{j=1}^n[g'(x_j)+\delta]c_j\\
                         &\leq\sum_{j=1}^n[1+2\delta]c_j
    \end{align*}
    and $(f:\phi)+(g:\phi)\leq (1+2\delta)(h:\phi)$.
    Divide by $(\psi_0:\phi)$ and (iii') and (iv') above to get
    \begin{equation*}
        I_\phi(f)+I_\phi(g)\leq(1+2\delta)I_\phi(h)\leq(1+2\delta)[I_\phi(f+h)+\delta I_\phi(k)]
    \end{equation*}
    Thus with sufficiently small $\delta$, $I_\phi(f)+I_\phi(g)<I_\phi(f+g)+\epsilon$.
\end{nmproof}
We are now in position to complete the proof.
\begin{claim}\label{cl:i2}
    Construction of the functional $I$.
\end{claim}
\begin{proof}
    Inequality (iv') tells us that
    \begin{equation*}
        x_\phi=(I_\phi(f))_{f\in C_c^+(G)}\in\prod_{f\in C_c^+(G)}[\frac{1}{(\psi_0:f)},(f:\psi_0)]=X
    \end{equation*}
    which, by Tychonoff, is compact.
    For $\phi,\phi'$ in $\Phi=\{\psi\in C_c^+(G):\psi(e)=1\}$, $\phi\leq\phi'$ if $\phi\geq\phi'$ pointwise, which is a preorder.
    Notice that $\phi\phi'\leq\phi\wedge\phi'$ (pointwise minimum), so that $(\Phi,\leq)$ is directed.
    Hence $(x_\phi)_{\phi\in\Phi}$ admits a converging subnet $x=\lim_{\mu\in M}x_{\phi_\mu}$ in $X$.

    Write $x=(I(f))_{f\in C_c^+(G)}$, so $I(f)=\lim_{\mu\in M}I_{\phi_\mu}(f)$ for each $f\in C_c^+(G)$.
    Then it follows that from (i'), (ii'), and (iii') that for $f,g$ in $C_c^+(G)$, we have
    \begin{align*}
        I(F\cdot x)&=I(f) & I(cf)&=cI(f) & I(f+g) &\leq I(f)+I(g)
    \end{align*}
    for $x\in G$, $c>0$.
    Moreover, by cofinality, if $V$ is a neighbourhood of $e$, then $\supp(\phi_\mu)\subseteq V$ for $\mu$ sufficiently large in $M$.
    Hence given $\epsilon>0$, by \cref{cl:1a}, $I_{\phi_\mu}(f)+I_{\phi_\mu}(g)<I_{\phi_\mu}(f+g)+\epsilon$ for $\mu$ sufficiently large in $M$.
    Since $\epsilon>0$ as arbitrary, we have $I(f)+I(g)\leq I(f+g)$.

    Let $I(0)=0$.
    If $f\in C_c^{\R}(G)$ and $f=f_1-f_2=g_1-g_2$ with $f_1,f_2,g_1,g_2\geq 0$, then $h=f_1+g_2=g_1+f_2$ satisfies that $I(h)=I(f_1)+I(g_2)=I(g_1)+I(f_2)$ and hence we may define $I(f)=I(f_1)-I(f_2)$, which do not depend on the choice of $f_1,f_2$.
    One may check that $I:C_c^{\R}(G)\to\R$ is $\R-$linear.
    Finally, if $f\in C_c(G)$, let $I(f)=I(\Re f)+iI(\Im f)$.
    It is left as an exercise to verify that $I:C_c(G)\to\C$ is $\C-$linear.

    Finally, the fact that $I(f\cdot x)=I(f)$ for $f\in C_c(G)$ and $x\in G$ follows for $f$ in $C_c^+(G)$ as above.
    If $f\in C_c^+(G)$, then (iv') tellus us that $I(f)\geq\frac{1}{(\psi_0:f)}>0$.
\end{proof}
\begin{remark}
    \begin{enumerate}[nl,r]
        \item In \cref{cl:i2}, $I_\phi(\psi_0)=1$ so $I(\psi_0)=1$.
        \item If $G$ is discrete, then $\psi_0=1_{\{e\}}=\min\Phi$.
            Then $I_{\psi_0}(f)=\frac{(f:\psi_0)}{(\psi_0:\psi_0)}=\sum_{x\in G}f(x)$ for $f\in C_c^+(G)$.
        \item If $G=\R$, let $\psi_0$ be the linear function which is $0$ on $(-\infty,-1/2-\delta)\cup(1/2+\delta,\infty)$, $1$ on $(-1/2+\delta,1/2-\delta)$, and continuied linearly on the remainder.
            Notice that $(\psi_0,\phi_n)\approx n$, so $\frac{(f:\phi_n)}{(\psi_0:\phi_n)}$ is approximately the Riemann-Darboux upper sum.
        \item Examine $\mathbb{O}_p$, $\psi_0=1_{\mathbb{O}_0}$, $\psi_n=1_{p^n\mathbb{O}_p}$.
    \end{enumerate}
\end{remark}
\begin{theorem}[Harr Measure]
    Let $\mathcal{B}(G)$ denote the Borel $\sigma-$algebra on $G$.
    Then there is a Radon measure $m:\mathcal{B}(G)\to[0,\infty]$ such that
    \begin{itemize}[nl]
        \item $m$ is left invariant: $m(xE)=m(E)$ for $x\in G$, $E\in\mathcal{B}(G)$
        \item $m(U)>0$ for $U\in\tau\setminus\{\emptyset\}$.
    \end{itemize}
\end{theorem}
\begin{proof}
    The Riesz Representation Theorem provides a measure $m:\mathcal{B}(G)\to[0,\infty]$ for which
    \begin{equation*}
        \int_G f \d{m}=I(f)
    \end{equation*}
    for $f\in C_c(G)$.
    Notice that
    \begin{equation*}
        \int_G f\cdot x\d{m}=I(f\cdot x)=\int_G f
    \end{equation*}
    for any $x\in G$, $f\in C_c(G)$.
    Thus if $U\in\tau$, $\supp f\subseteq U$ if and only if $\supp(f\cdot x)\subseteq x^{-1}U$ for $x\in G$ and $f\in C_c(G)$.
    Thus
    \begin{align*}
        m(U) &= \sup\{I(f):f\in C_c(G),0\leq f\leq 1\text{ and }\supp(f)\subseteq U\}\\
             &= \sup\{I(f\cdot x):f\in C_c(G),0\leq f\leq 1\text{ and }\supp(f)\subseteq U\}\\
             &= \sup\{I(g):g\in C_c(G),0\leq g\leq 1,\supp(g)\subseteq x^{-1}U\\
             &= m(x^{-1}U).
    \end{align*}
    Therefore, for any $E\in\mathcal{B}(G)$, we have
    \begin{align*}
        m(E)&=\inf\{m(U):E\in U,U\in\tau\}\\
            &=\inf\{m(xU):E\subseteq U,U\in\tau\}\\
            &=\inf\{m(xU):xE\subseteq xU,U\in\tau\}=m(xE).
    \end{align*}
    Finally, if $U\in\tau\setminus\{\emptyset\}$, there is $f\in C_c^+(G)$ with $0\leq f\leq 1$ and $\supp(f)\subseteq U$, so $m(U)\geq I(f)>0$.
\end{proof}
\begin{remark}
    If $E\in\mathcal{G}(G)$, $m(E)<\infty$, then $m(E)=\sup\{m(K):K\subseteq E,K\text{ compact}\}$.
    Inner regularity need not hold on infinite measure sets: taking $G=\R_d\times\R$, then $\R_d\times\{0\}$ is closed, and thus Borel.
    However, $m(E)=\infty$ while $m(K)=0$ for each compact $K\subset E$.
\end{remark}
\begin{theorem}[Uniqueness of Haar Measure]
    Let $m':\mathcal{B}(G)\to[0,\infty]$ be any Radon measure such that $m(xE)=m'(E)$ for $x\in G$ and $E\in\mathcal{B}(G)$.
    Then there is $c\geq 0$ such that $m'=cm$.
\end{theorem}
\begin{proof}
    Fix a symmetric neighbourhood $W=W^{-1}$ of $e$ with $\overline{W}$ compact.
    Given $f\in C_c^+(G)$, $\epsilon>0$, and $U$ a neighbourhood of $e$ such that $\norm{f-x\cdot f}_\infty<\epsilon$.
    Let $V=U\cap W$.
    Then let $x\in G$, and for any $x'\in G$ with $x'x^{-1}\in V$, we have
    \begin{align*}
        \left\lvert\int_G f(yx)dm'(y)-\int_G f(yx')dm'(y)\right\rvert&\leq\norm{x\cdot f-x'\cdot f}_\infty m(\supp(f)x^{-1}\cup\supp(f)Vx^{-1})\\
                                                                     &<\epsilon m(\supp(f)x^{-1}\cup\supp(f)Wx^{-1})
    \end{align*}
    and hence $x\mapsto\int_G x\cdot fdm'$ is continuous at each point in $G$.
    Thus
    \begin{equation*}
        D_f(x) = \frac{\int_Gx\cdot fdm'}{\int_G fdm}
    \end{equation*}
    defines a continuous function on $G$.

    If $f,g\in C_c^+(G)$, then $(x,y)\mapsto f(x)g(y^{-1})$ is non-negative, continuous, Borel measurable, and compactly supported on $G\times G$.
    Then by left-invariance and Tonelli's theorem,
    \begin{align*}
        \left(\int_Gfdm\right)\cdot\left(\int_G g(y^{-1})dm'(y)\right) &= \int_G\int_Gf(x)g(y^{-1})dm'(y)dm(x)\\
                                                                       &= \int_G\int_G f(x)g((x^{-1}y)^{-1})dm'(y)dm(x)\\
                                                                       &= \int_G\int_G f(x)g(y^{-1}x)dm(x)dm'(y)\\
                                                                       &= \int_G\int_G f(yx)g(x)dm(x)dm'(y)\\
                                                                       &= \int_Gg(y)\left(\int_G f(yx)dm'(y)\right)dm(x)
    \end{align*}
    Thus,
    \begin{equation*}
        \int_Gg(y^{-1})dm'(y) = \int_G g(x)D_f(x)dm(x).
    \end{equation*}
    But if we have any other $f'\in C_c^+(G)$, then we would have
    \begin{equation*}
        \int_Gg(x)D_f(x)dm(x)=\int_Gg(y^{-1})dm'(y)=\int_Gg(x)D_{f'}(x)dm(x)
    \end{equation*}
    so it follows that $D_f=D_{f'}$ $m-$a.e.
    Since $D_f$, $D_{f'}$ are continuous, we see that $D_f=D_{f'}$ everywhere.
    In particular, $D_f(e)=D_{f'}(e)$, or that
    \begin{equation*}
        \frac{\int_G fdm'}{\int_G fdm}=D_f(e)=D_{f'}(e)=\frac{\int_G f'dm'}{\int_G f'dm}
    \end{equation*}
    Let $c$ denote this common value, so $c\int_G fdm=\int_G fdm'$ for $f\in C_c^+(G)$.
    Hence $m'=cm$.
\end{proof}
\begin{example}
    \begin{enumerate}[nl,r]
        \item If $G$ is discrete, then $C_c(G)$ is composed of functions with finite support, and $m(E)$ is (a multiple of) the counting measure.
            In the finite case, we normalize by $|G|$.
        \item If $G=\R^n$, $I(f)=\int_{\R^n}f$ and $m$ is $n$-dimensional Lebesgue measure.
        \item Let $G=\GL_n(\R)$.
            \begin{enumerate}[nl,a]
                \item If $t\in\GL_n(\R)$, then for $f\in C_c(\R^n)$,
                    \begin{equation*}
                        \int_{\R^n}f\circ t(y)dy=\frac{1}{|\det t|}\int_{\R^n}f(y)dy.
                    \end{equation*}
                    Indeed, show that this holds for an elementary matrix $t$, and $\GL_n(\R)$ is the algebra generated by these elements.
                \item If $X\in\GL_n(\R)$, then $L_X:M_n(\R)\to M_n(R)$ is isomorphic to
                    \begin{equation*}
                        \begin{pmatrix}y_1\\\vdots\\y_n\end{pmatrix}\mapsto\begin{pmatrix}xy_1\\\vdots\\xy_n\end{pmatrix}:(\R^n)^n\to(\R^n)^n
                    \end{equation*}
                    and hence $\det L_X=\det X^n$.
                    Thus if $f\in C_c(M_n(\R))$, we have that
                    \begin{equation*}
                        \int_{M_n(\R)}f(xy)dy=\int_{M_n(\R)}f\circ L_X(y)dy=\frac{1}{|\det X|^n}\int_{M_n(\R)}f(y)dy.
                    \end{equation*}
                    Now since $\GL_n(\R)$ is open in $M_n(\R)$, so $C_c(\GL_n(\R))\subset C_c(M_n(\R))$, and we define for $f\in C_c(\GL_n(\R))$
                    \begin{equation*}
                        I(f)=\int_{\GL_n(\R)}f(y)\frac{1}{|\det y|^n}dy.
                    \end{equation*}
                    Then for $x\in\GL_n(\R)$, we have
                    \begin{align*}
                        I(f\cdot x)&=\int_{\GL_n(\R)}f(xy)\frac{1}{|\det xy|^n}\cdot|\det x|^ndy\\
                                   &=\int_{\GL_n(\R)}f(y)\frac{1}{|\det y|^n}\cdot\frac{|\det x|^n}{|\det x|^n}dy=I(f)
                    \end{align*}
                    If $M_{\GL_n(\R)}$ is the measure associated with $I$, then with $m$ the Lebesgue measure on $M_n(\R)\cong\R^{n^2}$, we have
                    \begin{equation*}
                        \frac{dm_{\GL_n(\R)}}{dm}(y)=\frac{1}{|\det y|^n}
                    \end{equation*}
                \item If we take $\R^\times\cong\GL_1(\R)$, then
                    \begin{equation*}
                        I(f)=\int_{\R^\times}f(y)\frac{dy}{|y|}
                    \end{equation*}
                \item Consider $\C^\times=\C\setminus\{0\}\subseteq\C\cong\R^2$.
                    Then $[L_{x+iy}]_{(1,i)}=\begin{pmatrix}x&-y\\y&x\end{pmatrix}$ so that $\det L_{x+iy}=|x+iy|^2$.
                    Thus we get an integral on $G=\C^\times$ by
                    \begin{equation*}
                        I(f)=\int_{\C^\times}f(z)\frac{dz}{|z|^2}
                    \end{equation*}
                \item On $\GL_n(\C)\subset\GL_{2n}(\R)$, we likewise find Haar integral
                    \begin{equation*}
                        I(f)=\int_{\GL_n(\C)}f(y)\frac{1}{|\det y|^{2n}}dy.
                    \end{equation*}
            \end{enumerate}
        \item Suppose $G$ admits an open (hence closed) subgroup $H$.
            If $m$ is a Haar measure on $G$, then $m_H=m|_{\mathcal{B}(H)}$ is a Haar measure on $H$.
            Let $T$ be a transversal for left cosets of $H$ (A of C), so $G=\bigcup_{t\in T}tH$.
            If $U\subset G$ is open with $m(U)<\infty$, then
            \begin{align*}
                \{t\in T:U\cap tH\neq\emptyset\} &= \{t\in T:M(U\cap tH)>0\}\\
                                                 &= \bigcup_{n=1}^\infty\{t\in T:m(U\cap tH)<\frac{1}{n}\}
            \end{align*}
            is countable, so if $E\in\mathcal{B}(G)$, $m(E)<\infty$, $E\subseteq\bigcup_{j=1}^\infty t_jH$ and then
            \begin{align*}
                m(E)&=\sum_{j=1}^\infty m(E\cap t_jH)=\sum_{j=1}^\infty m(t_j^{-1}(E\cap t_j H))\\
                    &=\sum_{j=1}^\infty m_H((t_j^{-1}E)\cap H)
            \end{align*}
        \item Suppose $H$ is a closed, non-open subgroup of $G$.
            We wish to see that for compact $K\subseteq H$, $m(K)=0$.
            \begin{enumerate}[nl,a]
                \item Given open $U$ with $K\subseteq U$, then there is open $V$ with $e\in V$ so $VK\subseteq U$.
                    Indeed, for $x\in K$, find open $U_x$ with $e\in U_x$, so $U_xx\subseteq U$.
                    Find open $V_x$, $e\in V_x$, so $V_x^2\subseteq U_x$, then $K\subseteq\bigcup_{j=1}^n V_{x_j}x_j$ where $x_1,\ldots,x_j\in K$.
                    Let $V=\bigcap_{j=1}^n V_{x_k}$.
                    If $x\in K$, $x\in V_{x_j}x_j$ for soe $j$ so $Vx\subseteq VV_{x_j}x_j\subseteq V_{x_j}^2x_j\subseteq U_{x_j}x_j\subseteq U$, i.e. $VK=\bigcup_{x\in K}V_x\subseteq U$.
                \item Suppose we had compact $K\subseteq H$ such that $m(K)>0$.
                    We may find open $U$ so $K\subseteq U$ and $m(U)<2m(K)$ (by outer regularity).
                    Take $V$ as above.
                    Since $H$ is non-open, there is $x\in V\setminus H$.
                    Then
                    \begin{itemize}[nl]
                        \item $K\cap xK=\emptyset$ as $K\subseteq H$, while
                        \item $K\cup xK\subseteq U$.
                    \end{itemize}.
                    Thus $2m(K)=m(K\cup xK)\leq m(U)<2m(K)$, a contradiction.

                    Thus, a closed non-open subgroup $H$ of $G$ is always $m$-locally null.
                    Hence, if $G$ is $\sigma$-compact, then closed non-open $H$ are $m$-null.
                    However, if $G=\R\times\R_d$, $H=\{0\}\times\R_d$ is closed, $m$-locally null, but not $m$-null.
            \end{enumerate}
        \item The measure on $(\Q_p,+)$ is determined by $(\mathbb{O}_p,+)$.
            Likewise, the measure $\GL_n(\Q_p)$is determined by $\GL_n(\mathbb{O}_p)=\{a\in M_n(\mathbb{O}_p):\det a\in\mathbb{O}_p^\times\}$
        \item On $\GL_n(\mathbb{O}_p)$, we have Haar integral
            \begin{equation*}
                I(f)=\int_{\GL_n(\Q_p)}f(y)\frac{1}{|\det y|_p^n}dy
            \end{equation*}
        \item $G$ is compact if and only if $m(G)<\infty$.
            The forward is clear since $m$ is Radon.
            If $G$ is not compact, let $U$ be a open neighbourhood of $e$ so $\overline{U}$ is compact, so $0<m(U)<\infty$.
            For any compact set $K$, $KU\subseteq K\overline{U}$ is compact, hence $KU\subsetneq G$.
            Inductively find $(x_n)_{n=1}^\infty\subset G$ so $x_{n+1}\notin\{x_1,\ldots,x_n\}U$.
            Notice that $x_jV\cap x_kV=\emptyset$ for $j\neq k$ for $V$ a neighbourhood of $e$ with $V=V^{-1},V^2\subset U$.
            Then $m(G)\geq nm(V)$ for any $n\in\N$, so $m(G)=\infty$.
    \end{enumerate}
\end{example}
Let $G$ be a locally compact group equipped with Haar measure $m$.
Then
\begin{equation*}
    L^1(G)=\{f:G\to\C:f\text{ measurable},\norm{f}_1=\int_G|f|dm<\infty\}/\sim_{m\quad a.e.}
\end{equation*}
This is a Banach space.
Recall that by definition of the Lebesgue integral
\begin{equation*}
    S^1(G)=\spn\{\chi_E:E\in\mathcal{B}(G),m(E)<\infty\}/\sim_{m\quad a.e.}
\end{equation*}
If $0<m(E)<\infty$, then, given $\epsilon>0$, there are compact $K\subseteq E$ and open $U\supseteq E$.
Hence Urysohn's lemma provides $f\in C_c^+(G)$ such that $f|_L=1$, $\supp(f)\subseteq U$, and $0\leq f\leq 1$.
Hence $\norm{f-1_E}_1<\epsilon$.
Note that if $f,g\in C_c(G)$, then $f=g$ $m$ a.e. if and only if $f=h$.
Thus $C_c(G)\subseteq L^1(G)$ is dense.
\subsection{The Modular Function}
If $x\in G$, define $m_x:\mathcal{B}(G)\to[0,\infty]$ by $m_x(E)=m(Ex)$.
Then since $R_x$ is a homeomorphism, one may verify that
\begin{itemize}[nl]
    \item $m_x$ is left invariant,
    \item $m_x(U)=m(Ux)>0$ if $U$ is non-empty and open, and
    \item $m_x(K)=m(Kx)<\infty$ if $K$ is compact.
\end{itemize}
Hence by uniqueness of Haar measure there exists some function $\Delta:G\to\R^\times$ such that $m_x=\Delta(x)m$.
In fact, $\Delta$ is a group homomorphism.
To see this, if $E\in\mathcal{B}(G)$ with $0<m(E)<\infty$ and $x,y\in G$, then
\begin{equation*}
    \Delta(xy)m(E)=m(Exy)=\Delta(y)m(Ex)=\Delta(x)\Delta(y)(E).
\end{equation*}
Denote by $x\cdot f$ the function $x\cdot f(y)=f(yx)$.
We then have the following result:
\begin{proposition}
    \begin{enumerate}[nl,r]
        \item For any $f\in L^1(G)$ and $x\in G$, $x\cdot f\in L^1(G)$ with
            \begin{equation*}
                \int_G f\d{m}=\Delta(x)\int_G x\cdot f\d{m}.
            \end{equation*}
        \item $\Delta$ is a continuous function.
    \end{enumerate}
\end{proposition}
\begin{proof}
    \begin{enumerate}[nl,r]
        \item Let $E\in\mathcal{B}(G)$ with $m(E)<\infty$.
            Then
            \begin{equation*}
                \Delta(x)\int \idc{E}\d{m}=\Delta(x)m(E)=m(Ex)=\int \idc{Ex}\d{m}=\int_G x^{-1}\cdot \idc{E}\d{m}
            \end{equation*}
            since $\idc{Ex}=x^{-1}\cdot\idc{E}$.
            Thus replacing $x$ by $x^{-1}$, we have
            \begin{equation*}
                \int\idc{E}\d{m}=\Delta(x)\int x\cdot \idc{E}\d{m}
            \end{equation*}
            so that the desired result holds for characteristic functions.
            Then by density of simple functions in $L^1$ and applying dominated convergence, the result holds for any $f\in L^1$.
        \item Let $f\in C_c^+(G)$, $\epsilon>0$, and $V=V^{-1}$ be a relatively compact neighbourhood of $e$ so $\norm{x\cdot f-f}_\infty<\epsilon$ for any $x\in V$.
            Then for $x\in V$, applying (i) above,
            \begin{align*}
                |\Delta(x)-1|&=\frac{|\Delta(x)\int f\d{m}-\int_Gf\d{m}|}{\int f\d{m}}\\
                             &\leq\frac{1}{\int f\d{m}}\int|x^{-1}\cdot f-f|\d{m}<\epsilon\frac{m(\supp(f)V)}{\int f\d{m}}.
            \end{align*}
            where $\supp(x^{-1}\cdot f-f)\subseteq\supp(f)V$ and $\supp(f)V$ has compact closure so that $m(\supp(f)V)<\infty$.
            Moreover, as $\epsilon\to 0$, we may arrange for $V$ to be decreasing, yielding continuity of $\Delta$ at $e$.
            Now if $x,y\in G$ are arbitrary, we have
            \begin{equation*}
                |\Delta(xy)-\Delta(y)|=|\Delta(x)-1|\Delta(y)
            \end{equation*}
            which shows that $\Delta$ is continuous at $y$.
    \end{enumerate}
\end{proof}
\begin{proposition}
    \begin{enumerate}[nl,r]
        \item The integral $f\mapsto\int_G f(x)\frac{1}{\Delta(x)}\d{x}$ on $C_c(G)$ is right invariant.
        \item For $f\in L^1(G)$,
            \begin{equation*}
                \int_G f(x^{-1})\frac{1}{\Delta(x)}\d{x}=\int_G f(x)\d{x}
            \end{equation*}
    \end{enumerate}
\end{proposition}
\begin{proof}
    \begin{enumerate}[nl,r]
        \item If $y\in G$ and $f\in C_c(G)$, then
            \begin{align*}
                \int_G y\cdot f(x)\frac{1}{\Delta(x)}\d{x}&=\int_G f(xy)\frac{1}{\Delta(x)}\d{x}=\int_G f(xy)\frac{1}{\Delta(xy)}\Delta(y)\d{x}\\
                                                          &= \int_G f(x)\frac{1}{\Delta(x)}\d{x}
            \end{align*}
        \item If $f\in C_c^+(G)$, then for any $y\in G$,
            \begin{align*}
                \int_G f\cdot y(x^{-1})\frac{1}{\Delta(x)}\d{x} &= \int_G f((xy^{-1})^{-1})\frac{1}{\Delta(x)}\d{x}\\
                                                                &= \int_G f(x^{-1})\frac{1}{\Delta(x)}\d{x}
            \end{align*}
            by the proof above.
            Hence by uniqueness of left Haar integral, there is $c>0$ so that
            \begin{equation*}
                \int_G f(x^{-1})\frac{1}{\Delta(x)}\d{x}=c\int_G f(x)\d{x}
            \end{equation*}
            for $f\in C_c(G)$.
            Notice, by continuity of $f\mapsto\int_G f\d{m}$ on $L^1(G)$, this holds for $f\in L^1(G)$.

            Now, if $c\neq 1$, there is a relatively compact neighbourhood $U=U^{-1}$ of $e$ such that $|\Delta(x)-1|<\frac{1}{2}|c-1|$ for $x\in U$.
            Then we have
            \begin{align*}
                0 &= \left\lvert\int_G\idc{U}(x^{-1})\frac{1}{\Delta(x)}\d{x}-c\int_G\idc{U}(x)\d{x}\right\rvert\\
                  &= \left\lvert\int_U\left(\frac{1}{\Delta(x)}-c\right)\d{x}\right\rvert\\
                  &= \left\lvert\int_U\left(\frac{1}{\Delta(x)}-1+1-c\right)\d{x}\right\rvert\\
                  &\geq m(U)\left\lvert|1-c|-\frac{1}{2}|c-1|\right\rvert=\frac{m(U)}{2}|1-c|>0
            \end{align*}
            which is a contradiction.
    \end{enumerate}
\end{proof}
For $f\in L^1(G)$, $x\in G$, we let
\begin{align*}
    x*f(y)&=f(x^{-1}y)\\
    f*x(y)&=f(yx^{-1})\frac{1}{\Delta(x)}\\
    f^*(x)&=\overline{f(x^{-1})}\frac{1}{\Delta(x)}
\end{align*}
Notice that $\norm{f}_1=\norm{x*f}_1=\norm{f*x}_1=\norm{f^*}_1$.
Moreover,
\begin{itemize}[nl]
    \item $x'*(x*f)=(x'x)*f$ and $(f*x)*x'=f*(xx')$
    \item $f^{**}=f$.
    \item $(x*f)^*=f^**x^{-1}$.
        Indeed, for $m$-a.e. $y$, we have
        \begin{align*}
            (x*f)^*(y) &= \overline{[x*f](y^{-1})}\frac{1}{\Delta(y)}\\
                       &= \overline{f(x^{-1}y^{-1})}\frac{1}{\Delta(y)}\\
                       &= \overline{f((yx)^{-1})}\frac{1}{\Delta(yx)}\frac{1}{\Delta(x^{-1})}\\
                       &= f^**x^{-1}(y)
        \end{align*}
\end{itemize}
\begin{proposition}
    For $f\in L^1(G)$, $\lim_{x\to e}\norm{x*f-f}_1=0=\lim_{x\to e}\norm{f*x-f}_1$.
\end{proposition}
\begin{proof}
    First, consider $g\in C_c(G)$ and $\epsilon>0$, and let $V=V^{-1}$ be a relatively compact neighbourhood of $e$ so $\norm{g\cdot x-g}_\infty<\epsilon$ for $x\in V$.
    Then
    \begin{equation*}
        \norm{x*g-g}_1=\int_G|g\cdot x^{-1}\d{m}\leq\norm{g\cdot x^{-1}-g}_\infty m(V\supp(g))<\epsilon m(V\supp(g))
    \end{equation*}
    so $\lim_{x\to e}\norm{x*g-g}_1=0$.
    If $f\in L^1(G)$, $\epsilon>0$, find $g\in C_c(G)$ such that $\norm{f-g}_1<\epsilon$.
    Then
    \begin{align*}
        \norm{x*f-f}_1 &\leq \norm{x*f-x*g}_1+\norm{x*g-g}_1+\norm{g-f}_1\\
                       &<2\epsilon+\norm{x*g-g}_1
    \end{align*}
    where $\norm{x*g-g}_1\to 0$ as $x\to e$.
    Since $\epsilon>0$ was arbitrary, we are done.

    Now, for $f,x$ as above,
    \begin{equation*}
        \norm{f_x-f}_1=\norm{(f*x-f)^*}_1=\norm{x^{-1}*f^*-f^*}_1
    \end{equation*}
    where $x^{-1}\to e$ as $x\to e$.
\end{proof}
\begin{theorem}[Weil Integral Formula]
    Let $N$ be a closed normal subgroup of $G$.
    \begin{enumerate}[nl,r]
        \item If $f\in C_c(G)$, then $x\mapsto\int_Nf(xn)dn:G\to\C$ is constant on cosets and hence defines a function $T_nf:\quot{G}{N}\to\C$.
            Furthermore, $T_N(C_c^*(G))\subseteq C_c^+(\quot{G}{N})$ and the operator $T_N:C_c(G)\to C_c(\quot{G}{N})$ is linear and covariant:
            \begin{equation*}
                (T_Nf)\cdot(yN)=T_N(f\cdot y)
            \end{equation*}
            for $f\in C_c(G)$ and $y\in G$.
        \item The functional on $C_c(G)$ given by $f\mapsto\int_{\quot{G}{N}}T_nf(xN)\d{x}N$ is hence a Haar integral on $G$, so we may write
            \begin{equation*}
                \int_{\quot{G}{N}}\int_Nf(xn)\d{n}\d{xN}=\int_G f(x)\d{x}
            \end{equation*}
    \end{enumerate}
\end{theorem}
\begin{proof}
    (ii) is a direct consequence of (i); let's see the proof of (i).

    The $N$-invariance of the first function is evident.
    Let $f\in C_c(G)$.
    We inspect the continuity of $T_Nf$ at $x$ in $G$.
    Given $\epsilon>0$, let $V=V^{-1}$ be a real compact neighbourhood of $e$, so $\norm{f\cdot y-f}_\infty<\epsilon$ for $y\in V$.
    Let $g\in C_c^+(G)$ satisfy that $0\leq g\leq 1$ and $g|_{Vx^{-1}\supp(f)}=1$.
    For $y\in V$, $yN=q_N(y)\in q_N(V)$ so
    \begin{align*}
        |T_Nf(yxN)-T_Nf(xN)| &\leq\int_N|f(yx)-f(xn)|g(n)\d{n}\\
                             &<\epsilon m_N(\supp(g)\cap N)
    \end{align*}
    Notice as $\epsilon\to 0$, we may shrink $V$ and hence $\supp(g)$.
    Hence $T_Nf$ is continuous at $xN$.
    Now, $\supp(T_Nf)\subseteq q_N(\supp f)$ so $T_Nf\in C_c(\quot{G}{N})$.

    If $g\in C_c^+(G)$, $x\in G$ is such that $f(x)>0$, let $U$ be a neigbourhood of $e$, $f(xy)>\frac{1}{2}f(x)$ for $y\in U$.
    Then
    \begin{equation*}
        T_Nf(xN)=\int_N f(xn)\d{n}\geq\frac{1}{2}f(x)m_N(U\cap N)>0
    \end{equation*}
\end{proof}
\begin{corollary}
    If $N$ is closed and normal in $G$, Then $\Delta_G|_N=\Delta_N$.
\end{corollary}
\begin{proof}
    Let $n'\in N$ and $f\in C_c^+(G)$.
    Then
    \begin{align*}
        \int_G n'\cdot f(x)\d{x} &=\int_{\quot{G}{N}}\int_Nf(xnn')\d{n}\d{xN}\\
                                 &= \int_{\quot{G}{N}}\frac{1}{\Delta_N(n')}\int_Nf(xn)\d{n}\d{xN}=\frac{1}{\Delta_N(n')}\int_Gf(x)\d{x}
    \end{align*}
    so that $\Delta_G(n')=\Delta_N(n')$.
\end{proof}
\begin{definition}
    We say that $G$ is \defn{unimodular} if $\Delta=1$ on $G$.
\end{definition}
\begin{proposition}
    $G$ is unimodular in any of the following cases:
    \begin{enumerate}[nl,r]
        \item $G$ is abelian, discrete, or compact
        \item $G$ is perfect: $G=\overline{[G,G]}$
        \item $\quot{G}{Z(G)}$ is unimodular.
        \item There is a closed, unimodular normal subgroup $N$ such that $\quot{G}{N}$ is compact.
    \end{enumerate}
\end{proposition}
\begin{proof}
    \begin{enumerate}[nl,r]
        \item This is (nearly) obvious for $G$ abelian or discrete.
            If $G$ is compact, then $\log\circ\Delta:G\to(\R,+)$ is a continuous homomorphism whose range is a compact subgroup.
        \item Any commutator $[x,y]\in xyx^{-1}y^{-1}\in\ker\Delta$.
        \item Let $Z=Z(G)$.
            If $y\in G$ and $f\in C_c(G)$, we have by Weyl's integral formula
            \begin{align*}
                \int_G y\cdot f(x)dx &= \int_{\quot{G}{Z}}\int_Z f(xz)\d{z}\d{xZ}\\
                                     &= \int_{\quot{G}{Z}}\int_Zf(xyz)\d{z}\d{xZ}\\
                                     &= \int_{\quot{G}{Z}}T_Zf(xyZ)\d{xZ}\\
                                     &= \int_{\quot{G}{Z}}T_Zf(xZyZ)dxZ\\
                                     &= \int_{\quot{G}{Z}}T_Zf(xZ)dxZ=\int_Gf(x)dx
            \end{align*}
        \item We have $\Delta_G|_N=\Delta_N=1$ by assumption, i.e. $N\in\ker\Delta_G$, so $\Delta_G$ induces a homomorphism $\overline{\Delta}:\quot{G}{N}\to(0,\infty)$ where $\Delta_G=\overline{\Delta}\circ\pi_N$.
            Verify that $\overline{\Delta}$ is continuous, so $\log\circ\overline{\Delta}:\quot{G}{N}\to\R^+$ is a continuous homomorphism, whose range is a closed subgroup.
            It follows that $\Delta_G=1$ on $G$.
    \end{enumerate}
\end{proof}
\begin{example}
    Here are some examples of unimodular groups.
    \begin{enumerate}[r]
        \item Let $\mathfrak{k}$ be a locally compact field with $|\mathfrak{k}|>3$.
            Then $\SL_n(\mathfrak{k})$ is perfect.

            Let $\{E_{ij}\}_{i,j=1}^n$ be a matrix unit for $M_n(\mathfrak{k})$, so $E_{ij}E_{k\ell}=\delta_{jk}E_{il}$.

            If $\lambda\in\mathfrak{k}$, $i,j,k$ distinct (i.e. $n\geq 3$), then
            \begin{equation*}
                [e+\lambda E_{ik},e+E_{kj}]=(e+\lambda E_{ik})(e+E_{kj})(e-\lambda E_{ik})(e-E_{kj})=e+\lambda E_{ij}.
            \end{equation*}
            If $n=2$, then
            \begin{equation*}
                \left[\begin{pmatrix}\alpha&0\\0&\alpha^{-1}\end{pmatrix},\begin{pmatrix}1&\beta\\0&1\end{pmatrix}\right]=e+\lambda E_{12}
            \end{equation*}
            where $\lambda=(1-\alpha^2)\beta$.

            If $S=\langle e+\lambda E_{ij}:\lambda\in\mathfrak{k},i\neq j\rangle$.
            Using only elemnentary operations induced by multiplying by elements of $S$ on the left, and element $a$ of $\SL_n(\mathfrak{k})$ satisfies
            (see pic)

            By an evident induction, we see that there are $s_1,s_2\in S$ so $s_1sas_2=e$.
            Thus $a=s^{-1}s_1^{-1}s_2^{-1}\in S$.
        \item Let $\mathfrak{k}=\R$ or $\C$.
            Consider $G=\GL_n(\mathfrak{k})$.
            Notice that $Z=Z(G)=\mathfrak{k}^\times e$.
            From (i), $\SL_n(\mathfrak{k})$ is perfect.

            Let $H=Z\cdot\SL_n(\mathfrak{k})$, which is closed (check!) and $\quot{H}{Z}\cong\quot{\SL_n(\mathfrak{k})}{Z\cap\SL_n(\mathfrak{k})}$ is perfect, being the quotient of a perfect group, hence unimodular.
            
            If $\mathfrak{k}=\C$ or $\mathfrak{k}=\R$ and $n$ is odd, $H=G$.
            Else if $\mathfrak{k}=\R$ and $n$ is even, then $H=\GL_n(\R)_e=\det^{-1}((0,\infty))$ (connnected component of $e$) and $G=\GL_n(\R)_e\cupdot(-e)\GL_n(\R)_e$, so $\quot{G}{H}\cong\{-1,1\}$ is compact.
        \item $E(n)=\R^n \rtimes \SO(n)$.
            Since $N=\R^n\rtimes\{e\}$ is closed, normal, and abelian, and $\quot{G}{N}=\SO(n)$ is compact.
        \item Consider
            \begin{equation*}
                \mathbb{H}=\left\{\begin{pmatrix}1&x&z\\0&1&y\\0&0&1\end{pmatrix}:x,y,z\in\R\right\}
            \end{equation*}
            then
            \begin{equation*}
                Z(\mathbb{H})=\left\{\begin{pmatrix}1&0&z\\0&1&0\\0&0&1\end{pmatrix}:z\in\R\right\}
            \end{equation*}
            has $\quot{\mathbb{H}}{Z}\cong\R^2$.
    \end{enumerate}
\end{example}
\begin{remark}
    In A1, a ``Braconnier'' modular function $\delta:\Aut(G)\to(0,\infty)$ is defined.
    \begin{enumerate}[nl,r]
        \item If $\gamma:G\to\Aut(G)$ has $\gamma(x)(y)=xyx^{-1}$, so $\gamma$ is a homomorphism.
            Then $\delta(\gamma(x))=\frac{1}{\Delta(x)}$.
        \item If $G$ is compact and $\alpha\in\Aut(G)$, then $\alpha(G)=G$ so $1=m(G)=m(\alpha(G))$ and $\delta(\alpha)=1$.
        \item If $G$ is discrete and $\alpha\in\Aut(G)$, then for any non-empty finite $F\subseteq G$, we have $|F|=|\alpha(F)|$, and it follows that $\delta(\alpha)=1$.
        \item If $G$ is unimodular and $H$ an open subgroup of $G$, then $H$ is unimodular
            However, there is some subtlety here:
            \begin{equation*}
                H=\left\{\begin{pmatrix}a&b\\0&a^{-1}\end{pmatrix}:a,b\in\R,a>0\right\}
            \end{equation*}
            is closed in $\SL_2(\R)$ and $H\cong\R\rtimes(0,\infty)$ is not unimodular.
            Moreover,
            \begin{equation*}
                N=\left\{\begin{pmatrix}1&b\\0&1\end{pmatrix}:b\in\R,a>0\right\}
            \end{equation*}
            is open, normal and abelian in $G=\left\{\begin{pmatrix}2^n &r\\0&1\end{pmatrix}n\in\Z,r\in\R\right\}$ and $G$ is closed in $\GL_2(\R)$.
    \end{enumerate}
\end{remark}
\subsection{The Convolution Algebra of Measures}
Let $G$ be a locally compact group.
Let
\begin{align*}
    M(G) &= \{\mu:\mathcal{B}(G)\to\C:\mu\text{ Radon measure}\}=\spn M_+(G)\\
    M_+(G) &= \{\mu:\mathcal{B}(G)\to[0,\infty)|\mu\text{ Radon}]
\end{align*}
If $\mu\in M_+(G)$ with $\mu(G)<\infty$ so $\mu$ is finite.

Recall the Hahn-Jordan Decomposition: each $\mu$ in $M(G)$ admits a decomposition $\mu=\sum_{k=0}^3 i^k\mu_k$ where each $\mu_i\in M_+(G)$, $\mu_0\perp\mu_2$, and $\mu_1\perp\mu_3$.
Any measures satisfying this decomposition are unique.
\begin{definition}
    If $\mu\in M(G)$, we define the $|\mu|:\mathcal{B}(G)\to[0,\infty)$ by
    \begin{equation*}
        |\mu|(E)=\sup\left\{\sum_{k=1}^\infty|\mu(E_k)|:E=\bigcupdot_{k=1}^\infty E_k,E_k\in\mathcal{B}(G)\right\}
    \end{equation*}
    and $|\mu|=M_+(G)$.
    If $\mu=\sum_{k=0}^3i^k\mu_k$ as in Hahn-Jordan, then $|\mu_0-\mu_2|=\mu_0+\mu_2$ and $|\mu_1-\mu_3|=\mu_1+\mu_3$.
    Furthermore,
    \begin{equation*}
        |\mu|\leq|\mu_0-\mu_2|+|\mu_1-\mu_3|\text{ and }|\mu_0-\mu_2||\mu_1-\mu_3|\leq|\mu|
    \end{equation*}
\end{definition}
\begin{theorem}[Riesz-Markov Duality]
    Let $C_0(G)=\overline{C_c(G)}\subseteq C_b(G)$ with the uniform topology.
    Then $C_0(G)^*\cong M(G)$ through the map $\mu\mapsto \langle\mu,\cdot\rangle$ where $\langle \mu,f\rangle=\int_G f\d{\mu}$.
    Moreover, $\norm{\langle\mu,\cdot\rangle}_{\text{op}}=|\mu|(G)$.
\end{theorem}
\begin{remark}
    Let $\mathcal{B}^\infty(G)=\{f:G\to\C:f\text{ bounded and Borel measurable}\}$, which is a Banach space under the uniform norm.
    Note that $\mathcal{B}^\infty(G)=\overline{\spn}\{1_E:E\in\mathcal{B}(G)\}$.
    We have
    \begin{equation*}
        \left\lvert\int_G f\d{\mu}\right\rvert\leq\int_G|f|d|\mu|\leq\norm{f}_\infty\norm{\mu}_1
    \end{equation*}
\end{remark}
If $\mu\in M(G)$ and $\epsilon>0$, then inner regularity provides compact $K\subseteq G$ such that $|\mu|(K)>|\mu(G)-\epsilon$.
Hence $|\mu|(G\setminus K)<\epsilon$.
Then $\norm{\mu-\mu_k}_1=\norm{\mu_{G\setminus K}}_1=|\mu|(G\setminus K)<\epsilon$.
\begin{lemma}[Continuous Fubini]
    Let $X,Y$ be locally compact spaces, $\mu\in M(X)$, $\nu\in M(Y)$.
    Then there is a measure $\mu\times\nu\in M(X\times Y)$ such that
    \begin{equation*}
        \int_{X\times Y}f\d{(\mu\times\nu)}+\int_Y\int_X f\d{\mu}\d{\nu}=\int_X\int_Yf\d{\nu}\d{\mu}
    \end{equation*}
    for any $f\in C_b(X)$.
\end{lemma}
\begin{proof}
    Let $\mathcal{A}=\spn\{\phi\times\psi:\phi\in C_0(X),\psi\in C_0(Y)\}$ where $\phi\times\psi(x,y)=\phi(x)\psi(y)$.
    Then $\overline{\mathcal{A}}=C_0(X\times Y)$ in the uniform topology.
    We define for $f\in\mathcal{A}$
    \begin{equation*}
        J(f) = \int_X\int_Y f\d{\nu}\d{\mu}\int_Y\int_X f\d{\mu}\d{\nu}
    \end{equation*}
    so that $J$ is linear on $\mathcal{A}$ and
    \begin{equation*}
        |J(F)|\leq\norm{f}_\infty|\mu|(X)|\nu|(Y)
    \end{equation*}
    so $J$ is bounded.
    Thus $J$ is uniformly continuous and hence extens uniquely to $C_0(X\times Y)$ as a bounded linear functional with $\norm{J}\leq\norm{\mu}_1\norm{\nu}_1$.
    By Riesz-Markov, there is $\mu\times \nu\in M(X\times Y)$ such that $J(f)=\int_{X\times Y}f\d{(\mu\times\nu)}$.
    Uniform limits are pointwise limits, so LDCT tells us that we have Fubini for $f\in C_0(X\times Y)$.

    By inner regularity, find $(K_n)_{n=1}^\infty$ and $(L_n)_{n=1}^\infty$ so that $\lim_{n\to\infty}\norm{\mu-\mu_{K_n}}_1=0=\lim_{n\to\infty}\norm{\nu-\nu_{L_n}}_1$.
    For each $n$, let $f_n\in C_c(X\times Y)$ be such that $f|_{K_n\times L_n}=f_n|_{K_n\times L_n}$ (Urysohn).
    We also notice that $(\mu\times\nu)_{K_n\times L_n}=\mu_{K_n}\times\nu_{L_n}$.
    Then check that $\lim_{n\to\infty}\norm{(\mu\times\nu)_{K_n\times L_n}-\mu\times\nu}_1=0$.
    Then for $f\in C_b(X\times Y)$,
    \begin{align*}
        \int_{X\times Y}f\d{(\mu\times\nu)} &= \lim_{n\to\infty}\int_{X\times Y}f\d{(\mu\times\nu)_{K_n\times L_n}}\\
                                            &= \lim_{n\to\infty}\int_{X\times Y}f_n\d{(\mu\times\nu)_{K_n\times L_n}}\\
                                            &= \lim_{n\to\infty}\int_X\int_Y f_n\d{\nu_{L_n}}\d{\mu_{K_n}}\\
                                            &= \lim_{n\to\infty}\int_X\int_Y f\d{\nu_{L_m}}\d{\mu_{K_n}}=\int_X\int_Y f\d{\nu}\d{\mu}
    \end{align*}
\end{proof}
\begin{theorem}
    Given $\mu,\nu$ in $M(G)$, there is a unique measure $\mu*\nu$ in $M(G)$ such that
    \begin{equation*}
        \int_G f\d{\mu*\nu}=\int_G\int_Gf(xy)\d{\mu(x)}\d{\mu(y)}
    \end{equation*}
    for $f\in C_0(G)$.
    The map $(\mu,\nu)\mapsto \mu*\nu$ is bilinear, associative, and satisfies $\norm{\mu*\nu}_1\leq\norm{\mu}_1\norm{\nu}_1$.
    Hence $(M(G),*)$ is a Banach algebra.
\end{theorem}
\begin{claim}
    Given $\mu\in M(G)$ and $f\in C_0(G)$, define $f\cdot\mu,\mu\cdot f:G\to\C$ by
    \begin{align*}
        f\cdot\mu(x) &=\langle \mu,x\cdot f\rangle=\int_G f(yx)\d{\mu(y)}\\
        \mu\cdot f(x)=\langle\mu,f\cdot x\rangle=\int_Gf(xy)\d{\mu(y)}
    \end{align*}
    Notice that $\mu\cdot f(e)=f\cdot\mu(e)$.
    Then $f\cdot\mu$, $\mu\cdot f$ in $C_0(G)$.
\end{claim}
\begin{nmproof}
    Indeed, let us check continuity of $\mu\cdot f$.
    Let $\epsilon>0$ and $V$ be a neighbourhood of $e$ such that $|f(x)-f(x')|<\epsilon$ for $x'x^{-1}\in V$ (uniform continuity, $\overline{C_c(G)}=C_0(G)$, uniform limit of uniformly cts is uniformly cts).
    Then
    \begin{equation*}
        |\mu\cdot f(x)-\mu\cdot f(x')|\leq\int_G|f(xy)-f(x'y)|\d{|\mu|(y)}\leq\epsilon|\mu|(G)
    \end{equation*}
    is uniformly continuous, hence continuous.
    Notice that $|\mu\cdot f(x)|\leq\int_G|f(xy)|\d{|\mu|(Y)}\leq\norm{f}_\infty\norm{\mu}_1$.
    Now, given $\epsilon>0$, let $K\subseteq G$ be compact so $\norm{\mu-\mu_K}_1<\epsilon$ and $f'\in C_c(G)$ be so $\norm{f-f'}_\infty<\epsilon$.
    Then
    \begin{align*}
        \norm{\mu\cdot f-\mu_K\cdot f'}_\infty&\leq\norm{\mu\cdot f-\mu_k\cdot f}_\infty+\norm{\mu_K\cdot f-\mu_K\cdot f'}_\infty\\
                                              &\leq\norm{\mu-\mu_K}_1\norm{f}_\infty+\norm{\mu_K}_1\norm{f-f'}_\infty\\
                                              &< (\norm{\mu}_1+\norm{f}_\infty)\epsilon
    \end{align*}
    and $\supp(\mu_K\cdot f')\subseteq\supp(f')K^{-1}$, so $u_k\cdot f\in C_c(G)$.
    Thus $\mu\cdot f\in C_0(G)$.
    Similarly, $f\cdot\mu\in C_0(G)$.
    It is evident that $(\mu,f)\mapsto\mu\cdot f$ and $(f,\mu)\mapsto f\cdot\mu$ are bilinear.
\end{nmproof}

\begin{claim}
    If $\mu,\nu\in M(G)$, then $\mu\cdot(f\cdot\nu)=(\mu\cdot f)\cdot\nu$ and $\langle\mu,f\cdot\mu\rangle=\langle\nu,\mu\cdot f\rangle$.
\end{claim}
\begin{nmproof}
    Use Fubini for continuous functions: for $x$ in $G$,
    \begin{align*}
        \mu\cdot(f\cdot\nu)(x) &= \int_G(f\cdot\nu)(xy)\d{\mu(y)}= \int_G\int_Gf(zxy)\d{\nu(z)}\d{\mu(y)}\\
                               &= \int_G\int_G f(zxy)\d{\mu(y)}\d{\nu(z)}\\
                               &= \int_G(\mu\cdot f)(zx)\d{\nu(z)}=(\mu\cdot f)\cdot\nu(x)
    \end{align*}
\end{nmproof}
\begin{claim}
    For $\mu,\nu$ in $M(G)$, define for $f$ in $C_0(G)$
    \begin{equation*}
        \langle\mu*\nu,f\rangle=\langle\mu,\nu\cdot f\rangle=\langle\nu,f\cdot\mu\rangle
    \end{equation*}
    Then $\mu*\nu$ is unique and satisfies the required properties.
\end{claim}
\begin{proof}
    Uniqueness follows by Riesz-Markov (...?)

    Moreover, $(\mu,\nu)\mapsto\mu*\nu$ is evidently bilinear and
    \begin{equation*}
        |\langle\mu*\nu,f\rangle|=|\langle\mu,\nu\cdot f\rangle|\leq\norm{\mu}_1\norm{\nu\cdot f}_\infty\leq\norm{\mu}_1\norm{\nu}_1\norm{f}_\infty.
    \end{equation*}
    This shows that
    \begin{itemize}[nl]
        \item $f\mapsto\langle\mu\times\nu,f\rangle$ is bounded, and hence $\mu*\nu$ is an element of $M(G)$ (by Riesz-Markov)
        \item $\norm{\mu*\nu}_1\leq\norm{\mu}_1\norm{\nu}_1$
    \end{itemize}
    It remains to see associativity.
    If $\rho\in M(G)$, then for $f\in C_0(G)$,
    \begin{align*}
        \langle\mu*(\nu*\rho)\rangle&=\langle\nu*\rho,f\cdot\mu\rangle=\langle\nu,\rho\cdot(f\cdot\mu)\rangle\\
                                    &=\langle\mu*\nu,\rho\cdot f\rangle
    \end{align*}
\end{proof}
\begin{remark}
    \begin{enumerate}[nl,r]
        \item If $\mu\in M(G)$, let $R_\mu,L_\mu:C_0(G)\to C_0(G)$ be given by $L_\mu f=\mu\cdot f$, $R_\mu f=f\cdot\mu$.
            Then for $\nu\in M(G)$, $\mu*\nu=R_\mu^*(\nu)=L_\nu^*(\mu)$ which shows $\nu\mapsto\mu*\nu$ or $\nu\mapsto\nu*\mu$ are each $w^*-w^*$-continuous.
            Note that $(\mu,\nu)\mapsto \mu*\nu$ may not be $w^*-w^*$-continuous.
        \item Let for $x$ in $G$ $\delta_x(E)$ be the point mass measure at $x$.
            Then $\langle\delta_x,f\rangle=f(x)$ for $f\in C_0(G)$.
            Then $\delta_x*\delta_y=\delta_{xy}$.
        \item If $\mu,\nu\in M_+(G)$, then $\mu*\nu\in M_+(G)$.
    \end{enumerate}
\end{remark}
\subsection{Atomic-Continuous and Lebesgue Decompositions}
Let $\mu\in M(G)$ and set
\begin{equation*}
    A(\mu) = \{x\in G:|\mu|(\{x\})>0\}=\bigcup_{n=1}\infty\left\{x\in G:|\mu(\{x\})>\frac{1}{n}\right\}
\end{equation*}
so $A(\mu)$ is countable.
For any $x\in G$, $|\mu|(\{x\})=|\mu(\{x\})|$ by definition of $|\mu|$ and hence
\begin{equation*}
    \sum_{x\in A(\mu)}|\mu(\{x\})|=\sum_{x\in A(\mu)}|\mu|(\{x\})=|\mu|(A(\mu))<\infty.
\end{equation*}
We then define the \defn{atomic} or \defn{discrete} part of $\mu$ by
\begin{equation*}
    \mu_d=\sum_{x\in A(\mu)}\mu(\{x\})\delta_x\in M(G)
\end{equation*}
and the \defn{continuous part} by
\begin{equation*}
    \mu_c=\mu-\mu_dd
\end{equation*}
Then $\mu_c\perp\mu_d$ so
\begin{equation*}
    \norm{\mu}_1=|\mu|(G)=|\mu_c|(G)+|\mu_d|(G)=\norm{\mu_c}_1+\norm{\mu_d}_1
\end{equation*}
The set $M_d(G)=\overline{\spn}\{\delta_x:x\in G\}$ is a subspace of $M(G)$, and $M_d(G)\cong\ell^1(G)$ isometrically.
Thus $M_c(G)=\im P_c$ is a closed subspace.

If $G$ is discrete, then $|\mu_c|(G)=0$ so $M(G)=M_d(G)\cong\ell^1(G)$.

If $G$ is not discrete, then $\{e\}$ is a closed, non-open subgroup, so for $x\in G$, $m(\{x\})=0$ where $m$ is the Haar measure.
Hence the measures absolutely continuous with respect to $m$ satisfy $M_d(G)\subseteq M_c(G)$.
We can employ the Lebesgue decomposition to write $\mu_c=\mu_a+\mu_{cs}$ where $\mu_a\ll m$ and $\mu_a\perp \mu_{cs}$.
Moreover, the Radon-Nikodym derivative has $\frac{\d{\mu}}{\d{m}}\in L^1(G)$.
To conclude,
\begin{align*}
    M(G) &= M_c(G)\oplus_1 M_d(G)\\
         &= M_a(G)\oplus_1 M_{cs}(G)\oplus_1 M_d(G)\\
         &\cong L^1(G)\oplus_1 M_{cs}(G)\oplus_1 \ell^1(G)
\end{align*}
Certainly $\ell^1(G)$ is a subalgebra.
We will show that the remaining components are also subalgebras.
\begin{remark}
    Given $\mu,\nu\in M(G)$, we formed a \defn{Radon product} $\mu\times\nu$ which satisfies
    \begin{equation*}
        \int_G\int_G f(x,y)\d{\mu(x)}\d{\mu(y)}=\int_{G\times G}f\d{(\mu\times\nu)}=\int_G\int_Gf(x,y)\d{\nu(y)}\d{\mu(x)}
    \end{equation*}
    for $f\in C_0(G)$.
    This extends to $f\in C_b(G\times G)$.
    Similarly,
    \begin{align*}
        \int_G f\d{(\mu*\nu)}&=\int_G\int_Gf(xy)\d{\mu(x)}\d{\nu(y)}=\int_G\int_G f(xy)\d{\nu(y)}\d{\mu(x)}\\
                             &=\int_{G\times G}f\circ p\d{(\mu\times\nu)}
    \end{align*}
    where $p:G\times G\to G$ is the product.

    If $E\in\mathcal{B}(G)$, then $\pi^{-1}(E)\in\mathcal{B}(G\times G)$.
    Indeed, since $p$ is continuous, $p^{-1}(U)\in\tau_G\times\tau_G\subseteq\mathcal{B}(G\times G)$ for $U$ open, so the result follows.
\end{remark}
\begin{theorem}
    If $\mu,\nu\in M(G)$, $E\in\mathcal{B}(G)$, then
    \begin{equation*}
        \mu*\nu(E)=(\mu\times\nu)\circ p^{-1}(E)
    \end{equation*}
    for $E\in\mathcal{B}(G)$.
\end{theorem}
\begin{proof}
    First note that
    \begin{equation*}
        (\mu\times\nu)(p^{-1}(E))=\int_{G\times G}\idc{\pi^{-1}(E)}\d{(\mu\times\nu)}=\int_{G\times G}\idc{E}\circ p\d{(\mu\times\nu)}.
    \end{equation*}
    Now given Jordan decomposition $\mu=\sum_{k=0}^3 i^k\mu_k$ and $\nu=\sum_{j=0}^3 i^j\nu_j$, we have
    \begin{equation*}
        \mu*\nu=\sum_{k,j=0}^3i^{k+j}\mu_k*\nu_k
    \end{equation*}
    so it suffices to show the result for $\mu,\nu\in M_+(G)$.

    First let $K\subseteq G$ be compact and $\epsilon>0$.
    Find open $U$ so $K\subseteq U$ and $\mu*\nu(U\setminus K)<\epsilon$, and by Urysohn find $f\in C_c(G,[0,1])$ such that $f|_K=1$ and $\supp(f)\subseteq U$.
    Then
    \begin{align*}
        (\mu\times\nu)(p^{-1}(K)) &= \int_{G\times G}\idc{K}\circ p\d{(\mu\times\nu)}\\
                                  &\leq\int_{G\times G}f\circ p\d{(\mu\times\nu)}=\int_G f\d{(\mu\times\nu)}\\
                                  &\leq \int_G\idc{U}\d{(\mu\times\nu)}=\mu*\nu(U)<\mu*\nu(K)+\epsilon
    \end{align*}
    so that $(\mu\times\nu)\circ p^{-1}(K)\leq\mu *\mu(K)$.

    Now let $N$ be $\mu\times\nu$-null.
    If $K\subseteq p^{-1}(N)$ is compact, then
    \begin{align*}
        (\mu\times\nu)(K) &\leq(\mu\times\nu)(p^{-1}(p(K)))\\
                          &\leq \mu*\nu(\pi(K))\leq\mu*\nu(N))=0
    \end{align*}
    so that $(\mu\times\nu)(\pi^{-1}(N))=\sup\{\mu\times\nu(K):K\subseteq N,K\text{ compact}\}=0$.

    Now let $U\subseteq G$.
    For each $n\in\N$, find compact $K_n\subseteq U$ so $\mu*\nu(U)<\mu*\nu(K_n)+1/n$ and find $f_n\in C_c(G,[0,1])$ such that $f_n|_{K_n}=1$, $\supp(f_n)\subseteq U$, and $g_n=\max\{f_1,\ldots,f_n\}$.
    Set $F=\bigcup_{n=1}^\infty K_n$ so $\mu\times\nu(U\setminus F)=0$.
    Thus $g_n\to\idc{U}$ as $n\to\infty$ on $G\setminus(U\setminus F)$ and hence, by above, $g_n\circ p\to\idc{U}\circ p=\idc{p^{-1}(U)}$ $\mu\times\nu-\mae$.
    Hence by monotone convergence,
    \begin{align*}
        (\mu\times\nu)(p^{-1}(U))&=\int_{G\times G}\idc{U}\circ p\d{(\mu\times\nu)}=\lim_{n\to\infty}\int_{G\times G}g_n\circ p\d{(\mu\times\nu)}=\lim_{n\to\infty}\int_G g_n\d{(\mu*\nu)}\\
                                 &= \int_G\idc{U}\d{(\mu*\nu)}=\mu*\nu(U)
    \end{align*}

    Finally, let $E\in\mathcal{B}(G)$ be any Borel set.
    Find open $U_n\supseteq E$ such that $\mu*\nu(U_n)<\mu*\nu(E)+1/n$.
    Let $V_n=\bigcap_{k=1}^n U_k$, so $\idc{V_n}\to\idc{E}$ (non-increasing) on $G\setminus\left(\bigcap_{n=1}^\infty U_n\setminus E\right)$, i.e. $\mu*\nu-\mae$.
    Hence, again by above, $\idc{V_n}\circ p\to\idc{E}\circ p$ $\mu\times\nu-\mae$.
    Thus by LDCT (since our measures are finite),
    \begin{align*}
        \mu\times\nu(\pi^{-1}(E))&=\int_{G\times G}\idc{E}\circ\pi\d{(\mu\times\nu)} = \lim_{n\to\infty}\int_{G\times G}\idc{V_n}\circ p\d{(\mu\times\nu)}\\
                                 &= \lim_{n\to\infty}\int_G\idc{V_n}\d{(\mu*\nu)} = \int_G\idc{E}\d{(\mu*\nu)}=\mu*\nu(E)
    \end{align*}
\end{proof}
\begin{remark}
    If $U,g_n$ are as in the above proof, then
    \begin{align*}
        \mu*\nu(U)&=\int_G\idc{U}\d{(\mu*\nu)}=\lim_{n\to\infty}\int_G g_n\d{(\mu*\nu)}\\
                  &= \lim_{n\to\infty}\int_G\int_G g_n(xy)\d{\mu(x)}\d{\nu(y)}\\
                  &\leq \int_G\int_G \idc{U}(xy)\d{\mu(x)}\d{\nu(y)}
    \end{align*}
    Then for $E\in\mathcal{B}(G)$,
    \begin{equation*}
        \mu*\nu(E)\leq\int_G\int_G\idc{E}(xy)\d{\mu(x)}\d{\nu(y)}=\int_G\mu(Ey^{-1})\d{\mu(y)}.
    \end{equation*}
\end{remark}
\begin{corollary}
    Let $\mu,\nu\in M_+(G)$.
    If $N\in\mathcal{B}(G)$ has that $Ny^{-1}$ is $\nu$-null for $y\in G$, then $N$ is $\mu\times\nu$-null.
\end{corollary}
\begin{proof}
    Use the remark above.
\end{proof}
\begin{theorem}
    Each of $M_a(G)$ and $M_c(G)$ is a (two-sided) ideal in $M(G)$.
\end{theorem}
\begin{proof}
    If $N\in\mathcal{B}(G)$ is
    \begin{itemize}[nl]
        \item $m$-null, then so are $Ny^{-1}$, $x^{-1}N$ for $y,x\in G$.
        \item a singleyon, $N=\{z\}$, then $\{z\}y^{-1}=\{zy^{-1}$ and $x^{-1}\{z\}=\{x^{-1}z\}$
    \end{itemize}
    Thus if $\nu\in M_a(G)$ or $\nu\in M_c(G)$, then the same is true of $\mu*\nu$ and $\nu*\mu$ for $\mu\in M(G)$.
\end{proof}
\begin{remark}
    If $\nu\in M+a(G)$, then if $f=\frac{\d{\nu}}{\d{m}}$ satisfies $\int_G h\d{\nu}=\int_G hf\d{m}$ for $h\in C_0(G)$.
    What can we learn about $\frac{\d{(\mu*\nu)}}{\d{m}}$, $\frac{\d{(\nu*\mu)}}{\d{m}}$?
\end{remark}
\begin{theorem}
    Let $X$ be a locally compact space, $\mathcal{L}$ a Banach space, and let
    \begin{equation*}
        C_b(X,\mathcal{L}) =\left\{f:X\to\mathcal{L}:F\text{ continuous},\norm{f}_\infty=\sup_{x\in X}\norm{f(x)}<\infty\right\}.
    \end{equation*}
    Then there is a bilinear map
    \begin{equation*}
        (f,\mu)\mapsto\int_X f\d{\mu}:C_b(X,\mathcal{L})\times M(X)\to\mathcal{L}
    \end{equation*}
    such that
    \begin{itemize}[nl]
        \item $\norm{\int_X f\d{\mu}}\leq\norm{f}_\infty\norm{\mu}_1$ and
        \item $T\left(\int_X f\d{\mu}\right)=\int_X T\circ f\d{\mu}$ for $T\in\mathcal{B}(\mathcal{L},\mathcal{L}')$.
    \end{itemize}
\end{theorem}
\begin{proof}
    Let
    \begin{equation*}
        \mathcal{S}=\mathcal{S}(X,\mathcal{L})=\spn\{\idc{E}\xi:E\in\mathcal{B}(X),\xi\in\mathcal{L}\}.
    \end{equation*}
    If $\Phi\in\mathcal{L}$, then it admits a standard form $\Phi=\sum_{j=1}^n \idc{E_j}\xi_j$ where for $i\neq j$, $E_i\cap E_j=\emptyset$ and $\xi_i\neq\xi_j$.
    Note that $\norm{\Phi}_\infty=\sup_{x\in X}\norm{\Phi(x)}=\max_{j=1,\ldots,n}\norm{\xi_i}$.
    Then for $\Phi$ in standard form,
    \begin{equation*}
        (\Phi,\mu)\mapsto\int_X\Phi\d{\mu}:=\sum_{j=1}^n\mu(E_j)\xi_j
    \end{equation*}
    from $\mathcal{S}\times M(X\to\mathcal{L}$ is bilinear and
    \begin{equation*}
        \norm{\int_X\Phi\d{\mu}}\leq \sum_{j=1}^n |\mu(E_j)|\norm{\xi_j}\leq\norm{\mu}_1\norm{\Phi}_\infty.
    \end{equation*}
    Now let $\mathcal{S}^\infty$ denote the uniform closure of $\mathcal{S}$ and the bilinear pairing extends isometrically.

    Now assume $X$ is compact.
    Then for $F\in C(X,\mathcal{L})$, $F(X)$ is totally bounded in $\mathcal{L}$ so for $\epsilon>0$, $F(X)\subseteq\bigcup_{j=1}^n B(\xi_j,\epsilon)$.
    Let $E_1=F^{-1}(\mathcal{B}(\xi,\epsilon))$, $E_{i+1}=F^{1}(B(\xi_k,\epsilon))\setminus\bigcup_{j=1}^{k-1}F^{-1}(B(\xi_j,\epsilon))$.
    Then $\phi_\epsilon=\sum_{j=1}^n\idc{E_j}\xi_j$ satisfies $\norm{\Phi_\epsilon-F}_\infty<\epsilon$.
    Thus $C(X,\mathcal{L})\subseteq S^\infty(X,\mathcal{L})$.
    Thus we may define $\int_X F\d{\mu}$ for $\mu\in M(X)$ and $F\in C(X,\mathcal{L})$.

    Finally, let $\mu\in M(X)$ and let $(K_n)_{n=1}^\infty$ be a sequence of compact sets such that $|\mu|(X\setminus K_n)<1/n$, so $\norm{\mu-\mu_{K_n}}_1\to 0$ as $\to\infty$.
    Then let
    \begin{equation*}
        \xi_n=\int_{K_n}F\d{\mu}=\int_KF\d{\mu_{K_n}}
    \end{equation*}
    for any $K\supseteq K_n$.
    Then $\{\xi_n\}_{n=1}^\infty$ is Cauchy since $\norm{\xi_n-\xi_m}=\norm{\int_K F\d{(\mu_{K_n}-\mu_{K_m})}}\leq\norm{F}_\infty\norm{\mu_{K_n}-\mu_{K_m}}$.
    Let $\int_X F\d{\mu}=\lim_{n\to\infty}\xi_n$, which is independent of the choice of $K_n$.

    Now if $T\in\mathcal{B}(\mathcal{L},\mathcal{L}')$, first apply $T$ to $\Phi$ in $S(X,\mathcal{L})$, and then by approimations to $\Psi$ in $S^\infty(X,\mathcal{L})$, and then extend using the construction above.
\end{proof}
\begin{theorem}
    Let $G$ be a locally compact group, $\mathcal{L}$ a Banach space, and suppose there is an action
    \begin{equation*}
        (x,\xi)\mapsto x\cdot\xi:G\times\mathcal{L}\to\mathcal{L}
    \end{equation*}
    such that
    \begin{itemize}[nl]
        \item $x\mapsto x\cdot\xi$ is continuous for each $\xi$
        \item $\xi\mapsto x\cdot\xi$ is linear for each $x$
        \item there is $C>0$ such that $\norm{x\cdot\xi}\leq C\norm{\xi_1}$ for $x\in G$, $\xi\in\mathcal{L}$.
    \end{itemize}
    Then there is a bilinear map $(\mu,\xi)\mapsto\mu\cdot\xi:M(G)\times\mathcal{L}\to\mathcal{L}$ such that $\norm{\mu\cdot\xi}\leq C\norm{\mu}_1\norm{\xi}$ and $(\mu*\nu)\cdot\xi=\mu\cdot(\nu\cdot\xi)$ for $\mu,\nu\in M_G$ and $\xi\in\mathcal{L}$.
\end{theorem}
\begin{proof}
    We let
    \begin{equation*}
        \mu\cdot\xi=\int_G x\circ\xi\d{\mu(x)}.
    \end{equation*}
    The bilinear and boundedness is clear.
    To check associaticity, let $w\in\mathcal{L}^*$ and check that $w\bigl((\mu*\nu)\cdot\xi\bigr)=w(\mu\cdot(\nu\cdot\xi))$ by Fubini for continuous integrands.
\end{proof}
\begin{definition}
    A \defn{Banach $G$-module} is an action $G\times X\to X$ which is
    \begin{itemize}[nl]
        \item linear in $X$
        \item multiplicative and continuous in $G$
        \item uniformly bounded in $G$:$ \norm{x\cdot\xi}\leq C\norm{\xi}$.
    \end{itemize}
    In addition, we say that the $G$-module is \defn{non-degenerate} of $e\cdot\xi=\xi$.
\end{definition}
In this context, the prior theorem essentially states that a Banach $G$-module is a Banach $M(G)$-module.
\begin{remark}
    A \defn{representation} of $G$ on a Banach space $X$ is a homomorphism $\pi:G\to\mathcal{B}(X)$ ($\mathcal{B}(X)$ is the set of bounded linear operators on $X$) such that $\pi(e)=I$.
    We typically also assume
    \begin{itemize}[nl]
        \item \defn{boundedness}: $\sup_{x\in G}\norm{\pi(x)}\leq C<\infty$
        \item \defn{strong operator continuity}: $x\mapsto\pi(x)\xi$ is continuous for any $\xi\in X$.
    \end{itemize}
\end{remark}
\begin{corollary}
    If $\pi:G\to\mathcal{B}(X)$ is a (bounded, SOT) representation of $G$, then $\pi$ induces a bounded homomorphism $\pi_M:M(G)\to\mathcal{B}(X)$ such that $\pi_M(\mu)\xi=\int_G\pi(x)\xi\d{\mu(x)}$ with $\pi_M(\delta_x)=\pi(x)$.
\end{corollary}
\begin{example}
    Recall that if $f\in L^1(G)$ and $x\in G$, we have actions $G\times L^1(G)\to L^1(G)$ by $(x,f)\mapsto x*f$, where $x*f(y)=f(x^{-1}y)$; and similarly $f*x(y)=f(yx^{-1})/\Delta(x)$.
    These make $L^1(G)$ both a left and right isometric $G$-module.
    Hence the last theorem provides us with a Banach $M(G)$-module structure
    \begin{align*}
        \mu*f&=\int_Gx*f\d{\mu(x)}\\
        f*\mu&=\int_Gf*x\d{\mu(x)}
    \end{align*}
\end{example}
\begin{lemma}
    \begin{itemize}[nl]
        \item $L^1\cap C_0(G)$ with norm $\norm{\cdot}_1+\norm{\cdot}_\infty$ is a Banach space, dense in $L^1(G)$, and which is a left Banach $G$-module (in $L^1(G)$).
        \item The space
            \begin{equation*}
                L^1\cap C^\delta(G)=\{f\in C(G):\norm{f}_1<\infty,\norm{f/\Delta}_\infty<\infty\}
            \end{equation*}
            is a Banach space with $\norm{\cdot}_1+\norm{\cdot/\Delta}_\infty$, which is dense in $L^1(G)$ and a right Banach $G$-module.
    \end{itemize}
\end{lemma}
\begin{proof}
    TODO.
\end{proof}
\begin{theorem}
    Let $\nu\in M_a(G)$ with $f=\frac{\d{\nu}}{\d{m}}\in L^1(G)$.
    \begin{enumerate}[nl,r]
        \item For $\mu\in M(G)$, $\frac{\d{(\mu*\nu)}}{\d{m}}=\mu*f$ and $\frac{\d{(\nu*\mu)}}{\d{m}}=f*\mu$
        \item If, further, $\mu\in M_a(G)$ with $g=\frac{\d{\mu}}{\d{m}}$, then $\frac{\d{(\mu*\nu)}}{\d{m}}=g*f$ where
            \begin{equation*}
                g*f=\int_Gg(x)x*fd{x}=\int_Gf(x)g*x\d{x}
            \end{equation*}
    \end{enumerate}
\end{theorem}
\begin{proof}
    Note that $C_c(G)\subseteq(L^1\cap C_0(G))\cap(L^1\cap C^\Delta(G))$.
    Let $(f_n)_{n=1}^\infty\subset C_c(G)$ such that $\lim_{n\to\infty}\norm{f-f_n}_1=0$, and then for $g\in C_c(G)$
    \begin{align*}
        \int_G h\d(\nu*\mu)&=\int_G\int_G h(xy)f(x)\d{x}\d{\mu(y)}\\
                           &= \int_G\int_G h(x)f(xy^{-1})\frac{1}{\Delta(y)}\d{x}\d{\mu(y)}\\
                           &= \lim_{n\to\infty}\int_G\int_Gh(x)f_n(xy^{-1})\frac{1}{\Delta(y)}\d{x}\d{\mu(y)}\\
                           &= \lim_{n\to\infty}\int_G h(x)\int_Gf_n(xy^{-1})\frac{1}{\Delta(y)}\d{\mu(y)}\d{x}\\
                           &= \lim_{n\to\infty}\int_G h(x)f_n*\mu(x)\d{x} &7 f_n\in C_c(G)\subseteq L^1\cap C^\delta(G)\\
                           &= \int_G h(x)f*\mu(x)\d{x} && \norm{f_n*\mu-f*\mu}_1\leq\norm{f_n-f}_1\norm{\mu}_1
    \end{align*}
    so $\frac{\d{(\nu*\mu)}}{\d{m}}=f*\mu$.
    The left case is similar, and (ii) is similar.
\end{proof}
Note that we may write for $m$-a.e. $x$,
\begin{equation*}
    f*g(x)=\int_Gf(y)g(y^{-1}x)\d{y}=\int_Gf(xy^{-1})g(y)\frac{1}{\Delta(y)}\d{y}
\end{equation*}
\begin{remark}
    In the finite group setting, representations of $G$ are in correspondence with submodules of $\C[G]$.
    A natural question to ask is: when does $M(G)$ replace $\C[G]$?

    If $Q=M(G)/M_c(G)\cong \ell^1(G)$, then
    \begin{equation*}
        x\cdot(\mu+M_c(G))=\delta_x*\mu+M_c(G)
    \end{equation*}
    so for $\alpha\in\ell^1(G)$,
    \begin{equation*}
        x\cdot\left(\sum_{y\in G}\alpha(y)\delta_y\right)=\sum_{y\in G}\alpha(y)\delta_{xy}=\sum_{y\in G}\alpha(x^{-1}y)\delta_y.
    \end{equation*}
    This is a bounded homomorphism of $G$ into $\mathcal{B}(\ell^1(G))\cong\mathcal{B}(Q)$ which is not strong operator continuous if $G$ is not discrete.
\end{remark}
A summability kernel in $L^1(G)$ is a net $(f_\alpha)$ introduced in A2.
We will show that
\begin{itemize}[nl]
    \item contractive summability kernals alwats exist: $\norm{f_\alpha}_1\leq 1$.
    \item If $(f_\alpha)$ is a summability kernel, then $\lim_\alpha f_\alpha*f=f=\lim_\alpha f*f_\alpha$ in $L^1$-norm, for $f$ in $L^1(G)$.
\end{itemize}
\begin{definition}
    Let $X$ be a Banach space.
    Then it is a \defn{Banach $L^1(G)$-module} if there is a bilinear action $(f,\xi)\mapsto f\cdot\xi:L^1(G)\times X\to X$ such that for $f,g\in L^1(G)$, $\xi\in X$,
    \begin{itemize}[nl]
        \item $f\cdot(g\cdot\xi)=(f*g)\cdot\xi$
        \item $\norm{f\cdot\xi}\leq c\norm{f}_1\norm{\xi}$ for some $c>0$.
    \end{itemize}
    Further, this is \defn{non-denenerate} if $X_0=\spn\{f\cdot\xi:f\in L^1(G)<\xi\in X\}$ is dense in $X$.
\end{definition}
\begin{theorem}
    If $X$ is a non-degenerate Banach $L^1(G)$-module, then it is a Banach $G$-module with
    \begin{equation*}
        \int_Gf(x)x\cdot\xi\d{x}=f\cdot\xi
    \end{equation*}
    for $f\in L^1(G)$ and $\xi\in X$.
\end{theorem}
\begin{proof}
    Let $(f_\alpha)$ be a contractive summability kernel in $L^1(G)$.
    We define for $x$ in $G$, $\xi_0=\sum_{j=1}^nf_j\cdot\xi_j$ in $X_0$,
    \begin{equation*}
        x\cdot\xi_0=\sum_{j=1}^n(x*f_j)\cdot\xi_j.
    \end{equation*}
    Notice that if $0=\sum_{j=1}^n f_j\cdot\xi_h$, then bilinearity of $X$ as an $L^1(G)$-module provides
    \begin{equation*}
        0=(x*f_\alpha)\cdot 0 =\sum_{j=1}^n(x*f_\alpha*f_j)\cdot\xi_j\to\sum_{j=1}^n(x*f_j)\cdot \xi_j.
    \end{equation*}
    Thus this map is well-defined.

    If $\xi_0=\sum_{j=1}^nf_j\cdot\xi_j\in X_0$ and $x\in G$, then
    \begin{align*}
        \norm{x\cdot\xi_0}&=\norm{\lim_\alpha\sum_{j=1}^n(x*f_\alpha*f_j)\cdot\xi_j}\\
                          &= \lim_\alpha\norm{(x*f_\alpha)\cdot\xi_0}\\
                          &\leq\lim_\alpha C\norm{x*f_\alpha}_1\norm{\xi_0}=C\norm{\xi_0}
    \end{align*}
    Hence, define $\pi_0(x)\in\mathcal{L}(X_0)$, $\pi_0(x)\xi_0=x\cdot\xi_0$ for $x\in G$, $\xi_0\in X_0$ satisfies that $\{\pi_0(x):x\in G\}$ is a group of operators bounded by $C$< and hence $\pi_0(x)$ extends uniquely to a linear bounded operator $\pi(x)$ on $X$.
    Thus $\{\pi(x):x\in G\}$ is a group in $\mathcal{B}(X)$, bounded by $C$.
    For $x\in G$ and $\xi\in X$, define $x\cdot\xi=\pi(x)\xi$.

    We wish to see strong operator continuity.
    Let $\epsilon>0$.
    If $\xi\in X$, there is $\xi_0=\sum_{j=1}^n f_j\cdot\xi_j\in X_0$ with $\norm{\xi-\xi_0}<\epsilon$.
    Then
    \begin{align*}
        \limsup_\alpha\norm{f_\alpha\cdot\xi-\xi}&\leq\limsup_{\alpha}\left[\underbrace{\norm{f_\alpha\cdot\xi-f_\alpha\cdot \xi_0}}_{\leq\norm{f_\alpha}_1\norm{\xi-\xi_0}}+\norm{f_\alpha\cdot\xi_0-\xi_0}+\norm{\xi_0-\xi}\right]\\
                                                 &\leq(C+1)\epsilon
    \end{align*}
    so that $\lim_\alpha f_\alpha*\xi=\xi$.
    Now let $\alpha$ be so $\norm{f_\alpha\cdot\xi-\xi}<\epsilon$ and, for $x_0\in G$,
    \begin{align*}
        \limsup_{x\to x_0}\norm{x\cdot\xi-x_0\cdot\xi}&\leq\limsup_{x\to x_0}\left[\underbrace{\norm{x\cdot\xi-(x*f_\alpha)}}_{\leq C\norm{\xi-f_\alpha\cdot\xi}}+\underbrace{\norm{(x*f_\alpha)\cdot\xi-(x_0*f_\alpha)\cdot\xi}}_{\leq C\norm{x*f_\alpha-x_0*f_\alpha}\norm{\xi}}+\underbrace{\norm{(x_0*f)\cdot\xi-x_0\cdot\xi}}_{\leq C\norm{f_\alpha\cdot\xi-\xi}}\right]\\
                                                      &\leq 2C\epsilon
    \end{align*}
\end{proof}
\subsection{Unitary Representations}
Let $\mathcal{H}$ be a hilbert space, and $\mathcal{U}(\mathcal{H})=\{U\in\mathcal{B}(H):U^*U=UU^*=I\}$ denote the unitary group.
This is a topological group with respect to the operator norms.
On $\mathcal{B}(\mathcal{H})$, we define two coarser topologies:
\begin{itemize}[nl]
    \item \defname{strong operator}
        The initial topology
        \begin{equation*}
            \tau_{\mathrm{so}}=\sigma\left(\mathcal{B}(\mathcal{H}),\{T\mapsto T\xi:\mathcal{B}(\mathcal{H})\to (\mathcal{H},\tau_{\norm{\cdot}}\}_{\xi\in\mathcal{H}}\right).
        \end{equation*}
    \item \defname{weak operator}
        The initial topology
        \begin{align*}
            \tau_{\mathrm{wo}} &= \sigma\left(\mathcal{B}(\mathcal{H}),\{T\mapsto\langle T\xi,\eta\rangle\}_{\xi,\eta\in\mathcal{H}}:\mathcal{B}(\mathcal{H})\to\C\right)\\
                               &= \sigma\left(\mathcal{B}(\mathcal{H}),\{T\mapsto T\xi:\mathcal{B}(\mathcal{H})\to(\mathcal{H},w)\}_{\xi\in\mathcal{H}}\right)
        \end{align*}
\end{itemize}
Notice that $\tau_{\mathrm{wo}}\subseteq\tau_{\mathrm{so}}$ on $\mathcal{B}(\mathcal{H})$.
That is, strong operator convergence implies weak operator convergence (in nets).
\begin{example}
    Let $B(\mathcal{B}(\mathcal{H}))=\{T\in\mathcal{B}(\mathcal{H}):\norm{T}\leq 1\}$, which is a semigroup in $\mathcal{B}(\mathcal{H})$.
    Then set
    \begin{itemize}[nl]
        \item\defname{unilateral shift}
            $S\in\mathcal{B}(\ell^2(\N))$, $S\delta_n=\delta_{n+1}$, so $S^*\delta_n=\delta_{n-1}$ if $n>1$, and $S^*\delta_0=0$ of $n=1$.
        \item\defname{bilateral shift}
            $U\in\mathcal{B}(\ell^2(\Z))$, $U\delta_n=\delta_{n+1}$ so $U^*\delta_n=\delta_{n-1}$
    \end{itemize}
    We now have
    \begin{enumerate}[nl,a]
        \item $\tau_{\mathrm{wo}}\subsetneq\tau_{\mathrm{so}}$ on $B(\mathcal{B}(\mathcal{H}))$, if $\dim\mathcal{H}=\infty$, since $S^n\to 0$ in $\tau_{\mathrm{wo}}$, while $S^n$ does not converge to anything in the strong operator topology.
        \item $T\mapsto T^*:\mathcal{B}(\mathcal{H})\to\mathcal{B}(\mathcal{H})$ is wo-wo continuous, but not so-so continuous if $\dim\mathcal{H}=\infty$.
            For example, $(S^n)^*=(S^*)^n\to 0$ but $S^n$ des not converge to $0$
        \item If $T_0\in\mathcal{B}(\mathcal{H})$, $T\mapsto TT_0$, $T\mapsto T_0T:\mathcal{B}(\mathcal{H})\to\mathcal{B}(\mathcal{H})$ are each wo-wo continuous.
            However, if $\dim\mathcal{H}=\infty$, then $(T,T')\mapsto TT'$ is not $\tau_{\mathrm{wo}}-\tau_{\mathrm{wo}}-\tau_{\mathrm{wo}}$ continuous.
            Note that $U^n,(U^*)^n\to 0$ as $n\to\infty$, but $U^nU^{*n}=I$.
    \end{enumerate}
\end{example}
\begin{proposition}
    \begin{enumerate}[nl,r]
        \item $(S,T)\mapsto ST:B(\mathcal{B}(\mathcal{H})\times B(\mathcal{B}(\mathcal{H})\to B(\mathcal{B}(\mathcal{H}))$ is $\tau_{\mathrm{so}}\times\tau_{\mathrm{so}}-\tau_{\mathrm{so}}$-continuous.
        \item $\tau_{\mathrm{so}}|_{\mathcal{U}(\mathcal{H})}=\tau_{\mathrm{wo}}|_{\mathcal{U}(\mathcal{H})}$
    \end{enumerate}
    Hence $(\mathcal{U}(\mathcal{H}),\tau_{\mathrm{wo}})=(\mathcal{U}(\mathcal{H}),\tau_{\mathrm{so}})$ is a topological group.
\end{proposition}
\begin{proof}
    \begin{enumerate}[nl,r]
        \item Let $T_\alpha \to T$ and $S_\alpha\to T$ strong operator in $B(\mathcal{B}(\mathcal{H})$.
            Then for $\xi\in\mathcal{H}$,
            \begin{align*}
                0\leq \norm{S_\alpha T_\alpha\xi-ST\xi}\leq\underbrace{\norm{S_\alpha T_\alpha\xi-S_\alpha T\xi}}_{\norm{T_\alpha\xi -T\xi}}+\norm{S_\alpha T\xi-ST\xi}\to 0
            \end{align*}
        \item Let $U_\alpha\to U$ in $\mathcal{U}(\mathcal{H})$.
            Then for $\xi\in\mathcal{H}$,
            \begin{align*}
                \norm{U_\alpha\xi-U\xi}^2 &= \langle U_\alpha\xi-U\xi,U_\alpha\xi-U\xi\rangle\\
                                          &= 2\norm{\xi}&2-2\Re\langle U_\alpha\xi,U\xi\rangle\\
                                          &\to 2\norm{\xi}^2-2\Re\langle U\xi,U\xi\rangle=0
            \end{align*}
            so that $U_\alpha\to U$ strong operator in $\mathcal{U}(\mathcal{H})$.
    \end{enumerate}
    We thus have that $(U,V)\mapsto UV:\mathcal{U}(\mathcal{H})\times\mathcal{U}(\mathcal{H})\to\mathcal{U}(\mathcal{H})$ is strong operator continuous, and hence weak operator continuous by (ii).
    In (b) above, we remarked that $U\mapsto U^{-1}=U^*$ is wo-wo continuous.
\end{proof}
\begin{remark}
    If $\dim\mathcal{H}=\infty$, then $(\mathcal{U}(\mathcal{H}),\tau_{\mathrm{wo}})$ is not locally compact.
\end{remark}
\begin{proposition}
    $\mathcal{U}(\mathcal{H})$ is the largest subgroup in $B(\mathcal{B}(\mathcal{H}))$.
\end{proposition}
\begin{proof}
    If $U,U^{-1}\in B(\mathcal{B}(\mathcal{H}))$, then for $\xi\in\mathcal{H}$,
    \begin{equation*}
        \norm{\xi}=\norm{U^*U\xi}\leq\norm{U\xi}\leq\norm{\xi}
    \end{equation*}
    so that $\norm{U\xi}=\norm{\xo}$.
    Hence $\langle\xi,\xi\rangle=\norm{\xi}^*=\norm{U\xi}^*2=\langle U^*U\xi,\xi\rangle$.
    Now for $\xi,\eta\in\mathcal{H}$, the polarization identity gives
    \begin{align*}
        4\langle\xi,\eta\rangle &= \sum_{k=0}^3i^k\langle\xi+i^k\eta,\xi+i^k\eta\rangle\\
                                &= \sum_{k=0}^3 i^k\langle U^*U(\xi+i^k\eta),\xi+i^k\eta\rangle\\
                                &= 4\langle U^*U\xi,\eta\rangle
    \end{align*}
    so $U^*U=I$.
    Then $U^*=U^*UU^{-1}=U^{-1}$.
\end{proof}
\begin{definition}
    Let $G$ be a locally compact group.
    A \defn{unitary representation} is a homomorphism $\pi:G\to\mathcal{U}(\mathcal{H})$ which is $\tau_G$-wo continuous.
\end{definition}
\begin{example}
    Consider $\lambda:G\to\mathcal{U}(L^2(G))$ given by $\lambda(x)f(y)=f(x^{-1}y)$ $m$-a.e., $f\in L^2(G)$.

    If $G$ is not a discrete group, then $\lambda:G\to(\mathcal{U}(\mathcal{H}),\tau_{\norm{\cdot}})$ is not continuous.
    However, $\lambda:G\to(\mathcal{U}(\mathcal{H}),\tau_{\mathrm{so}})$ is continuous (proof just like for translation on $L^1(G)$).
\end{example}
\begin{theorem}
    There is a bijective correspondence between any two of
    \begin{enumerate}[nl,r]
        \item unitary representations of $G$
        \item contractive representations of $G$ on Hilbert spaces
        \item non-degenerate bounded $*$-homomorphisms from $L^1(G)$ to $\mathcal{B}(\mathcal{H})$ where $\mathcal{H}$ is a Hilbert space
        \item non-degenerate bounded contractive homomorphisms from $L^1(G)$ to $\mathcal{B}(\mathcal{H})$, where $\mathcal{H}$ is Hilbert.
    \end{enumerate}
\end{theorem}
\begin{proof}
    \bij{i}{ii}
    Last proposition

    \bij{ii}{iv}
    Coincidence of Banach $G$-modules with Banach $L^1(G)$-modules, ($C=1$), $\pi:G\to B(\mathcal{B}(\mathcal{H}))$ is a continuous homomorphism with $\pi(e)=I$, then $\pi_1(f)\xi=\int_G f(x)\pi(x)\xi\d{x}$.

    If $\sigma:L^1(G)\to\mathcal{B}(\mathcal{H})$ homomorphism, $\norm{\sigma}\leq 1$, then define $\pi(x)=\lim_\alpha\sigma(x*f_\alpha)$ (weak operator) where $(f_\alpha)$ is a contractive summability kernel in $L^1(G)$ (A2Q1).

    \bij{i}{iii}
    Let $\pi:G\to\mathcal{U}(\mathcal{H})$ be a unitary representation.
    Then for $f\in L^1(G)$, $\xi,\eta\in\mathcal{H}$,
    \begin{align*}
        \langle\pi_1(f)^*\xi,\eta\rangle &= \langle\xi,\pi_1(f)\eta\rangle\\
                                         &= \int_G\langle\xi,\pi(x)\eta\rangle\overline{f(x)}\d{x}\\
                                         &= \int_G\langle\pi(x^{-1})\xi,\eta\rangle\overline{f(x)}\d{x}\\
                                         &= \int_G\langle\pi(x)\xi,\eta\rangle\overline{f(x^{-1})}\frac{1}{\Delta(x)}\d{x}=\langle\pi_1(f^*)\xi,\eta\rangle
    \end{align*}
    so $\pi_1(f)^*=\pi_1(f^*_$.

    Conversely, if $\sigma:L^1(G)\to\mathcal{B}(\mathcal{H})$ is a bounded $*$-homomorphism, then with $\pi$ as before, we have for $x$ in $G$
    \begin{align*}
        \pi(x)^*=\mathrm{wo}-\lim_{\alpha}\sigma(x*f_\alpha)^{-1}=\mathrm{wo}-\lim_\alpha(f_\alpha^**x^{-1})=\pi(x^{-1})
    \end{align*}
    (check last step!).
\end{proof}
\begin{definition}
    Let $G$ be a group.
    A function $u:G\to\C$ is \defn{of positive type} (or positive definite) if for any $x_1,\ldots,x_n\in G$, $[u(x_i^{-1}x_j)]$ is a positive semidefinite matrix.
    If $G$ is a locally compact group, let $\mathcal{B}^+(G)=\{u:G\to\C:u\text{ continuous positive type}\}$.
\end{definition}
\begin{theorem}[Gelfand-Naimark]
    Let $G$ be a locally compact group.
    Then $u\in B^+(G)$ if and only if there is a unitary representation $\pi:G\to\mathcal{U}(\mathcal{H})$ and $\xi\in\mathcal{H}$ such that $u(x)=\langle \pi(x)\xi,\xi\rangle$.
\end{theorem}
\begin{proof}
    We will use that $G$ is a topological group, not necessarily locally compact.

    \impl
    If $u=\langle\pi(\cdot)\xi,\xi\rangle$, then for $x_i$ in $G$ and $\lambda_i$ in $\C$, we have
    \begin{align*}
        \sum_{i=1}^n\sum_{j=1}^n\overline{\lambda_j}\lambda_iu(x_i^{-1}x_j)=\langle\sum_{j=1}^n\lambda_i\pi(x_j)\xi,\sum_{i=1}^n\lambda_i\pi(x_i)\xi\rangle\geq 0
    \end{align*}

    \impr
    Let $\C[G]=\left\{\sum_{i=1}^n\alpha_xx,\alpha_x\in\C,x\in G,\alpha_x\neq 0\text{ for finitely many }x\}$ denote the free $\C$-vector space over $G$.
    Define $\Lambda:G\to \mathcal{L}(\C[G])$ by
    \begin{equation*}
        \Lambda(x)\sum_{y\in G}\alpha_yy=\sum_{y\in G}\alpha_y(xy)=\sum_{y\in G}\alpha_{x^{-1}y}y.
    \end{equation*}
    Then $\Lambda(xx')=\Lambda(x)\Lambda(x')$ for $x,x'$ in $G$, and $\Lambda(e)=I$.

    On $\C[G]\times\C[G]$, define
    \begin{equation*}
        \left[\sum_{x\in G}\alpha_xx,\sum_{y\in G}\beta_yy\right]_y = \sum_{x\in G}\sum_{y\in G}\alpha_x\overline{\beta_y}u(y^{-1}x).
    \end{equation*}
    Notice that $[\cdot,\cdot]_u$ is positie and Hermitian, since for $x,y$ in $G$,
    \begin{equation*}
        \begin{pmatrix}
            u(e) & u(y^{-1}x)\\
            u(x^{-1} y) & u(e)
        \end{pmatrix}
    \end{equation*}
    is positive semidefinite, hence Hermitian, so $u(x^{-1}y)=u(y^{-1}x)$.
    Hence $[\cdot,\cdot]_u$ has Cauchy schwarz inequality $|[\alpha,\beta]_u|\leq[\alpha,\alpha]_u^{1/2}[\beta,\beta]_u^{1/2}$.
    Hence $\mathcal{K}_u=\{\alpha\in\C[G]:[\alpha,\alpha]_n=0\}$ is a subspace of $\C[G]$.

    Note that for $x\in G$, $\alpha,\beta\in\C[G]$,
    \begin{equation*}
        [\Lambda(x)\alpha,\Lambda(x)\beta]_u = \sum_{y\in G}\sum_{z\in G}\alpha_y\overline{\beta_z}u((xz)^{-1}xy)=[\alpha,\beta]_y.
    \end{equation*}
    In particular, $\Lambda(x)\mathcal{K}_u\subseteq\mathcal{K}_u$ for each $x\in G$.
    Hence we may define $\pi_0:G\to\mathcal{K}(\mathcal{H}_0)$ where $\mathcal{H}_0=\quot{\C[G]}{\mathcal{K}_u}$ and $\pi_0(x)(\alpha+\mathcal{K}_u)=\Lambda(x)\alpha+\mathcal{K}_u$ is well-defined.
    Furthermore, $\pi_0(xx')=\pi_0(x)\pi_0(x')$ for $x,x'\in G$, and $\pi_0(e)=I$.
    Define on $\mathcal{H}_0\times\mathcal{H}_0$
    \begin{equation*}
        \langle\alpha+\mathcal{K}_u,\beta+\mathcal{K}_u\rangle_u=[\alpha,\beta]_u
    \end{equation*}
    which is an inner product on $\mathcal{H}_0$.
    We note from above that each $\pi_0(x)$ is unitary on $\mathcal{H}_0$: $\pi_0(x^{-1})=\pi_0(x)$ and
    \begin{align*}
        \langle \pi_0(x)(\alpha+\mathcal{K}_u),\pi_0(x)(\beta+\mathcal{K}_u)\rangle_u = [\Lambda(x)\alpha,\Lambda(x)\beta]_u = [\alpha,\beta]_u =\langle\alpha+\mathcal{K}_u,\beta+\mathcal{K}_u\rangle
    \end{align*}
    We let $\mathcal{H}=\overline{\mathcal{H}_0}$ be the completion with respect to $\norm{\xi}=\langle\xi,\xi\rangle_u^{1/2}$, so $\mathcal{H}$ is a Hilbert space.
    Each element of the group of operators $\{\pi_0(x):x\in G\}$ extends to a unitary on $\mathcal{H}$, so we get a group of unitaries $\{\pi(x):x\in G\}$.
    Notive that for $x\in G$,
    \begin{equation*}
        \langle\pi(x)(e+\mathcal{K}_u),e+\mathcal{K}_u\rangle=[x,e]_u=u(x)
    \end{equation*}
    so we let $\xi=e+\mathcal{K}_u$.
    Notice then, that
    \begin{equation*}
        |u(x)|=|\langle\pi(x)\xi,\xi\rangle|\leq\norm{\pi(x)\xi}\norm{\xi}\leq\norm{\xi}^2=u(e)
    \end{equation*}
    so $u$ is bounded.
    If $\alpha,\beta\in\C[G]$,
    \begin{equation*}
        \langle\pi(x)(\alpha+\mathcal{K}_u),\beta+\mathcal{K}_u\rangle=\sum_{y\in G}\sum_{z\in G}\alpha_y\overline{\beta_z}u(z^{-1}xy)
    \end{equation*}
    so $x\mapsto\langle\pi(x)(\alpha+\mathcal{K}_u),\beta+\mathcal{K}_u\rangle$ is continuous.
    
    If $\xi,\eta\in\mathcal{H}$, $\epsilon>0$, find $\alpha,\beta\in\C[G]$ so $\norm{(\alpha+\mathcal{K}_u)-\xi}\leq\epsilon$ and $\norm{(\beta+\mathcal{K}_u)-\eta}<\epsilon$ and then
    \begin{align*}
        |\langle\pi(x)\xi,\eta\rangle - \langle\pi(x)(\alpha+\mathcal{K}_u),\beta+\mathcal{K}_u\rangle| &\leq |\langle\pi(x)(\xi-(\alpha+\mathcal{K}_u)),\eta\rangle|+|\langle\pi(x)(\alpha+\mathcal{K}_u),\eta-(\beta+\mathcal{K}_u)\rangle|\\
                                                                                                        &\leq \epsilon\norm{\eta}+(\norm{\xi}+\epsilon)\epsilon
    \end{align*}
    so by taking limits of bounded continuous functions, we see that $\pi:G\to(\mathcal{U})\mathcal{H},\tau_{\mathrm{wo}})$ is continuous.
\end{proof}
\begin{proposition}
    Let $\pi:G\to\mathcal{U}(H)$ be a unitary representation and $K\subseteq H$ a closed subspace.
    Let $P=P_K$ denote the orthogonal projection onto $K$.
    Then $\pi(G)K\subseteq K$ ($K$ is $\pi$-invariant) if and only if $P\pi(x)=\pi(x)\circ P$ for any $x\in G$.
\end{proposition}
\begin{proof}
    We have
\end{proof}
\begin{definition}
    We say that a unitary representation $\pi$ is \defn{irreducible} if it admits no closed invariant subspaces.
\end{definition}
\begin{lemma}
    Let $\pi:G\to\mathcal{U}(H)$ be a unitary representation.
    Then $\pi$ is irreducible if and only if
    \begin{equation*}
        \pi(G)'=\{T\in\mathcal{B}():T\pi(x)=\pi(x) T\text{ for all }x\in G\}=\C U.
    \end{equation*}
\end{lemma}
\begin{proof}
    \impr
    Let $K$ be $\pi$-invariant, $\{0\}\subsetneq K\subsetneq K$.
    Then $P_K\in\pi(G)'\setminus \C I$.

    \impl
    Let $T\in\pi(G)'$.
    Then for $x$ in $G$,
    \begin{equation*}
        T^*\pi(x) = (\pi(x^{-1}T)^*=(T\pi(x^{-1}))^*=\pi(x)T^*
    \end{equation*}
    so that $T^*\in\pi(G)'$.
    Thus if $T\in\pi(G)'\setminus \C I$, at least one of $\Re T=\frac{1}{2}(T+T^*)$ or $\im T=\frac{1}{2i}(T-T^*)$ is not in $\C I$.
    Thus there is $S=S^*\in\pi(G)'\setminus\C I$.
    Since normal operators with singleton spectrum are always always multiples of the identity, we must have $|\sigma(H)|\geq 2$.

    Let $U$ be a non-empty non-dense open set in $\sigma(S)$ and find $f\in C(\sigma(S))$ such that $f|_U=0$.
    If $g\in C(\sigma(S))$ is non-zero with $\supp(g)\subseteq U$, then$f(H)g(H)=fg(H)=0$, so $\im g(H)\subseteq\ker f(H)\neq\{0\}$.
    THen $\ker f(H)$ is $\pi$-invariant: if $x\in G$ ahd $\xi\in\ker f(H)$ is arbitrary, then $f(H)\pi(x)\xi=\pi(x)f(H)\xi=0$ so $\pi(x)\xi\in\ker f(H)$.

    % Recall that there is an isometry $\Phi:C(\sigma(S))\to\mathcal{B}(H)$ such that there is a linear isometry $\Phi:C(\sigma(S))\to\mathcal{B}(H)$ such that
\end{proof}
\begin{corollary}
    If $G$ is abelian, then every irreducible unitary representation is one-dimensional.
\end{corollary}
\begin{proof}
    If $\pi$ is an irreducible representation, then $\pi(G)\subseteq\pi(G)'=\C I$.
    Hence for $x\in G$, there is $\sigma(x)\in\C$ such that $\pi(x)=\sigma(x) I$.
    Notice that for also $y\in G$,
    \begin{equation*}
        \sigma(xy)I=\pi(xy)I=\pi(x)\pi(y)I=\sigma(x)\sigma(y)I
    \end{equation*}
    and
    \begin{equation*}
        \overline{\sigma(x)}=(\sigma(x)I)^*=\pi(x)^*=\pi(x^{-1})=\sigma(x^{-1})I
    \end{equation*}
    so $\sigma(x)\in\pi$.
\end{proof}
\section{Gelfand Theory}
\begin{definition}
    We say that $\mathcal{A}$ is a \defn{commutative Banach algebra} if $\mathcal{A}$ is a Banach space with a commuting associative product such that $\norm{ab}\leq\norm{a}\norm{b}$.
\end{definition}
\begin{example}
    \begin{enumerate}[nl,r]
        \item Let $X$ be a locally compact Hausdorff space, $\mathcal{A}=C_0(X)$.
            This is unital if and only if $X$ is compact.
        \item Consider $L^1(G)$ where $G$ is an abelian locally compact group.
            This is unital if and only if $G$ is discrete.
            The reverse is clear; forwardly, if $f$ acts as a unit, i.e. $f*g=g$ for all $g\in L^1(G)$, let $(f_\alpha)_\alpha$ be a summability kernel so $f=\lim_\alpha f*f_\alpha=\lim_\alpha f_\alpha$.
            Thus if $h\in C_0(G)$, then
            \begin{equation*}
                \int_G hf\d{m}=\lim_\alpha\int_G hf_\alpha\d{m}=h(e)=\int_G h\delta_e
            \end{equation*}
            and hence $m_f=\delta_e$.
            Thus $M_a(G)\cap M_d(G)\neq\emptyset$, so $G$ is discrete.
        \item Let $S$ be an abelian semigroup, and define a convolution product on $\ell^1(S)$ by
            \begin{equation*}
                \left(\sum_{s\in S}\alpha(s)\delta_s\right)*\left(\sum_{t\in S}\beta(t)\delta_t\right)=\sum_{u\in G}\left(\sum_{\substack{(s,t)\in S\times S\\st=u}}\alpha(s)\beta(t)\right)\delta_u.
            \end{equation*}
            If $S$ is unital, then $\ell^1(S)$ is unital.
            Otherwise, consider $S=\{s_1,\ldots,s_n\}$ by $s_is_j=s_i$ if $i\neq j$, and 0 otherwise.
            Then the unit has norm $n$ (?)
        \item If $G$ is an abelian locally compact group, $M(G)$ is a commutative Banach algebra.
        \item Let $\mathbb{D}=\{z\in\C:|z|<1\}$ and let $\mathcal{A}(\mathcal{D})=\{f\in C(\overline{\mathbb{D}}):f\text{ holomorphic}\}$ (\TODO{related to semigroup algebra $\ell^1(\{0\}\cup\N)$ from analytic functions}.)
    \end{enumerate}
\end{example}
\begin{definition}
    The \defn{(Gelfand) spectrum} of a commutative Banach algebra $\mathcal{A}$ is $\hat\mathcal{A}=\{\chi:\mathcal{A}\to\C:\chi\text{ linear and multiplicative}\}\setminus\{0\}$.
    We refer to $\chi$ as \defn{characters}.
\end{definition}
\begin{proposition}
    Let $\mathcal{A}$ be a unital commutative Banach algebra.
    Then for $\chi\in\hat{\mathcal{A}}$,
    \begin{enumerate}[nl,r]
        \item $\chi(1_{\mathcal{A}})=1$
        \item If $a\in\mathcal{A}^\times$, then $\chi(a)\neq 0$
        \item $|\chi(a)|\leq\norm{a}$, i.e. $\chi\in\mathcal{A}^*$, $\norm{\chi}\leq 1$.
    \end{enumerate}
\end{proposition}
\begin{proof}
    \begin{enumerate}[nl,r]
        \item There is $a$ so that $\chi(a)\neq 0$, so $\chi(1_{\mathcal{A}})\chi(a)=\chi(a)$ so $\chi(1_{\mathcal{A}})=1$.
        \item $\chi(a)\chi(a^{-1})=1$
        \item If $\lambda\in\C$, then $|\lambda|>\norm{a}$ for some $a\in \mathcal{A}$, then $\norm{\frac{1}{\lambda}a}<1$ so
            \begin{equation*}
                (\lambda 1_{\mathcal{A}}-1)^{-1}=[\lambda(1_{\mathcal{A}}-\frac{1}{\lambda}a)]^{-1}=\frac{1}{\lambda}\sum_{n=0}^\infty\frac{1}{\lambda^n}a^n.
            \end{equation*}
            Thus $\lambda-\chi(a)=\chi(\lambda 1_{\mathcal{A}}-a)\neq 0$.
            But $\chi(\chi(a)1_{\mathcal{A}}-a)=0$ so we cannot have $|\chi(a)|\geq\norm{a}$.
    \end{enumerate}
\end{proof}
\begin{corollary}
    If $\mathcal{A}$ is unital, then $\hat{\mathcal{A}}$ is $w^*$-compact in $\mathcal{A}$.
\end{corollary}
\begin{proof}
    Since $\hat{\mathcal{A}}\subseteq B(\mathcal{A}^\times)$, it suffices to show that $\hat{\mathcal{A}}$ is $w^*$-closed.

    If $\chi\in\overline{\hat{A}}^{w^*}$, say $\chi=\lim_\alpha\chi_\alpha$ for $\chi_\alpha\in\hat{\mathcal{A}}$.
    Then for $a,b\in\mathcal{A}$,
    \begin{align*}
        \chi(ab) &= \lim_\alpha\chi_\alpha(ab)=\lim_\alpha\chi_\alpha(a)\chi_\alpha(b)=\chi(a)\chi(b)\\
        \chi(1_{\mathcal{A}}) &= \lim_\alpha\xhi_\alpha(1_{\mathcal{A}})=1
    \end{align*}
    so $\chi\neq 0$.
\end{proof}
\begin{lemma}
    Let $\mathcal{A}$ be a unital commutative Banach algebra, and $\mathcal{I}\subsetneq\mathcal{A}$ an ideal.
    Then
    \begin{enumerate}[nl,r]
        \item$\mathcal{I}\cap\mathcal{A}^\times=\emptyset$,
        \item $\overline{\mathcal{I}}\subsetneq\mathcal{A}$ is an ideal, and
        \item $\mathcal{I}$ is contained in a proper maximal ideal $M$
        \item if $\mathcal{I}$ is maximal, it is closed.
    \end{enumerate}
\end{lemma}
\begin{proof}
    \begin{enumerate}[nl,r]
        \item Standard.
        \item If $\norm{b}<1$ in $\mathcal{A}$, then $1_{\mathcal{A}}-b\in\mathcal{A}^\times$ with $\left(1_{\mathcal{A}-b}^{-1}\right)=\sum_{n=0}^\infty b^n$.
            Thus $\mathcal{\mathcal{I}}\cap\left(1_{\mathcal{A}}+B^0(\mathcal{A})\right)=\emptyset$.
            Thus $\overline{I}\cap(1_{\mathcal{A}}+B^0(\mathcal{A}))=\emptyset$ too.
            If $a\in\overline{\mathcal{I}}$, then $a=\lim_{n\to\infty}a_n$ with each $a_n\in\mathcal{I}$.
            If $b\in\mathcal{A}$, then $ba=\lim_{n\to\infty}ba_n$ with each $ba_n\in\mathcal{I}$.
        \item Standard.
        \item $\mathcal{I}\subseteq\overline{\mathcal{I}}\subsetneq\mathcal{A}$
    \end{enumerate}
\end{proof}
\begin{theorem}
    If $\mathcal{A}$ is a unital Banach algebra, then the ``spectrum'' $\sigma(a)=\{\lambda\in \C:\lambda 1_{\mathcal{A}}-a\in \mathcal{A}\setminus\mathcal{A}^\times\}$ is a non-empty compact subset of $\C$.
\end{theorem}
\begin{proof}
    Just like for $\mathcal{B}(X)$.
\end{proof}
\begin{theorem}[Mazur]
    If $\mathcal{A}^\times=\mathcal{A}\setminus\{0\}$, then $\mathcal{A}\cong\C$.
\end{theorem}
\begin{proof}
    If there were $a\in\mathcal{A}\setminus \C1_{\mathcal{A}}$, then $\lambda 1_{\mathcal{A}}-a\neq 0$ for all $\lambda\in\C$, so $\lambda 1_{\mathcal{A}}-a\in\mathcal{A}^*$.
    This contradicts the last theorem.
\end{proof}
\begin{theorem}[Maximal Ideals]
    Let $\mathcal{A}$ be a unital commutative Banach algebra.
    Then $\{\ker\chi:\chi\in\hat{\mathcal{A}}\}$ is the collection of distinct maximal ideals.
\end{theorem}
\begin{proof}
    Each $\quot{\mathcal{A}}{\ker\chi}\cong\chi(\mathcal{A})=\C$, so $\ker\chi$ is maximal.
    If $\ker\chi=\ker\chi'$, then $\chi(a)1_{\mathcal{A}}-a\in\ker\chi=\ker\chi'$ so $0\chi'(\chi(a)1_{\mathcal{A}}-a)=\chi(a)-\chi'(a)$ for $a\in\mathcal{A}$.

    Conversely, if $\mathcal{M}$ is a maximal idea in $\mathcal{A}$, then $\quot{\mathcal{A}}{\mathcal{M}}$ is a field.
    By Mazur's theorem, $\quot{\mathcal{A}}{\mathcal{M}}\cong\C$, so the projection map is a character.
\end{proof}
\begin{corollary}
    \begin{enumerate}[nl,r]
        \item $\mathcal{A}\setminus\mathcal{A}^\times=\bigcup_{\chi\in\hat{\mathcal{A}}}\ker\chi$
        \item $\sup_{\chi\in\hat{\mathcal{A}}}|\chi(a)|=\lim_{n\to\infty}\norm{a^n}^{1/n}$
    \end{enumerate}
\end{corollary}
\begin{proof}
    \begin{enumerate}[nl,r]
        \item We already saw $\mathcal{A}^\times\subseteq\mathcal{A}\setminus\bigcup_{\chi\in\hat{\mathcal{A}}}\ker\chi$.
            If $a\in\mathcal{A}\setminus\mathcal{A}^\times$, then $a\mathcal{A}$ is a proper ideal contained in a maximal ideal $\ker\chi$.
        \item If $a\in\mathcal{A}$ and $\lambda\in\C$, then $\lambda\in\sigma(a)$ if and only if $\lambda=\chi(a)$.
            Then apply the spectral radius formula.
    \end{enumerate}
\end{proof}
\section{Abelian Locally Compact Groups}
Let $G$ be an abelian locally compact group.
Then $M(G)$ and $L^1(G)$ are commutative.
Let
\begin{equation*}
    \widehat{G} = \{\sigma:G\to\mathbb{T}|\sigma\text{ continuous group homomorphism}\}
\end{equation*}
Then Schur's lemma tells us that $\widehat{G}$ is the set of all irreducible unitary representations of $G$, as $U(\C)=\mathbb{T}$.
\begin{theorem}
    Let $G$ be an abelian locally compact group.
    \begin{enumerate}[nl,r]
        \item $\widehat{L^1(G)}\cong\widehat{G}$, where each element of $\widehat{L^1(G)}$ is given by $f\mapsto\langle f,\sigma\rangle=\int_G f\sigma\d{m}$.
        \item $\widehat{G}\cup\{0\}$ is $w^*$-compact in $L^\infty(G)\cong L^1(G)^*$, and thus $\widehat{G}$ is locally compact
        \item $(\widehat{G},w^*)$ is a locally compact group under pointwise operations ($\sigma^{-1}=\overline{\sigma}$).
    \end{enumerate}
\end{theorem}
\begin{proof}
    \begin{enumerate}[nl,r]
        \item If $\sigma\in\widehat{G}$, so $\sigma:G\to\pi = U(\C)$, then $\chi=\sigma_1:L^1(G)\to\mathcal{B}(\C)\cong\C$ is a non-degenerate, and hence non-zero multiplicative functional, i.e. in $\widehat{L^1(G)}$.
            Also, if $\sigma\neq\tau$, then $\sigma_1\neq\tau_1$, since there exists $f$ such that $\int_G f\sigma\neq\int_G f\tau$.
            
            Now, let $\chi\in\widehat{L^1(G)}$.
            If $G$ is discrete, then $L^1(G)\cong\ell^1(G)$ is unital, and hence we have $\norm{\chi}\leq 1$.
            If $G$ is not discrete, we let $\mathcal{A}=L^1(G)\oplus \C\delta_e\subset M(G)$.
            Then define $\tilde\chi:\mathcal{A}\to\C$ by $\tilde\chi(f+\alpha\delta_e)=\chi(f)+\alpha$.
            It is straightforward that $\tilde\chi\in\widehat{\mathcal{A}}$, so $\norm{\chi}\leq\norm{\tilde\chi}\leq 1$ from a proposition last class.
            Thus $\chi:L^1(G)\to\mathcal{B}(\C)\cong\C$ is a contractive representation, and hence self-adjoint, and hence there is $\sigma:G\to U(\C)\cong\pi$ such that $\chi=\sigma_1$.
        \item It suffices to show that $\widehat{G}\cup\{0\}$ is $w^*$-closed in $L^\infty(G)$.
            Let $\sigma$ be an element of the $w^*$-closure of $\widehat{G}$, so $\sigma=w^*-\lim_\alpha\sigma_\alpha$ for each $\sigma_\alpha\in\widehat{G}$.
            Then for $f,g\in L^1(G)$,
            \begin{equation*}
                \langle f*g,\sigma\rangle=\lim_\alpha\langle f*g,\sigma_\alpha\rangle=\lim_\alpha\langle f,\sigma_\alpha\rangle\langle g,\sigma_\alpha\rangle=\langle f,\sigma\rangle\langle g,\sigma\rangle
            \end{equation*}
            so $\sigma_1\in\widehat{L^1(G)}\cup\{0\}$, i.e. $\sigma\in\widehat{G}\cup\{0\}$.

            Since the $w^*$-topology is Hausdorff, there is a neighbourhood $U$ of $X$ and $V$ of $)$ so that $U\cap V\neq\emptyset$.
            Hence $0\in\overline{U}^{w^*}$, so $U$ is a relatively compact neighbourhood of $X$ by the spectral radius formula.
        \item Let $\pi_m:L^\infty(G)\to\mathcal{B}(L^2(G))$ be given by $\pi_m(\phi)f=\phi f$.
            Then for $f,g\in L^2(G)$, then $f\overline{g}\in L^1(G)$ by Cauchy-Schwarz and
            \begin{equation*}
                \langle \pi_m(\phi)f,g\rangle=\int_G\phi f\overline{g}\d{m}=\langle\phi,f\overline{g}\rangle.
            \end{equation*}
            On the other hand, if $f\in L^1(G)$, $f=\sgn f|f|^{1/2}|f|^{1/2}$ with
            \begin{equation*}
                \langle\phi,f\rangle=\int_G\phi f\d{m}=\langle\pi_m(\phi)\sgn f|f|^{1/2},|f|^{1/2}\rangle.
            \end{equation*}
            Hence $\phi_\alpha\to \phi$ $w^*$ in $L^\infty(G)$ if and only if $\pi_m(\phi_\alpha)\to\pi_m(\phi)$ in the weak operator topology, i.e. $\pi_m$ is a $w^*-w.o.$ homeomorphism.

            Then $\pi_m(\widehat{G})\subseteq\mathcal{U}(L^2(G))$ so $\widehat{G}$ embeds homeomorphically into a topological group, and is thus a topological group.
    \end{enumerate}
\end{proof}
\begin{proposition}
    \begin{enumerate}[nl,r]
        \item If $G$ is discrete, then $\widehat{G}$ is compact.
        \item If $G$ is compact, then $\widehat{G}$ is discrete.
    \end{enumerate}
\end{proposition}
\begin{proof}
    \begin{enumerate}[nl,r]
        \item We saw that $L^1(G)\cong \ell^1(G)$ is unital with $\widehat{G}\cong\widehat{\ell^1(G)}$.
        \item If $\sigma\in\widehat{G}\setminus\{1\}$, we normalize $m(G)=1$.
            Given $y\in G$ so $\sigma(y)\neq 1$, we have
            \begin{equation*}
                \int_G\sigma(x)\d{x}=\ing_G\sigma(yx)\d{x}=\sigma(y)\int_G\sigma(x)\d{x}
            \end{equation*}
            so that $\langle\sigma,1\rangle=\int_G\sigma(x)\d{x}=0$.
            Also, $\langle 1,1\rangle=\int_G 1\d{x}=1$.
            Hence
            \begin{equation*}
                \{\tau\in\widehat{G}:\langle\tau,1\rangle-\langle 1,1\rangle<1/2\}=\{1\}
            \end{equation*}
            is $w^*$-open.
    \end{enumerate}
\end{proof}
\begin{example}
    \begin{enumerate}
        \item Let $G=\Z$.
            If $\sigma\in\hat{\Z}$, let $z=\sigma(1)$, then $z^n=\sigma(n)$ for any $n\in\Z$.
            Write $\sigma=\sigma_z$.
            Then $z\mapsto\sigma_z:\pi\to\widehat{\Z}$ is a bijection.
            If $f=\sum_{n\in\Z}f(n)\delta_n\in\ell^1(\Z)$, then $\langle f,\sigma_z\rangle=\sum_{n\in\Z}f(n)z^n$, so $z\mapsto\langle f,\sigma_z\rangle$ is continuous, so $z\mapsto\sigma_z:\mathbb{T}\to\widehat{Z}$ is continuous, and thus a homeomorphism.
        \item Let $G=\R$ and $\sigma\in\widehat{R}$, so $\sigma(0)=1$.
            Hence there is $y_0>0$ such that $\int_0^y\sigma(x)\d{x}\neq 0$ for $y\in[-y_0,y_0]$.
            For such $y$,
            \begin{equation*}
                0\neq\int_0^y\sigma(x)\d{x}=\int_y^{2y}\sigma(y+x)\d{x}=\sigma(y)\int_y^{2y}\sigma(x)\d{x}
            \end{equation*}
            so that
            \begin{equation*}
                \sigma(y) = \frac{\int_0^y\sigma(x)\d{x}}{\int_y^{2y}\sigma(x)\d{x}}.
            \end{equation*}
            Thus the fundamental theorem of calculus shows that $\sigma$ is differentiable in a neighbourhood of $0$.
            For $x\in\R$, we have $\sigma'(x)=\sigma(x)\sigma'(0)$.
            Let $f(x)=e^{-\sigma'(0)x}\sigma(x)$ so $f(0)=1$, $f'(x)=0$, so $f(x)=1$ by the mean value theorem and in fact $\sigma(x)=e^{\sigma'(0)x}$.
            Since $\sigma(\R)\subseteq\mathbb{T}$, we have that $\sigma'(0)=it\in i\R$.
            Write $\sigma=\sigma_t$, $\sigma_t(x)=e^{ixt}$.

            The map $t\mapsto\sigma_t:\R\to\widehat{R}$ is an injective homomorphism and surjective, from above.
            If $t_n\to t_0$ in $\R$, $\sigma_{t_n}\to\sigma_{t_0}$ pointwise so by LDCT,
            \begin{equation*}
                \langle f,\sigma_{tn}\rangle=\int_{\R}f(x)e^{it_nx}\d{x}\fto{n\to\infty}\int_{\R}f(x)e^{it_0 x}\d{x}\langle f,\sigma_{t_0}\rangle
            \end{equation*}
            so $t\mapsto\sigma_t$ from $\R\to\widehat{R}$ is continuous.

            To see that the map is open, consider a $w^*$-open neighbourhood of $1=\sigma_0$: for $0<\epsilon<1-2/\pi$,
            \begin{align*}
                U_\epsilon &= \left\{t\in\R:\left\lvert\langle\idc{[-1,1]},\sigma_t\rangle-\langle\idc{[-1,1]},\sigma_0\rangle\right\rvert<\epsilon\right\}\\
                           &= \left\{t\in\R:\left\lvert\int_{-1}^1(e^{itx}-1)\d{x}\right\rvert<\epsilon\right\}\\
                           &= \left\{t\in\R:2\left\lvert\frac{\sin(t)}{t}-1\right\right<\epsilon\rvert\}=(-\delta,\delta)
            \end{align*}
            for some $0<\delta<\pi/2$.
            Thus $t\mapsto\sigma_t:R\to\widehat{\R}$ is open at 1.
        \item Let $G=\mathbb{T}$.
            Let $\sigma_1:\R\to\mathbb{T}$ be given by $\sigma_1(x)=e^{ix}$, which is continuous, surjective, and has $\ker\sigma_1=2\pi\Z$.
            If $\tau\in\widehat{\mathbb{T}}$, then $\tau\circ\sigma\in\widehat{\R}$, so $\tau\circ\sigma_1(x)=e^{itx}$ for all $x\in\R$.
            Also, $\ker\tau\supseteq\ker\sigma_1=2\pi\Z$, and hence $t\in\Z$.
            Write $t=n$, and for $z=\sigma_1(x)\in\pi$,
            \begin{equation*}
                \tau(z)=\tau\circ\sigma_1(x)=e^{inx}=z^n.
            \end{equation*}
            Write $\tau=\sigma_n$ for $n\in\Z$.
            We have that $n\mapsto\sigma_n$ is a bijection.
            It is a homeomorphism as both $\Z$ and $\widehat{\mathbb{T}}$ are discrete.
    \end{enumerate}
\end{example}
\begin{definition}[Fourier Transform]
    If $f\in L^1(G)$, define $\hat{f}:\widehat{G}\to\C$ by
    \begin{equation*}
        \hat{f}(\sigma)=\int_G f(x)\overline{\sigma(x)}\d{x}=\langle f,\overline{\sigma}\rangle.
    \end{equation*}
\end{definition}
\begin{proposition}[Riemann-Lebesgue, Gelfand]
    The map $f\mapsto\hat{f}:L^1(G)\to C_0(\widehat{G})$ is an injective homomorphism with
    \begin{enumerate}[nl,r]
        \item $\norm{\hat{f}}_\infty=\lim_{n\to\infty}\norm{f^{*n}}_1^{1/n}\leq\norm{f}_1$
        \item $\mathcal{A}(\widehat{G}) = \{\hat{f}:f\in L^1(G)\}$ is dense in $C_0(\widehat{G})$.
    \end{enumerate}
\end{proposition}
\begin{proof}
    \TODO{show injective}

    Since $\sigma\mapsto\overline{\sigma}$ is continuous on $\widehat{G}$, $\hat{f}$ is continuous.
    Letting $\hat{f}(0)=0$, we see for $\epsilon>0$, that $g_\epsilon=\max\{|\hat{f}|-\epsilon 1\}$ is supported on
    \begin{equation*}
        (\widehat{G}\cup\{0\})\setminus\{\sigma\in\widehat{G}\cup\{0\}:|\hat{f}(\sigma)|=|\langle f,\overline{\sigma}\rangle|<\epsilon\}
    \end{equation*}
    and hence compactly supported with $\norm{|\hat{f}|-g_\epsilon}_\infty\leq\epsilon$ so it follows that $\hat{f}\in C_0(\widehat{G})$.

    The formula in (i) is the spectra radius formula.
    We note that $\mathcal{A}$ is an
    \begin{itemize}[nl]
        \item algebra: $\langle f*g,\overline{\sigma}\rangle=\langle f,\overline{\sigma}\rangle\langle g,\overline{\sigma}\rangle$
        \item conjugate closed: $\widehat{f^*}(\sigma)=\overline{\sigma}_1(f^*)=\overline{\overline{\sigma}_1(f)}=\overline{\hat{f}(\sigma)}$
        \item point separating: $\sigma\neq\tau$ implies $\overline{\sigma}\neq\overline{\tau}$ so $\langle f,\overline{\sigma}\rangle\neq\langle f,\overline{\tau}\rangle$ for some $f\in L^1(G)$.
        \item separates points from $0$: $L^1(G)^*\cong L^\infty(G)\supseteq\hat{G}$.
    \end{itemize}
    Hence by Stone-Weierstrass, $\mathcal{A}(\widehat{G})$ is dense in $C_0(\widehat{G})$.

\end{proof}
\begin{theorem}
    Let $G$ be an abelian locally compact group.
    There is a bijective correspondence between
    \begin{enumerate}[nl]
        \item unitary representations $\pi:G\to\mathcal{U}(\mathcal{H})$, and
        \item non-degenerate contractive homomorphisms $\Pi:C_0(\widehat{G})\to\mathcal{B}(\mathcal{H})$ such that $\Pi(\overline{f})=\Pi(f)^*$
    \end{enumerate}
\end{theorem}
In the context of this theorem, we will commonly identifty $\pi:G\to\mathcal{U}(\mathcal{H})$ with integrated forms $\pi_1:L^1(G)\to\mathcal{B}(\mathcal{H})$ and $\pi_0:C_0(\widehat{G})\to\mathcal{B}(\mathcal{H})$.
\begin{proof}
    \imp{i}{ii}
    Recall that $\pi_1:L61(G)\to\mathcal{B}(\mathcal{H})$ satisfies that $\pi_1$ is contractive and $\pi_1(f^*)=\pi_1(f)^*$.
    Notice that $\pi_1(f)^*\pi_1(f)=\pi_1(f^**f)=\pi_1(f*f^*)=\pi(1(f)\pi_1(f)^*$, so $\pi_1(f)$ is normal.
    Hence
    \begin{equation*}
        \norm{\pi_1(f)}=\norm{\pi_1(f)^{2n}}^{1/(2n)}\leq\norm{f^{*(2n)}}_1^{1/(2n)}\to\norm{\hat{f}}_\infty.
    \end{equation*}
    Thus we may uniquely define linear continuous $\pi_0:C_0(\widehat{G})\to\mathcal{B}(\mathcal{H})$ such that $\pi_0(\hat{f})=\pi_1(f)$.
    Notice that $\pi_0(\overline{\hat{f}})=\pi_0(\hat{f^*})=\pi_1(f^*)=\pi_1(f)^*=\pi_0(\hat{f})^*$ and $\pi_0(\overline{\phi})=\pi_0(\phi)^*$ for $\phi\in C_0(\widehat{G})$.

    \imp{ii}{i}
    We let $\Pi_1:L^1(G)\to\mathcal{B}(\mathcal{H})$ be given by $\Pi_1(f)=\pi(\hat{f})$, so $\Pi_1$ is a homomorphism $(\widehat{f*g}=\hat{f}\hat{g})$ with $\norm{\Pi_1(f)}\leq \norm{\hat{f}}_\infty\leq\norm{f}_1$, and $\Pi_1(f^*)=\pi(\overline{\hat{f}})=\pi(\hat{f})^*=\Pi_1(f)^*$.
    The density of $A(\widehat{G})$ gives that $\Pi_1$ is non-degenerate.
    Hence there is $\pi:G\to\mathcal{U}(\mathcal{H})$ such that $\pi_1=\Pi_1$.
\end{proof}
\begin{example}
    Let $\mu\in M_+(\widehat{G})\setminus\{0\}$ and define $\pi_0^\mu:C_0(\widehat{G})\to\mathcal{B}(L^2(\widehat{G},\mu))$ be given by
    \begin{equation*}
        \pi_0^\mu(\phi)h=\phi(h),\phi\in C_0(\widehat{G}),h\in L^2(\widehat{G},\mu).
    \end{equation*}
    Clearly, $\pi_0^\mu$ is a contractive homomorphism with $\pi_0^\mu(\overline{\phi})=\pi_0^\mu(\phi)^*$.
    Let
    \begin{equation*}
        C_0^+(\widehat{G}) = \{\phi\in C_0^+(\widehat{G}):\phi(1)=1=\norm{\phi}_\infty\}
    \end{equation*}
    be directed by pointwise comparison.
    Then by inner regularity, $\lim\pi_0^\mu(\phi)h=h$: given $\epsilon>0$, let $K\subseteq\widehat{G}$ be compact so $\int_{\widehat{G}\setminus K}|h|^2<\epsilon^2$, eventually $\phi|_K=1$, so
    \begin{equation*}
        \left(\int_{\widehat{G}}|\phi h-h|^2\right)^{1/2}=\left(\int_{\widehat{G}}(\phi^2-1)|h|^2\right)^{1/2}\leq\left(\int_K|h|^2\right)^{1/2}<\epsilon.
    \end{equation*}
    Hence $\pi_0^\mu$ induces $\pi^\mu:G\to\mathcal{U}(L^2(G,\mu))$ such that $\pi^\mu(x)\pi_0^\mu(\hat{f})=\pi_0^\mu(\widehat{x*f})$.
    Now,
    \begin{equation*}
        \widehat{x*f}(\sigma)=\int_Gf(x^{-1}y)\overline{\sigma(y)}\d{y}=\int_G f(y)\overline{\sigma(xy)}=\overline{\sigma(x)}\hat{f}(\sigma).
    \end{equation*}
    Let $\hat{x}:\widehat{G}\to\mathbb{T}$ be given by $\hat{x}(\sigma)=\overline{\sigma(x)}$, so $\widehat{x*f}=\hat{x}\hat{f}$.
    Thus
    \begin{equation*}
        \pi^\mu(x)\pi_0^\mu(\hat{f})h=\pi_0^\mu(\hat{x}\hat{f})h=\hat{x}\hat{f}h
    \end{equation*}
    and since $A(\widehat{G})$ is dense in $C_0(\widehat{G})$, and since $\lim\pi_0^\mu(\phi)h=h$, we have
    \begin{equation*}
        \pi^\mu(x)h=\hat{x}h=\overline{\sigma(x)}h(\sigma).
    \end{equation*}
    Alternatively, we may denote $\pi^\mu=\int_{\widehat{G}}^\oplus\overline{\sigma}\d{\mu(\sigma)}$.

    Since $\mu$ is finite, $j:C_0(\widehat{G})+\C1\to L^2(\widehat{G},\mu)$ is a bounded linear map, putting functions into their $\mu$-a.e. equiv classes.
    Since $\mu$ is Radin, $j(C_0(\widehat{G}))$ is dense in $L^2(\widehat{G},\mu)$.
    Furthermore, $\pi_0^\mu(C_0(\widehat{G}))j(1)=j(C_0(\widehat{G}))$ is dense in $L^2(\widehat{G},\mu)$, i.e. $j(1)$ is a \defn{cyclic vector} for $\pi_0^\mu$.
\end{example}
\begin{theorem}[Stone]
    Let $G$ be an abelian locally compact group.
    Let $\pi:G\to\mathcal{U}(\mathcal{H})$ which admits a cyclic vector $\xi$, i.e. $\overline{\spn}\pi(g)\xi)=\mathcal{H}$.
    Then there is $\mu\in M_+(\widehat{G})$ and a unitary $U:L^2(\widehat{G},\mu)\to\mathcal{H}$ such that $U\pi^\mu(x)=\pi(x)U$ for $x$ in $G$ and $Uj(1)=\xi$.
\end{theorem}
\begin{proof}
    Let $\mu\in M(\widehat{G})\cong C_0(\widehat{G})^*$ be given by
    \begin{equation*}
        \int_{\widehat{G}}\phi\d{\mu}=\langle\pi_0(\phi)\xi|\xi\rangle.
    \end{equation*}
    Notice if $\phi\geq 0$, then
    \begin{equation*}
        \int_{\widehat{G}}\phi\d{\mu}=\langle\pi_0(\overline{\phi^{1/2}}\phi^{1/2})\xi|\xi\rangle=\norm{\pi_0(\phi^{1/2})\xi}^2\geq 0
    \end{equation*}
    so $\mu\in M_+(\widehat{G})$.

    Define $U_0:J(C_0(\widehat{G}))\to\pi_0(C_0(\widehat{G}))\xi$ by $U_0j(\phi)=\pi_0(\phi)\xi$ and note that
    \begin{equation*}
        \norm{\pi_0(\phi)\xi}^2=\langle\pi_0(|\phi|^2)\xi|\xi\rangle=\int_{\widehat{G}}|\phi|^2\d{\mu} = \norm{j(\phi)}_2^2.
    \end{equation*}
    Now $\overline{\spn}\pi(G)\xi=\mathcal{H}$ so $\overline{\pi_1(L^1(G))\xi}=\mathcal{H}$.
    To see this, write $h=\sum_{j=1}^n\alpha_j\pi(x_j)\xi$, where $(f_\alpha)$ is a summability kernel in $L^1(G)$, then
    \begin{equation*}
        \pi_1(f_\alpha)h=\sum_{j=1}^n\alpha_j\pi_1(f_\alpha*x_j)\xi\to\sum_{j=1}^n\alpha_j\pi(x_j)\xi.
    \end{equation*}

    Thus, $\overline{\pi_0(C_0(\widehat{G}))\xi}=\mathcal{H}$, i.e. $\xi$ is cyclic for $|pi_0$.
    Thus $U_0j(C_0(\widehat{G}))=\pi_0(C_0(\widehat{G}))\xi$ is dense in $\mathcal{H}$.
    Hence $U_0$ uniquely extends to a surjective isometry, i.e. unitary, $U:L^2(\widehat{G},\mu)\to\mathcal{G}$.
    Now for $\phi\in C_0(\widehat{G})$, $x\in G$,
    \begin{equation*}
        \pi(x)Uj(\phi)=\pi(x)\pi_0(\phi)\xi=\pi_0(\hat{x}\phi)\xi=Uj(\hat{x}\phi)=U\pi^\mu(x)j(\phi)
    \end{equation*}
    so $\pi(x)U=U\pi^\mu(x)$ on $\mathcal{H}$.
    Let $(f_\alpha)_\alpha$ be a continuous summability kernel for $L^1(G)$, then for $f\in L^1(G)$, $\hat{f_\alpha}\hat{f}=\widehat{f_\alpha*f}\to\hat{f}$ in $\norm{\cdot}_\infty$ and since $A(\widehat{G})$ is dense in $C_0(\widehat{G})$, $\lim_\alpha\widehat{f_\alpha}\phi=\phi$ uniformly in $C_0(\widehat{G})$ for $\phi$ in $C_0(\widehat{G})$.
    Hence $j(\widehat{f_\alpha})j(\phi)=j(\hat{f}_\alpha\phi)\to j(\phi)$ in $L^2(\widehat{G},\phi)$ and by density of $j(C_0(\widehat{G}))$ in $L^2(\widehat{G},\mu)$. $j(\hat{f}_\alpha)h\to h$, $h\in L^2(\widehat{G},\mu)$.
    
    In particular, $j(\hat{f}_\alpha 1)\to j(1)$.
    Thus
    \begin{equation*}
        Uj(1)=\lim_\alpha Uj(\hat{f}_\alpha)=\lim_\alpha\pi_0(\hat{f}_\alpha)\xi=\lim_\alpha\pi_1(f_\alpha)\xi=\xi
    \end{equation*}
\end{proof}
Reverse transform: if $\mu\in M(\widehat{G})$.
Define $\mu^\vee:G\to\C$ by $\mu^\vee(x)=\int_{\widehat{G}}\sigma(x)\d{\mu(\sigma)}$.
\begin{lemma}
    The map $(x,\sigma)\mapsto\sigma(x):G\times\widehat{G}\to\mathbb{T}$ is continuous.
\end{lemma}
\begin{proof}
    Fix $\sigma\in\widehat{G}$, $x\in G$, and let $f\in L^1(G)$ be so $\widehat{f}(\sigma)=1$.
    We have for $y\in G$ and $\tau\in\widehat{G}$
    \begin{align*}
        |\sigma(x)-\tau(y)| &\leq|f(\sigma)\overline{\sigma(x)}-\hat{f}(\tau)\overline{\tau(y)}| + |\hat{f}(\tau)\overline{\tau(y)}-\hat{f}(\sigma)\overline{\tau(y)}|\\
                            &= |\widehat{f*x}(\sigma)-\widehat{f*y}(\tau)|+|\hat{f}(\tau)-\hat{f}(\sigma)|\\
                            &\leq |\widehat{f*x}(\sigma)-\widehat{f*x}(\tau)|+|\widehat{f*x}(\tau)-\widehat{f*y}(\tau)|+|\hat{f}(\tau)-\hat{f}(\sigma)|\\
                            &\leq \norm{f*x-f*y}_1
    \end{align*}
    which converges to $0$.
\end{proof}
\begin{theorem}[Bochner]
    $B^+(G)=\{\check\mu:\mu\in M_+(\widehat{G})\}$.
    Hence $B(G)=\{\check{\mu}:\mu\in M(\widehat{G})\}$ is an algebra of (uniformly) continuous functions on $G$.
\end{theorem}
\begin{proof}
    Note that $\mu\in B^+(G)$ if and only if $\tilde\mu\in B^+(G)$ where $\tilde\mu(x)=\mu(x^{-1})$ for $x\in G$.
    If $\mu\in M_+(\widehat{G})$, then for $x\in G$,
    \begin{equation*}
        \check\mu(x^{-1})=\int_{\widehat{G}}\sigma(x^{-1})\d{\mu(\sigma)}=\int_{\widehat{G}}\hat{x}\d{\mu}=\langle\pi(x)^Mj(1)|j(1)\rangle
    \end{equation*}
    where $\pi^\mu:G\to U(L^2(\widehat{G},\mu))$ is given by $\pi^\mu(x)f=\hat{x}f$, so that $\check{\mu}\in B^+(G)$.

    If $\mu\in B^+(G)$, then Gelfand Naimark provides a representation $\pi:G\to \mathcal{U}(H)$ and $\xi\in H$ (which, by const. is cyclic) such that $u=\langle\pi(\cdot)\xi|\xi\rangle$.

    Let $\mu\in M_+(\widehat{G})$ and $U:L^2(\widehat{G},\mu)\to H$ so $U\pi^M(\cdot)=\pi(\cdot)U$ and $Uj(1)=\xi$.
    Then for $x\in G$,
    \begin{align*}
        u(x) &= \langle\pi(x)\xi|\xi\rangle = \langle\pi(x) Uj(x) | Uj(1)\rangle\\
             &= \langle U\pi^\mu(x) j(1) |Uj(1)\rangle=\langle\pi^\mu(x) j(1) | j(1)\rangle\\
             &= \int_G\hat{x}1\overline{1}\d{\mu}=\int_G\overline{\sigma(x)}\d{\mu}=\check{\mu}(x^{-1})
    \end{align*}
    i.e. $\tilde{u}=\check{\mu}$.

    We saw $\u\in M_+(\widehat{G})$ implies that $\check{\mu}\in B^+(G)$ and hence uniformly continuous.
    Hence if $\mu\in M(\widehat{G})=\spn M_+(\widehat{G})$, $\check{\mu}$ is uniformly continuous as well.
    Note that for $\mu,\nu\in M(G)$,
    \begin{equation*}
        (\mu*\nu)^\check(x)=\int_{\widehat{G}}\sigma(x)\d{(\mu*\nu)}=\int_{\widehat{G}}\int_{\widehat{G}}\sigma\tau(x)\d{\mu(\sigma)}\d{\mu(\tau)}=\check{\mu}(x)\check{\nu}(x).
    \end{equation*}
\end{proof}
\begin{proposition}
    The map $\mu\mapsto\check{\mu}:M(\widehat{G})\to B(G)$ is injective.
\end{proposition}
\begin{proof}
    If $f\in L^1(G)$, then for $\mu\in M(\widehat{G})$,
    \begin{align*}
        \int_{\widehat{G}}\hat{f}\d{\mu} &= \int_{\widehat{G}}\int_G f(x)\overline{\sigma(x)}\d{x}\d{\mu(\sigma)}\\
                                         &= \int_Gf(x)\int_{\widehat{G}}\sigma(x^{-1})\d{\mu(\sigma)}\d{x}\\
                                         &= \int_G f(x)\check{\mu}(x^{-1})\d{x}.
    \end{align*}
    Hence if $\check{\mu}=0$, then $\mu=0$ since $A(\widehat{G})=\{\hat{f}:f\in L^(G)\}$ is dense in $C_0(\widehat{G})$.
    Since $\mu\mapsto\check{\mu}$ is linear, we are done.
\end{proof}
\begin{example}
    \begin{enumerate}[r]
        \item Let $\lambda:G\to U(L^2(G))$ be given by $\lambda(x)f(y)=f(x^{-1}y)$.
            Just like to the fact that $x\mapsto x*f$ is continuous, $\lambda$ is continuous.

            Then $\lambda$ is called the \defn{left regular representation}.
            If $f\in L^2(G)$, $\langle\lambda(\cdot)f|f\rangle\in B^+(G)$.
            We note
            \begin{equation*}
                \langle\lambda(x)f|f\rangle=\int_G f(x^{-1}y)\overline{f(y)}\d{y}=\int_Gf(y)\overline{f(xy)}\d{y}.
            \end{equation*}
            If $f\in L^1\cap L^2(G)$, then since $G$ is unimodular,
            \begin{equation*}
                \langle\lambda(x)f|f\rangle=\int_Gf(y)\overline{f(xy)}\d{y}=\int_Gf(y)f^*(y^{-1}x^{-1})\d{y}=f^**f(x^{-1})
            \end{equation*}
            and $f^**f$ is (a.e. equal to) a continuous function on $G$.
        \item If $\phi\in C_0^+(G)$< then $U_\phi=\{x\in G:\phi(x)>0\}$ so $\overline{U_\phi}=\supp(\phi)$.
            Then $\phi^**\phi(x)=\int_G\phi(y)\phi(xy)\d{y}$ has $\phi^*\phi(x)>0$ if $xU_\phi\cap U_\phi\neq\emptyset$, i.e. $x\in U_\phi U_\phi^{-1}$.
            Furthermore, $\supp(\phi^**\phi)=\overline{U_\phi U_\phi^{-1}}=\supp(\phi)\supp(\phi)^{-1}$.
    \end{enumerate}
\end{example}
\begin{theorem}[Inversion]
    Let $B^1(G)=B\cap L^1(G)$.
    \begin{enumerate}[nl,r]
        \item $f\in B^1(G)$ implies that $\hat{f}\in L^1(\widehat{G})$
        \item With suitable normalization of Haar measure on $G$, $\widehat{G}$, we have for $f\in B^1(G)$ that $f(x)=\int_{\widehat{G}}\hat{f}(\sigma)\sigma(x)\d{\sigma}$, i.e. $\hat{f}\in L^1(\widehat{G})\cong M_a(\widehat{G})$, so $(\hat{f})^\check=f$.
    \end{enumerate}
\end{theorem}
\begin{proof}
    If $h\in L^1(G)$ and $f=\check{\mu}\in B^1(G)$, then
    \begin{equation*}
        \int_{\widehat{G}}\hat{h}\d{\mu}=\int_G h(x)\check{\mu}(x^{-1})\d{x}.
    \end{equation*}
    Now if also $g=\check{\nu}\in B^1(G)$, applying Fubini,
    \begin{align*}
        \int_{\widehat{G}}\hat{h}(\check{\nu})^\hat{}\d{\mu} &= \int_{\widehat{G}}\widehat{h*\check{\nu}}\d{\mu}=\int_G h*\check{\nu}(x)\check{\mu}(x^{-1})\d{x}\\
                                                             &= \int_G\int_G h(y)\check{\nu}(y^{-1}x)\check{\mu}(x^{-1})\d{y}\d{x}\\
                                                             &= \int_G\int_G h(xy)\check{\nu}(y^{-1})
    \end{align*}
    Since $A(\widehat{G})$ is dense in $C_0(\widehat{G})$ (and $C_c(G)$ is dense in $L^1(G)$), we find that $(\check{\nu})^{\hat{}}\d{\mu}=(\check{\mu})^{\hat{}}\d{\nu}$.

    We now build the Haar functional on $C_0(\widehat{G})$.
    Fix $\psi\in C_c(\widehat{G})$.
    For each $\sigma\in\supp(\psi)$, find $\mu\in C_c(G)\subseteq L^1(G)$ so $\hat{u}(\sigma)\neq 0$.
    Hence $\widehat{u^**u}=\hat{u^*}\hat{u}=|\hat{u}|>0$ in a neighbourhood of $\sigma$.
    Hence we can find $u_1,\ldots,u_n$ in $C_c(G)$ so
    \begin{enumerate}[a]
        \item $g=\sum_{j=1}^nu_j^**u_j\in B^+(C_c(G))\subseteq B^1(G)$, so $g=\check{\nu}$ for some $\nu\in M_+(\widehat{G})$ by Bochner's theorem.
        \item $\supp\psi\subseteq U_{\hat{g}}=\{\sigma\in\widehat{G}:\hat{g}(\sigma)>0\}$.
    \end{enumerate}
    Now let
    \begin{equation*}
        J(\psi) = \int_{\widehat{G}}\frac{\psi}{(\check{\nu})^{\hat{}}}\d{\nu}.
    \end{equation*}
    If also $f=\check{\mu}\in B^1(G)$ which satisfies (a),(b) above, then
    \begin{align*}
        J(\psi)&= \int_{\widehat{G}}\frac{\psi}{(\check{\nu})^{\hat{}}(\check{\mu})^{\hat{}}}(\check{\mu})^{\hat{}}\d{\nu}\\
               &= \int_{\widehat{G}}\frac{\psi}{(\check{\nu})^{\hat{}}(\check{\mu})^{\hat{}}}(\check{\nu})^{\hat{}}\d{\nu}\\
               &= \int_{\widehat{G}}\frac{\psi}{(\check{\mu})^{\hat{}}}\d{\nu}
    \end{align*}
    which shows a certain independence of the definition of $J$ from $g=\check{\nu}$.

    If $\phi,\psi\in C_c(\widehat{G})$, then using $g=\check{\nu}\in B^1(G)$, so (a), (b) satisfied for both $\phi,\psi$, linearity of $J$ is clear.
    Also, if $g=\hat{\nu}\in B^+\cap L^1(G)\setminus\{0\}$, then there is $\psi\in C_c^+(\widehat{G})$ so
    \begin{equation*}
        0< \int_{\widehat{G}}\psi\d{\nu} = \int_{\widehat{G}}\frac{\psi}{(\check{\mu})^{\hat{}}}(\check{\mu})^{\hat{}}\d{\nu}=\int_{\widehat{G}}\frac{\psi}{(\check{\mu})^{\hat{}}}(\check{\nu})^{\hat{}}\d{\nu}
    \end{equation*}
    and hence $J\neq 0$.
    Also, initial choice of $g=\hat{\nu}$, given $\psi\in C_c^+(G)$ shows that $J\geq 0$.
    Now let $\psi\in C_c(\widehat{G})$, $\tau\in\widehat{G}$.
    Find $g=\hat{\nu}\in B^+\cap L^1(G)$, so $U_{\widehat{g}}\supseteq\supp(\psi)\cup\supp(\psi\tau)$.
    Then
    \begin{align*}
        J(\psi\cdot\tau) = \int_{\widehat{G}}\frac{\psi(\tau\sigma)}{(\check{\nu})^{\hat{}}(\sigma)}\d{\nu(\sigma)}=\int_{\widehat{G}}\frac{\psi(\sigma)}{(\check{\nu})^{\hat{}}(\overline{\tau}\sigma)\d{\nu(\overline{\tau}\sigma)}.
    \end{align*}
    If $\mu=\delta_\tau *\nu$, we have
    \begin{itemize}[nl]
        \item $\d{\mu(\sigma)}=\d{\nu(\overline{\tau}\sigma)}$
        \item for $x\in G$, $\check{\mu}(x)=\check{\delta_\tau}(x)\check{\nu}(x)=\int_G\sigma(x)\d{\delta_\tau(\sigma)}\check{\nu}(x)=\tau(x)\chec{\nu}(x)$
    \end{itemize}
    so for $\sigma\in\widehat{G}$,
    \begin{equation*}
        (\check{\mu})^{\hat{}}(\sigma)=\int_G\overline{\sigma(x)}\check{\mu}(x)\d{x}=\int_G\overline{\sigma(x)}\tau(x)\check{\nu}(x)\d{x}=\int_G\overline{\tau\sigma}(x)\check{\nu}(x)\d{x}=(\check{\nu})^{\hat{}}(\overline{\tau}\sigma).
    \end{equation*}
    Thus $J(\psi\tau)=J(\psi)$, so that $J$ is the Haar integral.

    Next, if $f=\check{\mu}\in B^+\cap L^1(G)$, we have for $\psi\in C_c(\widehat{G})$, then a computation similar to above shows
    \begin{equation*}
        \int_{\widehat{G}}\psi\d{\mu}=J(\psi(\check{\mu})^{\hat{}}=J(\psi(\check{\mu})^{\hat{}})=\int_{\widehat{G}}\psi(\sigma)(\check{\mu})^{\hat{}}(\sigma)\d{\sigma}
    \end{equation*}
    i.e. $\mu\in M_a(\widehat{G})\cong L^1(\widehat{G})$, i.e. $\hat{f}=(\check{\mu})^{\hat{}}\in L^1(\widehat{G})$.
    Since $B^1(G)=\spn B^+\cap L^1(G)$, this gives (i)

    Finally, we show (ii).
    From analogy to (**), if $f=\check{\mu}\in B^1(G)$, then
    \begin{align*}
        f(x) = \hat{\mu}(x)=\int_{\widehat{G}}\sigma(x)\d{\mu(x)}=\int_{\widehat{G}}\sigma(x)(\check{\mu})^{\hat{}}(\sigma)\d{\sigma}=\int_{\widehat{G}}\sigma(x)\hat{f}(\sigma)\d{\sigma}=(\hat{f})^{\check{}}.
    \end{align*}
\end{proof}

\begin{enumerate}[nl]
    \item If $G$ is compact and $m_G(1)=1$.
        Then $\hat{1}(\sigma)=1$ if $\sigma=1$, and $0$ if $\sigma\neq 1$ for $\sigma\in\widehat{G}$, i.e $\hat{1}=\idc{\{1\}}$ on $G$.
        Then by inversion $1=1(e)=\int_G\hat{1}(\sigma)\sigma(e)\d{\sigma}=m_{\widehat{G}}(\{1\})$, i.e. $m_{\widehat{G}}$ is counting measure.
    \item Suppose $G$ is discrete.
        Then $\idc{\{e\}}=\idc{\{e\}}^**\idc{\{e\}}\in B^1(G)$.
        Then
        \begin{equation*}
            \widehat{\idc{\{e\}}}(\sigma)=\sum_{x\in G}\overline{\sigma(x)}\idc{\{e\}}(x)=1_{\widehat{G}}(\sigma).
        \end{equation*}
        Again, the inversion theorem gives
        \begin{equation*}
            m_{\widehat{G}}(\widehat{G}) = \int_{\widehat{G}}\idc{\widehat{G}}\d{m_{\widehat{G}}}=\int_{\widehat{G}}\hat{\idc{\{e\}}}(\sigma)\d{\sigma}=\int_{\widehat{G}}\hat{\idc{\{e\}}}(\sigma)\sigma(e)\d{\sigma}=1.
        \end{equation*}
    \item Let $m_{\R}$ be the standard normalization.
        We let $\alpha,\beta>0$ be so $\alpha m_{\R}$, $\beta m_{\hat{\R}}\cong\beta m_{\R}$ satisfy the inversion theorem.
        If $s\in\R$,
        \begin{equation*}
            \int_{\R}e^{-|x|}e^{-isx}\alpha\d{x}=2\alpha\int_0^\infty e^{-|x|}\cos(sx)\d{x}=\frac{2\alpha}{1+s^2}
        \end{equation*}
        so $s\mapsto (2\alpha)/(1+s^2)$ is of positive type, as $e^{-|x|}\d{x}$ is positive (Bochner).
        Inversion gives $e^{-|x|}=2\alpha\int_{\hat{\R}}\frac{e^{isx}}{1+s^2}B\d{s}$ for $x\in\R$.
        If $x=0$, we have $1=2\alpha\beta\int_{hat{\R}}\frac{\d{s}}{1+s^2}=2\pialpha\beta$.
        Typical normalizations give $\alpha=1$, $\beta=1/2\pi$ or $\alpha=\beta=\frac{1}{\sqrt{2\pi}}$.
\end{enumerate}
\section{Compact Groups}
\section{Introduction to Amenability Theory}
\end{document}
