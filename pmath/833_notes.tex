% header -----------------------------------------------------------------------
% Template created by texnew (author: Alex Rutar); info can be found at 'https://github.com/alexrutar/texnew'.
% version (1.13)


% doctype ----------------------------------------------------------------------
\documentclass[11pt, a4paper]{memoir}
\usepackage[utf8]{inputenc}
\usepackage[left=3cm,right=3cm,top=3cm,bottom=4cm]{geometry}
\usepackage[protrusion=true,expansion=true]{microtype}


% packages ---------------------------------------------------------------------
\usepackage{amsmath,amssymb,amsfonts}
\usepackage{graphicx}
\usepackage{etoolbox}
\usepackage{braket}

% Set enimitem
\usepackage{enumitem}
\SetEnumitemKey{nl}{nolistsep}
\SetEnumitemKey{r}{label=(\roman*)}

% Set tikz
\usepackage{tikz, pgfplots}
\pgfplotsset{compat=1.15}
\usetikzlibrary{intersections,positioning,cd}
\usetikzlibrary{arrows,arrows.meta}
\tikzcdset{arrow style=tikz,diagrams={>=stealth}}

% Set hyperref
\usepackage[hidelinks]{hyperref}
\usepackage{xcolor}
\newcommand\myshade{85}
\colorlet{mylinkcolor}{violet}
\colorlet{mycitecolor}{orange!50!yellow}
\colorlet{myurlcolor}{green!50!blue}

\hypersetup{
  linkcolor  = mylinkcolor!\myshade!black,
  citecolor  = mycitecolor!\myshade!black,
  urlcolor   = myurlcolor!\myshade!black,
  colorlinks = true,
}


% macros -----------------------------------------------------------------------
\DeclareMathOperator{\N}{{\mathbb{N}}}
\DeclareMathOperator{\Q}{{\mathbb{Q}}}
\DeclareMathOperator{\Z}{{\mathbb{Z}}}
\DeclareMathOperator{\R}{{\mathbb{R}}}
\DeclareMathOperator{\C}{{\mathbb{C}}}
\DeclareMathOperator{\F}{{\mathbb{F}}}

% Boldface includes math
\newcommand{\mbf}[1]{{\boldmath\bfseries #1}}

% proof implications
\newcommand{\imp}[2]{($#1\Rightarrow#2$)\hspace{0.2cm}}
\newcommand{\impe}[2]{($#1\Leftrightarrow#2$)\hspace{0.2cm}}
\newcommand{\impr}{{($\Longrightarrow$)\hspace{0.2cm}}}
\newcommand{\impl}{{($\Longleftarrow$)\hspace{0.2cm}}}

% align macros
\newcommand{\agspace}{\ensuremath{\phantom{--}}}
\newcommand{\agvdots}{\ensuremath{\hspace{0.16cm}\vdots}}

% convenient brackets
\newcommand{\brac}[1]{\ensuremath{\left\langle #1 \right\rangle}}
\newcommand{\norm}[1]{\ensuremath{\left\lVert#1\right\rVert}}
\newcommand{\abs}[1]{\ensuremath{\left\lvert#1\right\rvert}}

% arrows
\newcommand{\lto}[0]{\ensuremath{\longrightarrow}}
\newcommand{\fto}[1]{\ensuremath{\xrightarrow{\scriptstyle{#1}}}}
\newcommand{\hto}[0]{\ensuremath{\hookrightarrow}}
\newcommand{\mapsfrom}[0]{\mathrel{\reflectbox{\ensuremath{\mapsto}}}}
 
% Divides, Not Divides
\renewcommand{\div}{\bigm|}
\newcommand{\ndiv}{%
    \mathrel{\mkern.5mu % small adjustment
        % superimpose \nmid to \big|
        \ooalign{\hidewidth$\big|$\hidewidth\cr$/$\cr}%
    }%
}

% Convenient overline
\newcommand{\ol}[1]{\ensuremath{\overline{#1}}}

% Big \cdot
\makeatletter
\newcommand*\bigcdot{\mathpalette\bigcdot@{.5}}
\newcommand*\bigcdot@[2]{\mathbin{\vcenter{\hbox{\scalebox{#2}{$\m@th#1\bullet$}}}}}
\makeatother

% Big and small Disjoint union
\makeatletter
\providecommand*{\cupdot}{%
  \mathbin{%
    \mathpalette\@cupdot{}%
  }%
}
\newcommand*{\@cupdot}[2]{%
  \ooalign{%
    $\m@th#1\cup$\cr
    \sbox0{$#1\cup$}%
    \dimen@=\ht0 %
    \sbox0{$\m@th#1\cdot$}%
    \advance\dimen@ by -\ht0 %
    \dimen@=.5\dimen@
    \hidewidth\raise\dimen@\box0\hidewidth
  }%
}

\providecommand*{\bigcupdot}{%
  \mathop{%
    \vphantom{\bigcup}%
    \mathpalette\@bigcupdot{}%
  }%
}
\newcommand*{\@bigcupdot}[2]{%
  \ooalign{%
    $\m@th#1\bigcup$\cr
    \sbox0{$#1\bigcup$}%
    \dimen@=\ht0 %
    \advance\dimen@ by -\dp0 %
    \sbox0{\scalebox{2}{$\m@th#1\cdot$}}%
    \advance\dimen@ by -\ht0 %
    \dimen@=.5\dimen@
    \hidewidth\raise\dimen@\box0\hidewidth
  }%
}
\makeatother


% macros (theorem) -------------------------------------------------------------
\usepackage[thmmarks,amsmath,hyperref]{ntheorem}
\usepackage[capitalise,nameinlink]{cleveref}

% Numbered Statements
\theoremstyle{change}
\theoremindent\parindent
\theorembodyfont{\itshape}
\theoremheaderfont{\bfseries\boldmath}
\newtheorem{theorem}{Theorem.}[section]
\newtheorem{lemma}[theorem]{Lemma.}
\newtheorem{corollary}[theorem]{Corollary.}
\newtheorem{proposition}[theorem]{Proposition.}

% Claim environment
\theoremstyle{plain}
\theorempreskip{0.2cm}
\theorempostskip{0.2cm}
\theoremheaderfont{\scshape}
\newtheorem{claim}{Claim}
\renewcommand\theclaim{\Roman{claim}}
\AtBeginEnvironment{theorem}{\setcounter{claim}{0}}

% Un-numbered Statements
\theorempreskip{0.1cm}
\theorempostskip{0.1cm}
\theoremindent0.0cm
\theoremstyle{nonumberplain}
\theorembodyfont{\upshape}
\theoremheaderfont{\bfseries\itshape}
\newtheorem{definition}{Definition.}
\theoremheaderfont{\itshape}
\newtheorem{example}{Example.}
\newtheorem{remark}{Remark.}

% Proof / solution environments
\theoremseparator{}
\theoremheaderfont{\hspace*{\parindent}\scshape}
\theoremsymbol{$//$}
\newtheorem{solution}{Sol'n}
\theoremsymbol{$\blacksquare$}
\theorempostskip{0.4cm}
\newtheorem{proof}{Proof}
\theoremsymbol{}
\newtheorem{nmproof}{Proof}

% Format references
\crefformat{equation}{(#2#1#3)}
\Crefformat{theorem}{#2Thm. #1#3}
\Crefformat{lemma}{#2Lem. #1#3}
\Crefformat{proposition}{#2Prop. #1#3}
\Crefformat{corollary}{#2Cor. #1#3}
\crefformat{theorem}{#2Theorem #1#3}
\crefformat{lemma}{#2Lemma #1#3}
\crefformat{proposition}{#2Proposition #1#3}
\crefformat{corollary}{#2Corollary #1#3}


% macros (algebra) -------------------------------------------------------------
\DeclareMathOperator{\Ann}{Ann}
\DeclareMathOperator{\Aut}{Aut}
\DeclareMathOperator{\chr}{char}
\DeclareMathOperator{\coker}{coker}
\DeclareMathOperator{\disc}{disc}
\DeclareMathOperator{\End}{End}
\DeclareMathOperator{\Fix}{Fix}
\DeclareMathOperator{\Frac}{Frac}
\DeclareMathOperator{\Gal}{Gal}
\DeclareMathOperator{\GL}{GL}
\DeclareMathOperator{\Hom}{Hom}
\DeclareMathOperator{\id}{id}
\DeclareMathOperator{\im}{im}
\DeclareMathOperator{\lcm}{lcm}
\DeclareMathOperator{\Nil}{Nil}
\DeclareMathOperator{\rank}{rank}
\DeclareMathOperator{\Res}{Res}
\DeclareMathOperator{\Spec}{Spec}
\DeclareMathOperator{\spn}{span}
\DeclareMathOperator{\Stab}{Stab}
\DeclareMathOperator{\Tor}{Tor}

% Lagrange symbol
\newcommand{\lgs}[2]{\ensuremath{\left(\frac{#1}{#2}\right)}}

% Quotient (larger in display mode)
\newcommand{\quot}[2]{\mathchoice{\left.\raisebox{0.14em}{$#1$}\middle/\raisebox{-0.14em}{$#2$}\right.}
                                 {\left.\raisebox{0.08em}{$#1$}\middle/\raisebox{-0.08em}{$#2$}\right.}
                                 {\left.\raisebox{0.03em}{$#1$}\middle/\raisebox{-0.03em}{$#2$}\right.}
                                 {\left.\raisebox{0em}{$#1$}\middle/\raisebox{0em}{$#2$}\right.}}


% macros (analysis) ------------------------------------------------------------
\DeclareMathOperator{\M}{{\mathcal{M}}}
\DeclareMathOperator{\B}{{\mathcal{B}}}
\DeclareMathOperator{\ps}{{\mathcal{P}}}
\DeclareMathOperator{\pr}{{\mathbb{P}}}
\DeclareMathOperator{\E}{{\mathbb{E}}}
\DeclareMathOperator{\supp}{supp}
\DeclareMathOperator{\sgn}{sgn}

\renewcommand{\Re}{\ensuremath{\operatorname{Re}}}
\renewcommand{\Im}{\ensuremath{\operatorname{Im}}}
\renewcommand{\d}[1]{\ensuremath{\operatorname{d}\!{#1}}}


% file-specific preamble -------------------------------------------------------
\newcommand{\defname}[1]{{\textit{(#1)}:}}
\newcommand{\exname}[1]{{\textit{#1}:}}
\newcommand{\defn}[1]{{\boldmath\bfseries #1}}
% \usepackage{therefore}
\newcommand{\TODO}[1]{[\textit{\textbf{TODO: #1}}]}
\newcommand{\NOTE}[1]{[\textit{\textbf{NOTE: #1}}]}
\DeclareMathOperator*{\esssup}{ess\,sup}
\DeclareMathOperator{\ext}{ext}
\DeclareMathOperator{\conv}{conv}
\DeclareMathOperator{\dist}{dist}
\DeclareMathOperator{\Pol}{Pol}
\newcommand{\cwx}{\ensuremath{\overline{\operatorname{conv}}^{w^*}\,}}
\newcommand{\idc}{\mathbf{1}}
\newcommand{\FA}{\ensuremath{\operatorname{F}\!\operatorname{A}}}
\newcommand{\cw}{\ensuremath{\overline{\operatorname{conv}}\,}}

% Tons of notation:
% \newcommand{\Lip}[1]{\ensuremath{\operatorname{Lip}_{\F}(#1)}}
\newcommand{\Lipspace}{\ensuremath{\operatorname{Lip}_{\F}(X,d)}}


\newcommand{\lp}[1]{\ensuremath{\ell^{#1}}}
\newcommand{\csn}{\ensuremath{\mathbf{c}}}
\newcommand{\csz}{\ensuremath{\mathbf{c}_0}}
\newcommand{\lpspace}[1]{\ensuremath{\ell^{#1}_{\F}}}
\newcommand{\Lp}[1]{\ensuremath{L^{#1}_{\F}}}
% \newcommand{\Lpm}{\ensuremath{L^{#1}_{\F}(X,\mathcal{M},\mu)}}
\DeclareMathOperator{\Lip}{Lip}
\newcommand{\lbr}[1]{\ensuremath{\left[#1\right]}}
\newcommand{\inr}[1]{\ensuremath{\left(#1\right)}}


% constants --------------------------------------------------------------------
\newcommand{\subject}{Harmonic Analysis}
\newcommand{\semester}{Winter 2020}


% formatting -------------------------------------------------------------------
% Fonts
\usepackage{kpfonts}
\usepackage{dsfont}

% Adjust numbering
\numberwithin{equation}{section}
\counterwithin{figure}{section}
\counterwithout{section}{chapter}
\counterwithin*{chapter}{part}

% Footnote
\setfootins{0.5cm}{0.5cm} % footer space above
\renewcommand*{\thefootnote}{\fnsymbol{footnote}} % footnote symbol

% Table of Contents
\renewcommand{\thechapter}{\Roman{chapter}}
\renewcommand*{\cftchaptername}{Chapter } % Place 'Chapter' before roman
\setlength\cftchapternumwidth{4em} % Add space before chapter name
\cftpagenumbersoff{chapter} % Turn off page numbers for chapter
\maxtocdepth{subsection} % table of contents up to section

% Section / Subsection headers
\setsecnumdepth{subsection} % numbering up to and including "subsection"
\newcommand*{\shortcenter}[1]{%
    \sethangfrom{\noindent ##1}%
    \Large\boldmath\scshape\bfseries
    \centering
\parbox{5in}{\centering #1}\par}
\setsecheadstyle{\shortcenter}
\setsubsecheadstyle{\large\scshape\boldmath\bfseries\raggedright}

% Chapter Headers
\chapterstyle{verville}

% Page Headers / Footers
\copypagestyle{myruled}{ruled} % Draw formatting from existing 'ruled' style
\makeoddhead{myruled}{}{}{\scshape\subject}
\makeevenfoot{myruled}{}{\thepage}{}
\makeoddfoot{myruled}{}{\thepage}{}
\pagestyle{myruled}
\setfootins{0.5cm}{0.5cm}
\renewcommand*{\thefootnote}{\fnsymbol{footnote}}

% Titlepage
\title{\subject}
\author{Alex Rutar\thanks{\itshape arutar@uwaterloo.ca}\\ University of Waterloo}
\date{\semester\thanks{Last updated: \today}}

\begin{document}
\pagenumbering{gobble}
\hypersetup{pageanchor=false}
\maketitle
\newpage
\frontmatter
\hypersetup{pageanchor=true}
\tableofcontents*
\newpage
\mainmatter


% main document ----------------------------------------------------------------
\chapter{Harmonic Analysis}
\section{Locally Compact Groups}
\begin{definition}
    Let $G$ be a group.
    A topology $\tau$ on $G$ is a \defn{group topology} provided that
    \begin{itemize}[nl]
        \item $x\mapsto x^{-1}:G\to G$ is continuous, and
        \item $(x,y)\mapsto xy:G\times G\to G$ is continuous.
    \end{itemize}
    We call $(G,\tau)$ a \defn{topological group} where we omit $\tau$ when it is clear from context.
\end{definition}
Equivalently, we may assert that $(x,y)\mapsto xy^{-1}$ is $\tau\times\tau-\tau-$continuous.
Write $L_g(x)=gx$ and $R_g(x)=xg$ to denote the left and right multiplication maps; then it is easy to see that $L_g$ and $R_g$ are homeomorphisms.
Similarly, $x\mapsto x^{-1}$ is a homeomorphism.
\begin{definition}
    We say that a subset $A\subset G$ is \defn{symmetric} if $A^{-1}=A$.
\end{definition}
We have the following basic properties:
\begin{proposition}\label{p:tgrp}
    Let $(G,\tau)$ be a topological group.
    \begin{enumerate}[nl,r]
        \item If $\emptyset\neq A\subseteq G$ and $U$ is open, then $AU=\{ay:a\in A,y\in U\}$ and likewise $UA$ are open.
        \item Given $U\in\tau$ and $x\in U$, then there is a symmetric $V\in\tau$ with $e\in V$ such that $VxV\subseteq U$.
            In particular, if $e\in U$, then we can find symmetric $V$ so that $V^2\subseteq U$.
        \item If $H$ is a subgroup of $G$, then $\overline{H}$ is also a subgroup.
        \item An open subgroup is automatically closed.
        \item If $K,L\subseteq G$ are compact, then $KL$ is compact.
        \item If $K$ is compact and $C$ is closed in $G$, then $KC$ is closed.
    \end{enumerate}
\end{proposition}
In $(\R,+)$, then $\Z+\sqrt{2}\Z$ is not closed, so it is necessary to assume compactness in (vi).
\begin{proof}
    \begin{enumerate}[nl,r]
        \item $AU=\bigcup_{a\in A}L_a(U)$ is a union of open sets.
        \item Consider the continuous map $(y,z)\mapsto yxz$.
            Since $exe=x\in U$, there is a $\tau\times\tau-$neighbourhood of $(e,e)$ which maps into $U$ have a basic neighbourhood $V_1\times V_2$.
            Let $V=V_1\cap V_2$.
            Moreover, we may replace $V$ by $V^{-1}\cap V$. to attain symmetry.
        \item Let $x,y\in\ol{H}$, $U\in\tau$ with $xy\in U$.
            Then (ii) provides $V$ with $VxyV\subseteq U$.
            But $Vx\cap H\neq\emptyset$ and $\neq yV\cap H$ so there are $_1\in Vx\cap H$, $h_2\in yV\cap H$, and $h_1h_2\in VxyV\subseteq U$.
            Thus $U\cap H\neq\emptyset$.
            Thus $xy\in\ol{H}$.

            To use nets for inverses, if $x\in\ol{H}$, then $x=\lim_\alpha x_\alpha$ where $(x_\alpha)_{\alpha\in A}\subset H$ is a net.
            Then $x^{-1}=\lim_\alpha x_\alpha^{-1}\in\ol{H}$ as each $x_\alpha^{-1}\in H$.
        \item If $H$ is an open subgroup, then $H=G\setminus\bigcup_{x\in G\setminus H}xH$ is closed.
        \item $K\times L$ is compact, and hence so is its image under multiplication.
        \item If $x\in\ol{KC}$, then $x=\lim_\alpha k_\alpha x_\alpha$ where $k_\alpha\in H$ and $x_\alpha\in C$.
            Since $K$ is compact, we may assume (passing to a subnet if necessary) $k=\lim_\alpha k_\alpha$ exists in $K$.
            Then
            \begin{equation*}
                k^{-1}x=\lim_\alpha k_\alpha^{-1}\cdot\lim_\alpha k_\alpha x_\alpha=\lim_\alpha k_\alpha^{-1}k_\alpha x_\alpha=\lim x_\alpha\in C
            \end{equation*}
            so $x=kk^{-1}x\in KC$.
    \end{enumerate}
\end{proof}
\subsection{Homogenous Spaces}
Let $(G,\tau)$ be a topological group, $H$ a subgroup of $G$, and $\quot{G}{H}=\{xH;x\in G\}$.
Let $\pi:G\to \quot{G}{H}$ be given by $\pi(x)=xH$ be the projection map.
The \defn{quotient topology} on $\quot{G}{H}$ is $\tau_{\quot{G}{H}}=\{W\in \quot{G}{H}:\pi^{-1}(W)\in\tau\}$.
Notice that if $U\in\tau\setminus\{\emptyset\}$, then $UH=\pi^{-1}(\pi(U))$ is open, so $\pi:G\to \quot{G}{H}$ is an open map.
\begin{proposition}
    Let $(G,\tau)$, $H$ be as above.
    \begin{enumerate}[nl,r]
        \item The map $(x,yH)\mapsto xyH:G\times\quot{G}{H}\to\quot{G}{H}$ is $\tau\times\tau_{\quot{G}{H}}-\tau_{\quot{G}{H}}$ continuous and open.
        \item If $H$ is normal, then $(\quot{G}{H},\tau_{\quot{G}{H}})$ is a topological group.
        \item If $H$ is closed, then $\tau_{\quot{G}{H}}$ is Hausdorff.
    \end{enumerate}
\end{proposition}
\begin{proof}
    \begin{enumerate}[nl,r]
        \item Let $x,y\in G$, $W\in\tau_{\quot{G}{H}}$ satisfy $xyH=\pi(xy)\in W$.
            Then $xy\in\pi^{-1}(W)$ and by \cref{p:tgrp} we have $V\in\tau$ with $e\in V$ such that $VxyV\subseteq \pi^{-1}(W)$.
            But then $(x,\pi(y))\in Vx\times\pi(yV)\in \tau\times\tau_{\quot{G}{H}}$ and the latter set maps into $\pi(VxyV)\subseteq W$.

            Also, if $U\in\tau\times\tau_{\quot{G}{H}}$, then $U=\bigcup_{(x,yH)\in U}V_x\times W_{yH}$ and
            \begin{equation*}
                \pi(U)=\bigcup_{(x,yH)\in U}\pi(V_x\pi^{-1}(W_{yH}))
            \end{equation*}
            since $\pi$ is open.
        \item Recall that $(xH)(yH)=xyH$ is our multiplication operation on $\quot{G}{H}$ and $\pi$ is a group homomorphism.
            Then the following diagram commutes:
        % \begin{tikzcd}
            % M'\arrow[r,"f"]\arrow[rd,swap,"u"] & M\arrow[r,"g"]\arrow[d,"\alpha"] & M''\\
            % & M'\oplus M'' \arrow[ru,swap,"u"]
        % \end{tikzcd}
            % \begin{center}
                % \begin{tikzcd}
                    % G\times\quot{G}{H}\arrow[rd,"(x,yH)\mapsto xyH"]\arrow[d,"\pi\times\id"]\\
                    % \quot{G}{H}\times\quot{G}{H}\arrow[r,"(xH,yH)\mapsto xyH"] & \quot{G}{H}
                % \end{tikzcd}
            % \end{center}
            We have that $\pi\times\id$ is open and $(x,yH)\mapsto xyH$ is open from (i), so the multiplication from $\quot{G}{H}\times\quot{G}{H}\to\quot{G}{H}$ must be open and continuous.
        \item If $x,y\in G$ with $\pi(x)\neq\pi(y)$, then $e\notin xHy^{-1}$.
            Now $xHy^{-1}=L_x(R_{y^{-1}}(H))$ so $xHy^{-1}$ is closed.
            Hence by the last proposition, there is a symmetric open $V$ with $e\in V$ so $V^2\subseteq G\setminus(xHy^{-1})$.
            But then $e\notin (VxH)(VyH)^{-1}=VxHy^{-1}V$: if we had $e=vxhy^{-1}u$ with $v,u\in V$ and $h\in H$, then $v^{-1}u^{-1}=xhy^{-1}\in V^2\cap(xHy^{-1})=\emptyset$, a contradiction.
            Hence $VxH\cap VyH=\emptyset$ so $\pi(Vx)$, $\pi(Vy)$ is a pair of separating neighbourhoods of $\pi(x)$, $\pi(y)$.
    \end{enumerate}
\end{proof}
\begin{corollary}
    $G$ is Hausdorff if and only if there exists $x\in G$ so that $\{x\}$ is closed.
\end{corollary}
\begin{proof}
    In a Hausdorff space, points are closed.
    Conversely, if $\{x\}$ is closed, $\{e\}=L_{x^{-1}}(\{x\})$ is closed and a normal subgroup.
    Then $G\cong\quot{G}{\{e\}}$ is Hausdorff.
\end{proof}
If $(G,\tau)$ is not Hausdorff, then $\{e\}\subsetneq\overline\{e\}$ is the smallest closed subgrup in $G$.
Thus $\overline{\{e\}}\subseteq\bigcap_{x\in G}x\overline{\{e\}}x^{-1}\subseteq\overline{\{x\}}$ so $\overline{\{e\}}$ is normal.
In particular, $\quot{G}{\overline{\{e\}}}$ is Hausdorff.
\begin{definition}
    A \defn{locally compact group} is a Hausdorff topological group $(G,\tau)$ which is locally compact.
\end{definition}
\begin{enumerate}[nl,r]
    \item If there is any $U\in\tau\setminus\{\emptyset\}$ such that $\overline{U}$ is compact, then for any $x\in U$, we have $e\in x^{-1}U\subseteq L_{x^{-1}}(\overline{U})$ so $\overline{x^{-1}U}$ is compact.
        If $V\in\tau$ with $e\in V$ and $\overline{V}$ compact, then for any $x\in H$, $x\in xV$ and $\overline{xV}\subseteq L_x(\overline{V})$ and $\overline{xV}$ is compact.
        In particular, $(G,\tau)$ is locally compact if and only if there is some $U\in\tau\setminus\{\emptyset\}$ with $\overline{U}$ compact.
    \item If $(G,\tau)$ is locally cmpact and $N$ is a closed normal subgroup, then $(\quot{G}{N},\tau_{\quot{G}{N}})$ is locally compact.
        Indeed, $U\in\tau\setminus\{e\}$ with $\overline{U}$ compact, then $\overline{\pi(U)}\subseteq\pi(\overline{U})$ is compact.
\end{enumerate}
\begin{example}
    \begin{enumerate}[nl,r]
        \item If $G$ is any group and $\tau$ is the discrete topology, then $(G,\tau_d)$ is locally compact.
        \item If $((\R,+),\tau_{\norm{\cdot}})$ is locally compact.
        \item If $\{G_i\}_{i\in I}$ is a family of locally compact groups, then $\prod_{i\in I}G_i$ is a locally compact group if and only if all but finitely many $(G_i,\tau_i)$ are compact.
        \item $((\R^n,+),\tau_{\norm{\cdot}})$ is a locally compact group
        \item Suppose $\{F_i\}_{i\in I}$ is an infinite family of finite groups (with discrete topologies), then $G=\prod_{i\in I}F_i$ is a compact group.
            If $F\subset I$ is finite, then $N_F=\{(x_i)_{i\in I}\in G:x_i=e\text{ for }i\in F\}$ is open and a normal subgroup.
            $\{N_F:F\subset I\text{ finite}\}$ is a base for the topology at $e$.
    \end{enumerate}
\end{example}
\begin{example}[$p-$adic numbers]
    Let $p$ be a prime in $\N$.
    We will establish product structures and topologies on
    \begin{align*}
        \mathbb{O}_p&=\left\{\sum_{k=0}^\infty a_kp^k:a_k\in\{0,1,\ldots,p-1\}\right\}\cong\{0,1,\ldots,p-1\}^{\N}\\
        \mathbb{Q}_p&=\left\{\sum_{k=N}^\infty a_kp^k:N\in\Z, a_k\in\{0,1,\ldots,p-1\}\right\}
    \end{align*}
    are topological rings and a topological field respectively.
    Let $R_p=\prod_{n=0}^\infty\quot{\Z}{p^{n+1}\Z}$ which is a ring under pointwise operations.
    \begin{lemma}
        The map $\rho:R_p\times R_p\to[0,1]$ given by
        \begin{align*}
            \rho(x,y)&=\sum_{n\in\N_0}\frac{\rho_n(x_n,y_n)}{p^n} & &\rho_n(x_n,y_n)=\begin{cases}1 &:x_n=y_n\\ 0 &:x_n\neq y_n\end{cases}
        \end{align*}
        is a metric on $R_p$ which satisfies
        \begin{itemize}[nl]
            \item \defname{additively invariant}
                $\rho(x+z,y+z)=\rho(x,y)$ for $x,y,z\in R_p$
            \item $\tau_\rho$ is the product topology
        \end{itemize}
    \end{lemma}
    \begin{proof}
        Additive invariance is routine.
        Notice that if $\frac{1}{p^m}\geq\epsilon>\frac{1}{p^{m+1}}$, then the open $\epsilon-$ball around a point $x$ is $\{x_0\}\times\cdots\{x_m\}\times\prod_{n=m+1}^\infty\quot{\Z}{p^{n+1}\Z}$ is product-open.
        Conversely, any product-open set is a finite union of such $\epsilon-$balls.
    \end{proof}
    \begin{corollary}
        The function $\norm{x}_p=\rho(x,0)$ in $R_p$ satisfies
        \begin{itemize}[nl]
            \item $\norm{x}_p=0$ if and only if $x=0$
            \item $\norm{x+y}_p\leq\norm{x}_p+\norm{y}_p$
            \item $\norm{xy}_p\leq\norm{x}_p\norm{y}_p$
            \item $\norm{-x}_p=\norm{x}_p$
        \end{itemize}
        Hence $(R_p,\tau_\rho)$ is a compact topological ring.
    \end{corollary}
    \begin{proof}
        The properties follow directly using additive invariance.
        To see that $R_p$ is a topological ring, if $(x_\alpha),(y_\alpha)$ have $x=\lim x_\alpha$ and $y=\lim y_\alpha$, then, for example,
        \begin{align*}
            \norm{xy-x_\alpha y_\alpha}_p&\leq\norm{xy-x_\alpha y}_p+\norm{x_\alpha y-x_\alpha y_\alpha}_p\\
                                         &\leq\norm{x-x_\alpha}_p+\norm{y-y_\alpha}_p
        \end{align*}
        as $\norm{y}_p$, $\norm{x_\alpha}_p\leq 1$.
    \end{proof}
    We now view $\mathbb{O}_p$ as a closed subring of $R_p$.
    Define $\alpha:\mathcal{O}_p\to R_p$ be given on $a=\sum_{k=0}^\infty a_kp^k$ by
    \begin{equation*}
        \alpha(a) = \left(\sum_{k=0}^n a_kp^k+p^{n+1}\Z\right)_{n=0}^\infty.
    \end{equation*}
    This map is an injection with range $\alpha(\mathcal{O}_p)=\{(x_n)_{n=0}^\infty\in R_p:x_n=\pi_n(x_{n+1})\text{ for all }n\}$ where $\pi_n:\quot{\Z}{p^{n+2}\Z}\to\quot{\Z}{p^{n+1}\Z}$ is the canonical quotient map.
    In fact, this is called an inductive limit with respect to the maps $\pi_n$.
    Hence it is routine to show that
    \begin{itemize}[nl]
        \item $\alpha(\mathbb{O}_p)$ is a subring of $R_p$, and
        \item $\alpha(\mathbb{O}_p)$ is closed in $R_p$ (just check net limits in product topology)
    \end{itemize}
    If $a,b\in\mathbb{O}_p$, define $a+b=\alpha^{-1}(\alpha(a)+\alpha(b))$.
    \begin{remark}
        \begin{enumerate}[nl,r]
            \item $1+\sum_{k=1}^\infty 0\cdot p^k$ is the multiplicative identity in $\mathbb{O}_p$.
                Then $-1=\sum_{k=0}^\infty(p-1)p^k$.
            \item If $n\in\N$, we can uniquely write $n=\sum_{k=0}^{m(n)} a_kp^k$ with $a_k\in\{0,\ldots,p-1\}$.
                Then $n\cdot 1=\sum_{k=0}^{m(n)}a_kp^k\in\mathbb{O}_p$.
                In particular, $n\mapsto n\cdot 1:\N\to\mathbb{O}_p$ is an additive semigroup homomorphism with dense ring.
                Hence $n\mapsto n\cdot 1:\Z\to \mathbb{Q}_p$ has dense range.
                We call $\mathbb{O}_p$ the \defn{$p-$adic integers}.
        \end{enumerate}
    \end{remark}
\end{example}
\section{Abelian Locally Compact Groups}
\section{Compact Groups}
\section{Introduction to Amenability Theory}
\end{document}
