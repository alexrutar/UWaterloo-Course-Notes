% header -----------------------------------------------------------------------
% Template created by texnew (author: Alex Rutar); info can be found at 'https://github.com/alexrutar/texnew'.
% version (1.13)


% doctype ----------------------------------------------------------------------
\documentclass[11pt, a4paper]{memoir}
\usepackage[utf8]{inputenc}
\usepackage[left=3cm,right=3cm,top=3cm,bottom=4cm]{geometry}
\usepackage[protrusion=true,expansion=true]{microtype}


% packages ---------------------------------------------------------------------
\usepackage{amsmath,amssymb,amsfonts}
\usepackage{graphicx}
\usepackage{etoolbox}
\usepackage{braket}

% Set enimitem
\usepackage{enumitem}
\SetEnumitemKey{nl}{nolistsep}
\SetEnumitemKey{r}{label=(\roman*)}

% Set tikz
\usepackage{tikz, pgfplots}
\pgfplotsset{compat=1.15}
\usetikzlibrary{intersections,positioning,cd}
\usetikzlibrary{arrows,arrows.meta}
\tikzcdset{arrow style=tikz,diagrams={>=stealth}}

% Set hyperref
\usepackage[hidelinks]{hyperref}
\usepackage{xcolor}
\newcommand\myshade{85}
\colorlet{mylinkcolor}{violet}
\colorlet{mycitecolor}{orange!50!yellow}
\colorlet{myurlcolor}{green!50!blue}

\hypersetup{
  linkcolor  = mylinkcolor!\myshade!black,
  citecolor  = mycitecolor!\myshade!black,
  urlcolor   = myurlcolor!\myshade!black,
  colorlinks = true,
}


% macros -----------------------------------------------------------------------
\DeclareMathOperator{\N}{{\mathbb{N}}}
\DeclareMathOperator{\Q}{{\mathbb{Q}}}
\DeclareMathOperator{\Z}{{\mathbb{Z}}}
\DeclareMathOperator{\R}{{\mathbb{R}}}
\DeclareMathOperator{\C}{{\mathbb{C}}}
\DeclareMathOperator{\F}{{\mathbb{F}}}

% Boldface includes math
\newcommand{\mbf}[1]{{\boldmath\bfseries #1}}

% proof implications
\newcommand{\imp}[2]{($#1\Rightarrow#2$)\hspace{0.2cm}}
\newcommand{\impe}[2]{($#1\Leftrightarrow#2$)\hspace{0.2cm}}
\newcommand{\impr}{{($\Longrightarrow$)\hspace{0.2cm}}}
\newcommand{\impl}{{($\Longleftarrow$)\hspace{0.2cm}}}

% align macros
\newcommand{\agspace}{\ensuremath{\phantom{--}}}
\newcommand{\agvdots}{\ensuremath{\hspace{0.16cm}\vdots}}

% convenient brackets
\newcommand{\brac}[1]{\ensuremath{\left\langle #1 \right\rangle}}
\newcommand{\norm}[1]{\ensuremath{\left\lVert#1\right\rVert}}
\newcommand{\abs}[1]{\ensuremath{\left\lvert#1\right\rvert}}

% arrows
\newcommand{\lto}[0]{\ensuremath{\longrightarrow}}
\newcommand{\fto}[1]{\ensuremath{\xrightarrow{\scriptstyle{#1}}}}
\newcommand{\hto}[0]{\ensuremath{\hookrightarrow}}
\newcommand{\mapsfrom}[0]{\mathrel{\reflectbox{\ensuremath{\mapsto}}}}
 
% Divides, Not Divides
\renewcommand{\div}{\bigm|}
\newcommand{\ndiv}{%
    \mathrel{\mkern.5mu % small adjustment
        % superimpose \nmid to \big|
        \ooalign{\hidewidth$\big|$\hidewidth\cr$/$\cr}%
    }%
}

% Convenient overline
\newcommand{\ol}[1]{\ensuremath{\overline{#1}}}

% Big \cdot
\makeatletter
\newcommand*\bigcdot{\mathpalette\bigcdot@{.5}}
\newcommand*\bigcdot@[2]{\mathbin{\vcenter{\hbox{\scalebox{#2}{$\m@th#1\bullet$}}}}}
\makeatother

% Big and small Disjoint union
\makeatletter
\providecommand*{\cupdot}{%
  \mathbin{%
    \mathpalette\@cupdot{}%
  }%
}
\newcommand*{\@cupdot}[2]{%
  \ooalign{%
    $\m@th#1\cup$\cr
    \sbox0{$#1\cup$}%
    \dimen@=\ht0 %
    \sbox0{$\m@th#1\cdot$}%
    \advance\dimen@ by -\ht0 %
    \dimen@=.5\dimen@
    \hidewidth\raise\dimen@\box0\hidewidth
  }%
}

\providecommand*{\bigcupdot}{%
  \mathop{%
    \vphantom{\bigcup}%
    \mathpalette\@bigcupdot{}%
  }%
}
\newcommand*{\@bigcupdot}[2]{%
  \ooalign{%
    $\m@th#1\bigcup$\cr
    \sbox0{$#1\bigcup$}%
    \dimen@=\ht0 %
    \advance\dimen@ by -\dp0 %
    \sbox0{\scalebox{2}{$\m@th#1\cdot$}}%
    \advance\dimen@ by -\ht0 %
    \dimen@=.5\dimen@
    \hidewidth\raise\dimen@\box0\hidewidth
  }%
}
\makeatother


% macros (theorem) -------------------------------------------------------------
\usepackage[thmmarks,amsmath,hyperref]{ntheorem}
\usepackage[capitalise,nameinlink]{cleveref}

% Numbered Statements
\theoremstyle{change}
\theoremindent\parindent
\theorembodyfont{\itshape}
\theoremheaderfont{\bfseries\boldmath}
\newtheorem{theorem}{Theorem.}[section]
\newtheorem{lemma}[theorem]{Lemma.}
\newtheorem{corollary}[theorem]{Corollary.}
\newtheorem{proposition}[theorem]{Proposition.}

% Claim environment
\theoremstyle{plain}
\theorempreskip{0.2cm}
\theorempostskip{0.2cm}
\theoremheaderfont{\scshape}
\newtheorem{claim}{Claim}
\renewcommand\theclaim{\Roman{claim}}
\AtBeginEnvironment{theorem}{\setcounter{claim}{0}}

% Un-numbered Statements
\theorempreskip{0.1cm}
\theorempostskip{0.1cm}
\theoremindent0.0cm
\theoremstyle{nonumberplain}
\theorembodyfont{\upshape}
\theoremheaderfont{\bfseries\itshape}
\newtheorem{definition}{Definition.}
\theoremheaderfont{\itshape}
\newtheorem{example}{Example.}
\newtheorem{remark}{Remark.}

% Proof / solution environments
\theoremseparator{}
\theoremheaderfont{\hspace*{\parindent}\scshape}
\theoremsymbol{$//$}
\newtheorem{solution}{Sol'n}
\theoremsymbol{$\blacksquare$}
\theorempostskip{0.4cm}
\newtheorem{proof}{Proof}
\theoremsymbol{}
\newtheorem{nmproof}{Proof}

% Format references
\crefformat{equation}{(#2#1#3)}
\Crefformat{theorem}{#2Thm. #1#3}
\Crefformat{lemma}{#2Lem. #1#3}
\Crefformat{proposition}{#2Prop. #1#3}
\Crefformat{corollary}{#2Cor. #1#3}
\crefformat{theorem}{#2Theorem #1#3}
\crefformat{lemma}{#2Lemma #1#3}
\crefformat{proposition}{#2Proposition #1#3}
\crefformat{corollary}{#2Corollary #1#3}


% macros (algebra) -------------------------------------------------------------
\DeclareMathOperator{\Ann}{Ann}
\DeclareMathOperator{\Aut}{Aut}
\DeclareMathOperator{\chr}{char}
\DeclareMathOperator{\coker}{coker}
\DeclareMathOperator{\disc}{disc}
\DeclareMathOperator{\End}{End}
\DeclareMathOperator{\Fix}{Fix}
\DeclareMathOperator{\Frac}{Frac}
\DeclareMathOperator{\Gal}{Gal}
\DeclareMathOperator{\GL}{GL}
\DeclareMathOperator{\Hom}{Hom}
\DeclareMathOperator{\id}{id}
\DeclareMathOperator{\im}{im}
\DeclareMathOperator{\lcm}{lcm}
\DeclareMathOperator{\Nil}{Nil}
\DeclareMathOperator{\rank}{rank}
\DeclareMathOperator{\Res}{Res}
\DeclareMathOperator{\Spec}{Spec}
\DeclareMathOperator{\spn}{span}
\DeclareMathOperator{\Stab}{Stab}
\DeclareMathOperator{\Tor}{Tor}

% Lagrange symbol
\newcommand{\lgs}[2]{\ensuremath{\left(\frac{#1}{#2}\right)}}

% Quotient (larger in display mode)
\newcommand{\quot}[2]{\mathchoice{\left.\raisebox{0.14em}{$#1$}\middle/\raisebox{-0.14em}{$#2$}\right.}
                                 {\left.\raisebox{0.08em}{$#1$}\middle/\raisebox{-0.08em}{$#2$}\right.}
                                 {\left.\raisebox{0.03em}{$#1$}\middle/\raisebox{-0.03em}{$#2$}\right.}
                                 {\left.\raisebox{0em}{$#1$}\middle/\raisebox{0em}{$#2$}\right.}}


% macros (analysis) ------------------------------------------------------------
\DeclareMathOperator{\M}{{\mathcal{M}}}
\DeclareMathOperator{\B}{{\mathcal{B}}}
\DeclareMathOperator{\ps}{{\mathcal{P}}}
\DeclareMathOperator{\pr}{{\mathbb{P}}}
\DeclareMathOperator{\E}{{\mathbb{E}}}
\DeclareMathOperator{\supp}{supp}
\DeclareMathOperator{\sgn}{sgn}

\renewcommand{\Re}{\ensuremath{\operatorname{Re}}}
\renewcommand{\Im}{\ensuremath{\operatorname{Im}}}
\renewcommand{\d}[1]{\ensuremath{\operatorname{d}\!{#1}}}


% file-specific preamble -------------------------------------------------------
% \usepackage{therefore}
\newcommand{\TODO}[1]{[\textit{\textbf{TODO: #1}}]}
\DeclareMathOperator*{\esssup}{ess\,sup}
\DeclareMathOperator{\ext}{ext}
\DeclareMathOperator{\conv}{conv}
\DeclareMathOperator{\dist}{dist}
\newcommand{\cwx}{\ensuremath{\overline{\operatorname{co}}^{w^*}\,}}
\newcommand{\idc}{\mathbf{1}}
\newcommand{\FA}{\ensuremath{\operatorname{F}\!\operatorname{A}}}
\newcommand{\cw}{\ensuremath{\overline{\operatorname{co}}\,}}

% Tons of notation:
% \newcommand{\Lip}[1]{\ensuremath{\operatorname{Lip}_{\F}(#1)}}
\newcommand{\Lipspace}{\ensuremath{\operatorname{Lip}_{\F}(X,d)}}


\newcommand{\lp}[1]{\ensuremath{\ell^{#1}}}
\newcommand{\lpspace}[1]{\ensuremath{\ell^{#1}_{\F}}}
\newcommand{\Lp}[1]{\ensuremath{L^{#1}_{\F}}}
% \newcommand{\Lpm}{\ensuremath{L^{#1}_{\F}(X,\mathcal{M},\mu)}}
\DeclareMathOperator{\Lip}{Lip}


% constants --------------------------------------------------------------------
\newcommand{\subject}{Functional Analysis}
\newcommand{\semester}{Fall 2019}
\newcommand{\lbr}[1]{\ensuremath{\left[#1\right]}}
\newcommand{\inr}[1]{\ensuremath{\left(#1\right)}}


% formatting -------------------------------------------------------------------
% Fonts
\usepackage{kpfonts}
\usepackage{dsfont}

% Adjust numbering
\numberwithin{equation}{section}
\counterwithin{figure}{section}
\counterwithout{section}{chapter}
\counterwithin*{chapter}{part}

% Footnote
\setfootins{0.5cm}{0.5cm} % footer space above
\renewcommand*{\thefootnote}{\fnsymbol{footnote}} % footnote symbol

% Table of Contents
\renewcommand{\thechapter}{\Roman{chapter}}
\renewcommand*{\cftchaptername}{Chapter } % Place 'Chapter' before roman
\setlength\cftchapternumwidth{4em} % Add space before chapter name
\cftpagenumbersoff{chapter} % Turn off page numbers for chapter
\maxtocdepth{subsection} % table of contents up to section

% Section / Subsection headers
\setsecnumdepth{subsection} % numbering up to and including "subsection"
\newcommand*{\shortcenter}[1]{%
    \sethangfrom{\noindent ##1}%
    \Large\boldmath\scshape\bfseries
    \centering
\parbox{5in}{\centering #1}\par}
\setsecheadstyle{\shortcenter}
\setsubsecheadstyle{\large\scshape\boldmath\bfseries\raggedright}

% Chapter Headers
\chapterstyle{verville}

% Page Headers / Footers
\copypagestyle{myruled}{ruled} % Draw formatting from existing 'ruled' style
\makeoddhead{myruled}{}{}{\scshape\subject}
\makeevenfoot{myruled}{}{\thepage}{}
\makeoddfoot{myruled}{}{\thepage}{}
\pagestyle{myruled}
\setfootins{0.5cm}{0.5cm}
\renewcommand*{\thefootnote}{\fnsymbol{footnote}}

% Titlepage
\title{\subject}
\author{Alex Rutar\thanks{\itshape arutar@uwaterloo.ca}\\ University of Waterloo}
\date{\semester\thanks{Last updated: \today}}

\begin{document}
\pagenumbering{gobble}
\hypersetup{pageanchor=false}
\maketitle
\newpage
\frontmatter
\hypersetup{pageanchor=true}
\tableofcontents*
\newpage
\mainmatter


% main document ----------------------------------------------------------------
\chapter{Analysis in Metric Spaces}
\section{Topology}
Let $X$ denote a non-empty set, and $\mathcal{P}(X)$ denote the power set of $X$.
\begin{definition}
    A \textbf{topology} on a set $X$ is a set $\tau$ of subsets of $X$ such that
    \begin{enumerate}[nl,r]
        \item $\emptyset,X\in\tau$
        \item If $U_\alpha\in\tau$ for all $\alpha\in A$, then $\bigcup_{\alpha\in A}U_\alpha\in\tau$.
        \item If $n\in\N$ and $U_i\in\tau$ for each $1\leq i\leq n$, then $\bigcap_{i=1}^n U_i\in\tau$.
    \end{enumerate}
    The sets $U\in\tau$ are the \textbf{open sets} in $X$, and sets $X\setminus U$ for some open set $U$ are the \textbf{closed sets} in $X$.
    The pair $(X,\tau)$ is called a \textbf{topological space}.
\end{definition}
\begin{example}
    \begin{enumerate}[nl,r]
        \item \textit{Sorgenfry line:}
            Set $X=\R$, and consider
            \begin{equation*}
                \sigma=\Set{V\subseteq\R | \text{ for any }s\in V,\text{ there is }\delta=\delta(s)>0\text{ s.t. }[s,s+\delta)\subseteq V}
            \end{equation*}
            It is a straightforward exercise to verify that $\tau_{\abs{\cdot}}\subsetneq\sigma$.
            We say that $\sigma$ is \textbf{finer} than $\tau_{\abs{\cdot}}$.
        \item \textit{Relative or subset topology}: let $(X,\tau)$ be a topological space and $\emptyset\neq A\subseteq X$.
            Then we can define a topology $\tau|_A=\{U\cap A:U\in\tau\}$.
    \end{enumerate}
\end{example}
\subsection{Metric Topology}
A metric space $(X,d)$ is naturally a topological space, where the topology is given by
\begin{equation*}
    \tau_d = \Set{U\subseteq X |\text{ for each }x_0\in U\text{, there is }\delta=\delta(x_0)\text{ s.t. }\\B_\delta(x_0)\subseteq U}.
\end{equation*}
Given two metrics $d,\rho$ on $X$, we say that $d\sim\rho$ are \textbf{equivalent} if and only if there are $c,C>0$ such that
\begin{equation*}
    cd(x,y)\leq\rho(x,y)\leq Cd(x,y)\text{ for any }x,y\in X
\end{equation*}
Note that $d\sim\rho$ implies that $\tau_d=\tau_\rho$, but the reverse implication is not true.
An example of this are the metrics on $X=\R$ given by $d(x,y)$ and $\rho(x,y)=\frac{|x-y|}{1+|x-y|}$.
Then $d\not\sim\rho$ but $\tau_d=\tau_\rho$.
Let $(X,d)$, $(Y,\rho)$ be metric spaces.
A map $f:X\to Y$ is an \textbf{isometry} if for any $x,y\in X$, $d(x,y)=\rho(f(x),f(y))$.
By non-degeneracy, $f$ is automatically injective.
In particular, when $(X,d)$ is complete, then $(f(X),\rho|_{f(X)})$ is a complete metric space.
\begin{definition}
    Let $(X,\tau)$ and $(Y,\sigma)$ be topological spaces, and $f:X\to Y$.
    We say that $f$ is \mbf{$(\tau-\sigma-)$continuous at $x_0$} in $X$ if for any $V\in\sigma$ such that $f(x_0)\in V$, then there exists $U\in\tau$ such that $x_0\in U$ and $f(U)\subseteq V$.
    We say that $f$ is \mbf{$(\tau-\sigma-)$continuous} if it is continuous at each $x_0$ in $X$.
\end{definition}
An easy application of definitions yields the following:
\begin{proposition}
    Let $(X,\tau)$, $(Y,\sigma)$ be topological spaces and $f:X\to Y$.
    Then $f$ is continuous if and only if for any $U\in\sigma$, $f^{-1}(U)\in\tau$.
\end{proposition}
\begin{lemma}
    If $x_0\in X$ where $(X,\tau)$ is a topological space, then
    \begin{equation*}
        \mathcal{I}(x_0)= \Set{f\in C_b(X) | f(x_0)=0}
    \end{equation*}
    is closed, hence complete, subspace of $C_b(X)$.
\end{lemma}
\begin{proof}
    If $(f_n)_{n=1}^\infty\subseteq\mathcal{I}(x_0)$ and $f=\lim_{n\to\infty}f_n$ with respect to $\norm{\cdot}_\infty$ in $C_b(X)$, then $f(x_0)=\lim_{n\to\infty}f_n(x_0)=0$.
    Thus $f\in\mathcal{I}(x_0)$, and closed subsets of complete spaces are themselves complete.
\end{proof}
\chapter{Basic Elements of Functional Analysis}
\section{Banach Spaces}
Throughout, we denote by $\F$ either the field $\R$ or the field $\C$.
\begin{definition}
    Let $X$ be a vector space over $\F$.
    A \textbf{seminorm} is a functional $\norm{\cdot}:X\to\R$ such that it is
    \begin{itemize}[nl]
        \item \textit{(non-negative)} $\norm{x}\geq 0$ for any $x\in X$
        \item \textit{(subadditive)} $\norm{x+y}\leq\norm{x}+\norm{y}$ for $x,y\in X$
        \item \textit{($\abs{\cdot}-$homogenous)} $\norm{\alpha x}=\abs{\alpha}\norm{x}$ for $\alpha\in\F$, $x\in X$.
    \end{itemize}
    If in addition, $\norm{\cdot}$ satisfies the added requirement
    \begin{itemize}[nl]
        \item \textit{(non-degenerate)} $\norm{x}=0$ if and only if $x=0$
    \end{itemize}
    we call $\norm{\cdot}$ a \textbf{norm} for $X$.
    In this case, the pair $(X,\norm{\cdot})$ a \textbf{normed vector space}.
    We say that $(X,\norm{\cdot})$ is a \textbf{Banach space} provided that $X$ is complete with respect to the metric $\rho(x,y)=\norm{x-y}$ induced by the norm.
\end{definition}
\begin{example}
    Here are some standard examples of Banach spaces:
    \begin{enumerate}[r]
        \item $(\F,\abs{\cdot})$ is probably the simplest example of a Banach space.
        \item \textit{Finite-dimensional space:} denoted $(\F^d,\norm{\cdot}_p)$ with points $x=(x_j)_{j=1}^n$ equipped with the $p-$norm
            \begin{equation*}
                \norm{x}_p=\begin{cases}
                    \left(\sum_{i=1}^n\abs{x_j}^p\right)^{1/p} & 1\leq p<\infty\\
                    \max_{j=1,\ldots,n}|x_j| & p=\infty
                \end{cases}
            \end{equation*}
            is a Banach space
        \item If you have a background in basic measure theory, the space $L_{p,\F}(\Omega)$, where $\Omega$ is a compact domain.
            For a concrete example, take for example
            \begin{equation*}
                \quot{L^p{\F}([0,1])=\Set{f:[0,1]\to\F | f\text{ is Lebesgue measurable}, \left(\int_0^1\abs{f}^p\right)^{1/p}<\infty}}{\sim_{\text{a.e.}}}
            \end{equation*}
            where $1\leq p<\infty$.
            To enforce non-degeneracy, we must mod out by equivalence almost everywhere.
        \item The space of essentially bounded functions, $\displaystyle L_\infty^{\F}[0,1]$, $\norm{f}_\infty=\esssup_{t\in[0,1]}|f(t)|$.
        \item \textit{Function spaces:} let $(X,d)$ be a metric space, and define
            \begin{align*}
                C_b(X,\F)&=\Set{f:X\to\F | f\text{ is continuous and bounded}},&\norm{f}_\infty&=\sup_{x\in X}|f(x)|.
            \end{align*}
    \end{enumerate}
\end{example}
Here, we define a more involved example.
\begin{example}
    Let $(X,d)$ be a metric space.
    We define the space of \textit{Lipschitz functions}
    \begin{equation*}
        \Lipspace=\Set{f:X\to\F | f\text{ is bounded}, L(f)=\sup_{\substack{x,y\in X\\x\neq y}}\frac{|f(x)-f(y)|}{d(x,y)}<\infty}
    \end{equation*}
    Note that for any $f:X\to\F$, $f\in\Lipspace$ if and only if there is some $L\geq 0$ such that $|f(x)-f(y)|\leq Ld(x,y)$ for all $x,y$ in $X$.
    One may verify that $L(f)$ is the infimum over all values of $L$ for which this inequality holds over $X$.

    It is an easy exercise to see that $\Lipspace$ is a vector space and that $L:\Lipspace\to\R$ is a seminorm.
    However, we do not have non-degeneracy - for example, if $f$ is constant, then $L(f)=0$.
    To define a norm on the space of Lipschitz functions, we essentially force non-degeneracy by construction: we define the \textit{Lipschitz norm}
    \begin{equation*}
        \norm{f}_{\Lip}=\norm{f}_\infty+L(f).
    \end{equation*}
\end{example}
In this case, we do in fact have what we want:
\begin{proposition}
    $(\Lipspace,\norm{\cdot}_{\Lip})$ is a Banach space.
\end{proposition}
\begin{proof}
    Let $(f_n)_{n=1}^\infty$ be a Cauchy sequence in $(\Lipspace,\norm{\cdot}_{\Lip})$.
    Since $\norm{\cdot}_\infty\leq\norm{\cdot}_{\Lip}$ on $\Lipspace$, this sequence is uniformly Cauchy and hence converges to some $f\in C_b(X,\F)$ with respect to the uniform norm.
    Moreover, if $x,y\in X$, then
    \begin{align*}
        |f(x)-f(y)| &= \lim_{n\to\infty}|f_n(x)-f_n(y)|\leq \sup_{n\in\N}|f_n(x)-f_n(y)|\\
                    &\leq \sup_{n\in\N} L(f_n)d(x,y)\leq\sup_{n\in\N}\norm{f_n}_{\Lip} d(x,y).
    \end{align*}
    Since Cauchy sequences are bounded in norm, we have that $|f(x)-f(y)|\leq Ld(x,y)$ where $L=\sup_{n\in\N}\norm{f_n}_{\Lip}<\infty$, so in fact $f\in\Lipspace$.
    It is easy to verify that $\lim_{n\to\infty}\norm{f-f_n}_{\Lip}=0$.
\end{proof}
\subsection{Sequence Spaces}
Since we do not assume the background of measure theory in this treatment, one of our main basic examples of Banach spaces will be the sequence spaces.
Let $\F^{\N}$ denote the set of all sequences in $\F$, and define
\begin{equation*}
    \lp{1}=\Set{x=(x_j)_{j=1}^\infty\in\F^{\N} | \norm{x}_1=\sum_{j=1}^\infty|x_j|<\infty}.
\end{equation*}
It is easy to see that $(\ell_1,\norm{\cdot}_1)$ is a normed vector space.

More generally, for $1<p<\infty$, we may define
\begin{equation*}
    \lp{p}=\Set{x\in\F^{\N} |\norm{x}_p=\left(\sum_{j=1}^\infty|x_j|^p\right)^{1/p}<\infty}.
\end{equation*}
As always, it is easy to verify that the $\lp{p}-$spaces, for $1\leq p<\infty$, are in fact normed vector spaces.
The interesting work is in proving that they are Banach spaces.

Let $q=p/(p-1)$ so that $1/p+1/q=1$.
Then $q$ is called the \textbf{conjugate index} to $p$.
We have a number of standard inequalities on $\ell_p-$spaces, the proofs of which can be found in general in \TODO{eventually link measure theory result}.
\begin{proposition}[Inequalities in $\lp{p}-$spaces]
    Throughout, let $1<p,q<\infty$ be conjugate exponents.
    \begin{itemize}
        \item \textit{Young's Inequality:} If $a,b\geq 0$ in $\R$, then $ab\leq \frac{a^p}{p}+\frac{b^q}{q}$, with equality if and only if $a^p=b^q$.
        \item \textit{Hölder's Inequality:} If $x\in\lp{p}$ and $y\in\lp{q}$, then $xy=(x_iy_i)_{i=1}^\infty\in\ell_1$, with
            \begin{equation*}
                \sum_{i=1}^\infty \abs{x_iy_i}\leq\norm{x}_p\norm{y}_q.
            \end{equation*}
            Note that equality holds if and only if the following two conditions hold:
            \begin{enumerate}[nl,r]
                \item $\sgn(x_iy_i)=\sgn(x_ky_k)$ for all $j,k\in\N$ where $x_iy_i\neq 0\neq x_ky_k$, and
                \item $|x|^p=(|x_j|^p)_{j=1}^\infty$ and $|y|^q$ are linearly dependent in $\ell_1$.
            \end{enumerate}
        \item \textit{Minkowski's Inequality:} If $x,y\in\ell_p$, then $\norm{x+y}_p\leq\norm{x}_p+\norm{y}_p$ with equality exactly when one of $x$ or $y$ is a non-negative scalar combination of the other.
    \end{itemize}
\end{proposition}
In particular, Minkowski's Inequality \TODO{cite certain labels by name? and also link - would be nice}

\subsection{Bounded Continuous Functions into a Normed Space}
Let $(Y,\norm{\cdot})$ be a normed space and $\tau=\tau_{\norm{\cdot}}$ the topology induced by $\norm{\cdot}$.
Let $(X,\tau)$ be any topological space.
We define the space
\begin{equation*}
    C_b(X,Y) = \Set{f:X\to Y | f\text{ is bounded  and }\tau-\tau_{\norm{\cdot}}-\text{continuous}}
\end{equation*}
With pointwise operations, we see that $C_b(X,Y)$ is a vector space.
We also define for $f\in C_b(X,Y)$, $\norm{f}_\infty=\sup\{\norm{f(x)}:x\in X\}$, making $(C_b(X,Y),\norm{\cdot}_\infty)$ a normed vector space.
\begin{theorem}
    If $(Y,\norm{\cdot})$ is a Banach space, then $(C_b(X,Y),\norm{\cdot}_\infty)$ is a Banach space.
\end{theorem}
\begin{proof}
    Let $(f_n)_{n=1}^\infty$ be a Cauchy sequence in $(C_b(X,Y),\norm{\cdot}_\infty)$.
    Then for any $x\in X$, we have that $(f_n(x))_{n=1}^\infty$ is Cauchy in $(Y,\norm{\cdot})$ since $\norm{f_n(x)-f_m(x)}\leq\norm{f_n-f_m}_\infty$, and hence admis a limit $f(x)$.
    This defines a pointwise limit $f:X\to Y$.
    Fix $x_0\in X$: we must show that $f$ is continuous at $x_0$.
    Given $\epsilon>0$, set
    \begin{itemize}[nl]
        \item $n_1$ so that whenever $n,m\geq n_1$, $\norm{f_n-f_m}_\infty<\epsilon/4$.
        \item $n_2$ so that whenever $n\geq n_2$, $\norm{f_n(x_0)-f(x_0)}<\epsilon/4$.
        \item $N=\max\{n_1,n_2\}$.
        \item $U\in \tau$, $x_0\in U$ such that $f_N(U)\subseteq B_{\epsilon/4}(f(x_0))\subset Y$.
    \end{itemize}
    Then for $x\in U$, we let $n_x$ be so $n_x\geq n_1$ and $n\geq n_x$, so that $\norm{f_n(x)-f(x)}<\epsilon/4$.
    We then have
    \begin{align*}
        \norm{f(x)-f(x_0)} &\leq \norm{f(x)-f_{n_x}(x)} + \norm{f_{n_x}(x)-f_N(x)} + \norm{f_N(x)-f_N(x_0)} + \norm{f_N(x_0)-f(x_0)}\\
                           &<\frac{\epsilon}{4}+\norm{f_{n_x}-f_N}_\infty+\frac{\epsilon}{4}+\frac{\epsilon}{4}<\epsilon
    \end{align*}
    in other words that $f(U)\subseteq B_\epsilon(f(x_0))$ so that $f$ is continuous.

    Now let us check that $\norm{f}_\infty<\infty$.
    Since $|\norm{f_n}_\infty-\norm{f_m}_\infty|\leq\norm{f_n-f_m}_\infty$, $(\norm{f_n}_\infty)_{n=1}^\infty\subseteq\R$ is Cauchy, hence bounded.
    If $x\in X$, then
    \begin{equation*}
        \norm{f(x)} = \lim_{n\to\infty}\norm{f_n(x)}\leq\sup_{n\in\N}\norm{f_n(x)}\leq\sup_{n\in\N}\norm{f_n}_\infty<\infty
    \end{equation*}
    so $\norm{f}_\infty=\sup_{x\in X}\norm{f(x)}<\infty$.

    Finally, to show that the limit indeed converges approriately, if $\epsilon, n_1, x_0,N$ are as above, we have for $n\geq n_1$
    \begin{equation*}
        \norm{f_n(x_0)-f(x_0)} \leq \norm{f_n(x_0)-f_N(x_0)}+\norm{f_N(x_0)-f(x_0)}<\frac{\epsilon}{2}
    \end{equation*}
    so $\norm{f_n-f}_\infty=\sup_{x_0\in X}\norm{f_n(x_0)-f(x_0)}\leq\epsilon/2<\epsilon$.
    The convergence is uniform since $n_1$ is chosen uniformly in $X$.
\end{proof}
\begin{corollary}
    $(C_b(X,\F),\norm{\cdot}_\infty)$ is a Banach space.
\end{corollary}
\begin{example}
    \begin{enumerate}[nl,r]
        \item Let $T$ be a non-empty set and let
            \begin{equation*}
                \lp{\infty}(T) = \Set{x=(x_t)_{t\in T}\in\F^T | \norm{x}_\infty}<\infty
            \end{equation*}
            With pointwise operations, $(\ell_\infty,\norm{\cdot}_\infty)$ is a normed space.
            In fact, it is a Banach space, since
            \begin{align*}
                f\mapsto (f(t))_{t\in T}:C_b(T,\mathcal{P}(T))\to\ell_\infty(T)
            \end{align*}
            is a surjective linear isometry, and the result follows.
        \item Let $c=\Set{x\in\ell_\infty | \lim_{n\to\infty}x_n\text{ exists}}$.
            Then $(c,\norm{\cdot}_\infty)$ is a Banach space.
            Consider the topological space given by $\omega=\N\cup\{\infty\}$, with topology
            \begin{equation*}
                \tau_\omega=\mathcal{P}(\N)\cup\bigcup_{n\in\N}\{k\in\N:k\geq n\}
            \end{equation*}
            The map $f\mapsto(f(n))_{n=1}^\infty:C_b(\omega)\to c$ is a linear surjective isometry.
        \item Recall that $\mathcal{I}(\infty)$ is a closed, and hence complete, subspace of $c$.
            We may define $c_0=\Set{x\in\F^{\N} | \lim_{n\to\infty}x_n=0}\subseteq c\subseteq\ell_\infty$.
            In this case, $f\mapsto(f(n))_{n=1}^\infty:\mathcal{I}(\infty)\to c_0$ is a (linear) surjective isometry.
        \item Consider the Sorgenfry line $(\R,\sigma)$.
            One may verify that
            \begin{equation*}
                C_b((\R,\sigma),\F)=\Set{f:\R\to\F | f\text{ is bounded and }\lim_{t\to t_0^+} f(t)=f(t_0)\text{ for }t\in\R}
            \end{equation*}
    \end{enumerate}
\end{example}
\section{Linear Functionals and Operators}
Let $X,Y$ be vector spaces.
We let $\mathcal{L}(X,Y)=\Set{S:X\to Y | S\text{ is linear}}$; this is itself a vector space with pointwise operations.
Let $(X,\norm{\cdot})$ be a normed space.
We denote
\begin{align*}
    D(X) &= \{x\in X:\norm{x}<1\}\\
    S(X) &= \{x\in X:\norm{x}=1\}\\
    B(X) &= \{x\in X:\norm{x}\leq 1\}
\end{align*}
\textit{(Yes, this notation is confusion. No, I didn't choose it.)}

\begin{proposition}
    If $X,Y$ are normed spaces and $S\in\mathcal{L}(X,Y)$, then the following are equivalent:
    \begin{enumerate}[nl,r]
        \item $S$ is continuous
        \item $S$ is continuous at some $x_0\in X$
        \item $\norm{S}=\sup_{x\in D(X)}\norm{Sx}<\infty$.
    \end{enumerate}
    Moreover, in this case, we have 
    \begin{align*}
        \norm{S}&=\min\{L>0:\norm{Sx}\leq L\norm{x}\text{ for }x\in X\}\\
                &= \sup_{x\in S(X)}\norm{Sx}=\sup_{x\in B(X)}\norm{Sx}
    \end{align*}
\end{proposition}
\begin{proof}
    \imp{i}{ii}
    By definition.

    \imp{ii}{iii}
    Note that
    \begin{equation*}
        Sx_0+D(Y) = \{Sx_0+y:t\in D(Y)\} = \{y\in Y:\norm{Sx_0-y'}<1\}
    \end{equation*}
    is a neighbourhood of $Sx_0$.
    By the definition of metric continuity, there is $\delta>0$ such that
    \begin{equation*}
        x_0+\delta D(X) = \{x_0+\delta x:x\in D(x)\}=\{x'\in X:\norm{x_0-x'}<\delta\}
    \end{equation*}
    such that
    \begin{equation*}
        Sx_0+\delta S(D(X)) = S(x_0+\delta D(x)\subseteq Sx_0+D(Y)
    \end{equation*}
    which implies that $\delta S(D(X))\subseteq D(Y)$ and $S(D(X))\subseteq D(Y)/\delta$, in other words that $\norm{Sx}\leq 1/\delta$ for $x\in D(X)$.

    \imp{iii}{i}
    If $x\in X$ and $\epsilon>0$, then
    \begin{equation*}
        \norm{Sx} = (\norm{x}+\epsilon)\norm{S\left(\frac{1}{\norm{x}+\epsilon}\norm{x}\right)}\leq(\norm{x}+\epsilon){\norm{S}}
    \end{equation*}
    Then, letting $\epsilon\to 0^+$, we see that
    \begin{equation*}
        \norm{Sx}\leq\norm{x}\norm{S}=\norm{S}\norm{X}
    \end{equation*}
    If $x,x'\in X$, then $\norm{Sx-S'x}\leq\norm{S}\norm{x-x'}$ is $S$ is Lipschitz, hence continuous.

    To complete the proof, the content of (iii) implies (i) tellus us that the Lipschitz constant $L(S)\leq\norm{S}$.
    Furthermore, if $\norm{x}=1$, the preceding proof gives us that $\norm{S}_{S(X)}$.

    Conversely,
    \begin{equation*}
        \norm{S} = \sup_{x\in D(X)\setminus\{0\}}\norm{Sx}=\sup_{x\in D(X)\setminus\{0\}}\norm{x}\norm{S\left(\frac{1}{\norm{x}}x\right)}\leq\sup_{x\in S(X)}\norm{Sx}
    \end{equation*}
    The remaining equivalence is obvious.
\end{proof}
We now let $\mathcal{B}(X,Y)=\Set{S\in\mathcal{L}(X,Y) | S\text{ is bounded}}$.
We will see that $\norm{\cdot}$, above, defines a norm on $B(X,Y)$.
\begin{theorem}
    If $X,Y$ are normed spaces, then $(\mathcal{B}(X,Y),\norm{\cdot})$ is a normed space.
    Furthermore, if $Y$ is a Banach spaces, then so to is $(\mathcal{B}(X,Y),\norm{\cdot})$.
\end{theorem}
\begin{proof}
    Define
    \begin{equation*}
        \Gamma:\mathcal{B}(X,Y)\to C_b^Y(B(X))
    \end{equation*}
    given by $\Gamma(S)=S|_{B(X)}$.
    Then, by definition, $\Gamma$ is linear, with
    \begin{equation*}
        \norm{\Gamma(S)}_\infty=\sup_{x\in B(X)}\norm{Sx}=\norm{S}
    \end{equation*}
    Thus $\norm{\cdot}$ is a norm: if $S,T\in\mathcal{B}(X,Y)$, $\alpha\in\F$,
    \begin{align*}
        \norm{S+T}&=\norm{\Gamma(S+T)}_\infty=\norm{\Gamma(S)+\Gamma(T)}_\infty\leq \norm{\Gamma(S)}_\infty+\norm{\Gamma(T)}_\infty=\norm{S}+\norm{T}\\
        \norm{\alpha S}&=\norm{\Gamma(\alpha S)}_\infty=|\alpha|\norm{\Gamma(S)}_\infty=|\alpha|\norm{S}.
    \end{align*}
    Furthermore, $\Gamma:\mathcal{B}(X,Y)\to C_b^Y(\mathcal{B}(X))$ is an isometry.

    Now suppose that $Y$ is a Banach space.
    We will show that $\Gamma(\mathcal{B}(X,Y))$ is closed in $C_b^Y(B(X))$, and hence $B(X,Y)=\Gamma^{-1}(\Gamma(\mathcal{B}(X,Y)))$ is complete.
    Let $(S_n)_{n=1}^\infty\subset\mathcal{B}(X,Y)$ be $\norm{\cdot}-$Cauchy.
    Then $(\Gamma(S_n))_{n=1}^\infty$ is $\norm{\cdot}_\infty-$Cauchy in $C_b^Y(B(X))$, and hence there is $f\in C_b^Y(B(X))$ such that $\lim_{n\to\infty}\norm{\Gamma(S_n)-f}_\infty=0$.
    Then we let $S:X\to Y$ be given by
    \begin{equation*}
        Sx=\begin{cases}
            \norm{x}f\left(\frac{x}{\norm{x}}\right) & x\neq 0\\
            0 & x=0
        \end{cases}
    \end{equation*}
    If $x,x'\in X$ and $\alpha\in\F$ are all such that $x,x',x+\alpha x'\neq 0$, then
    \begin{align*}
        S(x+\alpha x') &= \norm{x+\alpha x'}f\left(\frac{1}{x+\alpha x'}(x+\alpha x')\right)\\
                       &= \norm{x+\alpha x'}\lim_{n\to\infty} S_n\left(\frac{1}{x+\alpha x'}(x+\alpha x')\right)\\
                       &= \lim_{n\to\infty}(S_nx+\alpha S_nx')=\lim_{n\to\infty}\left[\norm{x}S_n\left(\frac{1}{\norm{x}}x\right)+\alpha\norm{x'}S_n\left(\frac{1}{\norm{x}}x'\right)\right]\\
                       &= \norm{x}f\left(\frac{x}{\norm{x}}\right)+\alpha\norm{x'}f\left(\frac{x'}{\norm{x}}\right)\\
                       &= Sx+\alpha Sx'
    \end{align*}
    The above computation is easily performed if any of $x$, $x'$, $x+\alpha x'$ are 0.
    Hence $S\in\mathcal{L}(X,Y)$.
    We se that $S$ is continuous (say, at a point on $S(X)$), so $S\in\mathcal{B}(X,Y)$.
    Finally, as $S|_{B(X)}=f=\lim_{n\to\infty}S_n|_{B(X)}$ (with respect to the uniform norm), we have
    \begin{equation*}
        \norm{S-S_n}=\sup_{x\in B(X)}\norm{(S-S_n)x}=\norm{f-\Gamma(S_n)}_\infty
    \end{equation*}
    goes to $0$ as $n$ goes to infinity.
\end{proof}
\begin{definition}
    Given a vector space $X$, let $X'=\mathcal{L}(X,\F)$ denote the \textbf{algebraic dual}.
    If further $X$ is a normed space, we let $X^*=\mathcal{B}(X,\F)$ denote the (continuous) dual.
\end{definition}
\begin{corollary}
    If $X$ is a normed spaces, then $X^*$ is always a Banach space.
\end{corollary}
\begin{theorem}
    Let for $x\in\ell_1$, $f_x:c_0\to\F$ be given by $f_x(y)=\sum_{j=1}^\infty x_jy_j$.
    Then $f_x\in c_0^*$ with $\norm{f_x}=\norm{x}_1$.
    Furthermore, every element of $c_0^*$ arises as above.
\end{theorem}
\begin{proof}
    If $x\in\ell_1$ and $y\in c_0\subseteq\ell_\infty$, then
    \begin{equation*}
        \sum_{j=1}^\infty|x_jy_j|\leq\sum_{j=1}^\infty|x_j|\norm{y}_\infty=\norm{x}_1\norm{y}_\infty<\infty
    \end{equation*}
    so $f_x(y)=\sum_{j=1}^\infty x_jy_j$ is well-defined.
    It is obvious that $f_x$ is linear: $f_x(y+\alpha y')=f_x(y)+\alpha f(y')$ for $y,yl\in c_0$ and $\alpha\in\F$.
    Also, $\norm{f_x}\leq\norm{x}_1$.
    We let $y^{n}=(\overline{\sgn x},\ldots,\overline{\sgn x_n},0,0,\ldots)\in c_0$, with $\norm{y^n}=1$.
    Then
    \begin{equation*}
        \norm{f_x}\geq|f_x(y^n)|=\sum_{j=1}^n x_i\overline{\sgn x_i}=\sum_{j=1}^n|x_i|
    \end{equation*}
    so that $\norm{f_x}\geq\norm{x}_1$, and hence equality holds.

    Now let $f\in c_0^*$, and write $e_n=(0,\ldots,0,1,0,0,\ldots)\in c_0$, and let $x_n=f(e_n)$.
    Then, let $y\in c_0$ and $y^n=(y_1,\ldots,y_n,0,0,\ldots)$ and we have
    \begin{equation*}
        \norm{y-y^n}_\infty=\sup_{j\geq n+1}|y_j|
    \end{equation*}
    which goes to 0 as $n$ goes to infinity.
    Then since $f$ is continuous, we have
    \begin{equation*}
        f(y)=\lim_{n\to\infty}f(y^n)=\lim_{n\to\infty}\sum_{j=1}^n y_jx_j=\sum_{j=1}^\infty x_jy_j=f_x(y)
    \end{equation*}
    We use sequence $(y^n)_{n=1}^\infty$ as in $y^n\in c_0$, to see that
    \begin{equation*}
        \sum_{j=1}^n|x_i|=|f(y^n)|\leq\norm{f}<\infty
    \end{equation*}
    so $x\in\ell_1$.
    Thus $f=f_x$, as desired.
\end{proof}
\begin{corollary}
    $\ell_1\cong c^*$ isometrically isomorphically.
\end{corollary}
\begin{proof}
    For $y\in c$, let $L(y)=\lim_{n\to\infty}y_n$.
    Given $y\in c$, let $y^n=(y_1,\ldots,y_n,L(y),L(y),\ldots)\in c$.
    Notice that $\norm{y-y^n}_\infty\to 0$ similarly as above.

    We let $1=(1,1,\ldots)$, and $1_n=(0,\ldots,0,1,1,\ldots)$.
    If $m<n$, then $1_n-1_m\in c_0$, so
    \begin{equation*}
        |f(1_n)-f(1_m)| = |f_x(1_n-1_m)|\leq \sum_{j=m+1}^n|x_j|
    \end{equation*}
    so that $(f(1_n))_{n=1}^\infty$ is Cauchy in $\F$.
    Let $x_0=\lim_{n\to\infty}f(1_n)$.
    Let $\tilde x=(x_0,x_1,\ldots)\in\ell_1$.
    Then letting $x_j=f(e_j)$, we see that
    \begin{equation*}
        f(y)=\lim_{n\to\infty}f(y^n)=\sum_{j=1}^\infty x_jy_j+x_0L(y)
    \end{equation*}
    Similarly as above, we may show that $\norm{f}=\norm{\tilde x}_1$.


\end{proof}
\begin{remark}
    We write $c_0^*\cong\ell_1$ isometrically.
\end{remark}
\begin{corollary}
    $(\ell_1,\norm{\cdot}_1)$ is complete.
\end{corollary}
\section{Axiom of Choice and the Hahn-Banach Theorem}
\begin{definition}
    Let $S$ be a non-empty set.
    A \textbf{partial ordering} is a binary relation $\leq$ on $S$ which satisfies for $s,t,n\in S$,
    \begin{enumerate}[nl,r]
        \item \textit{(reflexivity)} $s\leq s$
        \item \textit{(transitivity)} $s\leq t$, $t\leq u$ implies $s\leq u$
        \item \textit{(anti-symmetry)} $s\leq t$, $t\leq s$ implies $s=t$
    \end{enumerate}
    We call the pair $(S,\leq)$ a \textbf{partially ordered set}.
    We say that $(S,\leq)$ is \textbf{totally ordered} if, given $s,t\in S$, at least one of $s\leq t$ or $t\leq s$ holds.
    We say that $(S,\leq)$ is \textbf{well-ordered} if given any $\emptyset\neq S_0\subseteq S$, there is some $s_0\in S_0$ such that $s_0\leq s$ for $s\in S_0$.
    A \textbf{chain} in a poset $(S,\leq)$ is any $\emptyset\neq C\subseteq S$ such that $(S,\leq|_C)$ is totally ordered.
\end{definition}
\begin{example}
    \begin{enumerate}[nl,r]
        \item $X\neq\emptyset$, $(\mathcal{P}(X),\subseteq)$ is a poset
        \item $(\R,\leq)$ is a totally ordered set
        \item $(\N,\leq)$, $(\omega=\N\cup\{\infty\},\leq)$, are well-ordered sets.
    \end{enumerate}
\end{example}
\begin{theorem}
    The following are equivalent:
    \begin{enumerate}[nl,r]
        \item \textit{(Axiom of Choice 1)}: For any $x\neq\emptyset$, there is a function $\gamma:\mathcal{P}(X)\setminus\{\emptyset\}\to X$ such that $\gamma(A)\in A$ for each $A\in\mathcal{P})X_\setminus\{\emptyset\}$.
        \item \textit{(Axiom of Choice 2)}: Given any $\{A_\lambda\}_{\lambda\in\Lambda}$ where $A_\lambda\neq\emptyset$ for each $\lambda$,
            \begin{equation*}
                \prod_{\lambda\in\Lambda}A_\lambda = \left\{(a_\lambda)_{\lambda\in\Lambda}:a_\lambda\in A_\lambda\text{ for each }\lambda\right\}\neq \emptyset
            \end{equation*}
        \item \textit{(Zorn's Lemma)}: In a poset $(S,\leq)$, if each chain $C\subseteq S$ admits an upper bound in $S$, then $(S,\leq)$ admis a maximal element.
        \item \textit{(Well-ordering principle)}: Any $S\neq\emptyset$ admits a well-ordering
    \end{enumerate}
\end{theorem}
\begin{proof}
    Exercise.
\end{proof}
\begin{definition}
    Let $X$ be a vector space (over $k$).
    A subset $S\subseteq X$ is called
    \begin{itemize}[nl]
        \item \textbf{linearly independent} if for any distinct $x_1,\ldots,x_n\in S$, the equation $0=\alpha_1x_1+\cdots+\alpha_nx_n=0$ where $\alpha_i\in k$ implies $\alpha_1=\cdots=\alpha_n=0$.
        \item \textbf{spanning} if each $x\in X$ admits $x_i\in S$ and $\alpha_i\in k$ such that $x=\alpha_1x_1+\cdots+\alpha_nx_n$.
        \item \textbf{Hamel basis} if it is both linearly independent and spanning
    \end{itemize}
\end{definition}
\begin{proposition}
    Any vector space $X$ admits a Hamel basis.
\end{proposition}
\begin{proof}
    Let $\mathcal{L}=\{L\subseteq X:L\text{ is linearly independent}\}$.
    Then $(\mathcal{L},\subseteq)$ is a poset.
    Verify that for any chain $\mathcal{C}\subseteq\mathcal{L}$, that $U=\bigcup_{L\in\mathcal{C}}L\in\mathcal{L}$ and is an upper bound for $\mathcal{C}$.
    Apply Zorn to find a maximal element $M$ in $(\mathcal{L},\subseteq)$.
    Verify that $M$ is spanning for $X$.
\end{proof}
\begin{corollary}
    If $X$ is an infinite dimensional normed space, then there exists $f\in X'\setminus X^*$.
\end{corollary}
\begin{proof}
    Our assumption provides $\{e_n\}_{n=1}^\infty$ which is linearly independent.
    By normalizing each element, we may and will suppose that each $\norm{e_n}=1$.
    Let
    \begin{equation*}
        \spn\{e_n\}_{n=1}^\infty=\Set{\sum_{j=1}^m\alpha_je_{n_j}:m\in\N,\alpha_i\in\F, n_1<\cdots<n_m}
    \end{equation*}
    and let $B$ be any linearly independent set containing $\{e_n\}_{n=1}^\infty$.
    Define $f:X=\spn B\to\F$ be given for $x=\sum_{b\in B\setminus\{e_n\}_{n=1}^\infty}\alpha_b b+\sum_{j=1}^n\alpha_je_{n_j}$ by $f(x)=\sum_{j=1}^m\alpha_j n_j$.
    The point is that $f(e_n)=n$ and $f(e)=0$ for any other $e\in B$.
    Notice that
    \begin{equation*}
        \norm{f}=\sup_{x\in B(X)}|f(x)|\geq\sup_{n\in\N}|f(e_n)|=\sup_{n\in\N}n=\infty
    \end{equation*}
\end{proof}
\begin{definition}
    Let $X$ be a $\R-$vector space.
    A \textbf{sublinear functional} is any $\rho:X\to\R$ such that it satisies
    \begin{itemize}[nl]
        \item \textit{(non-negative homogenity)} $\rho(tx)=t\rho(x)$ for $t\geq 0$, $x\in X$.
        \item \textit{(subadditivity)} $\rho(x+y)\leq\rho(x)+\rho(y)$ for $x,y\in X$.
    \end{itemize}
\end{definition}

\begin{theorem}[Hahn-Banach]
    Let $X$ be a $\R-$vector space, $\rho:X\to\R$ a sublinear functional, $Y\subseteq X$ a subspace and $f\in Y'$ such that $f\leq\rho|_Y$.
    Then there exists $F\in X'$ such that $F|_Y=f$ and $F\leq\rho$ on $X$.
\end{theorem}
\begin{proof}
    We first do this for extensions by a single point $x\in X\setminus Y$.
    We wish to find $c\in\R$ such that
    \begin{equation*}
        f(y)+\alpha c\leq\rho(y+\alpha x)
    \end{equation*}
    for $y\in Y$ and $\alpha\in\R$.
    In this case, we let $F:\spn Y\cup\{x\}\to\R$ be given by $F(y+\alpha x)=f(y)+\alpha c$, and we have that $F$ is linear and satisfies $F\leq\rho$ on $\spn Y\cup\{s\}$.
    To do this, let $y_+,y_-$ in $Y$ and observe that $f(y_+)+f(y_-)=f(y_++y_-)\leq \rho(y_++y_-)\leq\rho(y_++x)+\rho(y_--x)$ so that $f(y_-)-\rho(y_--x)\leq\rho(y_++x)-f(y_+)$.
    It thus follows that
    \begin{equation*}
        \sup\{f(y)-\rho(y-x):y\in Y\}\leq\in\{\rho(y+x)-f(y):y\in Y\}
    \end{equation*}
    so we may find $c\in\R$ for which
    \begin{equation*}
        \sup\{f(y)-\rho(y-x):y\in Y\}\leq c\leq\inf\{\rho(y+x)-f(y):y\in Y\}
    \end{equation*}
    If $t>0$, then for $y\in Y$,
    \begin{equation*}
        c\leq\rho\left(\frac{1}{t}y+x\right)-f\left(\frac{1}{t}y\right)\Rightarrow tc\leq\rho(y+tx)-f(y)\Rightarrow f(y)+tc\leq\rho(y+tx)
    \end{equation*}
    and if $s>0$, then for $y\in Y$,
    \begin{equation*}
        f\left(\frac{1}{s}y\right)-\rho\left(\frac{1}{s}y-x\right)\leq c\Rightarrow sc\leq f(y)-\rho(y+sx)\Rightarrow f(y)-sc\leq\rho(y-sx)
    \end{equation*}
    Clearly, $f(y)+0\leq\rho(y+0x)$.
    Hence, we have our desired inequality.

    We now use Zorn's lemma to lift this result to the whole space.
    Consider the set of ``$p$-extensions'' of $f$,
    \begin{equation*}
        \mathcal{E} = \Set{(\mathcal{M},\psi) | Y\subseteq\mathcal{M}\subseteq X,\mathcal{M}\text{ is a subspace}, \psi\in\mathcal{M}',\psi|_Y=f,\psi\leq P|_{\mathcal{M}}}
    \end{equation*}
    Define a partial order on $\mathcal{E}$ by
    \begin{equation*}
        (\mathcal{M},\psi)\leq(\mathcal{N},\phi)\text{ iff }\mathcal{M}\subseteq\mathcal{N},\phi|_{\mathcal{M}}=\psi
    \end{equation*}
    Suppose $\mathcal{C}\subseteq\mathcal{E}$ is a chain with respect to $\leq$.
    We let
    \begin{itemize}[nl]
        \item $\mathcal{U}=\bigcup_{(\mathcal{M},\varphi)}\mathcal{M}$ which is a subspace, since $\mathcal{C}$ is a chain.
        \item and define $\phi:\mathcal{U}\to\R$ by $\phi(x)=\psi(x)$ whenever $x\in\mathcal{M}$, which is again well-defined since $C$ is a chain.
    \end{itemize}
    Furthermore, we see that $\phi\in U'$, since if $x,y\in\mathcal{U}$, get $x\in\mathcal{M}$, $y\in\mathcal{N}$ for some $(\mathcal{M},\psi)\leq(\mathcal{N},\psi')\in\mathcal{C}$.
    Then $\phi(x+y)=\psi'(x+y)=\psi'(x)+\psi'(y)=\phi(x)+\phi(y)$, etc.
    Likewise, $\psi\leq p|_{\mathcal{U}}$.
    Thus by Zorn's lemma, $\mathcal{E}$ admits a maximal element $\mathcal{M},F$
    Then $\mathcal{M}=X$, for if not, then we would find $x\in X\setminus\mathcal{M}$ and we apply step one to $\spn\mathcal{M}\cup\{x\}$ to get $F'$, a strictly larger element violating maximality.
\end{proof}
Trivially, any $\C-$vector siace is a $\R-$vector space.
\begin{lemma}
    Let $X$ be a $\C-$vector space.
    \begin{enumerate}[nl,r]
        \item If $f\in X'_{\R}$ into $\R$, then define $f_{\C}$ given by $f_{\C}(x)=f(x)-if(ix)$ defines an element of $X'=X'_{\C}$.
        \item If $g\in X'$, then $f=\Re g$ in $X'_{\R}$ satisfies $g=f_{\C}$.
        \item If $X$ is a normed $\C-$vector space, then for $f\in X'_{\R}$,
            \begin{equation*}
                f\in X^*_{\R}\text{ if and only if}f_{\C}\in X^*=X^*_{\C}\text{ with }\norm{f}=\norm{f_{\C}}
            \end{equation*}
    \end{enumerate}
\end{lemma}
\begin{proof}
    (i) and (ii) are straightforward exercises; let's see (iii).
    We let fr $x\in X$, $z=\sgn f_{\C}(x)$.
    Then
    \begin{align*}
        \R\ni|f_{\C}(x)|&=\overline{z}f_{\C}(x)=f_{\C}(\overline{z}x)=\Re f_{\C}(\overline{z}x)=f(\overline{z}x)=|f(\overline{z}x)|\\
                        &\leq\norm{f}\norm{\overline{z}x}=\norm{f}|\overline{z}|\norm{x}=\norm{f}\norm{x}
    \end{align*}
    so we see that $\norm{f_{\C}}\leq\norm{f}$.
    Conversely,
    \begin{equation*}
        |f(x)|=|\Re f_{\C}(x)|\leq|f_{\C}(x)|\leq\norm{f_{\C}}\norm{x}\text{ so that }\norm{f}\leq\norm{f_{\C}}
    \end{equation*}
\end{proof}
\begin{corollary}
    If $X$ is a normed space, $Y\subseteq X$ a subspace and $f\in Y^*$, then there exists $F\in X^*$ such that $F|_{Y}=f$ and $\norm{F}=\norm{f}$.
\end{corollary}
\begin{proof}
    Define $\rho:X\to\R$ be given by $p(x)=\norm{f}\cdot\norm{x}$, so $p$ is sublinear and $\Re f\leq p|_Y$.
    Apply Hahn-banach to to this data and get $\tilde F\in X_{\R}^*$ such that $\tilde F|_Y=\Re f$ and $\tilde F\leq p$, and let $F=\tilde F_{\C}$.
\end{proof}
\begin{corollary}
    If $X$ is a normed space, $x\in C$, then there is $f\in X^*$ such that
    \begin{equation*}
        \norm{x}=f(x)=|f(x)|\text{ and }\norm{f}=1
    \end{equation*}
\end{corollary}
\begin{proof}
    Let $f_0:\F x\to\F$ be given by $f_0(\alpha x)=\alpha\norm{x}$.
    If $x\neq 0$, then
    \begin{equation*}
        \norm{f_0}=\sup_{\norm{\alpha x}\leq 1}|f_0(\alpha x)|=\sup_{\norm{\alpha x}\leq 1}|\alpha|\norm{x}=1
    \end{equation*}
    and apply the previous corollary.
    If $x=0$, this is trivial.
\end{proof}
\begin{theorem}
    Let $X$ be a normed space and $X^{**}$ denote the bidual.
    For $x\in X$, define $\hat x:X^*\to\F$ by $\hat x(f)=f(x)$.
    Then $\hat x\in X^{**}$ with $\norm{\hat x}=\norm{x}$, so that $x\mapsto\hat x:X\to X^{**}$ is a linear isometry.
\end{theorem}
\begin{proof}
    Notice that $|\hat x(f)|=|f(x)\leq\norm{f}\norm{x}$ so $\norm{\hat x}\leq\norm{x}$.
    The last corollary provides for $x\in X$ an $f_x\in S(X^*)$ with $|f_x(x)|=\norm{x}$.
    Then $\norm{\hat x}\leq|\hat x(f_x)|=\norm{x}$.
    Hence $\norm{\hat x}=\norm{x}$.
    Clearly $x\mapsto\hat x$ is linear.
\end{proof}
\begin{remark}
    Since $X^{**}$, being a dual space, is complete, we have that $\hat X=\{\hat x:x\in X\}$ satisfies that its closure $\overline{\hat X}\subseteq X^{**}$ is complete.
    Hence $\overline{\hat X}$ is a Banach space containing a dense copy of $X$.
    Often, we will simply write $\overline{\hat X}=\overline{X}$ and call it the \textbf{completion} of $X$.
\end{remark}
\subsection{Geometric Hahn-Banach}
If $A,B\subset X$ with $A\cap B=\emptyset$ (and other suitable assumptions), we will find a $\R-$hyperplane between $A$ and $B$.
\begin{definition}
    In a vector space, a \textbf{hyperplane} is any set of the form $x_0+\ker f$ with $x_0\in X$ and $f\in X'$.
    Then a \textbf{$\R-$hyperplane} is any set of the form $x_0+\ker\Re f$.
\end{definition}
\begin{proposition}
    Let $X$ be a normed space.
    \begin{enumerate}[nl,r]
        \item If $f\in X^*\setminus\{0\}$, then $\ker f$ is closed and nowhere dense.
        \item if $f\in X'\setminus X^*$, then $\overline{\ker f}=X$.
    \end{enumerate}
    Thus a hyperplane in $X$ is either closed and nowhere dense, or it is dense.
\end{proposition}
\begin{proof}
    To see (i), $\ker f=f^{-1}(\{0\})$ is a closed set since $f$ is continuous.
    Furthermore, if $Y\subsetneq X$ is a proper closed subspace, then it is nowhere dense.
    If not, then there would exist $y_0\in T$ and $\delta>0$ such that $y_0+\delta D(X)\subseteq Y$.
    But then $D(X)\subseteq\frac{1}{\delta}(Y-y_0)=Y$, so $X=\spn D(X)\subseteq Y$, a contradiction.

    To see (ii), suppose that $\ker f$ is not dense in $X$.
    Then there would be $x_0\in X$ and $\delta>0$ such that $(x_0+\delta D(X))\cap \ker f=\emptyset$, so
    \begin{equation}\label{e:idx}
        0\notin f(x_0+\delta D(X))=f(x_0)+\delta f(D(X))\Longrightarrow \frac{1}{\delta}f(x_0)\notin -f(D(X))=f(D(X))
    \end{equation}
    But then $\norm{f}\leq\frac{1}{\delta}f(x_0)$, for if $\norm{f}>\frac{1}{\delta}f(x_0)$, there would be $x\in D(X)$ such that $|f(x)|>\frac{1}{\delta}|f(x_0)|$.
    Thus
    \begin{equation*}
        \left\lvert\frac{f(x_0)}{\delta f(x)}\right\rvert<1\Longrightarrow \frac{f(x_0)}{\delta f(x)}=\frac{1}{\delta}f(x)
    \end{equation*}
    contradicting the statement in \cref{e:idx}.
\end{proof}
\begin{definition}
    Let $\emptyset\neq A\subseteq X$.
    We say that $A$ is
    \begin{itemize}[nl]
        \item \textbf{convex} if for $a,b\in A$ and $0<\lambda<1$, $(1-\lambda)a+\lambda b\in A$.
        \item \textbf{absorbing} at $a_0\in A$ if for any $x\in X$, there is $\epsilon(a_0,x)>0$ such that $a_0+tx\in A$ for $0\leq t<\epsilon$.
    \end{itemize}
\end{definition}
For example, if $X$ is a normed space, then any open set is absorbing around any of its points.
\begin{lemma}[Minkowski Functional]
    Let $A\subset X$ be a convex set containing 0 and absorbing at 0.
    Define $p:X\to\R$ by $p(x)=\inf\{t>0:x\in tA\}$.
    Then $p$ is a sublinear functional.
    Moreover, we have that
    \begin{enumerate}[nl,r]
        \item $\{x\in X:p(x)<1\}\subseteq A\subseteq\{x\in X:p(x)\leq 1\}$; and
        \item if $X$ is normed and $A$ is a neighbourhood of $0$, then there is $N>0$ such that $p(x)\leq N\norm{x}$ for $x\in X$.
    \end{enumerate}
\end{lemma}
\begin{proof}
    First note, for any $x\in X$, if $A$ is absorbing at $0$, there is $s>0$ such that $sx\in A$, so $x\in \frac{1}{s} A$ and hence $0\leq p(x)<\infty$.

    Let's see non-negative homogeneity.
    Clearly $p(0)=0$.
    If $s>0$ and $x\in X$, then
    \begin{equation*}
        p(sx)=\inf\{t>0:sx\in tA\}=\inf\left\{t>0:x\in\frac{t}{s}A\right\}=s\cdot\inf\left\{\frac{t}{s}>0:x\in\frac{t}{s}\right\}=sp(x)
    \end{equation*}
    We also have subadditivity.
    First, note that if $s,t>0$ and $a,b\in A$, then
    \begin{equation*}
        sa+tb=(s+t)\left(\frac{s}{s+t}a+\frac{s}{s+t}b\right)\in(s+t)A\Longrightarrow sA+tA\subseteq(s+t)A
    \end{equation*}
    by convexity, and also $(s+t)A=\{(s+t)a:a\in A\}\subseteq\{sa+tb:a,b\in A\}=sA+tA$.
    Thus $sA+tA=(s+t)A$.
    Now for $x,y\in X$, we have
    \begin{align*}
        p(x)+p(y) &= \inf\{s>0:x\in sA\}+\inf\{t>0:y\in tA\}\\
                  &= \inf\{s+t:s>0,t>0,x\in sA,y\in tA\}\\
                  &\geq\inf\{s+t:s>0,t>0,x+y\in sA+tA=(s+t)A\}\\
                  &= \inf\{r>0:x+y\in rA\}=p(x+y)
    \end{align*}
    so that $p$ is a sublinear functional.
    Then
    \begin{enumerate}[nl,r]
        \item If $p(x)<1$, then there is $0<t<1$ so $x\in tA$; i.e. $\frac{1}{t}x\in A$ and $x=(1-t)=+t\frac{1}{t}x\in A$.
            The second inclusion is obvious.
        \item The assumptions provide $\delta>0$ so $\delta D(X)\subseteq A$.
            Then for $x\in X$ and $\epsilon>0$,
            \begin{equation*}
                x\in(\norm{x}+\epsilon)D(X)=\frac{\norm{x}+\epsilon}{\delta}\delta D(X)\subseteq\frac{\norm{x}+\epsilon}{\delta}A
            \end{equation*}
            so $p(x)\leq\frac{\norm{x}+\epsilon}{\delta}$ so $p(x)\leq\frac{1}{\delta}\norm{x}$; the result follows with $N=1/\delta$.
    \end{enumerate}
\end{proof}
\begin{theorem}[Hyperplane Separation]
    Let $X$ be an $\F-$vector space, $A,B\subset X$ be convex with $A\cap B=\emptyset$ and $A$ absorbing at some $a_0$.
    Then there are $f\in X'$ and $\alpha\in\R$ such that
    \begin{equation*}
        \Re f(a)\geq\alpha\geq\Re f(b)
    \end{equation*}
    for $a\in A$ and $b\in B$.
    Moreover, if $X$ is normed, then
    \begin{itemize}[nl]
        \item If $A$ is a neighbourhood of $a_0$, we have $f\in X^*$; and
        \item if $A$ is absorbing around each of its points (for example if $A$ is open), then we have $\Re f(a)>\alpha\geq\Re f(b)$.
    \end{itemize}
\end{theorem}
\begin{proof}
    We first re-centre at $0$.
    Let $A-B=\{a-b:a\in A,b\in B\}$.
    Then it is easy to verify that
    \begin{enumerate}[nl,r]
        \item $A-B$ is absorbing at any $a_0-b$, $b\in B$
        \item $A-B$ is convex
        \item if $X$ is normed and $A$ a neighbourhood of $a_0$, then $A-B$ is a neighbourhood of each $a_0-b$, $b\in B$; and if $A$ is absorbing around any of its points (resp. open), then $A_B$ is absorbing around any of its points (resp. open).
    \end{enumerate}
    Let $x_0=a_0-b_0$ for some $b_0\in V$, and set $C=x_0-(A-B)$, so we have $0=x_0-x_0\in C$.
    Then by the above points, $C$ is absorbing at 0, convex, and if $X$ is normed and $A$ a neighbourhood of $a_0$, then $C$ is a neighbourhood of $0$; and if $A$ is absorbing at any of its points (resp. $A$ is open), then $C$ is absorbing at each of its points (resp. open).

    Let $p$ be the Minkowski functional of $C$.
    Notice that since $A\cap B=\emptyset$, $0\notin A-B$ so $x_0\notin C$.
    Thus by (i) of the lemma, $p(x_0)>1$.

    Let us find $f$ and $\alpha$.
    Let $f_0:\R x_0\to\R$, by $f_0(sx)=sp(x_0)$.
    Hence $f_0$ is linear and $f_0\leq p|_{Rx_0}$, so by Hahn-Banach, get $f\in X_{\R}'$ such that $f\leq p$ on $X$.
    If $a\in A$ and $b\in B$, then $x_0-(a-b)\in C$, so by (i) of the lemma, since $p(x_0)\geq 1$, we have $f(x_0-(a-b))\leq p(x_0-(a-b))\leq 1$.
    Thus $f(x_0)+f(b)\leq 1+f(a)$ so in fact $f(b)\leq f(a)$.
    Thus there exists some $\alpha\in\R$ such that
    \begin{equation*}
        \sup\{f(b):b\in B\}\leq\alpha\leq\inf\{f(a):a\in A\}
    \end{equation*}
    If $\F=\R$, we are done; otherwise, we shall replace $f$ by $f_{\C}$

    For the remainder of the proof, we suppose $X$ is a normed space, and $A$ is a neighbourhood of $a_0$.
    Then part (ii) of the lemma provides $N>0$ so that $p(x)\leq N\norm{x}$.
    Then for $x\in X$, $f(x)\leq p(x)\leq N\norm{x}$ and $-f(x)=p(-x)\leq N\norm{-x}=N\norm{x}$ so $|f(x)|\leq N\norm{x}$, in other words that $\norm{f}\leq N$ and $f\in X^*$.
    If $A$ is absorbing around any of its points, then $f(a)>\alpha$ for any $a\in A$.
    Indeed, suppose $f(a)=\alpha$.
    Then there would be $t>0$ so $a+t(-x_0)\in A$.
    But then $\alpha\leq f(a-tx_0)=f(a)-tf(x_0)<\alpha$, a contradiction.
\end{proof}
\begin{definition}
    If $\emptyset\neq S\subset X$, then its \textbf{convex hull} is given by
    \begin{equation*}
        \conv(S) = \{\sum_{i=1}^n\lambda_jx_j:n\in\N,x_1,\ldots,x_n\in S\text{ and }\lambda_1,\ldots,\lambda_n\geq 0\text{ with }\sum_{j=1}^n\lambda_j=1\}
    \end{equation*}
    One can verify that $\conv(S)$ is in fact convex, and is the smallest convex set containing $S$, i.e.
    \begin{equation*}
        \conv(S)=\bigcap\{C:S\subseteq C\subseteq X,C\text{ convex}\}
    \end{equation*}
    If $X$ is normed, we let $\overline{\conv}(S)$ denote the \textbf{closed convex hull}, i.e. the closure of the convex hull.
\end{definition}
\begin{definition}
    A \textbf{half-space} of $X$ is any set of the form $H=\{x\in X:\Re f(x)\leq \alpha\}$ for some $f\in X'$, $\alpha\in\R$.
\end{definition}
If $X$ is normed, then the last proposition shows $H$ is closed if and only if $f$ is bounded.
\begin{theorem}
    If $X$ is a normed vector space and $\emptyset\neq S\subset X$, then $\overline{\conv}(S)=\cap\{H:S\subseteq H\subset X, H\text{ a closed half space}\}$.
\end{theorem}
\begin{proof}
    It is immediate that $\overline{\conv}(S)\subseteq\cap\{H:S\subseteq H\subset X,H\text{ a closed half-space}\}$.
    Thus suppose $x_0\notin \overline{\conv}(S)$.
    Then there is $\delta>0$ such that $(x_0+\delta D(X))\cap\overline{\conv}(S)=\emptyset$.
    Since $x_0+\delta D(X)$ is open and convex, hyperplace separation gives provides $f\in X^*$ and $\alpha\in\R$ so $\Re f(a)>\alpha\geq\Re f(b)$ for $a\in x_0+\delta D(X)$ and $b\in\overline{\conv}(S)$.
    Then $S\subset H=\{y\in X:\Re f(x)\leq\alpha\}$ but $x_0\notin H$.
\end{proof}
\section{Some Applications of Baire Category Theorem}
\begin{theorem}[Baire Category I]
    If $(X,d)$ is a complete metric space and $\{U_n\}_{n=1}^\infty$ is a countable collection of dense, open subsets, then $\bigcap_{n=1}^\infty U_n$ is dense in $X$.
\end{theorem}
\begin{definition}
    Let $(X,d)$ be a metric space.
    A subset $F\subset X$ is \textbf{nowhere dense} if $X\setminus F$ is dense in $X$; equivalently, $\overline{F}$ contains no non-trivlal open subsets.
    We say that a subset $M\subseteq X$ is \textbf{meagre} (1st category) if $M=\bigcup_{n=1}^\infty F_n$ and each $F_n$ is nowehere dense; and a set is \textbf{non-meagre} (2nd category) otherwise.
\end{definition}
\begin{theorem}[Baire Category II]
    Let $(X,d)$ be a complete metric space.
    Then a non-empty open $U\subseteq X$ is non-meagre.
\end{theorem}
\begin{proof}
    Suppose not, so $U=\bigcup_{n=1}^\infty F_n\subseteq\bigcup_{n=1}^\infty\overline{F}_n$, each $F_n$ (hence $\overline{F_n}$) nowhere dense.
    Then each $V_n=X\setminus\overline{F_n}$ is open and dense, and hence by BCT I, $G=\bigcap_{n=1}^\infty V_n$ is dense in $X$, and hence $U\cap G\neq\emptyset$, violating assumption
\end{proof}
\begin{theorem}[Banach-Steinhaus]
    Let $X,Y$ be normed spaces, $U\subseteq X$ be non-meagre, and $\mathcal{F}\subset\mathcal{B}(X,Y)$ be such that for each $x\in U$, $\sup\{\norm{Tx}:T\in\mathcal{F}\}<\infty$ (pointwise bounded).
    Then $\mathcal{F}$ is uniformly bounded, i.e. $\sup\{\norm{T}:T\in\mathcal{F}\}<\infty$.
\end{theorem}
\begin{proof}
    Let for each $n\in\N$
    \begin{equation*}
        F_n=\bigcap_{T\in\mathcal{F}} T^{-1}(nB(Y))=\{x\in X:\norm{Tx}\leq n\text{ for all }T\in\mathcal{F}\}
    \end{equation*}
    so each $F_n$ is closed and, by the pointwise boundedness assumption, $U\subseteq\bigcup_{n=1}^\infty F_n$.
    By assumption of non-meagreness of $U$, at least one $F_{n_0}$ admis an interior point: there is $x_0\in F_{n_0}$ and $\delta>0$ such that $x_0+\delta D(X)\subseteq F_{n_0}$.
    Then if $x\in D(X)$, we have
    \begin{equation*}
        Tx=\frac{1}{\delta}\left[T\left(x_0+\frac{\delta}{2}x\right)-T\left(x_0-\frac{\delta}{2}x\right)\right]
    \end{equation*}
    so $\norm{Tx}\leq\frac{2}{\delta}n_0$, in other words
    \begin{equation*}
        \norm{T}=\sup_{x\in D(x)}\norm{Tx}\leq\frac{2n_0}{\delta}<\infty
    \end{equation*}
    where the bound is independent of $T$.
\end{proof}
\begin{theorem}[Open Mapping]
    Let $X,Y$ be Banach spaces, and $T\in B(X,Y)$ surjective.
    Then $T$ is an open map; i.e. $T(U)$ is open in $Y$ whenver $U$ is open in $X$.
\end{theorem}
\begin{remark}
    Given $x\in X$ and $\alpha\in\F\setminus\{0\}$, non-empty $A\subset X$, we have that $\overline{x+\alpha A}=x+\alpha\overline{A}$.
    Indeed, note that for $(a_k)_{k=1}^\infty\subset A$, we have
    \begin{equation*}
        a_k\to a\in\overline{A}\text{ if and only if }x+\alpha a_k\to x+\alpha a\in x+\alpha\overline{A}
    \end{equation*}
\end{remark}
\begin{lemma}
    With the assumptions as above, we have that if $\overline{T(D(X)}\supset rB(Y)$ for some $r>0$, then $T(D(X))\supseteq rD(Y)$.
\end{lemma}
\begin{proof}
    Let $z\in rD(Y)$ and let $0<\delta<1$ be so $\norm{z}<r(1-\delta)<r$.
    Set $y=z/(1-\delta)$ so $\norm{y}<r/(1-\delta)$.
    It suffices to show that $y\in\frac{1}{1-\delta}T(D(X))$.
    To begin, let $A=T(D(X))\cap rB(Y)$, so $\overline{A}=rB(Y)$.
    Indeed, if $y\in rB(Y)\subseteq\overline{T(D(X))}$, then there is $(y_k)_{k=1}^\infty\subset\overline{T(D(X))}$, so $y=\lim y_k$.
    But then there is $x_k\in D(X)$ so each $\norm{y_k-T(x_k)}<1/k$ so $y=\lim T(x_k)$ with each $x_k\in D(X)$.

    Now we inductively build a sequence $(y_n)_{n=1}^\infty$ as follows.
    \begin{itemize}[nl]
        \item Since $y\in rD(Y)\subseteq\overline{A}$, there is $y_1\in A\cap(y+\delta rD(Y))$
        \item $y\in y_1+\delta r(D(Y))\subseteq y_1+\delta\overline{A}=\overline{y_1+\delta A}$, so there is $y_2\in(y_1+\delta A)\cap (y+\delta^2rD(Y))$
        \item $y\in y_n+\delta^n rD(Y)\subseteq\overline{y_n+\delta^nA}$, so there is $y_{n+1}\in(y_n+\delta^n A)\cap(y+\delta^{n+1}rD(Y))$
    \end{itemize}
    By construction, $y_{n+1}-y_n\in\delta^n A$, so $\norm{y_{n+1}-y_n}\leq\delta^n r$ and there is $x_n\in\delta^n D(X)$ such that $y_{n+1}-y_n=Tx_n$.
    Likewise, $y_1\in A\subseteq T(D(X))$ so $y=T(x_0)$ for some $x_0\in D(X)$.
    Notice that each $y_n\in y+\delta^nr\in D(Y)$, so $\norm{y_n-y}\leq\delta^n r\to 0$.
    Since $X$ is complete, we let $x=\sum_{n=0}^\infty x_n$, and by construction
    \begin{equation*}
        \norm{x}\leq\sum_{n=0}^\infty\norm{x_n}<\sum_{n=0}^\infty\delta^n=\frac{1}{1-\delta}
    \end{equation*}
    Then by linearity and continuity of $T$, we have
    \begin{equation*}
        Tx=\sum_{n=0}^\infty Tx_n=y_1+\sum_{n=1}^\infty(y_{n+1}-y_n)=y_N+\sum_{n=N}^\infty(y_{n+1}-y_n)\to y
    \end{equation*}
    so that indeed $T(x)=y$, as required.
\end{proof}
\begin{remark}
    So far, we've only used completeness of $X$ and continuity and linearity of $T$.
\end{remark}
We now proceed with the proof of the open mapping theorem.
\begin{proof}
    It suffices to see that $T(D(X))$ contains a neighbourhood of $0$ in $Y$.
    Indeed, if $\emptyset\neq U\subseteq X$ is open, $x\in U$, then there is $\delta>0$ such that $x+\delta D(X)\subseteq U$, so $U-x\supseteq\delta D(X)$.
    If $T(D(X))\supseteq rD(Y)$, then $T(U-x)\supseteq\delta T(D(X))\supseteq r\delta D(Y)$ so that $Tx+r\delta D(Y)\subseteq T(U)$.
    In other words, $T(U)$ is a neighbourhood of any of its points, and thus open.

    Now write $X=\bigcup_{n=1}^\infty nD(X)$, and we assume that $T(X)=Y$.
    Hence $Y=\bigcup_{n=1}^\infty nT(D(X))$, so $Y=\bigcup_{n=1}^\infty n\overline{T(D(X))}$.
    But $Y$ is complete, so by Baire category theorem, there is some $n$ so that $n\overline{T(D(X))}$ has non-empty interior.
    Since $nT(D(X))$ is convex and symmetric, and hence $n\overline{T(D(X))}$ is convex and symmetric as well.
    Thus if $y\in D(Y)$, then $y_0\pm\epsilon\in y_0+\epsilon D(Y)$ so
    \begin{equation*}
        \epsilon y=\frac{1}{2}\left[y_0+\epsilon y-(y_0-\epsilon y)\right]\in n\overline{T(D(X))}
    \end{equation*}
    and $\frac{\epsilon}{n}y\in\overline{T(D(X))}$, i.e. $\frac{\epsilon}{n} D(Y)\subseteq\overline{T(D(X))}$.
    Thus applying the main lemma, $\frac{\epsilon}{n}D(Y)\subseteq T(D(X))$.
\end{proof}
\begin{theorem}[Inverse Mapping]
    If $X,Y$ are Banach spaces and $T\in \mathcal{B}(X,Y)$ is invertible, $T^{-1}\in\mathcal{B}(Y,X)$
\end{theorem}
\begin{proof}
    Direct application of the open mapping theorem.
\end{proof}
Let $X,Y$ be normed spaces.
Then we define for $(x,y)\in X\oplus Y$, and we let $\norm{(x,y)}_1=\norm{x}+\norm{y}$.
It is easy to check that $\norm{\cdot}_1$ is a norm on $X\oplus Y$, and if $X,Y$ are Banach, then so is $(X\oplus Y,\norm{\cdot}_1)$.
In this case, we write $X\oplus_1 Y$.
\begin{theorem}[Closed Graph]
    Let $X,Y$ be Banach spaces and $T\in\mathcal{L}(X,Y)$.
    Then $T\in\mathcal{B}(X,Y)$ if and only if $\Gamma(T)=\{(x,Tx):x\in X\}$ is closed in $X\oplus_1Y$.
\end{theorem}
\begin{proof}
    Let $T\in\mathcal{B}(X,Y)$.
    If $(x_n)\to x$ in $X$, then $Tx_n\to Tx$ in $Y$.
    Thus if $(x,y)\in\overline{\Gamma(T)}$, then $(x,y)=\lim(x_n,Tx_n)$ where $(x_n,Tx_n)\in\Gamma(T)$.
    But then
    \begin{equation*}
        \norm{y-Tx}\leq \norm{y-Tx_n}+\norm{Tx_n-Tx}\leq\norm{x-x_n}+\norm{y-Tx_n}+\norm{Tx_n-tx}=\norm{(x-y)-(x_n,Tx_n)}_1
    \end{equation*}
    so in fact $y=Tx$ so $(x,y)=(x,Tx)$.

    Conversely, if $\Gamma(T)$ is closed in $X\oplus_1 Y$, then $\Gamma(T)$ is a Banach space.
    Define $S:\Gamma(T)\to X$ by $S(x,Tx)=x$.
    Notice that $S$ is linear, and
    \begin{equation*}
        \norm{S(x,Tx)}=\norm{x}\leq\norm{(x,Tx)}_1
    \end{equation*}
    so $\norm{S}\leq 1$, so $S$ is bounded.
    It is also clear that $S$ is bijective, with $S^{-1}:X\to\Gamma(T)$ given by $S^{-1}(x)=(x,Tx)$.
    Thus the inverse mapping theorem gives that $S^{-1}$ is also bounded.
    Hence for any $x\in X$,
    \begin{equation*}
        \norm{Tx}\leq\norm{(x,Tx)}_1=\norm{S^{-1}x}\leq\norm{x}\norm{S^{-1}}
    \end{equation*}
    so that $T$ is in fact bounded.
\end{proof}
\begin{theorem}[Closed graph test]
    Given normed spaces and $T\in\mathcal{L}(X,Y)$, we have that $\Gamma(T)$ is closed in $X\oplus_1 Y$ if and only if whenever $x_n\to 0$ for which we may assume that $Tx_n$ converges in $Y$, say $y=\lim Tx_n$, then $y=0$ too.
\end{theorem}
\begin{proof}
    We have $(x_n,Tx_n)\to(x,z)\in\overline{\Gamma(T)}$ if and only if $(x_n-x,T(x_n-x))\to(x,z)-(x,Tx)=(0,z-Tx)$.
    Set $y=z-Tx$.
    We have $(x,z)\in\Gamma(T)$ if and only if $z=Tx$ if and only if $y=0$.
\end{proof}
\subsection{Testing hypothesis of OMT}
\begin{enumerate}[nl,r]
    \item Let $1\leq p<r<\infty$.
        We have that $\ell_p\subseteq\ell_r$, with $\norm{x}_r\leq\norm{x}_p$ for $x\in\ell_p$.
        First, suppose $x\in B(\ell_p)$, so for each $k$, $|x_k|\leq\norm{x}_p\leq 1$ so $|x_k|^{r/p}\leq|x_k|$.
        Hence
        \begin{equation*}
            \norm{x}_r = \left(\sum_{k=1}^\infty|x_k|^r\right)^{1/r}\leq \left(\sum_{k=1}^\infty|x_k|^p\right)^{1/r}=\norm{x}_p^{p/r}\leq 1
        \end{equation*}
        so if $x\in\ell_p\setminus\{0\}$, then the result follows.

        Let $S:(\ell_p,\norm{\cdot}_p)\to(\ell_p,\norm{\cdot}_r)$ be the identity map.
        Then $\norm{S}\leq 1$, and furthermore $S$ is bijective.
        If $S$ were open, then by the proof of inverse mapping theorem, we would see that $\norm{S^{-1}}<\infty$.
        Define $x^{(n)}\in\ell_p$ by
        \begin{equation*}
            x_k^{(n)}=
            \begin{cases}
                \frac{1}{ck^{1/p}} & k\leq n\\
                0 & k>n
            \end{cases}
            ,
            c=\sum_{k=1}^\infty\frac{1}{k^{r/p}}
        \end{equation*}
        We compute that $\norm{x^{(n)}}_r<1$ while $\norm{x^{(n)}}_p=\frac{1}{c}\left(\sum_{k=1}^n\frac{1}{k}\right)^{1/p}$.
        In other words, $\norm{S^{-1}x^{(n)}}_p$ goes to infinity, while $\norm{x^{(n)}}_r<1$, contradicting $\norm{S^{-1}}<\infty$.
        The moral of this is that if the range space is not complete, then OMT may not hold.
    \item Take $X=C_b(0,1)$, $X_0=\{f\in X:f\text{ is diferentiable on }(0,1),f'\in C_b(0,1)\}$.
        We have $X_0\subseteq X$, and we put the uniform norm $\norm{\cdot}_\infty$ on both spaces.
        We let $D:X_0\to X$, $Df=f'$.
        If $h_n(t)=t^n$, then $\norm{h_n}_\infty=1$ while $\norm{Dh_n}_\infty=n$, so $D$ is not bounded.
        Despite this, we have that $\Gamma(D)=\{(f,f'):f\in X_0\}$ is closed in $X_0\oplus_1 X$.
        We apply the closed graph test: let $(f_n,f_n')\to (0,g)$ in $X_0\oplus_1 X$.
        Notice that $\norm{f_n'}_\infty<\infty$, so $f_n$ is Libschitz on $(0,1)$, so $f_n$ is uniformly continuous on $(0,1)$, so $f_n(0^+)=\lim_{t\to 0^+}f(t)$ exists.
        Thus by the fundamental theorem of calculus, $f_n(t)=f_n(0^+)+\int_0^t f_n'$ for $t\in(0,1)$.
        In particular,
        \begin{itemize}[nl]
            \item $f_n\to 0$ uniformly, so $f_n(0^+)\to 9$
            \item $f_n'\to g$ uniformly, so for each $t\in(0,1)$,
                \begin{equation*}
                    \int_0^tg=\lim_{n\to\infty}\int_0^tf_n'=\lim_{n\to\infty}[f_n(t)-f_n(0^+)]=0
                \end{equation*}
        \end{itemize}
        and again, by the FT of C, $g(t)=0$.
        Thus $g=0$, so $\Gamma(D)$ is closed.
        We say that $D:X_0\to X$ is a \textbf{closed} operator.
        The moral here is that if the domain is not complete, then closedness of the graph does not imply boundedness of the operator.

        Now, let $J:X\to X_0$ have $Jg(t)=\int_0^t g$ for $t\in(0,1)$.
        By the FT of C, $D\circ J(G)=g$, in other words that $D\circ J=I$.
        We have for $g\in X$,
        \begin{equation*}
            \norm{Jg}_\infty=\sup_{t\in(0,1)}|\int_0^tg|\leq\sup_{t\in(0,1)}t\norm{g}_\infty\leq\norm{g}_\infty
        \end{equation*}
        so $\norm{J}\leq 1$.
        Hence $J(D(X))\subseteq D(X_0)$, and we apply $D$ to see $D(X)\subseteq D(D(X_0))$, in other words, that $D$ is open.
        As an exercise, show that $C_b(0,1)=X$ is not separable, while $X_0$ is separable.
\end{enumerate}
Let $X\subsetneq Y$ be $\F-$vector spaces.
We can always find a subspace $Z\subset Y$ so $X+Z=Y$ and $X\cap Z=\{0\}$.
Indeed, let $B$ be a basis for $X$, and $B'=B\cup B'$ is a basis for $Y$, and take $Z=\spn B'$.
\begin{theorem}
    Let $Y$ be a Banach space and $X\subsetneq Y$ a closed subspace.
    Then $X$ admis a closed complement $Z$ if and only if there is some $P\in\mathcal{B}(Y)$ such that $P\circ P=P$ and $\im P=P(Y)=X$.
\end{theorem}
\begin{remark}
    We say that $X\subsetneq Y$ is \textbf{boundedly complemented} if either of the above conditions hold.
\end{remark}
\begin{proof}
    $(\Leftarrow)$ Let $Z=\ker P$, which is closed.
    If $y\in Y$, then $y=Py+(I-Py$ where $Py\in X$ and $P(I-P)y=0$ so $(I-P)y\in\ker P$.
    If $z\in Z\cap X$, then $z=Py$ for some $y\in Y$ so $Pz=P^2y=Py=z$, but $z\in\ker P$, so $z=Pz=0$.

    $(\Rightarrow)$ Let $S:X\oplus_1 Z\to Y$ be given by $S(x,z)=x+z$.
    Then $S$ is surjective and if $(x,z)\in\ker S$, then $x+z=0$ so $x=-z\in X\cap Z=\{0\}$, hence $S$ is injective.
    Furthermore,
    \begin{equation*}
        \norm{S(x+z)}=\norm{x+z}\leq\norm{(x,z)}_1
    \end{equation*}
    so $\norm{S}\leq 1$.
    Hence $S$ is a bounded bijection between Banach space and hence $S^{-1}$ is bounded by the inverse mapping theorem.
    Let $P_1:X\oplus_1 Z\to X$ be given by $P_1(x,z)=x$; and $J:X\to Y$ by $Jx=x$.
    Notice that $\norm{P_1}=1$ and $\norm{J}=1$.
    Define $P:Y\to Y$ by $Py=JP_1S^{-1}y$.
    Then
    \begin{itemize}[nl]
        \item $\im J=X$, and each of $P_1,S^{-1}$ are surjective, so $\im P=X$
        \item If $y\in Y$, $\norm{Py}=\norm{JP_1S^{-1}y}\leq\norm{S^{-1}}\norm{y}$ so $\norm{P}\leq\norm{S^{-1}}$
        \item Clearly $P^2=JP_1S^{-1}JP_1S^{-1}=P$
    \end{itemize}
\end{proof}
\begin{theorem}
    $c_0$ is not boundedly complemented in $\ell_\infty$.
\end{theorem}
\begin{proof}
    Let us assume otherwise; hence, there is $P=P^2\in\mathcal{B}(\ell_\infty)$ such that $\im P=c_0$.
    Note that $c_0=\ker(I-P)$.
    As in A2, we let $\mathcal{F}\subset\mathcal{P}(\N)$ be a family of infinite subsets such that for $E\neq F$ in $\mathcal{F}$, $|E\cap F|<\infty$ and $|\mathcal{F}|=\mathfrak{c}$.
    For each $F\in\mathcal{F}$, we let $y_F=(I_P)\chi_F\neq 0$.
    If $\alpha_1,\ldots,\alpha_n\in F$ are pairwise distinct, $F_1,\ldots,F_m\in\mathcal{F}$, then
    \begin{equation*}
        \sum_{i=1}^n\alpha_i\chi_{F_i}=\underbrace{\sum_{i=1}^m\alpha_i\chi_{F_i\setminus\bigcup_{j\in[m]\setminus\{i\}}F_j}}_{:= z}+\underbrace{\sum_{k=2}^m\sum_{1\leq i<\cdots<i_k\leq m}(\alpha_{i_1}+\cdots+\alpha_{i_k})\chi_{F_{i_1}\cap\cdots\cap F_{i_k}}}_{\in c_0}
    \end{equation*}
    where $\norm{z}_\infty=\max_{k=1,\ldots,m}|\alpha_k|$.
    Hence
    \begin{equation}\label{e:ab}
        \norm{\sum_{i=1}^m\alpha_iy_{F_i}}=\norm{(I-P)z}\leq\norm{I-P}\norm{z}=\norm{I-P}\max_{k=1,\ldots,m}|\alpha_k|
    \end{equation}
    Now, let for $n,k\in\N$, $\mathcal{F}_{n,k}=\{F\in\mathcal{F}:|\delta_k(y_F)|\geq\frac{1}{n}\}$m where $\delta_k(x_i)_{i=1}^\infty=x_k$, so $\delta_k\in\ell_\infty^*$ with $\norm{\delta_k}\leq 1$.
    Let $F_1,\ldots,F_m$ be pairwise disjoint in $\mathcal{F}_{n,k}$, and $\alpha_i=\overline{\sgn\delta_k(y_{F_i})}$.
    Then we have each $|\alpha_i|=1$, so by \cref{e:ab}, we find
    \begin{equation*}
        \norm{I-P}\geq\norm{\sum_{i=1}^\infty\alpha_iy_{F_i}}_\infty\geq|\delta_k\sum_{i=1}^n\alpha_iy_{F_i}|=\sum_{i=1}^m|\delta_k(y_{F_i})|\geq\frac{m}{n}
    \end{equation*}
    so $m\leq n\norm{I-P}$ and it folows that $\mathcal{F}_{n,k}$ is finite.
    Since each $y_F\neq 0$ for $F\in\mathcal{F}$, we see that $\mathcal{F}=\bigcup_{n=1}^\infty\bigcup_{k=1}^\infty$, which contradicts that $|\mathcal{F}|=\mathfrak{c}$.
    Hence such a $P$ must not exist.
\end{proof}
\begin{theorem}
    If $X$ is a finite dimensional vector space over $\F$, then any two norms are equivalent.
\end{theorem}
\begin{proof}
    Let $\norm{\cdot}$ be a norm on $X$.
    Fix a basis $(e_1,\ldots,e_n)$ for $X$, and let $x=\sum_{k=1}^nx_ke_k$, $x_i\in\F$, $\norm{x_k}_\infty=\max_{k=1,\ldots,n}|x_k|$.
    This is easily checked to be a norm.
    Moreover, $B_\infty=\{x\in X:\norm{x}_\infty\leq 1\}$ admits a homeomorphic identification
    \begin{equation*}
        B_\infty=
        \begin{cases}
            [-1,1]^n &\F=\R\\
            \overline{D}^n & \F=\C
        \end{cases}
    \end{equation*}
    and hence is compact.
    Thus $S_\infty=\{x\in X:\norm{x}_\infty=1\}$ is compact as well.
    Hence, for $x=\sum_{k=1}^\infty x_ke_k$, we have
    \begin{equation*}
        \norm{x}\leq\sum_{k=1}^n|x_k|\norm{e_k}\leq\norm{x}_\infty\underbrace{\sum_{k=1}^n\norm{e_k}}_{:=M}
    \end{equation*}
    Now for $x,y\in X$, we have $|\norm{x}-\norm{y}|\leq\norm{x-y}\leq M\norm{x-y}_\infty$ so $\norm{\cdot}$ is Lipschitz with respect to $\norm{\cdot}_\infty$, and hence $\tau_{\norm{\cdot}_\infty}-$continuous.
    Thus the extreme value theorem tells us that $m=\inf_{x\in S_\infty}\norm{x}>0$.
    Hence for $x\in X\setminus\{0\}$, $\norm{x}=\norm{x}_\infty\cdot\norm{\frac{1}{\norm{x}_\infty}x}\geq\norm{x}_\infty m$.
    In general, $m\norm{x}_\infty\leq\norm{x}\leq M\norm{x}_\infty$.
    We thus have that $\norm{\cdot}\sim\norm{\cdot}_\infty$, so any norms are equivalent.
\end{proof}
\begin{corollary}
    Let $(X,\norm{\cdot})$ be a finite dimensional normed space.
    Then
    \begin{enumerate}[nl,r]
        \item $K\subseteq X$ is compact if an only if $K$ is closed and bounded.
        \item $(X,\norm{\cdot})$ is a Banach space
        \item For any normed space $Y$, we have $\mathcal{L}(X,Y)=\mathcal{B}(X,Y)$
        \item We have $X'=X^*$.
    \end{enumerate}
\end{corollary}
\begin{proof}
    \begin{enumerate}[nl,r]
        \item The forward direction is immediate.
            If $K$ is closed and bounded, is contained in some scaled copy of $B_\infty$, which is compact.
        \item Cauchy sequences are bounded, and thus contained in some scaled copy of $B_\infty$, which is compact.
        \item Let $T\in\mathcal{L}(X,Y)$, and let $\norm{x}_0=\norm{x}+\norm{Tx}$.
            Then the result follows by equivalence of norms.
        \item Immediate.
    \end{enumerate}
\end{proof}
\begin{proposition}
    A finite dimensional subspace of normed space is always closed and boundedly complemented.
\end{proposition}
\begin{proof}
    Let $Y\subseteq X$ be so $Y$ is finite dimensional and $X$ a normed space.
    We can find a basis $(e_1,\ldots,e_n)$ for $Y$.
    We may assume that each $\norm{e_k}=1$.
    We define $f_1,\ldots,f_n\in Y'=Y^*$ by
    \begin{equation*}
        f_k\left(\sum_{j=1}^n\alpha_je_j\right)=\alpha_k
    \end{equation*}
    By Hahn-Banach, get $F_1,\ldots,F_n\in X^*$ such that $F_k|_Y=f_k$ and $\norm{F_k}=\norm{f_k}$.
    Define $P:X\to X$ by $Px=\sum_{k=1}^n F_k(x)e_k$.
    Notice that $\im P\subseteq Y$ and by choice of $F_k|_Y=f_k$, we have $P|_Y=I_Y$.
    Thus $P^2=P$.
    Finally, for $x\in X$, $\norm{Px}\leq\sum_{k=1}^n\norm{f_k}\norm{x}$ so $\norm{P}\leq\sum\norm{f_k}<\infty$, i.e. $P$ is bounded.
    Closedness of $Y$ thus follows from the last corollary.
    Alternatively, $Y=\ker(I-P)$.
\end{proof}
\section{On Compactness of the Unit Ball}
\begin{lemma}
    Let $X$ be a normed space and $Y\subsetneq X$ a closed subspace.
    Then given $\epsilon\in(0,1)$ there is $x_0\in D(X)\subseteq B(X)$ such that $d(x_0,Y)>1-\epsilon$.
\end{lemma}
\begin{proof}
    Let $x\in X\setminus Y$ and let $f:Y+\F x\to\F$ be given by $f(y+\alpha x)=\alpha$, $y\in Y$, $\alpha\in \F$.
    Then $f$ is linear and $\ker f=Y$ is closed, $Y\subsetneq Y+\F x$, so $f$ is bounded.
    Let $F\in X^*$ be any Hahn-Banach extension of $f$ with $\norm{F}=\norm{f}$.

    Now, we find $x_0\in D(X)$ such that $|F(x_0)|>(1-\epsilon)\norm{F}$.
    Since $Y\subseteq\ker F$, we have for $y\in Y$ that $\norm{F}\norm{x_0-y}\geq|f(x_0-y)|=|F(x_0)|>(1-\epsilon)\norm{F}$, so $\norm{x_0-y}>1-\epsilon$.
    Hence $d(x_0,Y)=\inf_{y\in Y}\norm{x_0-y}\geq 1-\epsilon$.
\end{proof}
\begin{theorem}
    Let $X$ be a normed space.
    Then $B(X)$ is compact if and only if $X$ is finite dimensional.
\end{theorem}
\begin{proof}
    The reverse implication is standard.
    Thus suppose $X$ is not finite dimensional.
    Let $\epsilon\in(0,1)$ and let $x_1\in B(X)\setminus\{0\}$.
    Inductively,
    \begin{itemize}[nl]
        \item Find $x_2\in B(X)$ such that $\dist(x_2,\F x_1)\geq 1-\epsilon$
        \item Find $x_3\in B(X)$ such that $\dist(x_3,\spn\{x_1,x_2\})\geq 1-\epsilon$
        \item Find $x_{n+1}\in B(X)$ such that $\dist(x_{n+1},\spn\{x_1,\ldots,x_n\})\geq 1-\epsilon$
    \end{itemize}
    Hence we have $\{x_n\}_{n=1}^\infty\subset B(X)$ such that for $m<n$,
    \begin{equation*}
        \norm{x_n-x_m}\geq d(x_n,\spn\{x_1,\ldots,x_{n-1}\})\geq 1-\epsilon
    \end{equation*}
    so the sequence admis no converging subsequence and $B(X)$ is not compact.
\end{proof}
\section{More Topology}
\begin{definition}
    Let $(X,\tau)$ be a topological space.
    A \textbf{base} for $\tau$ is any family $\beta\subseteq\tau$ such that for any $U\in\tau$ and $x\in U$, there is $B\in\beta$ such that $x\in B\subseteq U$.
    A \textbf{subbase} for $\tau$ is any family $\alpha\subseteq\tau$ such that $\{\bigcap_{k=1}^n U_k:n\in\N,U_1,\ldots,U_n\in\alpha\}$ is a base for $\tau$.
\end{definition}
Note that if $\emptyset\neq X$ and $\beta\subseteq\mathcal{P}(X)$ for which $\bigcup_{B\in \beta}B=X$ and $\beta$ is closed under finite intersections, then
\begin{equation*}
    \tau_\beta=\{\bigcup_{i\in I}B_i:\{B_i\}_{i\in I}\subset B,I\text{ any index set with }|I|\leq|\beta|\}
\end{equation*}
is a topology.
\begin{definition}
    Let $X\neq\emptyset$.
    Suppose we are given
    \begin{itemize}[nl]
        \item a family $\{(X_\alpha,\tau_\alpha)\}_{\alpha\in A}$ of topological spaces, and
        \item for each $\alpha\in A$, a function $f_\alpha:X\to X_\alpha$
    \end{itemize}
    Then the \textbf{initial topology} on $X$ given this data is denoted
    \begin{equation*}
        \sigma=\sigma(X,(f_\alpha)_{\alpha\in A})=\sigma(X,(f_\alpha,\tau_\alpha)_{\alpha\in A})
    \end{equation*}
    and is the topology with base
    \begin{equation*}
        \bigcap_{k=1}^n f_{\alpha_k}^{-1}(U_{\alpha_k}),n\in\N,\alpha_1,\ldots,\alpha_n\in A,\text{ each }U_{\alpha_k}\in\tau_{\alpha_k}
    \end{equation*}
\end{definition}
In particular, $\{f_\alpha^{-1}(U_\alpha):U_\alpha\in\tau_\alpha,\alpha\in A\}$ is a subbase for $\sigma$.
\begin{remark}
    The topology is chosen so that each $f_\alpha:X\to X_\alpha$ is $\sigma-\tau_\alpha-$continuous.
    Furthermore, if $\tau\subseteq\mathcal{P}(X)$ is any topology for which every $f_\alpha$ is $\sigma-\tau_\alpha-$continuous, then $\sigma\subseteq\tau$.
    We say that $\sigma$ is the \textbf{coarsest} topology so that all the $f_\alpha$ are continuous.
\end{remark}
\begin{example}
    \begin{enumerate}[nl,r]
        \item \textit{Metric topology:} If $(X,d)$ is a metric space, for each $x\in X$, let $d_x$ be given by $d_x(x')=d(x,x')$.
            Then $\sigma(X,(d_x)_{x\in X})=\tau_d$.
        \item \textit{Relative topology:} If $(Y,\tau)-$topological space, $\emptyset\neq X\subseteq Y$, and $i:X\to Y$ is the inclusion map.
            Then $\tau|_X=\sigma(X,\{i\})$.
        \item \textit{Product topology:} Let $\{(X_\alpha,\tau_\alpha)\}_{\alpha\in A}$ be a family of topological spaces.
            Let $X=\prod_{\alpha\in A}X_\alpha$.
            Let for $\alpha\in A$, $p_\alpha:X\to X_\alpha$ denote the projection map onto the component $\alpha$.
            Then the product topology $\pi=\sigma(X,\{p_\alpha\}_{\alpha\in A})$.
            Hence, $V\in\mathcal{P}(X)$, then $V\in\pi$ if and only if for any $x\in V$, there is $\alpha_1,\ldots,\alpha_n\in A$ and $U_{\alpha_k}\in\tau_{\alpha_k}$ such that $x_{\alpha_k}=p_{\alpha_k(x)}\in U_{\alpha_k}$ and $x\in \bigcap_{k=1}^n p_{\alpha_k}^{-1}(U_{\alpha_j})\subseteq V$.

            Note that if $X=\prod_{n=1}^\infty X_n$, each $(X_n,\tau_n)$ is a topological space, then the basic open sets look like $U_1\times U_2\times\cdots\times U_m\times X_{m+1}\times X_{m+2}\times\cdots$.
        \item \textit{Linear topology:} Let $X$ be a vector space and $Z\subseteq X'$ a subspace.
            Then $\sigma(X,Z)$ is the coarsest topology allowing each $f\in Z$ to be continuous, $f:X\to\F$.
            The basic open sets are given as follows: let $x_0\in X$, $\epsilon>0$, and $D=D(\F)$, and we consider for $f\in Z$
            \begin{equation*}
                f^{-1}(f(x_0)+\epsilon D)=\underbrace{\{x\in X:|f(x)-f(x_0)|<\epsilon\}}_{\text{"affine hypertube"}}=\{x\in X:|\frac{1}{\epsilon}f(x)-\frac{1}{\epsilon}f(x_0)|<1\}
            \end{equation*}
            so that
            \begin{equation*}
                \left\{\bigcap_{k=1}^n\{x\in X:|f_k(x)-f_k(x_0)|<1\}:f_1,\ldots,f_n\in Z,n\in\N\right\}
            \end{equation*}
            is a base for $\sigma(X,Z)$.
        \item Now let $X$ be a normed space.
            Then the \textbf{weak topology} on $X$ is $\omega=\sigma(X,X^*)$.
            Certainly $\omega\subseteq\tau_{\norm{\cdot}}$.
            Similarly, the \textbf{weak*-topology} on $X^*$ is $\omega^*=\sigma(X^*,\hat X)$ (recall for $x\in X$, $\hat x(f)=f(x)$).
            Since $\hat X\subseteq X^{**}$, we have $\omega^*\subseteq\omega=\sigma(X^*,X^{**})\subseteq\tau_{\norm{\cdot}}$.
    \end{enumerate}
\end{example}
Let $(X,\tau)$ be a topological space.
\begin{definition}
    A subset $K\subseteq X$ is called \textbf{compact} if for any collection $\{U_\alpha\}_{\alpha\in A}\subseteq\tau$ with $\bigcup_{\alpha\in A}U_\alpha\supseteq K$, there exists some finite $U_1,\ldots,U_n$ covering $K$.
    If $X$ itself is $\tau-$compact, we call $(X,\tau)$ a compact space.
\end{definition}
\begin{definition}
    A set $F\subseteq X$ is \textbf{closed} if $X\setminus F\in\tau$.
    If $S\subseteq X$, then the \textbf{closure} of $S$ is $\overline{S}=\cap\{F\subseteq X:S\subseteq F,X\setminus F\in\tau\}$.
\end{definition}
Note that $\overline{S}=\{x\in X:\text{for any }U\in\tau\text{ with }x\in U,U\cap S\neq\emptyset\}$.
\begin{definition}
    A family $\mathcal{F}\subseteq\mathcal{P}(X)$ has the \textbf{finite intersection property} if for any $F_1,\ldots,F_n\in\mathcal{F}$, $\bigcap_{l=1}^n F_k\neq\emptyset$.
\end{definition}
\begin{proposition}
    Let $(X,\tau)$ be a topological space.
    Then $(X,\tau)$ is compact if and only if any $\mathcal{F}\subseteq\mathcal{P}(X)$ with the finite intersection property has $\bigcap_{F\in\mathcal{F}}\overline{F}\neq\emptyset$.
\end{proposition}
\begin{proof}
    Suppose $X$ is compact and $\mathcal{F}\subset\mathcal{P}(X)$ has the finite intersection property but with $\bigcap_{F\in\mathcal{F}}\overline{F}$, then $\{X\setminus\overline{F}\}_{F\in\mathcal{F}}$ is an open cover of $X$ with no finite subcover.

    Conversely, if $\mathcal{O}\subseteq\tau$ is an open cover of $X$, then $\mathcal{F}=\{X\setminus U\}_{U\in\mathcal{O}}$ satisfies $\bigcap_{F\in\mathcal{F}}=\emptyset$, so there is $F_1,\ldots,F_n\in\mathcal{F}$ with $\bigcap_{k=1}^n F_k=\emptyset$.
    Then $\{X\setminus F_i\}_{i=1}^k$ is a finite subcover.
\end{proof}
\begin{definition}
    Let $X$ be a non-empty set.
    An \textbf{ultrafilter} is a family $\mathcal{U}\subset\mathcal{P}(X)$ such that
    \begin{itemize}[nl]
        \item $\mathcal{U}$ has the finite intersection property
        \item If $A\in\mathcal{P}(X)$, then either $A\in\mathcal{U}$ or $X\setminus A\in\mathcal{U}$.
    \end{itemize}
\end{definition}
\begin{example}
    \begin{enumerate}[nl,r]
        \item \textit{Principal / trivial ultrafilter:} If $x_0\in X$, let $U_{x_0}=\{U\subseteq X:x_0\in U\}$.
    \end{enumerate}
\end{example}
\begin{lemma}[Ultrafilter]
    If $\mathcal{F}\subseteq\mathcal{P}(X)$ is any set with the finite intersection property, then there is an ultrafilter $\mathcal{U}$ with $\mathcal{F}\subset\mathcal{U}$.
\end{lemma}
\begin{proof}
    Let $\Phi=\{\mathcal{G}\subseteq\mathcal{P}(X):\mathcal{F}\subseteq\mathcal{G},\mathcal{G}\text{ has f.i.p.}\}$.
    Then $\Phi$ is partially ordered by inclusion.
    If $\Gamma\subseteq\Phi$ is a chain, then $\mathcal{G}_\Phi=\bigcup_{\mathcal{G}\in\Gamma}\mathcal{G}$ contains $\mathcal{F}$ and has the finite intersection property.
    Hence $\Phi$ admits a maximal element $\mathcal{U}$.
    Let $A\in\mathcal{P}(X)\setminus\mathcal{U}$.
    Then $U\cup\{A\}\supsetneq\mathcal{U}$, so $\mathcal{U}\cup\{A\}$ fails the finite intersection property.
    Hence get $U_1,\ldots,U_n$ so $A\cap\bigcap_{k=1}^n U_k=\emptyset$.
    Now if $V_1,\ldots,V_m\in\mathcal{U}$, then $\bigcap_{j=1}^n V_j\cap\bigcap_{k=1}^n U_j\subseteq\bigcap_{k=1}^n U_k\subseteq X\setminus A$, so $(X\setminus A)\cap\bigcap_{j=1}^m V_j$.
    Thus $\mathcal{U}\cup\{X\setminus A\}$ has finite intersection property, so $X\setminus A\in\mathcal{U}$ by maximality.
\end{proof}
\begin{corollary}
    If $U\subseteq\mathcal{P}(X)$ is an ultrafilter, then
    \begin{enumerate}[nl,r]
        \item If $A\in\mathcal{P}(X)$, $A\in\mathcal{U}$ if and only if $A\cap U\neq\emptyset$ for each $U\in\mathcal{U}$
        \item If $A,B\in\mathcal{P}(X)$, then $A\cup B\in\mathcal{U}$ implies at least one of $A$ or $B$ is in $\mathcal{U}$
        \item If $A\in\mathcal{U}$ and $A\subseteq V$ implies $V\in\mathcal{U}$
    \end{enumerate}
\end{corollary}
\begin{proof}
    The forward implication of (i) follows since $\mathcal{U}$ has finite intersection.
    Conversely, $X\setminus A\notin\mathcal{U}$, so $A\in\mathcal{U}$.
    (ii) and (iii) follow consequently.
\end{proof}
\begin{corollary}
    If $X$ is an infinite set, it admits a non-principle ultrafilter.
\end{corollary}
\begin{proof}
    Let $\mathcal{F}=\{F\in\mathcal{P}(X):X\setminus F\text{ is finite}\}$.
    Then $\mathcal{F}$ has the finite intersection property.
    Apply the lemma.
\end{proof}
\begin{proposition}
    There are at least $\mathfrak{c}$ many ultrafilters in $\mathcal{P}(\N)$.
\end{proposition}
\begin{proof}
    We let $\mathcal{F}\subset\mathcal{P}(\N)$ be a collection of infinite sets such that $E\neq F$ in $\mathcal{F}$ implies $|E\cap F|<\infty$, and $|\mathcal{F}|=\mathfrak{c}$.
    For each $F\in\mathcal{F}$, we let $\mathcal{F}_F=\mathcal{F}_0\cup\{F\}$, which has the finite intersection property.
    Moreover, if $E\in\mathcal{F}\setminus\{F\}$, then $\mathcal{F}_F\cup\{E\}$ would fail f.i.p.
    Hence, for $F\in\mathcal{F}$, let $\mathcal{U}_F$ be any ultrafilter containing $\mathcal{F}_F$, giving $\mathfrak{c}$ many ultrafilters.
\end{proof}
\begin{remark}
    It can be shown (with a lot more work) that $\N$ admits $2^\mathfrak{c}$ ultrafilters.
\end{remark}
Let $\mathcal{U}\subset\mathcal{P}(\N)$ be a non-principal ultrafilter.
Define $\delta_{\mathcal{U}}:\mathcal{P}(\N)\to\{0,1\}\subset\R$ by $\delta_{\mathcal{U}}(A)=1$ if $A\in\mathcal{U}$, and $0$ if $X\setminus A\in\mathcal{U}$.
Since $\N\in\mathcal{U}$, we see that $\delta_{\mathcal{U}}(\emptyset)=0$.
If $\emptyset\neq A,B\in\mathcal{P}(\N)$ with $A\cap B=\emptyset$, then if $A\cup B\in\mathcal{U}$, then exactly one of $A$ or $B$ is in $\mathcal{U}$.
Thus $\delta_U(A\cup B)=\delta_U(A)+\delta_U(B)$.
If $E_1,\ldots,E_n\subseteq\N$ with $E_j\cap E_k=\emptyset$ for $j\neq k$, then $\sum_{k=1}^n|\delta_{\mathcal{U}}(E_k)|\leq 1$ so $\norm{\delta_{\mathcal{U}}}_{\text{var}}\leq 1$.
Since $\delta_{\mathcal{U}}(\N)=1$, we have $\norm{\delta_{\mathcal{U}}}_{\text{var}}=1$.
Let $L_{\mathcal{U}}\in\ell_\infty^*$ be the linear functional associated to $\delta_{\mathcal{U}}$.
We then have (with some verification possibly needed)
\begin{enumerate}[nl,r]
    \item $L_{\mathcal{U}}(1)=1$, $\norm{L_{\mathcal{U}}}=1$
    \item $L_{\mathcal{U}}|_{\mathbf{c_0}}=0$, so if $x\in\ell_\infty^{\R}$, then $\liminf_{n\to\infty}x_n\leq L_{\mathcal{U}}\leq\limsup_{n\to\infty}x_n$
    \item Exactly one of $2\N$ and $2\N-1$ is in $\mathcal{U}$, so $L(\chi_{2\N})\neq L_{\mathcal{U}}(\chi_{2\N-1})$, so $L_{\mathcal{U}}$ is not translation invariant.
    \item Let $S\in\mathcal{B}(\ell_\infty)$ be given by $Sx=\left(\frac{x_1+\cdots+x_n}{n}\right)_{n=1}^\infty$.
        Then $L_{\mathcal{U}}\circ S$ is a Banach limit.
\end{enumerate}
\begin{definition}
    If $(X,\tau)$ is a topological space, $\mathcal{U}$ an ultrafilter on $X$, we say that $x_0\in X$ is a $(\tau-)$limit point for $\mathcal{U}$ if for each $U\in\tau$ with $x_0\in U$, we have $U\in\mathcal{U}$.
\end{definition}
\begin{proposition}
    Let $(X,\tau)$ be a topological space.
    Then $(X,\tau)$ is compact if and only if any ultrafilter on $X$ admits a $\tau-$limit point.
\end{proposition}
\begin{proof}
    Let us begin with an observation: if $x\in X$ and $\mathcal{U}$ is an ultrafilter on $X$, then
    \begin{align*}
        x\in\bigcap_{V\in\mathcal{U}}\overline{V} &\Leftrightarrow\text{for any }U\in\tau\text{ with }x\in U, U\cap V\neq\emptyset\text{ for each }V\in \mathcal{U}\\
                                                   &\Leftrightarrow x\text{ is a $\tau-$limit point of }\mathcal{U}
    \end{align*}
    
    If $(X,\tau)$ is compact, then $\bigcap_{V\in\mathcal{U}}\overline{V}\neq\emptyset$.
    If $\mathcal{F}\subseteq\mathcal{P}(X)$ has the finite intersection property, then there exists an ultrafilter $\mathcal{U}\supseteq\mathcal{F}$, so $\bigcap_{F\in\mathcal{F}}\overline{F}\supseteq\bigcap_{V\in\mathcal{U}}\overline{V}\neq\emptyset$.

\end{proof}
\begin{theorem}[Tychonoff]
    Let $\{(X_\alpha,\tau_\alpha)\}_{\alpha\in A}$ be a family of compact spaces, and $X=\prod_{\alpha \in A}X_\alpha$ with the product topology $\pi$.
    Then $(X,\pi)$ is compact.
\end{theorem}
\begin{proof}
    Let $\mathcal{U}$ be an ultrafilter on $X$; we will show that it admits a $\pi-$limit point.
    Fix $\alpha\in A$ and let $\mathcal{U}_\alpha=\{p_\alpha(V):V\in\mathcal{U}\}$, where $p_\alpha$ is the coordinate projection onto $\alpha$.
    If $\emptyset\neq S_\alpha\subseteq X_\alpha$, then $S_\alpha=p_\alpha^{-1}(p_\alpha^{-1}(S_\alpha))$, so $S_\alpha\in\mathcal{U}_\alpha$ if and only if $p^{-1}(S_\alpha)\in\mathcal{U}$, and since $p^{-1}$ commutes with complementation, $\mathcal{U}_\alpha$ is an ultrafilter.
    The last proposition provides a $\tau_\alpha-$limit point $x_\alpha$ for $\mathcal{U}_\alpha$.
    Now let $x=(x_\alpha)_{\alpha\in A}$, where $x_\alpha$ is found as above.
    If $W\in\pi$ with $x\in W$, then there are $\alpha_1,\ldots,\alpha_n$ in $A$, $U_{\alpha_i}\in\tau_{\alpha_i}$ with $x\in\bigcap_{k=1}^n p_{\alpha_k}^{-1}(U_{\alpha_k})\subseteq W$.
    Since each $x_{\alpha_k}$ is a $\tau_{\alpha_k}-$limit point of $\mathcal{U}_{\alpha_k}$, we see that each $U_{\alpha_k}\in\mathcal{U}_{\alpha_k}$, so $p_{\alpha_k}^{-1}(U_{\alpha_k})\in\mathcal{U}$.
    Thus we see that $W\in\mathcal{U}$, so $x$ is a $\pi-$limit point of $\mathcal{U}$.
\end{proof}
\begin{remark}
    \begin{enumerate}[r]
        \item Tychonoff's theorem implies the axiom of choice.
            Given $\{X_\alpha\}_{\alpha\in A}$ be a family of non-empty sets.
            Find $y$ which is not a member of any $X_\alpha$, and let $Y_\alpha=X_\alpha\cup\{y\}$ and $\tau_\alpha=\{\emptyset,\{y\},X_\alpha,Y_\alpha\}$, and $(Y_\alpha.\tau_\alpha)$ is compact.
            The constant element $y$ is an element of $Y$, so by Tychonoff, $(Y,\pi)$ is compact.
            Given $\alpha_1,\ldots,\alpha_n\in A$, then $\bigcup_{k=1}^np_{\alpha_k}^{-1}(\{y\})$.
            Since $\prod_{k=1}^n X_{\alpha_k}\neq 0$, we see that $Y\subsetneq\bigcup_{k=1}^np_{\alpha_k}^{-1}(\{y\})$.
            Hence by compactness, $Y\not\subseteq\bigcup_{\alpha\in A}p_\alpha^{-1}(\{y\})$.
            Hence $\prod_{x\in A}X_\alpha=Y\setminus\bigcup_{\alpha\in A}p_\alpha^{-1}(\{y\})\neq 0$.
        \item If we are given $(X_\alpha,\tau_\alpha)_{\alpha\in A}$ a family of topological spaces, $X=\prod_{\alpha\in A}X_\alpha$, we can define the \textbf{box topology}, i.e. the topology with base $\left\{\prod_{\alpha\in A}U_\alpha:U_\alpha\in\tau_\alpha\setminus\{\emptyset\}\text{ for each }\alpha\right\}$
            Of course, $\pi\subseteq\tau$, and the inclusion is proper on infinite products.
    \end{enumerate}
\end{remark}
\begin{proposition}
    Let $(X,\tau)$ be a compact space.
    \begin{enumerate}[nl,r]
        \item If $K\subseteq X$ is closed, then $K$ is compact.
        \item If $(Y,\sigma)$ is a topological space and $f:X\to Y$ is continuous, then $f(X)$ is compact.
    \end{enumerate}
\end{proposition}
\begin{proof}
    Immediate.
\end{proof}
\begin{remark}
    If $X$ is a normed space, $w^*=\sigma(X^*,\hat X)$, if $x\in X$, $\hat x\in X^{**}$, $\hat x(f)=f(x)$, $\hat X=\{\hat x:x\in X\}$.
    If $A,B$ are non-empty sets, $A^B\cong \{f:B\to A\}$.
\end{remark}
\begin{theorem}[Alaoglu]
    Let $X$ be a normed space.
    Then $B(X^*)$ is $w^*=\sigma(X^*,\hat X)-$compact
\end{theorem}
\begin{proof}
    Let $\Gamma:X^*\to\F^X$ be given by $\Gamma(f)=(f(x))_{x\in X}$, so $\Gamma$ is injective.
    Let $\pi=\sigma(\F^X,\{p_x\}_{x\in X})$ be the product topology.
    If $U_1,\ldots,U_n\subseteq\F$ are open and $x_1,\ldots,x_n\in X$, then
    \begin{equation*}
        \Gamma\left(\bigcap_{k=1}^n\hat x_n^{-1}(U_k)\right)=\bigcap_{k=1}^n\Gamma\left(\hat x^{-1}_n(U_k)\right)=\bigcap_{k=1}^n \hat x_n^{-1}(U_k)\cap\Gamma(X^*)
    \end{equation*}
    so $\Gamma$ is an open map onto its image in $\F^X$.
    Similarly, it is easy to show that $\Gamma^{-1}$ is also an open map, so in fact $\Gamma$ is a homeomorphism onto its image.

    We now consider $\overline{\Gamma(B(X^*))}\subset\F^X$.
    Let $g\in \overline{\Gamma(B(X^*))}$ and let $D=D(\F)$.
    Given $x,y\in X$ and $\alpha\in\F$, and then given $\epsilon>0$, we find $f\in B(X^*)$ such that
    \begin{equation*}
        \Gamma(f)\in p_x^{-1}\left(g(x)+\frac{\epsilon}{3}D\right)\cap p_y^{-1}\left(g(y)+\frac{\epsilon}{3(|\alpha|+1)}D\right)\cap p_{x+\alpha y}^{-1}\left(g(x+\alpha y)+\frac{\epsilon}{3}D\right)
    \end{equation*}
    We have that $f$ is linear with $\Gamma(f)(x)=f(x)$, etc. so we have
    \begin{equation*}
        |g(x)+\alpha g(y)-g(x+\alpha y)|\leq|g(x)-f(x)|+|\alpha||g(y)-f(y)|+|g(x+\alpha y)-f(x+\alpha y)|<\epsilon
    \end{equation*}
    and since $\norm{f}\leq 1$, we have $|g(x)|\leq|g(x)-f(x)|+|f(x)|<\epsilon/3+\norm{x}$.
    Then since $\epsilon>0$ is arbitrary, get $g\in X'$ and $|g(x)|\geq\norm{x}$, i.e. $g\in B(X^*)$.
    Hence we have that $g=\Gamma(g)$.

    Thus $\Gamma(B(X^*))\subseteq\prod_{x\in X}\norm{x}\overline{D}\subseteq\F^X$ is a closed subset of a compact subset of $\F^X$.
    Thus $B(X^*)$ is the continuous image of a compact set and hence compact.
\end{proof}
\begin{remark}
    If $r>0$, then we may replace $B(X^*)$ with $rB(X^*)$ in the proof above, with trivial modifications.
    Thus any ball is $w^*-$compact.
    Hence bounded $w^*-$closed sets in $X^*$ are automatically $w^*-$compact.
\end{remark}
\begin{definition}
    A topological space $(X,\tau)$ is Hausdorff if given $x\neq y$ in $X$, there are $U_x,V_y\in\tau$ such that $x\in U_x$ and $y\in V_y$ and $U_x\cap U_y=\emptyset$.
\end{definition}
\begin{example}
    \begin{enumerate}[nl,r]
        \item A metric space is Hausdorff.
        \item $X$ a normed space, $w=\sigma(X,X^*)$ is Hausdorff (by Hahn-Banach and A2Q1).
        \item If $X$ is a normed space, then $w^*=\sigma(X^*,\hat X)$ on $X^*$ is Hausdorff.
        \item $\{(X_\alpha,\tau_\alpha)\}_{\alpha\in A}$ family of topological spaces, $X=\prod_{\alpha\in A}X_\alpha$ with $\pi$ the product topology.
            Then $(X,\pi)$ is Hausdorff if and only if all $(X_\alpha,\tau_\alpha)$ are Hausdorff.
            (Straightfoward exercise).
    \end{enumerate}
\end{example}
\begin{proposition}
    Let $(X,\tau)$ be a Hausdorff space, $K\subseteq X$ $\tau-$compact.
    Then $K$ is $\tau-$closed.
\end{proposition}
\begin{proof}
    Straightforward exercise.
\end{proof}
\begin{proposition}
    Let $(X,\tau)$ be a compact space.
    \begin{enumerate}[nl,r]
        \item If $(Y,\sigma)$ is a Hausdorff space and $\phi:X\to Y$ is continuous and bijective, then $\phi^{-1}:Y\to X$ is continuous.
        \item If $\tau'\subseteq\tau$ is a Hausdorff topology on $X$, so $\tau'=\tau$.
    \end{enumerate}
\end{proposition}
\begin{proof}
    \begin{enumerate}[nl,r]
        \item If $F\subseteq X$ is $\tau-$closed, then it is $\tau-$compact.
            Hence $(\phi^{-1})^{-1}(F)=\phi(F)$ is $\sigma-$closed, so by A1Q1, $\phi^{-1}$ is continuous.
        \item $\id:X\to X$ is continuous, so if $U\in\tau'$, then $\id^{-1}(U)=U\in\tau$, so $\id$ is continuous.
            Hence by (1) $\id^{-1}$ is continuous so $\tau\subseteq\tau'$.
    \end{enumerate}
\end{proof}
\begin{theorem}[Metrization]
    If $X$ is a separable normed space, then $B(X^*)$ is $w^*-$metrizable, i.e. there exists a metric $\rho$ on $B(X^*)$ such that $w^*|_{B(X^*)}=\tau_\rho$.
\end{theorem}
\begin{proof}
    Let $\{x_n\}_{n=1}^\infty\subset B(X)$ be any set which is separating for $X^*$, i.e. if $f\in X^*\setminus\{0\}$, then $f(x_n)\neq 0$ for some $n$ (for example, take any dense subset of $D(X)\setminus\{0\}$).
    Let $\rho$ be given by
    \begin{equation*}
        \rho(f,g)=\sum_{k=1}^\infty\frac{|(f-g)(x_k)|}{2^k}\leq 2
    \end{equation*}
    It is easy to see that this is a metric.

    Given $f_0\in B(X^*)$, take $\epsilon>0$ and let
    \begin{itemize}[nl]
        \item $n$ be so $\sum_{k=n+1}^\infty\frac{2}{2^k}<\frac{\epsilon}{2}$, and
        \item $V=\bigcap_{k=1}^n\{f\in B(X^*):|\hat x_k(f)-\hat x_k(f_0)|<\epsilon/2\}\in w^*|_{B(X^*)}$, $f_0\in V$.
    \end{itemize}
    Then if $f\in V$,
    \begin{align*}
        g(f,f_0) &= \sum_{k=1}^n\frac{|f(x_k)-f_0(x_k)|}{2^k}+\sum_{k=n+1}^\infty\frac{|f(x_k)-f_0(x_k)|}{2^k}<\epsilon
    \end{align*}
    so $f_0\in V\subset B^\circ_{\rho,\epsilon}(f_0)$.
    Since $f_0$ is arbitrary, we have $\tau_\rho\subseteq w^*|_{B(X^*)}$, but since $w^*$ is compact and $\tau_\rho$ is Hausdorff, these must be equal.
\end{proof}
\begin{enumerate}[nl,r]
    \item Note that different separating families from $B(X)$ may produce different metrics, but always the same topology.
    \item The definition of $\rho$ above extends to all of $X^*\times X^*$.
        However, $X^*$ with the weak* topology is not in metrizable if $X$ is infinite dimensional.
    \item $X^*=\bigcup_{=1}^\infty nB(X^*)$, so each $nB(X^*)$ is metrizable and compact, and thus $w^*-$separable.
        Thus if $X$ is separable, then $X^*$ is itself separable.
\end{enumerate}
\section{Nets}
\begin{definition}
    A pair $(N,\leq)$ is a \textbf{preorder} on $N$ if
    \begin{itemize}[nl]
        \item $v\leq v$ for $v\in N$
        \item $v_1\leq v_2$ and $v_2\leq v_3$ implies $v_1\leq v_3$.
    \end{itemize}
    This pair is \textbf{cofinal} if for any $v_1,v_2\in N$, there is $v_3\in N$ so $v_1\leq v_3$ and $v_2\leq v_3$.
    Then $(N,\leq)$ is a \textbf{directed set} if $\leq$ is a cofinal preorder.
    Given a non-empty set $X$, a \textbf{net} is a function $x:N\to X$.
\end{definition}
\begin{definition}
    If $(x_0)_{v\in N}$ is a net in $X$, $A\subseteq X$, we say that $(x_0)_{n\in\N}$ is
    \begin{itemize}[nl]
        \item \textbf{eventually} in $A$ if there is $v_A\in N$ so $x_v\in A$ whenever $v\geq v_A$
        \item \textbf{frequently} in $A$ if for any $v\in N$, there is $v'\in N$ with $v'\geq g$ so $x_{v'}\in A$.
    \end{itemize}
\end{definition}
\begin{definition}
    Now, let $(M,\leq)$ be anther directed set
    A map $\phi:M\to N$ is \textbf{eventually cofinal} if for any $v\in N$, there is $\mu_v\in N$ s $\phi(u)\geq v$ whenever $\mu\geq \mu_0$.
    Given a net $(x_v)_{v\in N}$ and an eventually cofinal $\phi:M\to N$, we call $(x_{\phi(\mu)})_{\mu\in M}$ a \textbf{subnet}.
\end{definition}
\begin{definition}
    We call $\phi:M\to N$ a \textbf{directed map} if
    \begin{enumerate}[nl,r]
        \item $\mu\leq \mu'$ in $M$ implies $\phi(\mu)\leq\phi(\mu')$ in $N$
        \item For any $v\in N$, there is $\mu\in M$ s $v\leq\phi(\mu)$.
    \end{enumerate}
\end{definition}
Directed maps are always cofinal.
Different sources use directed maps over eventually cofinal maps.
\begin{example}
    \begin{enumerate}[nl,r]
        \item $(\N,\leq)$ is directed, and subsequences are special types of subnets.
        \item $(\R,\leq)$ is directed
        \item (\textit{Riemann sums}) Let $a<b$ in $\R$.
            We let
            \begin{equation*}
                N=\{(P,P^*):P=\{a=t_0<t_1<\cdots<t_n=b\},P^*=\{t_1^*,\ldots,t_n^*\},t_k^*\in[t_{k-1},t_k]\}
            \end{equation*}
            and say $(P,P^*)\leq (Q,Q^*)$ if $P\subseteq Q$.
            One can verify that this is a net (the Riemann sum net).
        \item (\textit{Nets from filtering families}).
            We say that $\mathcal{F}\subset\mathcal{P}(X)\setminus\{\emptyset\}$ is a \textbf{filtering family} if for each $F_1,F_2\in\mathcal{F}$, there is $F_3\in\mathcal{F}$ such that $F_3\subseteq F_1\cap F_2$.
            For example, an ultrafilter is a filtering family.
            Let
            \begin{equation*}
                N_{\mathcal{F}}=\{(x,F):x\in F,F\in\mathcal{F}\}
            \end{equation*}
            equipped with the pre-order $(x,F)\leq(x',F')$ if and only if $F\supseteq F'$.
            Since $\mathcal{F}$ is a filtering family, $(N_{\mathcal{F}},\leq)$ is directed.
            Let $x_{(x,F)}=x$, so $(x)_{(x,F)\in N_{\mathcal{F}}}$ is the net built from $\mathcal{F}$.
            Note that if $F\in\mathcal{F}$, then $(x)_{(x,F)\in\mathcal{F}}$ is eventually in $F$.

            An \textbf{ultranet} $(x_v)_{v\in N}\subset X$ is a net for which any $A\in\mathcal{P}(X)$, $(x_v)_{v\in N}$ is either eventually in $A$ or eventually in $X\setminus A$.
            If $\mathcal{F}$ is an ultrafilter, then $(x)_{(x,F)\in N_{\mathcal{F}}}$ is an \textbf{ultranet}.
    \end{enumerate}
\end{example}
\subsection{Nets and Topology}
Now, suppose $(X,\tau)$ is a topological space.
\begin{definition}
    We say that $x_0\in X$ is
    \begin{itemize}[nl]
        \item Some $x_0\in X$ is a \textbf{limit point} if for any $U\in\tau$ with $x_0\in U$, $(x_v)_{v\in N}$ is eventually in $U$.
            That is, there is $\nu_U$ such that $x_\nu\in U$ whenever $\nu\geq\nu_U$.
            We write $x_0=\lim_{v\in N}x_v$, the $\tau-$limit of $(x_v)_{v\in N}$.
            Note that this is an abuse of notation, since limit points need not be unique (when $(X,\tau)$ is not Hausdorff).
        \item Some $x_0\in X$ is a \textbf{cluster point} of $(x_v)_{v\in N}$ if for any $U\in\tau$ with $x_0\in U$, $(x_v)_{v\in N}$ is frequently in $U$.
    \end{itemize}
\end{definition}
\begin{proposition}
    If $(x_v)_{v\in N}$ is a net in $(X,\tau)$ and $x_0\in X$, then $x_0$ is a cluster point for $(x_v)_{v\in N}$ if and only if $x_0$ is a $\tau-$limit point of $x_{v_\mu}$ for some subnet $(x_{v_\mu})_{\mu\in M}$ of $(x_v)_{v\in N}$.
\end{proposition}
\begin{proof}
    \impr
    Suppose $x_0$ is a cluster point for $(x_\nu)_{\nu\in N}$.
    Then for each $v\in N$ and $U\in\tau$ containing $x_0$, define
    \begin{equation*}
        F_{\nu,U}=\{\nu'\in N:\nu'\geq \nu,x_{\nu'}\in U\}
    \end{equation*}
    which is non-empty since $x_0$ is a cluster point.
    Then set
    \begin{equation*}
        \mathcal{F}=\{F_{\nu,U}:\nu\in N,U\in\tau,x_0\in U\}\subset\mathcal{P}(N)
    \end{equation*}
    Let's see that $\mathcal{F}$ is filtering: suppose $F_{\nu,U}$ and $F_{\nu',U'}$ are in $\mathcal{F}$.
    Get $\mu\geq\nu$ and $\mu\geq\nu'$ by definition of a net and set $V=U\cap U'$, which is open and contains $x_0$.
    Then since $x_0$ is a cluster point, get some $\mu'\geq \mu$ such that $x_{\mu'}\in V$, so $F_{\mu',V}\subseteq F_{\nu,U}\cap F_{\nu',U'}$
    We then let $M=N_{\mathcal{F}}$ be the net construction from the filtering family and set $v_{(\nu,F)}=V$.

    Now set $N_{\mathcal{F}}=\{(\nu,F):\nu\in F,F\in\mathcal{F}\}$ with the standard preorder and $\nu_{(\nu,F)}=\nu$.
    Then the map $(\nu,F)\mapsto\nu$ from $N_{\mathcal{F}}\to N$ is eventually cofinal: if $\nu_0\in N$ is arbitrary, take any $F_0=F_{\nu_0,U}\in\mathcal{F}$.
    Then $F_0=\{\nu\in N:\nu\geq\nu_0,x_\nu\in U\}$, so if $F_{\mu,V}\in\mathcal{F}$ with $F_{\mu,V}\subseteq$
    We let $M=N_{\mathcal{F}}$ as in (iv) above, and $v_{v,\mathcal{F}}=v$.
    Check that $(x_v)_{(v,F)\in N_{\mathcal{F}}}$ is eventually in $U$ for any $U\in\tau$ with $x_0\in U$.
    [Check: $(v,F)\mapsto v:N_{\mathcal{F}}\to N$ is cofinal, but is not evidently directed]
    
    $(\Leftarrow)$ If for some subnet $(x_{v_\mu})_{\mu\in M}$ is eventually in $U$ for any $U\in\tau$ with $x_0\in U$, then $(x_v)_{v\in N}$ is frequently in $U$ for such $U$ by definition of a subnet.
\end{proof}
\begin{proposition}
    If $(Y,\sigma)$ is another topological space, then $f:X\to Y$ is continuous if and only if for any $x_0\in X$ and net $(x_v)_{v\in N}$ with having $x_0$ as a limit, $f(x_0)=\lim_{v\in N}f(x_v)$.
\end{proposition}
\begin{proof}
    If $V\in\sigma$ with $f(x_0)\in V$, then $f^{-1}(V)\in\tau$ with $x_0\in f^{-1}(V)$.
    Since $(x_v)_{v\in N}$ is eventually in $f^{-1}(V)$, so $(f(x_v))_{v\in N}$ is eventually in $V$.

    Conversely, let $\tau_{x_0}=\{U\in\tau:x_0\in U\}$, which is filtering on $X$.
    Let $N_{\tau_{x_0}}=\{(x,U):x\in U,U\in\tau_{x_0}\}$ be directed by $(x,U)\leq(x',U')$ if and only if $U\supseteq U'$ as in (iv) above.
    Then $x_0=\lim_{(x,U)\in N_{\tau_{x_0}}}x$.
    Now, let $V\in\sigma$ with $f(x_0)$.
    The assumptions on $f$ tell us there is $v-V\in N_{\tau_{x_0}}$ such that for $v\geq v_V$, we have $f(x_0)\in V$
    We have $v_V=(x,U)$ for some $U\in\tau_{x_0}$ and $x\in U$, so for any $x'\in U$, $(x',U)\geq(x,U)$ and $f(x')=f(x_{x',U})\in V$, so that $x_0\in U=\bigcup_{x'\in U}\{x'\}\subseteq f^{-1}(V)$, so $f$ is continuous at $x_0$.
    But $x_0\in X$ was arbitrary.
\end{proof}
\begin{remark}
    We get the following consequences of this result:
    \begin{enumerate}[nl,r]
        \item Given topologies $\tau,\tau'$ on $X$, $\tau'\subseteq\tau$ if and only if $\tau'-\lim_{v\in N}x_v=x_0$ whenever $\tau-\lim_{v\in N}x_v=x_0$ for any $x_0\in X$.
        \item (limits in product topology) $\{(x_\alpha,\tau_\alpha)\}_{\alpha\in A}$ be topological space and $X=\prod_{\alpha\in A}X_\alpha$ equipped with the product topology $\pi$.
            If $(x^{(v)})_{v\in N}$ is a net in $X$ and $x^{(0)}\in X$, then $\pi-\lim_{v\in N}x^{(v)}=x^{(0)}$ if and only if for every $\alpha\in A$, $\tau_\alpha-\lim_{v\in N}x_\alpha^{(v)}=x_\alpha^{(0)}$.
            Recall that $\pi$ is the coarsest topology making each $\mu_\alpha$ continuous.
        \item If $X$ is a normed space and $(f_v)_{v\in N}\subset X^*$, $f_0\in X^*$, then $w^*-\lim_{v\in N}f_v=f_0$ if and only if $\lim_{v\in N}f_v(x)=f_0(x)$ for each $x\in X$.
    \end{enumerate}
\end{remark}
\subsection{Roles of weak and weak* topologies in convexity}
\begin{theorem}[$w^*-$Separation]
    Let $X$ be a normed space, $A,B\subset X^*$ each be non-empty and convex, with $A\cap B=\emptyset$ and $B$ $w^*-$open.
    Then there is $x\in X$ and $\alpha\in\R$ such that
    \begin{equation*}
        \Re f(x)\leq \alpha < \Re g(x)
    \end{equation*}
    for $f\in A$ and $g\in B$.
\end{theorem}
\begin{proof}
    The separation theorem and the fact that $B$ is $\norm{\cdot}-$open (i.e. $w^*\subseteq\tau_{\norm{\cdot}}$) provides $F\in X^{**}$ and $\alpha\in\R$ such that $\Re F(f)\leq\alpha\Re F(g)$ for $f\in A$, $g\in B$.
    Since $B\in w^*=\sigma(X^*,\hat X)$, if $f_0\in B$, then there are $x_1,\ldots,x_n$ in $X$ such that
    \begin{equation*}
        f_0\in U=\bigcap_{i=1}^n \hat x_i^{-1}(f_0(x_i)+\mathbb{D})\subseteq B
    \end{equation*}
    Let $Y=\bigcap_{i=1}^n\ker\hat{x_i}\subseteq X^*$.
    Then for $i=1,\ldots,n$, $\hat x_i(f_0+Y)=\{f_0(x_i)\}\subset f_0(x_i)+\mathbb{D}$, so that $f_0+Y\subseteq U\subseteq B$.
    Thus if $f\in Y$, then $\Re F(f_0+f)>\alpha$ and hence $\Re F(f)>\alpha-\Re F(f_0)$ which implies that $f\in\ker\Re F$, so $f\in\ker F$.
    That is, $Y\subseteq\ker F$.
    The next lemma shows that $F\in\spn\{\hat x_1,\ldots,\hat x_n\}\subseteq\hat X$, i.e. $F=\hat x$ for some $x\in X$.
\end{proof}
\begin{lemma}
    In an $\F-$vector space, if $f_0,f_1,\ldots,f_\in X'$ with $\ker f_0\supseteq\bigcap_{i=1}^n\ker f_i$, then $f\in\spn\{f_1,\ldots,f_n\}$.
\end{lemma}
\begin{proof}
    Define $T:X\to\F^n$ by $Tx=(f_1(x),\ldots,f_n(x))$.
    Then $\ker T=\bigcap_{i=1}^n\ker f_i$.
    Let $\mathcal{R}=\im T\subseteq\F$ and $g_0\in\mathcal{R}'$ by $g_0(Tx)=f_0(x)$.
    Then $g_0$ is well-defined: if $Tx=Ty$, then $x-y\in\ker T\subseteq\ker f_0$, so $f_0(x-y)=0$ so $f_0(x)=f_0(y)$.
    Also $g_0$ is linear.
    Let $g\in(\F^n)'$ such that $g|_{\mathcal{R}}=g_0$.
    Hence there are $\alpha_1,\ldots,\alpha_n\in\F$ such that $g(y_1,\ldots,y_n)=\sum_{j=1}^n\alpha_jy_j$.
    Hence for $x\in X$,
    \begin{equation*}
        f_0(x)=g_0(Tx)=g(Tx)=g(f_1(x),\ldots,f_n(x))=\sum_{j=1}^n\alpha_jf_j(x)
    \end{equation*}
    so that $f_0=\sum_{j=1}^n\alpha_jf_j$.
\end{proof}
\begin{theorem}[$w^*-$Closed Convex Hull]
    If $S\subset X^*$, then
    \begin{equation*}
        \cwx S=\cap\{\{f\in X^*:\Re f(x)\leq \alpha\}\supseteq S:x\in X,\alpha\in\R\}
    \end{equation*}
\end{theorem}
\begin{proof}
    The set on the right is $w^*-$closed and convex being the intersection of such.
    Conversely, if $f\in X^*\setminus\cwx S$, which is open, then there is a basic $w^*-$open neighbourhood
    \begin{equation*}
        B=\bigcap_{j=1}^n\hat x_j^{-1}(f(x_j)+\mathbb{D})\subseteq X^*\setminus\cwx S
    \end{equation*}
    so that $B\cap\cwx S=\emptyset$.
    Also, $B$ is convex.
\end{proof}
\begin{remark}
    If $X$ is a normed spacee, a closed half space $H=\{x\in X:\Re f(x)\leq\alpha\}$ for some $f$ in $X^*$, $\alpha\in\R$.
    Hence, $H$ is weakly closed $(\Re f)^{-1}([\alpha,\infty))=f^{-1}(\{z\in\C:\Re z\geq\alpha\})$ is $w-$closed.
    Thus if $S\subset X$, we have $\cw S\in w=\sigma(X,X^*)\subseteq\tau_{\norm{\cdot}}$, so $\cw S$ is automatically weakly closed.
    Hence if $C\subseteq X$ is convex, then $C$ is norm closed if and only if $C$ is $w-$closed.
\end{remark}
\begin{definition}
    Let $X$ be a normed space.
    If $E\subseteq X$ (non-empty), the \textbf{polar} of $E$ is given by
    \begin{align*}
        E^\circ&=\{f\in X^*:\Re f(x)\leq 1\text{ for all $x$ in }E\}\subseteq X^*\\
               &= \bigcap_{x\in E}\{f\in X^*:\Re\hat x(f)\leq 1\}
    \end{align*}
    so $E^\circ$ is convex and $w^*-$closed in $X^*$, and $0\in E^\circ$.

    If $F\subseteq X^*$ (non-empty), let the \textbf{pre-polar} of $F$ be given by
    \begin{equation*}
        F_\circ=\{x\in X:\Re f(x)\leq 1\text{ for all $f$ in }F\}
    \end{equation*}
    so, like above, $F_0$ is convex, $(w-)$closed, and $0\in F_0$.
\end{definition}
\begin{theorem}[Bipolar]
    \begin{enumerate}[nl,r]
        \item If $\emptyset\neq E\subseteq X$, then $(E^\circ)_\circ=\cw(E\cup\{0\})$.
        \item If $\emptyset\neq F\subseteq X^*$, then $(F_\circ)^\circ=\cwx(F\cup\{0\})$.
    \end{enumerate}
\end{theorem}
\begin{proof}
    \begin{enumerate}[nl,r]
        \item Note that $E\cup\{0\}\subseteq (E^\circ)_\circ$, so $\cw(E\cup\{0\})\subseteq(E^\circ)_\circ$.
            If $x_0\in X\setminus\cw(E\cup\{0\})$, then the separation theorem provides $f\in X^*$, $\alpha\in \R$ so $\Re f(x_0)>\alpha\geq\Re f(x)$ for $x\in E\cup\{0\}$.
            Notice that $\alpha\geq \Re f(0)=0$, and we let $\beta=\frac{1}{2}[\Re f(x_0)+\alpha]>0$, so $\Re f(x_0)>\beta\geq\Re f(x)$ for $x\in E\cup\{0\}$, $\beta>0$.
            Let $g=\frac{1}{\beta}f$ and we see that $g\in E^\circ$ and as $\Re g(x_0)>1$, $x_0\notin (E^\circ)_\circ$.
        \item Similar, use $w^*-$separation.
    \end{enumerate}
\end{proof}
\begin{remark}
    Let $Y\subseteq X$ be a subspace.
    If $f\in Y^0$, then $\Re f(y)\leq 1$ for $y\in Y$ implies that $f(y)=0$ for all $y\in Y$.
    We write $Y^a=Y^0$, and $Y^a=\{f\in X^*:f|_Y=0\}$ is called the \textbf{annhilator} of $Y$.
    Likewise, if $Z\subseteq X^*$ is a subspace, then $Z_a=Z_0$ where $Z_a=\{x\in X:f(x)=0\text{ for each }f\in Z\}$ is called the \textbf{pre-annhilator}.
    Notice that $Y^a$ and $Z_a$ are subspaces.
\end{remark}
\begin{corollary}
    \begin{enumerate}[nl,r]
        \item If $Y\subseteq $ is a subspace, then $(X^a)_a=\overline{X}$.
        \item If $Z\subseteq X^*$ is a subspace, then $(Z_a)^a=\overline{Z}^{w^*}$.
    \end{enumerate}
\end{corollary}
\begin{lemma}
    If $X$ is a normed space, then $B(X)^0=B(X^*)$ and $B(X^*)_0=B(X)$.
\end{lemma}
\begin{proof}
    If $f\in B(X^0)$, then $\Re f(x)\leq 1$ for $x\in B(X)$.
    Thus for $x\in B(X)$, $|f(x)|=\overline{\sgn f(x)}f(x)=f(\overline{\sgn f(x)}x)\leq 1$, so $\norm{f}\leq 1$ and $f\in B(X^*)$.
    Conversely, if $f\in B(X^*)$, $x\in B(X)$, then $\Re f(x)\leq|f(x)|\leq 1$ so $f\in B(X)^\circ$.
    Then use the Bipolar theorem.
\end{proof}
\begin{theorem}[Goldstine]
    If $X$ is a normed space, then $\overline{B(\hat X)}^{w^*}=B(X^{**})$.
    Note that $w^*=\sigma(X^{**},\hat{X^*})$.
\end{theorem}
\begin{proof}
    The Bipolar theorem provides $\overline{B(\hat X)}^{w^*}=\cwx B(\hat x)=(B(\hat X)_\circ)^\circ$.
    But, in $X^*$,
    \begin{align*}
        B(X)^\circ &= \{f\in X^*:\Re f(x)\leq 1\text{ for $x$ in }B(X)\}\\
                   &= \{f\in \hat{X^*}:\Re \hat x(f)\leq 1\text{ for $x$ in }B(X)\}\\
                   &= B(\hat X)_\circ
    \end{align*}
    Hence we have, using the lemma,
    \begin{equation*}
        \overline{B(\hat X)}^{w^*}=(B(\hat X)_\circ)^\circ=(B(X)^\circ)^\circ=B(X^*)^\circ=B(X^{**})
    \end{equation*}
\end{proof}
\begin{example}
    \begin{enumerate}[nl,r]
        \item Recall that $c_0^*\cong\ell_1$ and $\ell_1^*\cong\ell_\infty$, wheren $c_0\subseteq\ell_\infty$.
            Thus by Goldstine, $\overline{B(c_0)}^{w^*}=B(\ell_\infty)$, so $w^*=\sigma(\ell_\infty,\ell_1)$.
            Since $\ell_1$ is separable, we have that $(B(\ell_\infty),w^*)$ is metrizable.
            In fact, if $x\in\ell_\infty$, then if $x^{(n)}=(x_1,\ldots,x_n,0,0,\ldots)\in c_0$, we have $x=w^*-\lim_{n\to\infty}x^{(n)}$.
        \item $\ell_\infty^*\cong\FA(\N)$.
            But $B(\FA(\N),w^*)$ is not metrizable.
            Since $\ell_1^*\cong\ell_\infty$, there is a natural isometric embedding $\ell_1\hookrightarrow\FA(\N)$.
            Then $y^{(n)}=\frac{1}{n}(1,1,\ldots)\in B(\ell_1)$, and $w^*-$cluster point of $(y^{(n)})_{n=1}^\infty\subset B(\FA(\N))$ is a Banach limit.
    \end{enumerate}
\end{example}
\begin{corollary}
    If $F\in X^{**}$, there always exists a net $(x_\nu)_{\nu\in N}\subset X$ such that
    \begin{equation*}
        F=w^*-\lim_{\nu\in N}\hat x_\nu\text{ and }\norm{x_\nu}\leq\norm{F}
    \end{equation*}
\end{corollary}
\begin{proof}
    If $F\neq 0$, $\frac{1}{\norm{F}}F\in B(X^{**})=\overline{B(\hat X)}^{w^*}$, and we may find $(y_\nu)_{\nu\in N}\subset B(X)$ such that $(\hat y_\nu)_{\nu\in N}\subset B(\hat X)$ and $\frac{1}{\norm{F}}F=w^*-\lim_{\nu\in N}\hat y_\nu$.
    Let $x_\nu=\norm{F}y_\nu$.
\end{proof}
Consider $\mathcal{F}=w^*_{\frac{1}{\norm{F}}F}=\{U\in w^*|_{B(X^{**})}:F\in U\}$ is a filtering family.
Each $U\in w^*_{\frac{1}{\norm{F}}F}$ has $U\cap B(\hat X)\neq\emptyset$ by Goldstine.
Let $N_{\mathcal{F}}=\{(x,U):x\in B(X),\hat x\in U,U\in\mathcal{F}\}$.
Then $(x_\nu)_{\nu\in N_{\mathcal{F}}}=(x)_{(x,U)\in N_{\mathcal{F}}}$ works.
\begin{definition}
    A normed space $X$ is \textbf{reflexive} if $\hat X=X^{**}$.
\end{definition}
Notice that $X^{**}=(X^*)^*$ is complete, and $x\mapsto\hat x$ is an isometry, so a reflexive space is always complete.
\begin{theorem}
    Let $X$ be a Banach space.
    The following are equivalent:
    \begin{enumerate}[nl,r]
        \item $X$ is reflexive
        \item $B(X)$ is $w-$compact
        \item $w^*=w$ on $X^*$
        \item $X^*$ is reflexive.
    \end{enumerate}
\end{theorem}
\begin{proof}
    The map $\iota:x\mapsto\hat x$ is a $w-w^*|_{\hat X}-$homeomorphism.
    Recall $w^*=\sigma(X^{**},\hat X^*)$, and $w^*|_{\hat X}=\sigma(\hat X,(\hat X)^*|_{\hat X})$ and we have for $x_0\in X$, net $(x_\nu)_{\nu\in N}$ in $X$,
    \begin{align*}
        w-\lim_{\nu\in N}x_\nu = x_0 &\iff \lim_{\nu\in N}f(x_\nu)=f(x_0)\forall f\in X^*\\
                                     &\iff \lim_{\nu\in N}\hat x_\nu(f)=\hat x_0(f)\forall f\in X^*\\
                                     &\iff \lim_{\nu\in N}\hat f(\hat x_\nu)=\hat f(\hat x_0)
    \end{align*}
    and having the same convergent nets means that the topologies are the same.

    \imp{i}{ii}
    By assumption, $\widehat{B(X)}=B(\hat X)=B(X^{**})$.
    Since $B(X^{**})$ is $w^*-$compact, and hence $\iota^{-1}(B(X^{**}))=B(X)$ is $w-$compact

    \imp{ii}{i}
    If $B(X)$ is $w-$compact, then since $x\mapsto\hat x:X\to X^{**}$ is continuous, we see that $B(\hat X)=\widehat{B(X)}$ is $w^*-$compact.

    \imp{i}{iii}
    We have $\hat X=X^{**}$ so on $X^*$, we have $w=\sigma(X^*,X^{**})=\sigma(X^*,\hat X)=w^*$.

    \imp{iii}{iv}
    $B(X^*)$ is compact, hence $w-$compct, so by $(ii)$ implies $(i)$ applied to $X^*$, we have that $X^*$ is reflexive.

    \imp{iv}{i}
    We assume $\widehat{X^*}=X^{***}$.
    Thus on $X^{***}$, we have $w=\sigma(X^{**},X^{***})=\sigma(X^{**},\widehat{X^*})=w^*$.
    Now $B(\hat X)=B(X^{**})\cap\hat X$ is norm-closed and convex, hence $w-$closed, by Closed Convex Hull theorem.
    Thus from above, $B(\hat X)$ is $w^*-$closed, so $B(\hat X)=\overline{B(\hat X)}^{w^*}=B(X^{**})$ by Goldstine, so $\hat X=X^{**}$.
\end{proof}
\begin{corollary}
    \begin{enumerate}[nl,r]
        \item Any finite dimensional normed space is reflexive.
        \item Any closed subspace $Y$ of a normed space $X$ is reflexive.
    \end{enumerate}
\end{corollary}
\begin{proof}
    \begin{enumerate}[nl,r]
        \item A finite dimensional normed space is complete, and its closed ball is compact, and thus $w-$compact as $\tau_{\norm{\cdot}}\supseteq w$.
        \item By Hahn-Banach, $Y^*=X^*|_Y$, so $\sigma(Y,Y^*)=\sigma(Y,X^*|_Y)=\sigma(X,X^*)|_Y$.
            Now $B(Y)=B(X)\cap Y$ is norm-closed and convex, hence $w-$closed in $B(X)$.
            But $B(X)$ is $w-$compact, so $B(Y)$ is a $w-$closed subset of a $w-$compact space and thus $w-$compact.
    \end{enumerate}
\end{proof}
\subsection{Extreme Points and the Krein-Milman Theorem}
\begin{definition}
    Let $X$ be a vector space and $C\subset X$ convex.
    A \textbf{face} $F$ of $C$ is any non-empty subset such that if $x\in F$, $x=(1-t)y+tz$, $t\in(0,1)$, $y,z\in C$ implies that $y,z\in F$.
    A \textbf{extreme point} of $C$ is a singleton face, i.e. $\ext C=\{x\in C:\{x\}\text{ is a face of }C\}$.
    Hence $x\in\ext C$ if for any $t\in(0,1)$ and $y,z\in C$, if $x=(1-t)y+tz$ then $x=y=z$.
\end{definition}
\begin{remark}
    \begin{enumerate}[nl,r]
        \item Faces of $C$ are not necessarily convex.
        \item A face $F'$ of a convex face $F$ of $C$ is itself a face of $C$.
        \item $\ext F\subseteq\ext C$.
        \item If $f\in X'$ and $\Re f(C)=[a,b]$, then $(\Re f)^{-1}(\{b\})$ is itself a face of $C$.
    \end{enumerate}
\end{remark}
\begin{theorem}[Krein-Milman]
    Let $X$ be a normed space and $C\subset X^*$ convex and $w^*-$compact.
    Then $C=\overline{co}^{w^*}\ext C$.
\end{theorem}
\begin{proof}
    We first verify that any $w^*-$closed face of $C$ admits an extreme point.
    We let $\mathcal{F}=\{F:F\text{ is a convex $w^*-$closed face of $C$}\}$, which is partially ordered by reverse inclusion.
    If $\mathcal{C}$ is a chain in $\mathcal{F}$ with $F_1,\ldots,F_n\in\mathcal{C}$, we may assume $F_1\supseteq\cdots\supseteq F_n$ so that $\mathcal{C}$ has the finite intersection property.
    Thus $\emptyset\neq F_0=\bigcap_{F\in\mathcal{C}}F$.
    If $x\in F_0$, $t\in(0,1),y,z\in C$ and $x=(1-t)y+tz$, then $x\in F$ for any $F\in\mathcal{C}$ so $y,z\in F$ for any $f\in\mathcal{C}$.
    Thus $y,z\in\bigcap_{F\in\mathcal{C}}F=F_0$.
    Also $F_0$ is closed, so $F_0\in\mathcal{F}$.
    Thus $F_0$ is an upper bound in $\mathcal{F}$ for $\mathcal{C}$, so by Zorn, get some maximal element $M$.

    Let $M$ be a minimal $w^*-$closed convex face of $F$.
    Then given $x\in X$, $\Re\hat x:X^*\to\R$ is $w^*-$continuous, and hence $\Re\hat x(M)=[a_x,b_x]$ since the only compact convex subsets of $\R$ are compact intervals.
    But then $F_x=(\Re \hat x)^{-1}(\{b_x\})\cap M$ is a $w^*-$closed convex face in $M$, so that $F_x=M$.
    If $f,g\in M$, then $\Re f(x)=\Re\hat x(f)=b_x=\Re\hat x(g)=\Re g(x)$, so $f=g$ and hence $M=\{f\}$ and $f\in\ext F$.

    Now let $f_0\in X^*\setminus\cwx\ext C$.
    Since $C$ is $w^*-$compact and convex, $\Re\hat x(C)=[a_x,b_x]$, so $C_x=(\Re\hat x)^{-1}(\{b_x\})\cap C$ is a $w^*-$closed convex face of $C$.
    Hence by above, there is $f\in\ext C_x\subseteq\ext C$ with $\Re\hat x(f)=b_x$.
    But then $\Re\hat x(f_0)>\alpha\geq\Re\hat x(f)=b_x$, so $\Re\hat x(f_0)\notin[a_x,b_x]=\Re\hat x(C)$, so $f_0\notin C$.
    Thus $C\subseteq\cwx\ext C$, where the converse inclusion is obvious.
\end{proof}
\begin{corollary}
    \begin{enumerate}[nl,r]
        \item If $C\subset X$ is a $w-$compact convex set, then $C=\cw\ext C$.
        \item If $C\subset X$ is a norm-compact convex set, then $C=\cw\ext C$.
    \end{enumerate}
\end{corollary}
\begin{proof}
    \begin{enumerate}[nl,r]
        \item We have that $x\mapsto\hat x:X\to\hat X\subseteq X^{**}$ is continuous.
            Hence $\hat C$ is $w^*-$compact in $X^{**}$, so $x\mapsto\hat x:C\to\hat C$ is a homeomorphism.
            In $\hat C$, we have
            \begin{equation*}
                \widehat{\cw^w\ext C}=\cwx\ext\hat C=\hat C
            \end{equation*}
            so that $C=\cw^w\ext C=\overline{co}\ext C$ by the closed convex hull theorem.
        \item Since $w\subseteq\tau_{\norm{\cdot}}$, any norm-compact is $w-$compact.
    \end{enumerate}
\end{proof}
\begin{remark}
    Let $X$ be a normed space.
    Then $\ext B(X)\subseteq S(X)$.
\end{remark}
\begin{proposition}
    Let $1<p<\infty$.
    Then $\ext B(\ell_p)=S(\ell_p)$.
\end{proposition}
\begin{proof}
    Let $x\in S(\ell_p)$, so $x=(1-t)y+tz$.
    Then
    \begin{equation*}
        1=\norm{x}_p\leq(1-t)\norm{y}_p+t\norm{z}_p\leq 1
    \end{equation*}
    so that $\norm{y}_p=\norm{z}_p=1$ and $\norm{x}_p=(1-t)\norm{y}_p+t\norm{z}_p$.
    Thus by the equality case for Minkowski, there is $s\geq 0$ so $s(1-t)y=tz$.
    Taking norms, we have $y=z$.
\end{proof}
\begin{proposition}
    We have $\ext B(c_0)=\emptyset$.
\end{proposition}
\begin{proof}
    Let $x=(x_1,x_2,\ldots)\in B(C_0)$.
    Since $\lim x_n=0$, get $n_0$ so $|x_{n_0}|\leq 1/2$.
    If $x_{n_0}\neq 0$, let $y=(x_1,\ldots,x_{n_0-1},2x_{n_0},x_{n_0+1},\ldots)$ and $z=(x_1,\ldots,x_{n_0-1},0,x_{n_0+1},\ldots)$, and similarly for $x_{n_0}=0$.
    Thus we have in each case that $y,z\in B(c_0)$ and $x=y/2+z/2$.
\end{proof}
\begin{corollary}
    There exists no normed space $X$ for which $c_0\cong X^*$.
\end{corollary}
\begin{proof}
    If there were such $X$, then $B(c_0)$ would be $w^*-$compact, and hence Krein-Milman would imply $\ext B(c_0)\neq\emptyset$.
\end{proof}
\begin{definition}
    Let $(X,\tau)$ be a compact Hausdorff space, and let
    \begin{equation*}
        P(X)=\{\mu\in B(C^{\R}(X,\tau)^*):\mu(1)=1\}
    \end{equation*}
\end{definition}
\begin{theorem}
    $\ext P(X)=\{\hat x:x\in X\}$, where $\hat x(f)=f(x)$.
    Furthermore, $\cwx\ext P(X)=P(X)$.
\end{theorem}
\begin{proof}
    Write $C=C^{\R}(X,\tau)$.
    Note that $P(X)=B(C^*)\cap\hat \idc^{-1}(\{1\})$ is $w^*-$compact and convex.
    Hence by Krein-Milman, we have that $\cwx\ext P(X)=P(X)$.
    It remains to describe $\ext P(X)$.
    
    (I) Some inequalities.
    Fix $\mu\in P(X)$.
    If $0\leq f\leq 1$ in $C$, then $0\leq \idc-f\leq 1$ so $\norm{f}_\infty$, $\norm{\idc -f}_\infty\leq 1$.
    Thus $|\mu(f)|\leq 1$ and $|\idc-\mu(f)|=|\mu(\idc-f)|\leq 1$.
    Thus $0\leq\mu(f)\leq 1$.
    Then if $g\neq 0$ and $g\geq 0$ in $C$, then we have $\mu(g/\norm{g}_\infty)\geq 0$, so $\mu(g)>0$; if $g\leq h$ in $C$, then $h-g\geq 0$ and $\mu(h)\geq\mu(g)$.

    If $g\in C$, $g^+=\max\{g,0\}$, $g^-=\max\{-g,0\}\in C$, and $g=g^+-g^-$ while $|g|=g^++g^-$.
    Hence if $0\leq f\leq 1$ in $C$ and let $\mu_f(g)=\mu(fg)$ for $g\in C$, we have
    \begin{align}\label{e:in1}
        |\mu_f(g)|&=|\mu_f(g^+-g^-)|=|\mu(fg^+)+\mu(fg^-)|\leq\mu(fg^+)+\mu(fg^-)=\mu(f(g))\nonumber\\
                  &\leq\mu(f\norm{g}_\infty)=\mu(f)\norm{g}_\infty
    \end{align}
    and, with $f=\idc$, we have
    \begin{equation}\label{e:in2}
        |\mu(g)|\leq\mu(|g|)
    \end{equation}

    (II) Let $\delta\in\ext P(X)$.
    We first show for $h,g$ in $C$ that $\delta(hg)=\delta(h)\delta(g)$.
    To see this, since $\delta\neq 0$, we may find $0\leq f\leq 1$ such that $0<\delta(f)<1$.
    Now let $\mu=\frac{1}{\delta(f)}\delta_f$ so, for $g\in C$, \cref{e:in1} provides
    \begin{equation*}
        |\mu(g)|=\frac{1}{\delta(f)}|\delta_f(g)|\leq\frac{1}{\delta(f)}\delta(f)\norm{y}_\infty=\norm{y}_\infty
    \end{equation*}
    so that $\mu\in B(C^*)$.
    We also know that $\mu(\idc)=1$.
    Hence $\mu\in P(X)$.
    Likewise, $\nu=\frac{1}{1-\delta(f)}\delta_{\idc-f}\in P(X)$.
    We have that
    \begin{equation*}
        \delta(f)\mu+(1-\delta(f))\nu=\delta
    \end{equation*}
    so by assumption on $\delta$, $\mu=\delta$.
    Thus $\frac{1}{\delta(f)}\delta(fg)=\mu(g)=\delta(g)$, so that $\delta(fg)=\delta(f)\delta(g)$.
    Note that $C=\spn\{f\in C:0\leq f\leq 1\}$, so we get multiplicativity of $\delta$.

    Suppose now for each $x\in X$, there exists some $f_x\in\ker\delta$ so that $f_x(x)\neq 0$.
    Let $U_x=f_x^{-1}(\R\setminus\{0\})$, so $X=\bigcup_{x\in X}\{x\}=\bigcup_{x\in X}U_x$ so there are $x_1,\ldots,x_n$ in $X$ so $X=\bigcup_{j=1}^n U_{x_j}$.
    Then $f=\sum_{j=1}^n f_{x_j}^2>0$ on $X$ (by definition of each $U_{x_j}$), so $1/f\in C$.
    Then
    \begin{equation*}
        1 = \delta(\idc)=\delta\left(\frac{1}{f}\right)\delta(f)=\delta\left(\frac{1}{f}\right)\sum_{j=1}^n\delta(f_{x_j})^2=0
    \end{equation*}
    since each $f_{x_i}\in\ker\delta$, which is absurd.
    Hence there is $x\in X$ so $f(x)=0$ whenever $f\in\ker\delta$, so $\ker\delta\supsetneq\ker\hat x$, so $\delta\in\R\hat x$ and since $\delta(\idc)=1=\hat x(\idc)$, so $\delta=\hat x$.

    (III) If $\hat x=(1-t)\mu+tv$ and $t\in(0,1)$, $\mu,\nu\in P(X)$, then by \cref{e:in2},
    \begin{equation*}
        t|\nu(f)|\leq t\nu(|f|)\leq\hat x(|f|)=|f(x)|
    \end{equation*}
    so $\ker\nu\supseteq\ker\hat x$ and as above, $\nu=\hat x$.
    Then $\mu=\hat x$.
\end{proof}
\begin{remark}
    For $\F=\R$ or $\F=\C$, it is similar to show that $\ext B(C^{\F}(X,\tau)^*)=\{z\hat x:z\in\F,|z|=1,x\in X^*\}$.
\end{remark}
Let $PA(\N)=\{\mu\in\FA(\N):\norm{\mu}_{\text{var}}\leq 1,\mu(\N)=1\}$ so, as above, $PA(\N)$ is a $w^*=\sigma(\FA(\N),\ell_\infty)-$compact set.
\begin{proposition}
    $\ext PA(\N)=\{\delta_{\mathcal{U}}:\mathcal{U}\text{ is an ultrafilter on}\N\}$
\end{proposition}
\begin{proof}
    If $\delta\in\ext PA(\N)$, let $f_\delta\in\ell_\infty^*$ be as in $A1$.
    As above, we compute that $f_\delta(\chi_E\chi_F)=f_\delta(\chi_E)f_\delta(\chi_F)$, and we have $\chi_E\chi_F=\chi_{E\cap F}$ and hence $\delta(E\cap F)=\delta(E)\delta(F)$.
    Hence
    \begin{equation*}
        \mathcal{U}=\{E\subseteq\N:\delta(E)\neq 0\}=\{E\subseteq\N:\delta(E)=1\}
    \end{equation*}
    is an ultrafilter.
    The converse is easy.
\end{proof}
\section{Euclidean and Hilbert Spaces}
\begin{definition}
    Let $X$ be a vector space over $\F$ ($\R$ or $\C$).
    A form $\lbr{\cdot,\cdot}:X\to\F$ is called \textbf{Hermitian} if for $x,x',y$ in $X$, $\alpha\in\F$, we have
    \begin{enumerate}[nl,r]
        \item $\lbr{x+\alpha x',y}=\lbr{x,y}+\alpha\lbr{x',y}$
        \item $\overline{\lbr{y,x}}=\lbr{x,y}$
    \end{enumerate}
    and furthermore \textbf{positive} if
    \begin{enumerate}[resume]
        \item $\lbr{x,x}\geq 0$ for all $x\in X$
    \end{enumerate}
    and \textbf{non-degenerate} if
    \begin{enumerate}[resume]
        \item $\lbr{x,y}=0$ for all $y\in X$ implies $x=0$.
    \end{enumerate}
\end{definition}
\begin{proposition}
    Let $[\cdot,\cdot]$ be a positive Hermitian form.
    Let $p(x)=\lbr{x,x}^{1/2}$, so $p:X\to[0,\infty)$.
    Then for $x,y\in X$ and $\alpha\in\F$, we have
    \begin{enumerate}[nl,r]
        \item $p(\alpha x)=|\alpha|p(x)$
        \item $\lbr{x,y}|\leq p(x)p(y)$
        \item $p(x+y)\leq p(x)+p(y)$
        \item $\lbr{\cdot,\cdot}$ is non-degenerate if and only if $\lbr{x,x}>0$ for $x\in X\setminus\{0\}$.
    \end{enumerate}
    Furthermore, in this case, we have
    \begin{itemize}[nl]
        \item Equality in (ii) if and only if $x,y$ are linearly dependent
        \item $\lbr{x,y}=p(x)p(y)$ if and only if there is $s\geq 0$ such that $x=sy$ or $y=sx$ if and only if equality holds in $(iii)$.
    \end{itemize}
\end{proposition}
\begin{proof}
    \begin{enumerate}[nl,r]
        \item $p(\alpha x)=(\alpha\overline{\alpha}\lbr{x,x})^{1,2}=|\alpha|p(x)$
        \item If $\alpha\in F$, then
            \begin{align*}
                0\leq \lbr{x-\alpha y,x-\alpha y}&=\lbr{x,x}-\overline{\alpha}\lbr{x,y}-\overline{\overline{\alpha}\lbr{x,y}}+|\alpha|^2\lbr{y,y}\\
                                                 &= p(x)^2-2\Re\overline{\alpha}\lbr{x,y}+|\alpha|p(y)^2
            \end{align*}
            Set $\alpha=\sgn\lbr{x,y}$ so that $\overline{\alpha}\lbr{x,y}=|\lbr{x,y}|$ so
            \begin{equation*}
                |\lbr{x,y}|\leq\frac{1}{2}\left(p(x)^2+p(y)^2\right)
            \end{equation*}
            Then if $t>0$, by (i),
            \begin{equation*}
                |\lbr{x,y}|=|\lbr{tx,\frac{1}{t}y}\leq\frac{1}{2}(t^2p(x)^2+\frac{1}{t^2}p(y)^2)
            \end{equation*}
            If $p(x)=0$, we let $t\to\infty$ so that $\lbr{x,y}=0$; if $p(y)=0$, we let $t\to 0^+$ and again that $\lbr{x,y}=0$.
            If $\lbr{x,y}\neq 0$, set $t=p(y)/p(x)$ and we are done.
        \item
            \begin{align*}
                p(x+y)^2&=\lbr{x+y,x+y}=p(x)^2+2\Re[x,y]+p(y)^2\\
                        &\leq p(x)^2+2|\lbr{x,y}|+p(y)^2\\
                        &\leq p(x)^2+2p(x)p(y)+p(y)^2=(p(x)+p(y))^2
            \end{align*}
        \item We see, by (iii), if $p(x)^2=\lbr{x,x}=0$, then $\lbr{x,y}=0$ for all $y$.
            Hence $\lbr{\cdot,\cdot}$ is non-degenerate if and only if $\lbr{x,x}>0$ for $x\in X\setminus\{0\}$.
            If $x,y$ are linearly dependant, then equality holds in (ii) by direct computation.
            If $x,y$ are not linearly dependent, then the choice of $\alpha=\sgn\lbr{x,y}$ in (ii) gives strict inequality.
            The condition $\lbr{x,y}=p(x)p(y)$ requires non-negativity of $\lbr{x,y}$, showing one is a $R_{\geq 0}$ multiple of the other.
            This is equivalent to having equality in (iii).
    \end{enumerate}
\end{proof}
\begin{definition}
    A non-degenerate positive Hermitian form on a vector space $\mathcal{E}$ is called an \textbf{inner product}.
    The pair $(\mathcal{E},(\cdot,\cdot))$ is called a Euclidean space.
    If, further, $\mathcal{E}$ is complete with respect to the induced norm $\norm{x}=\inr{x,x}^{1/2}$, then we call $(\mathcal{E},\inr{\cdot,\cdot})$ a \textbf{Hilbert space}.
\end{definition}
\begin{example}
    \begin{enumerate}[r]
        \item (Euclidean Space) $(C[0,1],\langle\cdot,\cdot\rangle)$ given by $\inr{f,g}=\int_0^1 f\overline{g}$
        \item (Euclidean Space) Recall $\ell=\{x\in\F^{\N}:x_n=0\text{ for all but finitely many }n\}$, and $(\ell,\langle\cdot,\cdot\rangle)$ with $\langle x,y\rangle=\sum_{j=1}^\infty x_j\overline{y}_j$
        \item (Hilbert Space) $(L_2[0,1],\inr{\cdot,\cdot})$, $\inr{f,g}=\int_{[0,1]}f\overline{g}$.
        \item (Hilbert Space) $(\ell_2,\inr{\cdot,\cdot})$, $\inr{x,y}=\sum_{j=1}^\infty x_j\overline{y}_j$ (convergence by Hölder's inequality)
        \item (Non-separable Hilbert Space) Let $\Gamma$ be an uncountable set.
            If $a=(a_\gamma)_{\gamma\in\Gamma}\in[0,\infty)^\Gamma$, we let $\mathcal{F}=\{F\subset\Gamma:|F|<\infty\}$.
            We define $\sum_{\gamma\in\Gamma}a_\gamma=\sup_{F\in\mathcal{F}}\sum_{\gamma\in F}a_\gamma=\lim_{F\in\mathcal{F}}\sum_{\gamma\in F}a_\gamma$ where $\mathcal{F}$ is pre-ordered by inclusion.
            Suppose that $\sum_{\gamma\in\Gamma}a_\gamma<\infty$.
            Let $\Gamma_n=\{\gamma\in\Gamma:a_\gamma\geq 1/n\}$ and we have
            \begin{equation*}
                \infty>\sum_{\gamma\in\Gamma}a_\gamma\geq\sup_{F\in\mathcal{F}}\sum_{\gamma\in F\cap\Gamma_n}a_\gamma\geq\sum_{F\in\mathcal{F}}\frac{F\cap\Gamma_n}{n}
            \end{equation*}
            so that $|\Gamma_n|<\infty$.
            Thus $\Gamma_a=\{\gamma\in\Gamma:a_\gamma>0\}=\bigcup_{n=1}^\infty\Gamma_n$ is countable.

            Now, define $\ell_2(\Gamma)=\{x=(x_\gamma)\in\F^{\Gamma}:\sum_{\gamma\in\Gamma}|x_\gamma|^2<\infty\}$.
            If $x,y\in\ell_2(\Gamma)$, then we may let $\Gamma_{|x|^2}\cup\Gamma_{|y|^2}\subseteq\{\gamma_k\}_{k=1}^\infty$ so Hölder's inequality for $\ell_2$ says that
            \begin{equation*}
                \sum_{k=1}^\infty|x_{\gamma_k}\overline{y}_{\gamma_k}|\leq\left(\sum_{k=1}^\infty|x_{\gamma_k}|^2\right)^{1/2}\left(\sum_{k=1}^\infty|y_{\gamma_k}|^2\right)^{1/2}<\infty.
            \end{equation*}
            Thus, $\sum_{k=1}^\infty x_{\gamma_k}\overline{y_{\gamma_k}}$ is absolutely converging.
            Write $\inr{x,y}=\sum_{\gamma\in\Gamma}x_\gamma\overline{y_\gamma}=\sum_{k=1}^\infty x_{\gamma_k}\overline{y_{\gamma_k}}$.
            Now if $(x^{(n)})_{n=1}^\infty\subset\ell_2(\Gamma)$ is $\norm{\cdot}_2-$Cauchy, then $\Gamma'=\bigcup_{n=1}^\infty\Gamma_{|x^{(n)}|^2}$ is countable.
            Then since $\ell_2(\Gamma')\cong\ell_2$ (up to counting $\Gamma'$), so the Cauchy sequence has a limit.
            Thus $\ell_2(\Gamma)$ is a Hilbert space.
            It is immediate that $(\ell_2(\Gamma),\norm{\cdot}_2)$ is non-separable.
        \item Let $w:\N\to(0,\infty)$.
            Let $\ell_2^w=\{x\in\F^{\N}:\sum_{k=1}^\infty|x_k|^2w(k)<\infty\}$.
            Notice that if $x,y\in\ell_2^w$, then $(x_kw(k)^{1/2})_{k=1}^\infty$, $(y_kw(k)^{1/2})_{k=1}^\infty\in\ell_2$, so it follows that
            \begin{equation*}
                \inr{x,y}_w=\sum_{k=1}^\infty x_k\overline{y_k}w(k)
            \end{equation*}
            defines an inner product, and $W:\ell_2^2\to\ell_2$ by $W(x_k)_{k=1}^\infty=(x_kw(k)^{1/2})_{k=1}^\infty$ is a surjective linear isometry, so $\ell_2^w$ is a hilbert space.
    \end{enumerate}
\end{example}
\subsection{Various Identities}
Let $\lbr{\cdot,\cdot}$ be a Hermitian form on $X$.
Then we have the \textit{polarization identitiy}: then over $\R$, $4[x,y]=[x+y,x+y]-[x-y,x-y]$, and over $\C$, $4[x,y]=\sum_{k=0}^3 i^k[x+i^ky,x+i^ky]$.

Now suppose $(\mathcal{E},\inr{\cdot,\cdot})$ is a Euclidean space.
We say that $x,y\in\mathcal{E}$ are \textbf{orthogonal} if $\inr{x,y}=0$ and write $x\perp y$.
We call a subset $E\subset\mathcal{E}$ \textbf{orthogonal} if $x\neq y\in E$ implies $x\perp y$.
We write $x\perp E$ if $x\perp y$ for each $y\in E$.
We have
\begin{itemize}[nl]
    \item \textit{Pythagoreans' identity}: if $\{x_1,\ldots,x_n\}\subset\mathcal{E}$ orthogonal, then $\norm{\sum_{j=1}^n x_j}^2=\sum_{j=1}^n\norm{x_j}^2$.
    \item \textit{Parallelogram law}: $\norm{x+y}^2+\norm{x-y}^2=2\norm{x}^2+2\norm{y}^2$.
\end{itemize}
Note that if $\F=\C$, $\inr{x,y}=\frac{1}{4}\sum_{k=0}^3 i^k\norm{x+i^ky}^2$ defines an inner product, for any norm satisfying the parallelogram law.
\begin{proposition}
    If $y\in\mathcal{E}$ with $(\mathcal{E},\inr{\cdot,\cdot})$ a Euclidean space, then $f_y:\mathcal{E}\to\F$ by $f_y(x)=\inr{x,y}$ is linear with $\norm{f_y}=\norm{y}$.
    Furthermore, $|f_y(x)|=\norm{y}$ for $y\neq 0$, $x\in B(\mathcal{E})$ if and only if $x=\frac{\zeta}{\norm{y}}y$ where $|\zeta|=1$.
\end{proposition}
\begin{proof}
    Linearity isfrom an assumption on $\inr{\cdot,\cdot}$.
    Furthermore, Cauchy-Schwarz tells us that
    \begin{equation*}
        |f_y(x)|=|\inr{x,y}|\leq\norm{x}\norm{y}\Rightarrow\norm{f_y}\leq\norm{y}
    \end{equation*}
    so the equality case for Cauchy-Schwarz provides the last statement of the proposition, and supplies $\norm{f_y}\geq\norm{y}$.
\end{proof}
\begin{definition}
    In a Euclidean space $(\mathcal{E},\inr{\cdot,\cdot})$, a set $E\subset\mathcal{E}$ is called \textbf{orthonormal} provided that for $e,e'\in E$,
    \begin{equation*}
        \inr{e,e'}=\begin{cases}
            1 &: e=e'\\
            0 &: e\neq e'
        \end{cases}
    \end{equation*}
\end{definition}
\begin{lemma}[Closest Approximation to Finite]
    Let $\{e_1,\ldots,e_n\}$ be orthonormal in a Euclidean space $(\mathcal{E},\inr{\cdot,\cdot})$ and $\mathcal{M}=\spn\{e_1,\ldots,e_n\}$.
    Then for $x\in\mathcal{E}$ we have that
    \begin{enumerate}[nl,r]
        \item $P_{\mathcal{M}}x=\sum_{j=1}^n \inr{x,e_j}e_j$ satisfies that $x-P_{\mathcal{M}}x\perp\mathcal{M}$ and hence $\norm{x}^2=\norm{P_{\mathcal{M}}}^2+\norm{x-P_{\mathcal{M}}x}^2$
        \item $d(x,\mathcal{M})=\norm{x-\sum_{j=1}^n(x,e_j)e_j}^{1/2}$
    \end{enumerate}
\end{lemma}
\begin{proof}
    \begin{enumerate}[nl,r]
        \item If $1\leq k\leq n$, we have
            \begin{equation*}
                \inr{x-P_{\mathcal{M}}x,e_k} = \inr{x,e_k}-\sum_{j=1}^n\inr{e,e_j}\inr{e_j,e_k}=\inr{x,e_k}-\inr{x,e_k}=0
            \end{equation*}
            and it follows that $x-P_{\mathcal{M}}x\perp\mathcal{M}$.
            Pythagoras' law provides the second formula.
        \item Endow $\F^n$ with the usual inner product $\norm{\cdot}_2$.
            By Cauchy-Schwarz, for $x\in\mathcal{E}$ and $\alpha\in\F^n$,
            \begin{equation*}
                \left\lvert\inr{((x,e_j))_{j=1}^n,\alpha}\right\rvert=\left\lvert\sum_{j=1}^n(x,e_j)\overline{\alpha}_j\right\rvert\leq\left(\sum_{j=1}^n|\inr{x,e_j}|^2\right)^{1/2}=\norm{P_{\mathcal{M}}x}\norm{\alpha}_2
            \end{equation*}
            so that
            \begin{align*}
                \norm{x-\sum_{j=1}^n\alpha_je_j}^2 &= \norm{x}^2-2\Re\sum_{i=1}^n\inr{x,e_j}\overline{\alpha}_j+\sum_{j=1}^n|\alpha_j|^2\\
                                                   &\geq\norm{x}^2-2\left\lvert\left((\inr{x,e_j})_{j=1}^n,\alpha\right)\right\rvert+\norm{\alpha}_2^2\\
                                                   &\geq\norm{x}^2-2\norm{P_{\mathcal{M}}x}\norm{\alpha}_2+\norm{\alpha}_2^2\\
                                                   &=\norm{x-P_{\mathcal{M}}x}^2+\left(\norm{P_{\mathcal{M}}x}-\norm{\alpha}_2\right)
            \end{align*}
            We get equality above if $x\perp\mathcal{M}$ or otherwise there is some $s\geq0$ so $\alpha_j=s\inr{x,e_j}$ for $j=1,\ldots,n$.
            Hence, in this case,
            \begin{equation*}
                \norm{x-\sum_{j=1}^n s\inr{x,e_j}e_j}^2=\norm{x-P_{\mathcal{M}}x}^2+\norm{P_{\mathcal{M}}x}^2(1-s)^2
            \end{equation*}
            which is minimized when $s=1$.
    \end{enumerate}
\end{proof}
\begin{remark}
    \begin{enumerate}[nl,r]
        \item The proof above shows that $P_{\mathcal{M}}x$ is the unique elemet of $\mathcal{M}$ satisfying $\dist(x,\mathcal{M})=\norm{x-P_{\mathcal{M}}x}$.
        \item It may be shown that $P_{\mathcal{M}}:\mathcal{E}\to\mathcal{E}$ is linear with $\im P_{\mathcal{M}}=\mathcal{M}$, $P_{\mathcal{M}}^2=P_{\mathcal{M}}$, and $\norm{P_{\mathcal{M}}}=1$ (in other words, this map is actually a projection operator)
    \end{enumerate}
\end{remark}
\begin{theorem}[Orthonormal Basis]
\item let $(\mathcal{E},\inr{\cdot,\cdot})$ be a Euclidean space, $E\subset\mathcal{E}$ an orthonormal set.
    Then the following are equivalent:
    \begin{enumerate}[nl,r]
        \item $\overline{\spn}E=\mathcal{E}$
        \item for $x\in\mathcal{E}$= $x=\sum_{e\in E}\inr{x,e}e=\lim_{F\in\mathcal{F}}\sum_{e\in F}\inr{x,e}e$, where $\mathcal{F}=\{F\subseteq E:|f|<\infty\}$, directed by inclusion (Bessel's identity)
        \item For $x,y\in\mathcal{E}$, $\inr{x,y}=\sum_{e\in E}\inr{x,e}\inr{e,y}$ (Parseval's identity).
    \end{enumerate}
\end{theorem}
\begin{proof}
    \imp{i}{ii}
    For $F\in\mathcal{F}$, let $\mathcal{E}_{F}=\spn F$, so that $\mathcal{E}_F\subseteq\mathcal{E}_{F'}$ if $F\subseteq F'$ in $\mathcal{F}$ and $\spn E=\bigcup_{F\in\mathcal{F}}\mathcal{E}_F$.
    Hence for $x\in\mathcal{E}$, we have
    \begin{equation*}
        0=\dist(x,\spn E)=\dist\left(x,\bigcup_{F\in\mathcal{F}}\mathcal{E}_F\right)=\inf_{F\in\mathcal{F}}\dist(x,\mathcal{E}_F)=\lim_{F\in\mathcal{F}}\dist(x,\mathcal{E}_F)
    \end{equation*}
    Thus by the f.d. approximation lemma, we have
    \begin{equation*}
        0 = \lim_{F\in\mathcal{F}}\dist(x,\mathcal{E}_F)=\lim_{F\in\mathcal{F}}\norm{x-\sum_{e\in F}\inr{x,e}e}
    \end{equation*}

    \impe{ii}{iii}
    We have
    \begin{align*}
        0&=\lim_{F\in\mathcal{F}}\norm{x-\sum_{e\in F}\inr{x,e}e}^2\\
         &=\lim_{F\in\mathcal{F}}\left(\norm{x}^2-2\Re\sum_{e\in F}\overline{\inr{x,e}}\inr{x,e}+\sum_{e\in F}\norm{\inr{x,e}}^2\right)\\
         &= \lim_{F\in\mathcal{F}}\left(\norm{x}^2-\sum_{e\in F}|\inr{x,e}^2|\right)\\
         &= \norm{x}^2-\sum_{e\in E}|\inr{x,e}|^2
    \end{align*}

    \imp{ii}{iv}
    Recall that $f_y=\inr{\cdot,y}\in\mathcal{E}^*$ so that
    \begin{equation*}
        \inr{x,y}=f_y\left(\lim_{F\in\mathcal{F}}\sum_{e\in F}\inr{x,e}e\right)=\lim_{F\in\mathcal{F}}\sum_{e\in F}\inr{x,e}f_y(e)=\sum_{e\in E}\inr{x,e}\inr{e,y}
    \end{equation*}
    
    \imp{iv}{ii}
    Take $x=y$.
    
    \imp{iii}{i}
    Obvious; $x=\lim_{F\in\mathcal{F}}\sum_{e\in F}\inr{x,e}e\in\overline{\spn E}$, i.e. $\mathcal{E}\subseteq\overline{\spn{E}}\subseteq\mathcal{E}$.
\end{proof}
\begin{definition}
    Any set $E\subset\mathcal{E}$ satisfying the above conditions is called a \textbf{orthonormal basis} for $\mathcal{E}$.
\end{definition}
\begin{theorem}[Gram-Schmidt]
    Let $(x_1,x_2,\ldots)$ be a linearly independent sequence in a euclidean space $(\mathcal{E},\inr{\cdot,\cdot})$.
    There exists an orthogonal sequence $(z_1,z_2,\ldots)$ which satisfies $\spn\{z_1,\ldots,z_n\}=\spn\{x_1,\ldots,x_n\}$ for $n=1,2,\ldots$ so that $\spn\{z_1,z_2,\ldots\}=\spn\{x_1,x_2,\ldots\}$.
\end{theorem}
\begin{proof}
    Let $\mathcal{E}_n=\spn\{x_1,\ldots,x_n\}$.
    We set
    \begin{align*}
        z_1&=x_1 & e_1 &=\frac{z_1}{\norm{z_1}}\\
        z_2&=x_2-P_{\epsilon_1}x_2 & e_2&=\frac{z_2}{\norm{z_2}}\\
           &\vdots\\
        z_{n+1}&=x_{n+1}-P_{\mathcal{E}_n}x_{n+1} & e_{n+1}&=\frac{z_{n+1}}{\norm{z_{n+1}}}
    \end{align*}
    where $P_{\mathcal{E}_n}x=\sum_{j=1}^n\inr{x,e_j}e_j$.
    Inductively, $z_n\in\mathcal{E}_n$ and $z_n\perp\mathcal{E}_k$ for $k=1,\ldots,n-1$.
    Hence each set $\{z_1,\ldots,z_n\}$is orthonormal and $\spn\{z_1,\ldots,z_n\}\subseteq\spn\{x_1,\ldots,x_n\}$ is of full dimension and hence equal.
\end{proof}
\begin{corollary}
    Any separable Euclidean space admits an orthonormal basis.
\end{corollary}
\begin{proof}
    Let $\{x_n\}_{n=1}^\infty$ be dense in $\mathcal{E}$.
    Let $n_1=\min\{n:x_n\neq 0\}$, and $n_{k+1}=\min\{n:x_n\notin\spn\{x_{n_1},\ldots,x_{n_k}\}\}$.
    Then $\{x_{n_1},x_{n_2},\ldots\}$ and normalize to get an orthonormal set $E=\{e_1,e_2,\ldots\}$ which satisfies $\overline{\spn E}=\overline{\spn}\{x_n\}_{n=1}^\infty=\mathcal{E}$.
\end{proof}
\begin{theorem}[Riesz Fischer]
    Let $(\mathcal{E},\inr{\cdot,\cdot})$ be a Euclidean space.
    Then $\mathcal{E}$ is a Hilbert space if and only if for any orthonrmal set $E$ and an $\alpha=(\alpha_e)_{e\in E}\in\ell_2(E)$, we have that $\sum_{e\in E}\alpha_ee\in\mathcal{E}$.
\end{theorem}
\begin{proof}
    \impr
    If $\alpha\in\ell_2(E)$ then $E_\alpha=\{e\in E:\alpha_e\neq 0\}$ is countable, and write $E_\alpha=(e_1,e_2,\ldots)$.
    If $m<n$, we have
    \begin{equation*}
        \norm{\sum_{k=1}^n\alpha_{e_k}e_k-\sum_{k=1}^m\alpha_{e_k}e_k}^2=\sum_{k=m+1}^n|\alpha_{e_k}|^2\leq\sum_{k=n+1}^\infty|\alpha_{e_k}|^2\to 0
    \end{equation*}
    so $x_\alpha=\sum_{k=1}^\infty \alpha_{e_k}e_k=\lim_{n\to\infty}\sum_{k=1}^n\alpha_{e_k}e_k$ converges.
    If $F\in\mathcal{F}$, $F\supseteq\{e_1,\ldots,e_n\}$, then
    \begin{equation*}
        \norm{x_\alpha-\sum_{e\in F}\alpha_ee}^2=\sum_{e=\{e_1,e_2,\ldots\}\setminus F}|\alpha_e|^2\leq\sum_{k=n+1}^\infty |\alpha_{e_k}|^2\to 0
    \end{equation*}
    so $x_\alpha=\sum_{e\in E}\alpha_ee=\lim_{F\in\mathcal{F}}\sum_{e\in F}\inr{x,e}e$.

    \impl
    Let $(x^{(n)})_{n=1}^\infty$ be Cauchy in $\mathcal{E}$.
    Let $\mathcal{M}=\overline{\spn\{x^{(n)}\}_{n=1}^\infty}\subset\mathcal{E}$ so $\mathcal{M}$ is separable and admits a countable orthonormal basis $E=(e_1,e_2,\ldots)$.
    Then we appeal to orthonormal basis to see that for any $x\in\mathcal{M}$, $\sum_{k=1}^\infty|\inr{x,e_k}|^2=\norm{x}^2<\infty$ and $x=\sum_{k=1}^\infty\inr{x,e_k}e_k$.

    Our present assumption show that $U:\ell_2(E)\to\mathcal{M}$ given by $U_\alpha=\sum_{k=1}^\infty \alpha_ke_k$ always converges in $\mathcal{M}\subseteq\mathcal{E}$.
    Then orthonormal basis theorem gives $\norm{U_\alpha}=\norm{\alpha}_2$ so $U$ is a surjective isometry.
    We let $\alpha^{(n)}=((x^{(n)},e_k))_{k=1}^\infty\in\ell_2(E)$, then $\norm{\alpha^{(n)}-\alpha^{(m)}}_2=\norm{U_\alpha^{(n)}-U_\alpha^{(m)}}=\norm{x^{(n)}-x^{(m)}}$ so $(\alpha^{(n)})_{n=1}^\infty$ is Cauchy and in $\ell_2(E)$ and hence admits a limit $\alpha$.
    Furthermore,
    \begin{equation*}
        \norm{\sum_{k=1}^\infty\alpha_ke_k-x^{(n)}}=\norm{U_\alpha-U_\alpha^{(n)}}=\norm{\alpha-\alpha^{(n)}}\to 0
    \end{equation*}
    as required.
\end{proof}
\begin{definition}
    If $\emptyset\neq S\subset\mathcal{E}$, $(\mathcal{E},\inr{\cdot,\cdot})$ a Euclidean space, we define its \textbf{perpindicular} by $S^\perp=\{y\in\mathcal{E}:(\inr{x,y}=0\text{ for any }x\in S)\}$.
\end{definition}
\begin{remark}
    \begin{enumerate}[nl,r]
        \item $S\subseteq T$ implies $T^\perp\subseteq S^\perp$
        \item $S^\perp=\bigcap_{x\in S}\ker f_x$ and is thus closed.
        \item $\overline{S}^\perp=S^\perp$, since $\overline{S}^\perp\subseteq\overline{S}^\perp$, and if $y\in S^\perp$ and $x\in\overline{S}$, then $x=\lim x_n$ with $x_n\in S$ so $\inr{x,y}=f_y(x)=f_y\lim x_n=\lim f_y(x_n)=\lim\inr{x_n,y}=0$.
        \item $(\overline{\spn}S)^\perp=S^\perp$.
            Notice that $(\spn S)^\perp=S^\perp$ (easy test) and use (iii)
        \item $\overline{\spn}S\cap S^\perp=\{0\}$.
    \end{enumerate}
\end{remark}
\begin{theorem}[Existence of Orthonormal Basis]
    Let $(H,\inr{\cdot,\cdot})$ be a Hilbert space.
    \begin{enumerate}[nl,r]
        \item Given an orthonormal set $E\subset H$, $P_E:H\to H$, $P_Ex=\sum_{e\in E}\inr{x,e}e$ satisfies
            \begin{equation*}
                \im P_E\subseteq\overline{\spn E}\text{ for }x\in H, x-P_Ex\in E^\perp
            \end{equation*}
        \item $H$ admits an orthonormal basis, i.e. an orthonormal set $M$ such that $\overline{\spn M}=H$.
    \end{enumerate}
\end{theorem}
\begin{proof}
    \begin{enumerate}[r]
        \item Let $\mathcal{F}=\{F\subseteq E:|F|<\infty\}$ be directed by inclusion, and for $F\in\mathcal{F}$, $\mathcal{E}_F=\spn F$.
            Then as in the proof of OMBT, we have for $x\in H$
            \begin{equation*}
                0\leq \dist(x,\spn E)^2=\lim_{F\in\mathcal{F}}\dist(x,\mathcal{E}_F)^2=\norm{x}^2-\sum_{e\in E}|\inr{x,e}|^2
            \end{equation*}
            so $\sum_{e\in E}|\inr{x,e}|^2\leq\norm{x}^2<\infty$.
            Thus appealing to Riesz-Fischer, $P_Ex=\sum_{e\in E}\inr{x,e}e$ converges in $H$.
            Since $P_Ex=\lim_{F\in\mathcal{F}}\sum_{e\in F}\inr{x,e},e$, we see that $P_Ex\in\overline{\spn}E$, so $\im P_E\subseteq\overline{\spn}E$.
            Moreover, if $e'\in E$, $f_{e'}=\inr{\cdot,e'}\in H^*$ so
            \begin{equation*}
                \inr{x-P_Ex,e'}=\inr{x,e'}-f_{e'}\left(\sum_{e\in E}\inr{x,e}e\right)=\inr{x,e'}-\sum_{e\in E}\inr{x,e}f_{e'}(e)=-
            \end{equation*}
            so $x-P_Ex\in E^\perp$.
        \item Let $\mathcal{O}=\{E\subseteq H:E\text{ is orthonormal}\}$, which is partially ordered by inclusion.
            Note that $\emptyset\in\mathcal{O}$ vacuously.
            If $\mathcal{C}\subseteq\mathcal{O}$ is a chain, then $\bigcup_{E\in\mathcal{C}}\in\mathcal{O}$ is an upper bound for $\mathcal{C}$.
            By Zorn' get a maximal element $M$.

            Suppose $\overline{\spn} M\subsetneq H$, and get $x\in H\setminus \overline{\spn} M$ and $y=x-P_Mx\in(\overline{\spn} M)^\perp\setminus\{0\}$.
            But then $M\subsetneq M\cup\{\frac{1}{\norm{y}}y\}$, violating maximality.
    \end{enumerate}
\end{proof}
\begin{corollary}
    If $H$ is a Hilber space with orthonormal basis $E$, then the map
    \begin{equation*}
        U:H\to\ell_2(E), Ux=(\inr{x,e})_{e\in E}
    \end{equation*}
    is a surjective isometry which respects inner products.
\end{corollary}
\begin{proof}
    We know $\norm{x}^2=\sum_{e\in E}|\inr{x,e}|^2=\norm{Ux}_2$ from ONBT.
    It is evident that $U$ is linear and $\im U$ is dense in $\ell_2(E)$ so that $U$ is surjective.
    Finally, if $x,y\in H$, then
    \begin{equation*}
        \inr{x,y}_H=\sum_{e\in E}\inr{x,e}\inr{e,y}=\sum_{e\in E}\inr{x,e}\overline{\inr{y,e}}=\inr{Ux,Uy}_{\ell_2(E)}
    \end{equation*}
    as required.
\end{proof}
\begin{remark}
    If each of $E$, $E'$ is an orthonormal basis for a Euclidean space $(\mathcal{E},\inr{\cdot,\cdot})$, then $|E|=|E'|$.
    We let $\mathfrak{k}$ be any countable dense subfield of $\F$.
    Then $D=\spn_{\mathfrak{k}}$, so $|D|=\aleph_0|E|=|E|$ when $|E|$ is infinite.
    Since for $e'$, $e''$ in $E'$, $\norm{e'-e''}=\sqrt{2}$, we have that any ball $e'+\frac{1}{\sqrt{2}}D(\mathcal{E})$ contains at least one element of $D$, and $d_{e'}\neq d_{e''}$ if $e'\neq e''$ in $E'$.
    This shows that $|E|\geq|E'|$.
    Likewise $|E'|\leq|E|$.
\end{remark}
\begin{corollary}[Orthogonal complementation]
    Let $(\mathcal{E},\norm{\cdot,\cdot})$ be a Euclidean space and $\mathcal{M}\subseteq\mathcal{E}$ a subspace which is complete with respect to the norm induced from $\inr{\cdot,\cdot}$, i.e. $(\mathcal{M},\inr{\cdot,\cdot})$ is a Hilbert space.
    Then there is a unique operator $P_{\mathcal{M}}=P:\mathcal{E}\to\mathcal{E}$ such that $\im P=\mathcal{M}$ and $\im(I-P)=\mathcal{M}^\perp$.
    Moreover,
    \begin{itemize}[nl]
        \item $P$ is linear
        \item $\norm{P}\leq 1$, $\norm{P}=1$ if $\mathcal{M}\neq\{0\}$
        \item $P^2=P$
        \item for $x,y\in\mathcal{E}$, $\inr{Px,y}=\inr{Px,Py}=\inr{x,Py}$.
    \end{itemize}
    Such an operator is called the \textbf{orthogonal projection}.
\end{corollary}
\begin{proof}
    The theorem above prvides an orthonormal basis $E$ for $\mathcal{M}$.
    Then $P_E$, as defined above, satisfies $\im P=\mathcal{M}$ and $\im(I-P)=\mathcal{M}^\perp$.
    Moreover, if $P$ satisfies those conditions, then for $x\in\mathcal{E}$,
    \begin{equation*}
        Px+x-Px=x=P_Ex+x-P_eX
    \end{equation*}
    so that
    \begin{equation*}
        Px-P_eX=[x-P_Ex]-[x-Px]\in\mathcal{M}\cap\mathcal{M}^\perp=\{0\}
    \end{equation*}
    so $Px=P_ex$.
    Then if $x,y\in\mathcal{E}$ and $\alpha\in\F$,
    \begin{equation*}
        P(x+\alpha y)+x+\alpha y-P(x+\alpha y)=x+\alpha y=Px+x-Px+\alpha[Py+y-Py]
    \end{equation*}
    so
    \begin{equation*}
        P(x+\alpha y)-[Px+\alpha py]=x-Px+\alpha[y-Py]-[x+\alpha y-P(x+\alpha y)]\in\mathcal{M}\cap\mathcal{M}^\perp=\{0\}
    \end{equation*}
    and we have linearity.

    If $x\in\mathcal{E}$, Pythagoras tells us that $\norm{x}^2=\norm{Px}^2+\norm{x-Px}^2$ so $\norm{Px}\leq\norm{x}$, i.e. $\norm{P}\leq 1$.
    If $e'\in E$, $Pe'=P_Ee'=\sum_{e\in E}\inr{e',e}e=e'$, so $P|_{\spn E}=\id$ and by uniqueness of extension of bounded linear functionals (uniformly continuous), we see that $P|_{\mathcal{M}}=\id_{\mathcal{M}}$.
    This shows that if $\mathcal{M}\neq\{0\}$, $\norm{P}=1$ and $P=P^2$.
    Furthermore, this also shows that $\im P=\mathcal{M}$.
    Finally,
    \begin{equation*}
        \inr{Px,y}=\inr{Px,Py+y-Py}=\inr{Px,Py}
    \end{equation*}
    and likewise $\inr{x,Py}=\inr{Px,Py}$.
\end{proof}
\begin{corollary}
    Let $H$ be a Hilbert space.
    \begin{enumerate}[nl,r]
        \item If $\mathcal{M}$ is a closed subspace, then $(\mathcal{M}^\perp)^\perp=\mathcal{M}$.
        \item If $\emptyset\neq S\subset H$, then $(S^\perp)^\perp=\overline{\spn}S$.
    \end{enumerate}
\end{corollary}
\begin{proof}
    \begin{enumerate}[nl,r]
        \item We have $\mathcal{M}\subseteq\mathcal{M}^{\perp\perp}$ and $\mathcal{M}$ is complete and thus admis an orthogonal projection $P_{\mathcal{M}}H\to H$ with $\im P_{\mathcal{M}}=\mathcal{M}$ and $\im(I-P_{\mathcal{M}})=M^\perp$.
            Now if $x\in\mathcal{M}^{\perp\perp}$, $P_{\mathcal{M}}x\in\mathcal{M}$ so that $x-P_{\mathcal{M}}x\in\mathcal{M}^{\perp\per}+\mathcal{M}=\mathcal{M}^{\perp\perp}$ so that $x-P_{\mathcal{M}}x\in\mathcal{M}^\perp$.
            Thus
            \begin{equation*}
                x-P_{\mathcal{M}}x\in\mathcal{M}^{\perp\perp}\cap\mathcal{M}^\perp=\{0\}
            \end{equation*}
            so that $x\in P_{\mathcal{M}}x\in\mathcal{M}$.
            Hence $\mathcal{M}^{\perp\perp}\subseteq\mathcal{M}$.
        \item We have $(\overline{\spn} S)^\perp=S^\perp$ and apply (i).
    \end{enumerate}
\end{proof}
\begin{theorem}[Riesz-Fréchet]
    If $H$ is a Hilbert space and $f\in H^*$, then there is a unique $x_0\in H$ such that $f=f_{x_0}$; i.e. $f(x)=(x,x_0)$ for all $x\in H$.
\end{theorem}
\begin{proof}
    If $f=0$, let $x_0=0$.
    If $f\neq 0$, $\ker f\subsetneq H$ so $(\ker f)^{\perp\perp}=\ler f$, so $(\ker f)^\perp\neq\{0\}$.
    If $x_1,x_2\in(\ker f)^\perp$, then $f(x_2)x_1-f(x_1)x_2\in(\ker f)^\perp\cap\ker f=\{0\}$, so that $\dim(\ker f)^\perp=1$ and $(\ker f)^\perp=\F x_1$.
    But then $f_{x_1}=\inr{\cdot,x_1}$ has $\ker f_{x_1}=(\F x_1)^\perp=(\ker f)^{\perp\perp}=\ker f$, so there is $\alpha\in\F$ so $f=\alpha f_{x_1}=f_{\overline{\alpha}x_1}$.
    Set $x_0=\overline{\alpha}x_1$.

    Uniqueness holds since $x\mapsto f_x:H\to H^*$ is an isometry and thus injective.
\end{proof}
\begin{remark}
    \begin{enumerate}[nl,r]
        \item Many results above may fail in a non-complete Euclidean space.
            Consider $(\ell,\inr{\cdot,\cdot})$ where $\ell$ is the space of finitely supported sequences.
            Define $f:\ell\to\F$ by $f(x)=\sum_{k=1}^\infty\frac{1}{k}x_k$.
            By Hölder, $|f(x)|\leq\left(\sum_{k=1}^\infty\frac{1}{k^2}\right)\norm{x}_2$ so that $f\in\ell^*$.
            If there were $x^{(0)}\in\ell$ so that $f=f_{x^{(0)}}$ for some $x^{(0)}\in(\ker f)^\perp\setminus\{0\}$, we would then have $x_k^{(0)}=(e_k,x^{(0)})=\frac{1}{k}$, which is non-zero for infinitely many $k$, giving a contradiction.
            In fact, $(\ker f)^{\perp}=\{0\}$ so that $(\ker f)^{\perp\perp}=\ell$.
        \item Let $(\mathcal{E},\inr{\cdot,\cdot})$ be a Euclidean space.
            Let $H=\overline{\mathcal{E}}$ be the metrical completion with respect to $\norm{x}_2$.
            If $x,y\in H$, then $x=\lim x_n=\lim x_n'$ with $x_n,x_n'\in\mathcal{E}$, and $y=\lim y_n=\lim y_n'$ similarly.
            Then
            \begin{align*}
                \left\lvert\inr{x_n,y_n}-\inr{x_n,y_n}\right\rvert&\leq |\inr{x_n,y_n}-\inr{x_n,y_m}|+|\inr{x_n,y_m}-\inr{x_n,y_n}|\\
                                                                  &\leq\norm{x_n}\norm{y_n-y_m}+\norm{x_n-x_m}\norm{y_m}
            \end{align*}
            so that $(\inr{x_n,y_n})_{n=1}^\infty\subset\F$ is Cauchy, and thus admits a limit.
            Moreover, $|\inr{x_n,y_n}-\inr{x_n',y_n'}|\leq\norm{x_n}\norm{y_n-y_n'}+\norm{x_n-x_n'}\norm{y_n'}$.
            Thus, $\inr{x,y}=\lim_{n\to\infty}\inr{x_n,y_n}=\lim_{n\to\infty}\inr{x_n',y_n'}$ is well-defined on $H\times H$.
            It is straightforward to verify that this is an inner product, and $\norm{x}=\lim_{n\to\infty}\norm{x_n}=\inr{x,x}^{1/2}$.
            Thus the completion of a Euclidean space is a Hilber space.
        \item As a consequence of (ii), we have $\mathcal{E}^*=\{f_x:x\in H\}$ where $H=\overline{\mathcal{E}}$, as above.
            Furthermore, $\overline{\mathcal{E}}\cong H^{**}$.
        \item If $H$ is a Hilbert space, the map $f\mapsto f_x$ from $H\to H^*$ is
            \begin{itemize}[nl]
                \item a conjugate linear map: $f_{x+\alpha y}=f_x+\overline{\alpha}f_y$
                \item an isometry: $\norm{f_x}=\norm{x}$
            \end{itemize}
    \end{enumerate}
\end{remark}
\section{Adjoint Operators}
\begin{definition}
    Let $X,Y$ be vector spaces over $\F$, and $T\in\mathcal{L}(X,Y)$.
    Define the \textbf{adjoint} of $T$, $T^*:Y'\to X'$ by $T^*f=f\circ T$.
\end{definition}
Notice that $T^*\in \mathcal{L}(Y',X')$.
\begin{proposition}
    Let $X,Y,Z$ be normed spaces, $T\in\mathcal{B}(X,Y)$ and $S\in\mathcal{B}(Y,Z)$.
    Then
    \begin{enumerate}[nl,r]
        \item $T^*\in\mathcal{B}(Y^*,X^*)$ with $\norm{T^*}=\norm{T}$
        \item $T\mapsto T^*:\mathcal{B}(X,Y)\to\mathcal{B}(Y^*,X^*)$ is linear
        \item $T^{**}:=(T^*)^*$ satisfies $T^{**}\in\mathcal{B}(X^{**},Y^{**})$ and $T^{**}\hat x=\widehat{Tx}$.
        \item $(S\circ T)^*=T^*\circ S^*\in\mathcal{B}(Z^*,X^*)$.
    \end{enumerate}
\end{proposition}
\begin{proof}
    \begin{enumerate}[nl,r]
        \item[(i),(iii)] If $f\in Y^*$, then
            \begin{equation*}
                \norm{T^*f}=\sup\{|T^*f(x)|:x\in B(X)\}\leq\sup\{\norm{f}\norm{Tx}:x\in B(X)\}\leq\norm{f}\norm{T}
            \end{equation*}
            so $\norm{T^*}\leq\norm{T}$.
            If $x\in X$ and $f\in Y^*$, then
            \begin{equation*}
                T^{**}\hat{x}(f)=\hat{x}(T^*f)=T^*f(x)=f(Tx)=\widehat{Tx}(f)
            \end{equation*}
            so that $\norm{T}=\norm{T^{**}|_{\hat X}}\leq\norm{T^{**}}\leq\norm{T^*}$.
        \item[(ii)] Immediate.
        \item[(iv)] Immediate.
    \end{enumerate}
\end{proof}
\begin{remark}
    If $H,K$ are Hilbert spaces and $T\in\mathcal{B}(H,K)$, then we define for $x\in K$, $T^*x$ by $f_{T^*x}=T^*f_x$.
    Notice that (i) and (iv) hold in this setting.
    However, (ii) is replaced by $T\mapsto T^*$ is conjugate linear.
    Notice that $T^*$ satisfies $(Tx,y)=(x,T^*y)$ for $x,y\in H$.
\end{remark}
\begin{theorem}[Kernel-Annhilator]
    If $X$ and $Y$ are Banach spaces, $T|in B(X,Y)$, then $\ker T=[\im(T^*)]_a$ and $\ker(T^*)=(\im T)^a$.
\end{theorem}
\begin{proof}
    We have
    \begin{equation*}
        \ker T = \{x\in X:Tx=0\}=\{x\in X:T^*g(x)=g(Tx)=0\text{ for all }x\in X\}=[\im(T^*)]_a
    \end{equation*}
    and
    \begin{equation*}
        \ker(T^*)= \{g\in Y^*:T^*g=0\}=\{g\in Y^*:g(Tx)=T^*g(x)=0\text{ for all }x\in X\}=[\im(T)]^a
    \end{equation*}
\end{proof}
\begin{remark}
    If $T\in \mathcal{B}(H,K)$ where $H,K$ are Hilbert spaces, then $\ker T=(\im T^*)^\perp$, identifying $T^{**}=T$ since Hilbert spaces are reflexive.
\end{remark}
\begin{theorem}[Characterization of Invertibility]
    Let $X,Y$ be Banach spaces, $T\in B(X,Y)$.
    Then TFAE:
    \begin{enumerate}[nl,r]
        \item $T$ is invertible
        \item $T^*$ is invertible
        \item $\overline{\im T}=Y$ and $\inf\{\norm{Tx}:x\in S(X)\}>0$, we say that $T$ is \textbf{bounded below}, and
        \item both $T$ and $T^*$ are bounded below.
    \end{enumerate}
\end{theorem}
\begin{proof}
    \imp{i}{ii}
    Let $T^{-1}\in \mathcal{B}(Y,X)$, so $I_Y=TT^{-1}$, $I_X=T^{-1}T$.
    Then $(T^{-1})^*T^*=(TT^{-1})^*=I_Y^*=I_{Y^*}$ and likewise for the reverse.

    \imp{ii}{iii}
    By the kernel-annhilator theorem, we have $(\im T)^a=\ker(T^*)=\{0\}$ in $Y^*$, so by annhilator-preannhilator, $\overline{\im T}=(\im T)^a_a=\{0\}_a=Y$.
    Now if $x\in S(X)$, find $f\in X^*$ so $f(x)=\norm{x}=1=\norm{f}$ (by Hahn-Banach).
    Then
   \begin{equation*}
        1=f(x)=[T^*(T^*{-1})f](x)=[(T^*)^{-1}f](Tx)\leq\norm{(T^*)^{-1}f}\norm{Tx}\leq\norm{(T^*)^{-1}}\norm{Tx}
    \end{equation*}
    so that $\norm{Tx}\geq\frac{1}{\norm{(T^*)^{-1}}}>0$ and $T$ is bounded below.

    \imp{iii}{i}
    Let $T$ be bounded below, and set $c=\inf\{\norm{Tx}x:x\in S(X)\}>0$, then for $x\in X\setminus\{0\}$, $\norm{Tx}=\norm{x}\norm{T\left(\frac{1}{\norm{x}}x\right)}\geq c\norm{x}$.
    If $y\in\overline{\im T}$, then $y=\lim y_n$, each $y_n=Tx_n\in\im T$.
    Then
    \begin{equation*}
        \norm{x_n-x_m}\leq\frac{1}{c}\norm{Tx_n-Tx_m}
    \end{equation*}
    so $(x_n)_{n=1}^\infty$ is Cauchy as $(Tx_n)_{n=1}^\infty$ converges.
    THen $x=\lim x_n\in X$ and by continuity of $T$, $y=Tx\in\im T$.
    Notive as well that bounded below implies $\ker T=\{0\}$.

    We assume that $T$ is bounded below and $\im T=\overline{\im T}=Y$, so $T$ is bijective, hence invertible.

    \imp{i,ii}{iv}
    Use (iii)

    \imp{iv}{iii}
    We suppose that $T$ is bounded below, and so is $T^*$.
    Then $\{0\}=\ker(T^*)$ in $Y^*$, so $Y=\{0\}_a=\ker(T^*)_a=\overline{\im T}$ and $T$ is bounded below provides $\im T=\overline{\im T}=Y$, so $\ker T=\{0\}$.
\end{proof}
\begin{remark}
    Reasons why $T\in\mathcal{B}(X,Y)$ is not intervible: $\ker T\supsetneq\{0\}$, $\im T\subsetneq Y$, $T$ is not bounded below.
\end{remark}
\begin{example}
    Let $T:\ell_p\to\ell_p$ be given by $T(x_n)_{n=1}^\infty=\left(\frac{1}{n}x_n\right)_{n=1}^\infty$, so $\norm{T}-1$.
    Notice that $\ker T=\{0\}$ and $\overline{\im T}=\ell_p$.
    However, $T$ is not bounded below.
\end{example}
\section{Spectral Theory for Bounded operators}
Let $X$ be a $\C-$Banach space, and $\mathcal{B}(X)=\mathcal{B}(X,X)$.
\begin{definition}
    If $T\in\mathcal{V}(X)$, we define the \textbf{resolvent} of $T$ by $\rho(T)=\{\lambda\in\C:\lambda I-T\text{ is invertible}\}$.
    Then the \textbf{spectrum} of $T$, $\sigma(T)$, is given by $\sigma(T)=\C\setminus\rho(T)$.
    We define the \textbf{point spectrum} $\sigma_p(T)=\{\lambda\in \C:\ker(\lambda I-T)\supsetneq\{0\}\}$, so $\sigma_p(T)\subseteq\sigma(T)$.
\end{definition}
\begin{example}
    \begin{enumerate}[nl,r]
        \item If $X$ is finite dimensional, then $\sigma(T)=\sigma_p(T)$.
        \item Let $1\leq p<\infty$ and define $S:\ell_p\to\ell_p$ by $S(x_1,x_2,\ldots)=(0,x_1,x_2,\ldots)$.
            Notive that $S$ is linear and $\norm{Sx}_p=\norm{x}_p$, so $\norm{S}=1$.
            Also, $\ker S=\{0\}$.
            Suppose $\lambda\in\sigma_p$, so there is $x\in\ker(\lambda I-S)\setminus\{0\}$.
            We let $k=\min\{n\in\N:x_n\neq 0\}$, we see $0=(S_x)_k=\lambda_{x_k}$, but $\ket S=\{0\}$, so $0\notin\sigma_p(T)$, but hence no $\lambda$ as above exists, so $\sigma_p(X)=\emptyset$.
    \end{enumerate}
\end{example}
For any $T\in\mathcal{B}(X)$, is $\sigma(T)\neq\emptyset$?

Let
\begin{equation*}
    \mathcal{G}(X)=\{T\in\mathcal{B}(X):T\text{ is invertible}\}
\end{equation*}
Notive that if $S,T\in\mathcal{G}(X)$, then $(ST)^{-1}=T^{-1}S^{-1}$, so $\mathcal{G}(X)$ is a group in $\mathcal{B}(X)$ with identity $I$.
Note that $\mathcal{B}(X)$ is compelte, and if $S,T\in\mathcal{B}(X)$, then $\norm{ST}\leq\norm{S}\norm{T}$, so that $S\mapsto ST$ and $S\mapsto TS$ for some $T\in\mathcal{B}(X)$ are continuous.
\begin{theorem}[Inversion]
    \begin{enumerate}[nl,r]
        \item If $T\in D(X)$, then $\sum_{k=0}^\infty T_k$ converges in $\mathcal{B}(X)$, and $\sum_{k=0}^\infty T^k=(I-T)^{-1}$
        \item If $S,T\in\mathcal{B}(X)$ such that $S\in\mathcal{G}(X)$ and $\norm{T-S}<\frac{1}{\norm{S^{-1}}}$, then $T\in\mathcal{G}(X)$ with $T^{-1}=S^{-1}+\sum_{k=1}^\infty[S^{-1}(S-T)]^kS$.
    \end{enumerate}
    Thus, we find that $\mathcal{G}(X)$ is open in $\mathcal{B}(X)$ and $T\mapsto T^{-1}$ on $\mathcal{G}(X)$ is continuous.
\end{theorem}
\begin{proof}
    \begin{enumerate}[nl,r]
        \item Let $S_n=\sum_{k=0}^\infty T^k$, so for $m<n$, we have
            \begin{equation*}
                \norm{S_n-S_m}\leq\sum_{k=m+1}^\infty\norm{T^k}\leq\sum_{k=m+1}^n\norm{T}^k=\frac{\norm{T}^{m+1}}{1-\norm{T}}\to 0
            \end{equation*}
            since $\norm{T}<1$, so $(S_n)_{n=1}^\infty$ is Cauchy, and thus convergent in $\mathcal{B}(X)$.
            Also,
            \begin{equation*}
                (I-T)S_n=I-T^{n+1}\to U\text{ as }T^{n+1}\to 0
            \end{equation*}
            since $\norm{T}<1$.
            Similarly, $S_n(I-T)\to I$, so that $S=\sum_{k=0}^\infty T^k$ has $S(I-T)=I=(I-T)S$.
        \item We have $\norm{S^{-1}S-T)}\leq\norm{S^{-1}}\norm{S-T}<1$ so by (i)
            \begin{equation*}
                T=S-(S-T)=S[I-S^{-1}(S-T)]\in\mathcal{G}(X)
            \end{equation*}
            Furthermore,
            \begin{equation*}
                T^{-1}=[I-S^{-1}(S-T)]^{-1}S^{-1}=\sum_{k=0}^\infty[S^{-1}(S-T)]^kS^{-1}
            \end{equation*}
    \end{enumerate}
    In particular, we see that for $S\in\mathcal{G}(X)$, $S+\frac{1}{\norm{S^{-1}}}D(X)\subseteq\mathcal{G}(X$, so (a) holds.
    From (ii), we see that
    \begin{equation*}
        \norm{T^{-1}-S^{-1}}\leq\sum_{k=1}^\infty\norm{[S^{-1}(T-S)]^kS}\leq\sum_{k=1}^\infty\norm{S^{-1}}^k\norm{T-S}^k\norm{S^{-1}}=\frac{\norm{S^{-1}}^2\norm{T-S}}{1-\norm{S^{-1}}\norm{T-S}}
    \end{equation*}
    so that $\lim_{T\so S}\norm{T^{-1}-S^{-1}}=0$.
\end{proof}
\begin{definition}
    Suppose $\mathcal{B}$ is a $\C-$Banach space, $U\subseteq\C$ and $F:U\to\mathcal{B}$.
    We say that $F$ is \textbf{holomorphic} if for any $z_0\in U$,
    \begin{equation*}
        F'(z_0)=\lim_{z\to z_0}\frac{1}{z-z_0}[F(z)-F(z_0)]
    \end{equation*}
\end{definition}
\begin{remark}
    Just as in calculus, a holomorphic funtion is continuous on its domain.
\end{remark}
\begin{proposition}
    Let $T\in\mathcal{B}(X)$.
    Then
    \begin{enumerate}[nl,r]
        \item $\rho(T)$ is open in $\C$
        \item $R(z)=R_T(z)=(zI-T)^{-1}$ defines a holomorphic function on $\rho(T)$, called the \textbf{resolvent function}, and
        \item $\sigma(T)\subseteq\norm{T}\overline{\mathbb{D}}$, and for $|z|>\norm{T}$, $R(z)\leq\frac{1}{|z|-\norm{T}}$.
    \end{enumerate}
\end{proposition}
\begin{proof}
    \begin{enumerate}[nl,r]
        \item Define $F:\C\to\mathcal{B}(X)$ by $F(z)=zI-T$.
            Then $F$ is continuous and $\rho(T)=F^{-1}(\mathcal{G}(X))$.
        \item If $z,z_0\in\rho(T)$, then
            \begin{align*}
                R(z)-R(z_0) &= (zI-T)^{-1}-(z_0I-T)^{-1}=(zI-T)^{-1}[(z_0I-T)-(zI-T)](z_0I-T)^{-1}\\
                            &= (z_0-z)(zI-T)^{-1}(z_0I-T)^{-1}
            \end{align*}
            Hence
            \begin{equation*}
                \frac{1}{z-z_0}[R(z)-R(z_0)]=-(zI-T)^{-1}(z_0I-T)^{-1}\to -(z_0I-T)^{-2}
            \end{equation*}
            by continuity of inversion.
        \item If $|z|>\norm{T}$, then $\norm{\frac{1}{z}T}<1$ so $zI-T=z(I-\frac{1}{z}T)\in\mathcal{G}(X)$, so $\sigma(T)\subseteq\norm{T}\overline{\mathbb{D}}$.
            Furthermore, for $|z|>\norm{T}$, we have
            \begin{equation*}
                R(z)=(zI-T)^{-1}=\fra{1}{z}(I-\frac{1}{z}T)^{-1}=\frac{1}{z}\sum_{k=0}^\infty\frac{1}{z^k}T^k
            \end{equation*}
    \end{enumerate}
\end{proof}
\begin{theorem}[Liouville]
    If $f:\C\to\C$ is holomorphic and bounded, then $f$ is constant.
\end{theorem}
\begin{proof}
    Apply Cauchy integral formula.
\end{proof}
\begin{theorem}[Liouville for Banach Spaces]
    If $F:\C\to\mathcal{B}$ is holomorphic and bounded, then $F$ is constant.
\end{theorem}
\begin{proof}
    Let $f\in\mathcal{B}^*$ and let $F_f=f\circ F:\C\to\C$.
    Notice that for $z,z_0\in\C$,
    \begin{equation*}
        \frac{F_f(z)-F_f(z_0)}{z-z_0}=f\left(\frac{1}{z-z_0}[F(z)-F(z_0)]\right)\to f(F^1(z_0))
    \end{equation*}
    by linearity and continuity of $f$, and hence $F_f'=f\circ F'$.
    Also, if $F$ is bounded, then for $z\in\C$, $|F_f(z)|=|f(F(z))|\leq\norm{f}\norm{F(z)}$ shows that $F_f$ is bounded, so by Liouville's theorem, is constant.
    In particular, if $z,z'\in\C$, $f(F(z)-F(z'))=F_f(z)-F_f(z')=0$.
    Thus by Hahn-Banach, we have $F(z)=F(z')$ for any $z,z'\in\C$, so $F$ is constant.
\end{proof}
\begin{theorem}
    If $T\in\mathcal{B}(X)$, then $\sigma(T)\neq\emptyset$ and compact.
\end{theorem}
\begin{proof}
    If $\sigma(T)=\emptyset$, then $R:\C\to\mathcal{B}(X)$ is holomorphic.
    Hence, as $\norm{T}\overline{\mathbb{D}}$ is compact in $\C$, $R$ is bounded on $\norm{T}\overline{\mathbb{D}}$; and if $|z|>\norm{T}$, we have
    \begin{equation*}
        \norm{R(z)}\leq\frac{1}{|z|-\norm{T}}\to 0
    \end{equation*}
    It follows that $R$ is bounded on $\mathcal{B}(X)$, and hence constant, and thus $R=0$.
    But $R(z)(zI-T)=I$, a contradiction.

    Moreover, $\rho(T)=\C\setminus\sigma(T)$ is open, and $\sigma(T)\subseteq\norm{T}\overline{\mathbb{D}}\subset\C$.
    Thus $\sigma(T)$ is a non-empty compact set.
\end{proof}
\begin{corollary}[Joke]
    $\C$ is algebraically closed.
\end{corollary}
\begin{proof}
    Let $p(x)\in\C[x]$ be an arbitrary irreducible polynomial with $p(x)=(x-r_1)\cdots(x-r_n)$ for some $r_i\in\overline{\C}$.
    Consider the operator $T:\C^n\to\C^n$ with diagonal $r_1,\ldots,r_n$ and hence characteristic polynomial $p(x)$.
    Then $\emptyset\neq\sigma(T)=\sigma_p(T)=\{x\in\C:p(x)=0\}$, so $p$ has some root in $\C$, so that $\deg p=1$.
\end{proof}
\begin{proposition}
    \begin{enumerate}[nl,r]
        \item If $X$ is a (non-Hilbert) Banach space, then $\sigma(T^*)=\sigma(T)$.
        \item If $H$ is a Hilbert space, $T\in\mathcal{B}(H)$, then $\sigma(T^*)=\{\overline{\lambda}:\lambda\in\sigma(T)\}$.
    \end{enumerate}
\end{proposition}
\begin{proof}
    \begin{enumerate}[nl,r]
        \item $(\lambda I_X-T)^*=\lambda I_{X^*}-T^*$ and is invertible if and only if $\lambda I_X-T$ is invertible
        \item Same.
    \end{enumerate}
\end{proof}
\begin{definition}
    We define the \textbf{point spectrum} $\sigma_p(T)=\{\lambda\in\C:\ker(\lambda I-T)\neq\{0\}\}$, the \textbf{approximate point spectrum} $\sigma_{ap}(T)=\{\lambda\in\C:\lambda I-T\text{ is not bounded below}\}$, and the \textbf{compression spectrum} $\sigma_{com}(T)=\{\lambda\in\C:\overline{\im}(\lambda I-T)\subsetneq X\}$.
\end{definition}
\begin{remark}
    \begin{enumerate}[nl,r]
        \item $\sigma_p(T)\subseteq\sigma_{ap}(T)$.
        \item We have $[\ran(\lambda I-T)]^a=\ker(\lambda I-T^*)$ by kernel-annhilator so $\overline{\im}(\lambda I-T)=\ker(\lambda I-T^*)$ by annhilator-preannhilator, so that $\sigma_{com}(T)=\sigma_p(T^*)$.
    \end{enumerate}
\end{remark}
\begin{lemma}
    If $(T_n)_{n=1}^\infty\subset\mathcal{G}(X)$ satisfies that
    \begin{itemize}[nl]
        \item $T=\lim_{n\to\infty}T_n$
        \item $M=\sup_{n\in\N}\norm{T_n^{-1}}<\infty$
    \end{itemize}
    then $T\in\mathcal{G}(X)$.
\end{lemma}
\begin{proof}
    Since $M>0$, for sufficiently large $n$, we have $\norm{T-T_n}\leq\frac{1}{M}\leq\frac{1}{\norm{T_n^{-1}}}$, so $T\in\mathcal{G}(X)$ by inversion theorem.
\end{proof}
\begin{proposition}
    \begin{enumerate}[nl,r]
        \item $\partial\sigma(T)\subseteq\sigma_{ap}(T)$
        \item $\sigma_{ap}(T)$ is closed
    \end{enumerate}
    Hence $\sigma_{ap}(T)$ is always a non-empty closed subset of $\C$.
\end{proposition}
\begin{proof}
    \begin{enumerate}[nl,r]
        \item Let $\lambda\in\partial\sigma(T)$, so there is $(\lambda_n)_{n=1}^\infty\subset\rho(T)=\C\setminus\sigma(T)$ such that $\lambda=\lim_{n\to\infty}\lambda_n$.
            Then $\norm{(\lambda_nI-T)-(\lambda I-T)}=|\lambda_n-\lambda|\to 0$, but $\lambda I-T\notin\mathcal{G}(X)$, so by the lemma, $\sup_{n\in\N}\norm{(\lambda_nI-T)^{-1}}=\infty$.
            Passing to a subsequence if necessary, we may suppose $\lim_{n\to\infty}\norm{(\lambda_n I-T)^{-1}}=\infty$.

            For each index $n$, let $x_n\in S(X)$ so $\alpha_n=\norm{(\lambda_n I-T)^{-1}x_n}>\norm{(\lambda n-T)^{-1}}-\frac{1}{n}$ so $\lim_{n\to\infty}\alpha_n=\infty$.
            Then $y_n=\frac{1}{\alpha_n}(\lambda_n I-T)^{-1}x_n$, so $y_n\in S(X)$ and
            \begin{align*}
                (\lambda I-T)y_n&=(\lambda_n I-T)y_n+(\lambda-\lambda_n)y_n\\
                                &= \frac{1}{\alpha_n}x_n+(\lambda-\lambda_n)y_n\to 0
            \end{align*}
            so $\lambda I-T$ is not bounded below.
        \item If $\lambda=\lim_{n\to\infty}\lambda_n$, each $\lambda_n\in\sigma_{ap}(T)$, for each $n$ find $x_n\in S(X)$ so $\norm{(\lambda_n I-T)x_n}<\frac{1}{n}$.
            Then
            \begin{align*}
                \norm{(\lambda I-T)x_n} &\leq\norm{(\lambda_n I-T)x_n}+\norm{(\lambda-\lambda_n)x_n}<\frac{1}{n}+|\lambda-\lambda_n|\to 0
            \end{align*}
            so $\lambda I-T$ is not bounded below.
    \end{enumerate}
\end{proof}
\begin{example}
    Let $S\in B(\ell_p)$, $1<p<\infty$, where $S(x_1,x_2,\ldots)=(0,x_1,x_2,\ldots)$, the \textit{unilateral shift} map.
    It is immediate that $\norm{Sx}_p=\norm{x}_p$ for $x\in\ell_p$, so $\norm{S}=1$.
    Recall that $\ell_p^*\cong\ell_q$ where $p,q$ are conjugate.
    Define a bilinear form on $\ell_p\times\ell_q$ by $\langle x,y\rangle=\sum_{k=1}^\infty x_ky_k$.
    We compute $\langle x,S^*y\rangle=\langle Sx,y\rangle=\sum_{k=1}^\infty x_ky_{k+1}=\langle x,(y_2,y_3,\ldots)\rangle$ so $S^*(y_1,y_2,\ldots)=(y_2,y_3,\ldots)$.
    Recall that $\sigma_p(S)=\emptyset$.
    However, if $\lambda\in\mathbb{D}$, let $y_\lambda=(1,\lambda,\lambda^2,\ldots)\in\ell_q$.
    Then $S^*(y_\lambda)=\lambda\cdot y_\lambda$.
    Hence $\sigma_p(S^*)\supseteq\mathbb{D}$.
    Furthermore, if $\lambda\in\sigma_p(S^*)$ and $y\in\ker(\lambda I-S^*)$, then $\lambda^ny=(S^*)^n\to 0$ so $\lambda^n\to 0$, forcing $|\lambda|<1$.
    Thus
    \begin{equation*}
        \mathbb{D}=\sigma_p(S^*)\subseteq\sigma(S^*)=\sigma(S)\subseteq\overline{D}
    \end{equation*}
    since $\norm{S}=1$, and since $\sigma(S)$ is compact, $\sigma(S)=\overline{D}$.

    We know that $\sigma_{ap}(S)\supseteq\partial\sigma(S)=\mathbb{S}$.
    If $\lambda\in\mathbb{D}$, then for $x\in\ell_p$, $\norm{(S-\lambda I)x}_p\geq\norm{Sx}_p-\norm{\lambda x}_p=(1-|\lambda|)\norm{x}_p$, so $S-\lambda I$ is bounded below.
    Thus $\sigma_{ap}(S)\cap\mathbb{S}=\emptyset$, so $\sigma_{ap}(S)=\mathbb{S}$.
    In conclusion,
    \begin{align*}
        \sigma(S)&=\mathbb{D} & \sigma_p(S)&=\emptyset\\
        \sigma_{ap}(S)&=\mathbb{S}=\partial\sigma(S) & \sigma_{com}(S)&=\sigma_p(S^*)=\mathbb{D}\\
        \sigma(S^*) &= \overline{\mathbb{D}} & \sigma_p(S^*)^=\mathbb{D}\\
        \sigma_{ap}(S^*)&=\partial\sigma(S^*)\cup\sigma_p(S^*)=\overline{\mathbb{D}} & \sigma_{com}(S^*)=\emptyset
    \end{align*}
\end{example}
\begin{remark}
    Let $\sigma_p(T)$, $\sigma_{com}(T)$ may be empty, and if non-empty need not be closed.
\end{remark}
\begin{remark}
    If $p=1$ and $S\in B(\ell_1)$ is the unilateral shift, as above, and $L\in\ell_\infty^*$ be a Banach limit.
    Then
    \begin{equation*}
        S^{**}L=L\circ S^L
    \end{equation*}
    so $\sigma_p(S^{**})\ni 1$.
    Thus $\sigma_{com}(S^*)=\sigma_p(S^{**})\neq\emptyset$.
\end{remark}
\begin{theorem}[Spectral Mapping]
    Let $T\in \mathcal{B}(X)$, $p\in\C[x]$, then $\sigma(p(T))=p(\sigma(T))$.
\end{theorem}
\begin{proof}
    We may assume that $p\neq 0$.
    Let $\lambda_0\in\C$ and write $p(t)-\lambda_0=\alpha\prod_{k=1}^n(t-\lambda_k)$.
    Then
    \begin{equation*}
        p(T)-\lambda_0 = \alpha\prod_{k=1}^n(T-\lambda_kI)
    \end{equation*}
    Thus $p(T)-\lambda_0I\notin\mathcal{G}(X)$ if and only some $T-\lambda_I\notin\mathcal{G}(X)$, so $\lambda_0\in\sigma(p(T))$ if and only if at least one $\lambda_k\in\sigma(T)$ if and only if $p(\lambda)-\lambda_0=0$ for some $\lambda\in\sigma(T)$, i.e. $\lambda_0=p(\lambda)\in p(\sigma(T))$.
\end{proof}
\begin{theorem}[Spectral Radius Formula]
    If $T\in\mathcal{B}(X)$, let $r(T)=\sup\{|\lambda|:\lambda\in\sigma(T)\}$.
    Then $r(T)=\lim_{n\to\infty}\norm{T^n}^{1/n}$.
\end{theorem}
\begin{proof}
    By the spectral mapping theorem, $r(T^n)=r(T)^n$.
    Moreover, $r(T^n)\leq\norm{T^n}$ since $\sigma(T^n_\subseteq\norm{T^n}\overline{\mathbb{D}}$.
    Thus $r(T)=r(T^n)^{1/n}\leq\norm{T^n}^{1/n}$ so that $r(T)\leq\liminf_{n\to\infty}\norm{T^n}^{1/n}$.

    Now, let $f\in\mathcal{B}(X)^*$.
    We recall that for $|z|>\norm{T}$,
    \begin{align*}
        R(z)&=(zI-T)^{-1}=\frac{1}{z}(I-\frac{1}{z}T)^{-1}=\frac{1}{z}\sum_{k=0}^\infty\frac{1}{z^k}T^k\\
            &= \sum_{k=1}^\infty\frac{1}{z^k}T^{k-1}.
    \end{align*}
    Thus the Holomorphic funciton $F_f=f\circ R:\rho(T)=\C\setminus\sigma(T)\to\C$ satisfies $F_f(z)=\sum_{k=1}^\infty f(T^{k-1})\frac{1}{z^k}$.
    From $\C-$analysis, the holomorphic function admits a Laurent series for all $z$ with $|z|>r(T)$, and by uniqueness of Laurent series, $F_f(z)=\sum_{n=1}^\infty f(T^{k-1})\frac{1}{z^k}$ for $|z|>r(T)$.
    Hence if $|z_0|>r(T)$, we have that $\left\{f\left(\frac{1}{z_0^k}T^k\right)\right\}_{k=1}^\infty$ is bounded in $\C$.
    Doing this for any $f\in\mathcal{B}(X)^*$, we may apply Banach-Steinhaus to see that $\left\{\frac{1}{z_0}^kT^k\right\}_{k=1}^\infty$ is bounded in $\mathcal{B}(X)$, so
    \begin{equation*}
        \sup_{n\in\N}\norm{\frac{1}{z_0^n}T^n}\leq M<\infty
    \end{equation*}
    Then $\norm{T^n}\leq M|z_0|^n$ so  $\norm{T^n}^{1/n}\leq M^{1/n}|z_0|$, so that
    \begin{equation*}
        \limsup_{n\to\infty}\norm{T^n}^{1/n}\leq|z_0|.
    \end{equation*}
    This applies for any $|z_0|>r(T)$, so the result follows.
\end{proof}
\section{Compact Operators}
\begin{definition}
    Let $X,Y$ be Banach spaces over $\F$.
    A linear opeartor $K:X\to Y$ is called \textbf{compact} if $\overline{K(B(X))}$ is compact in $Y$.
    We let
    \begin{equation*}
        \mathcal{K}(X,Y)=\{K\in\mathcal{L}(X,Y):K\text{ is compact}\}.
    \end{equation*}
    Since compact sets in $Y$ are bounded, we immediately see that $K(X,Y)\subseteq\mathcal{B}(X,Y)$.
\end{definition}
\begin{proposition}
    Let $X,Y$ be Banach spaces.
    Then
    \begin{enumerate}[nl,r]
        \item $\mathcal{K}(X,Y)$ is a closed subspace of $\mathcal{B}(X,Y)$, and
        \item If $W,Z$ are also Banach spaces, $S\in\mathcal{B}(Y,Z)$, $T\in\mathcal{B}(W,X)$, $K\in\mathcal{K}(X,Y)$, then $SKT\in\mathcal{K}(W,Z)$.
    \end{enumerate}
\end{proposition}
\begin{proof}
    \begin{enumerate}[nl,r]
        \item Let $K,L\in\mathcal{K}(X,Y)$, $\alpha\in\F$, and $(x_n)_{n=1}^\infty\subset B(X)$.
            Then $\overline{K(B(X))}\times\overline{L(B(X))}$ is compact in $Y\times Y\cong Y\oplus_1 Y$, so we have converging
            \begin{equation*}
                \left(\left(Kx_{n_j},Lx_{n_j}\right)\right)_{j=1}^\infty\subset\overline{K(B(X))}\times\overline{L(B(X))}
            \end{equation*}
            Now, $(K+\alpha L)(x_{n_j})=Kx_{n_j}+\alpha Lx_{n_j}\to y+\alpha y'$, so $((K+\alpha L)x_n)_{n=1}^\infty$ admits a converging subsequence in $Y$.
            
            Now let $K\in\overline{\mathcal{K}(X,Y)}\subseteq\mathcal{B}(X,Y)$, so $K=\lim_{n\to\infty}K_n$ with each $K_n\in\mathcal{K}(X,Y)$.
            Let $\epsilon>0$.
            Let $n_0\in\N$ be so $n\geq n_0$ implies $\norm{K-K_n}<\epsilon/3$.
            Since $K_{n_0}$ is compact, there are $\{x_1,\ldots,x_m\}\subset B(X)$ such that
            \begin{equation*}
                K_{n_0}(B(X))\subseteq\bigcup_{j=1}^m(K_{n_0}x_j+\frac{\epsilon}{3}B(Y))
            \end{equation*}
            Then for $x\in B(X)$, find $x_j$ so $K_{n_0}x\in K_{n_0}x_j+\frac{\epsilon}{3}B(X)$ and hence
            \begin{align*}
                \norm{Kx-Kx_j}&\leq\norm{Kx-K_{n_0}x}+\norm{K_{n_0}x-K_{n_0}x_j}+\norm{K_{n_0}x_j-Kx_j}\\
                              &\leq \norm{K-K_0}\norm{x}+\frac{\epsilon}{3}+\norm{K-K_{n_0}}\norm{x_j}<\epslion
            \end{align*}
            and we see that $K(B(X))\subseteq\bigcup_{j=1}^m(Kx+\epsilon B(Y))$, and hence is totally bounded.
        \item We have
            \begin{align*}
                \overline{SKT(B(W))}&\subseteq\overline{SK(\norm{T}B(X))}=\norm{T}\overline{SK(B(X))}\\
                                    &=\norm{T}\overline{S(K(B(X)))}\subseteq\norm{T}\overline{S(\overline{K(B(X))})}=\norm{T}S(\overline{K(B(X))})
            \end{align*}
            is a continuous image of a compact set and hence compact.
    \end{enumerate}
\end{proof}
\begin{example}
    \begin{enumerate}[nl,r]
        \item Given Banach spaces $X,Y$, let $\mathcal{F}(X,Y)=\{F\in\mathcal{B}(X,Y):\rank F<\infty\}$.
            Note that $\rank F=\dim(\im F)$.
            If $F\in\mathcal{F}(X,Y)$, then $F(B(X))\subseteq\norm{F}B(\im F)$ is subset of a compact set, so $F$ is compact.
            Thus $\overline{\mathcal{F}(X,Y)}\subseteq\mathcal{K}(X,Y)$.
        \item Also $W,Z$ Banach spaces, $S\in\mathcal{B}(Y,Z)$, $T\in\mathcal{B}(W,Z)$, then $SFT\in\mathcal{F}(W,Z)$ for $F\in\mathcal{F}(X,Y)$.
        \item Let $I=[0,1]$ and $k\in C(I^2)$.
            Then for $f\in C(I)$, $x\in I$, define
            \begin{equation*}
                Kf(x)=\int_0^1k(x,y)f(y)dy
            \end{equation*}
            This defines a compact operator in $\mathcal{K}(C(I),C(I))$.
    \end{enumerate}
\end{example}



\end{document}
