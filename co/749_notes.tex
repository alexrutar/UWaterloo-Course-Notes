% header -----------------------------------------------------------------------
% Template created by texnew (author: Alex Rutar); info can be found at 'https://github.com/alexrutar/texnew'.
% version (1.13)


% doctype ----------------------------------------------------------------------
\documentclass[11pt, a4paper]{memoir}
\usepackage[utf8]{inputenc}
\usepackage[left=3cm,right=3cm,top=3cm,bottom=4cm]{geometry}
\usepackage[protrusion=true,expansion=true]{microtype}


% packages ---------------------------------------------------------------------
\usepackage{amsmath,amssymb,amsfonts}
\usepackage{graphicx}
\usepackage{etoolbox}

% Set enimitem
\usepackage{enumitem}
\SetEnumitemKey{nl}{nolistsep}
\SetEnumitemKey{r}{label=(\roman*)}

% Set tikz
\usepackage{tikz, pgfplots}
\pgfplotsset{compat=1.15}
\usetikzlibrary{intersections,positioning,cd}
\usetikzlibrary{arrows,arrows.meta}
\tikzcdset{arrow style=tikz,diagrams={>=stealth}}

% Set hyperref
\usepackage[hidelinks]{hyperref}
\usepackage{xcolor}
\newcommand\myshade{85}
\colorlet{mylinkcolor}{violet}
\colorlet{mycitecolor}{orange!50!yellow}
\colorlet{myurlcolor}{green!50!blue}

\hypersetup{
  linkcolor  = mylinkcolor!\myshade!black,
  citecolor  = mycitecolor!\myshade!black,
  urlcolor   = myurlcolor!\myshade!black,
  colorlinks = true,
}


% macros -----------------------------------------------------------------------
\DeclareMathOperator{\N}{{\mathbb{N}}}
\DeclareMathOperator{\Q}{{\mathbb{Q}}}
\DeclareMathOperator{\Z}{{\mathbb{Z}}}
\DeclareMathOperator{\R}{{\mathbb{R}}}
\DeclareMathOperator{\C}{{\mathbb{C}}}
\DeclareMathOperator{\F}{{\mathbb{F}}}

% Boldface includes math
\newcommand{\mbf}[1]{{\boldmath\bfseries #1}}

% proof implications
\newcommand{\imp}[2]{($#1\Rightarrow#2$)\hspace{0.2cm}}
\newcommand{\impe}[2]{($#1\Leftrightarrow#2$)\hspace{0.2cm}}
\newcommand{\impr}{{($\Rightarrow$)\hspace{0.2cm}}}
\newcommand{\impl}{{($\Leftarrow$)\hspace{0.2cm}}}

% align macros
\newcommand{\agspace}{\ensuremath{\phantom{--}}}
\newcommand{\agvdots}{\ensuremath{\hspace{0.16cm}\vdots}}

% convenient brackets
\newcommand{\brac}[1]{\ensuremath{\left\langle #1 \right\rangle}}
\newcommand{\norm}[1]{\ensuremath{\left\lVert#1\right\rVert}}
\newcommand{\abs}[1]{\ensuremath{\left\lvert#1\right\rvert}}

% arrows
\newcommand{\lto}[0]{\ensuremath{\longrightarrow}}
\newcommand{\fto}[1]{\ensuremath{\xrightarrow{\scriptstyle{#1}}}}
\newcommand{\hto}[0]{\ensuremath{\hookrightarrow}}
\newcommand{\mapsfrom}[0]{\mathrel{\reflectbox{\ensuremath{\mapsto}}}}
 
% Divides, Not Divides
\renewcommand{\div}{\bigm|}
\newcommand{\ndiv}{%
    \mathrel{\mkern.5mu % small adjustment
        % superimpose \nmid to \big|
        \ooalign{\hidewidth$\big|$\hidewidth\cr$/$\cr}%
    }%
}

% Convenient overline
\newcommand{\ol}[1]{\ensuremath{\overline{#1}}}

% Big \cdot
\makeatletter
\newcommand*\bigcdot{\mathpalette\bigcdot@{.5}}
\newcommand*\bigcdot@[2]{\mathbin{\vcenter{\hbox{\scalebox{#2}{$\m@th#1\bullet$}}}}}
\makeatother

% Big and small Disjoint union
\makeatletter
\providecommand*{\cupdot}{%
  \mathbin{%
    \mathpalette\@cupdot{}%
  }%
}
\newcommand*{\@cupdot}[2]{%
  \ooalign{%
    $\m@th#1\cup$\cr
    \sbox0{$#1\cup$}%
    \dimen@=\ht0 %
    \sbox0{$\m@th#1\cdot$}%
    \advance\dimen@ by -\ht0 %
    \dimen@=.5\dimen@
    \hidewidth\raise\dimen@\box0\hidewidth
  }%
}

\providecommand*{\bigcupdot}{%
  \mathop{%
    \vphantom{\bigcup}%
    \mathpalette\@bigcupdot{}%
  }%
}
\newcommand*{\@bigcupdot}[2]{%
  \ooalign{%
    $\m@th#1\bigcup$\cr
    \sbox0{$#1\bigcup$}%
    \dimen@=\ht0 %
    \advance\dimen@ by -\dp0 %
    \sbox0{\scalebox{2}{$\m@th#1\cdot$}}%
    \advance\dimen@ by -\ht0 %
    \dimen@=.5\dimen@
    \hidewidth\raise\dimen@\box0\hidewidth
  }%
}
\makeatother


% macros (theorem) -------------------------------------------------------------
\usepackage[thmmarks,amsmath,hyperref]{ntheorem}
\usepackage[capitalise,nameinlink]{cleveref}

% Numbered Statements
\theoremstyle{change}
\theoremindent\parindent
\theorembodyfont{\itshape}
\theoremheaderfont{\bfseries\boldmath}
\newtheorem{theorem}{Theorem.}[section]
\newtheorem{lemma}[theorem]{Lemma.}
\newtheorem{corollary}[theorem]{Corollary.}
\newtheorem{proposition}[theorem]{Proposition.}

% Claim environment
\theoremstyle{plain}
\theorempreskip{0.2cm}
\theorempostskip{0.2cm}
\theoremheaderfont{\scshape}
\newtheorem{claim}{Claim}
\renewcommand\theclaim{\Roman{claim}}
\AtBeginEnvironment{theorem}{\setcounter{claim}{0}}

% Un-numbered Statements
\theorempreskip{0.1cm}
\theorempostskip{0.1cm}
\theoremindent0.0cm
\theoremstyle{nonumberplain}
\theorembodyfont{\upshape}
\theoremheaderfont{\bfseries\itshape}
\newtheorem{definition}{Definition.}
\theoremheaderfont{\itshape}
\newtheorem{example}{Example.}
\newtheorem{remark}{Remark.}

% Proof / solution environments
\theoremseparator{}
\theoremheaderfont{\hspace*{\parindent}\scshape}
\theoremsymbol{$//$}
\newtheorem{solution}{Sol'n}
\theoremsymbol{$\blacksquare$}
\theorempostskip{0.4cm}
\newtheorem{proof}{Proof}
\theoremsymbol{}
\newtheorem{nmproof}{Proof}

% Format references
\crefformat{equation}{(#2#1#3)}
\Crefformat{theorem}{#2Thm. #1#3}
\Crefformat{lemma}{#2Lem. #1#3}
\Crefformat{proposition}{#2Prop. #1#3}
\Crefformat{corollary}{#2Cor. #1#3}
\crefformat{theorem}{#2Theorem #1#3}
\crefformat{lemma}{#2Lemma #1#3}
\crefformat{proposition}{#2Proposition #1#3}
\crefformat{corollary}{#2Corollary #1#3}


% macros (algebra) -------------------------------------------------------------
\DeclareMathOperator{\Ann}{Ann}
\DeclareMathOperator{\Aut}{Aut}
\DeclareMathOperator{\chr}{char}
\DeclareMathOperator{\coker}{coker}
\DeclareMathOperator{\disc}{disc}
\DeclareMathOperator{\End}{End}
\DeclareMathOperator{\Fix}{Fix}
\DeclareMathOperator{\Frac}{Frac}
\DeclareMathOperator{\Gal}{Gal}
\DeclareMathOperator{\GL}{GL}
\DeclareMathOperator{\Hom}{Hom}
\DeclareMathOperator{\id}{id}
\DeclareMathOperator{\im}{im}
\DeclareMathOperator{\lcm}{lcm}
\DeclareMathOperator{\Nil}{Nil}
\DeclareMathOperator{\rank}{rank}
\DeclareMathOperator{\Res}{Res}
\DeclareMathOperator{\Spec}{Spec}
\DeclareMathOperator{\spn}{span}
\DeclareMathOperator{\Stab}{Stab}
\DeclareMathOperator{\Tor}{Tor}

% Lagrange symbol
\newcommand{\lgs}[2]{\ensuremath{\left(\frac{#1}{#2}\right)}}

% Quotient (larger in display mode)
\newcommand{\quot}[2]{\mathchoice{\left.\raisebox{0.14em}{$#1$}\middle/\raisebox{-0.14em}{$#2$}\right.}
                                 {\left.\raisebox{0.08em}{$#1$}\middle/\raisebox{-0.08em}{$#2$}\right.}
                                 {\left.\raisebox{0.03em}{$#1$}\middle/\raisebox{-0.03em}{$#2$}\right.}
                                 {\left.\raisebox{0em}{$#1$}\middle/\raisebox{0em}{$#2$}\right.}}


% macros (analysis) ------------------------------------------------------------
\DeclareMathOperator{\M}{{\mathcal{M}}}
\DeclareMathOperator{\B}{{\mathcal{B}}}
\DeclareMathOperator{\ps}{{\mathcal{P}}}
\DeclareMathOperator{\pr}{{\mathbb{P}}}
\DeclareMathOperator{\E}{{\mathbb{E}}}
\DeclareMathOperator{\supp}{supp}
\DeclareMathOperator{\sgn}{sgn}

\renewcommand{\Re}{\ensuremath{\operatorname{Re}}}
\renewcommand{\Im}{\ensuremath{\operatorname{Im}}}
\renewcommand{\d}[1]{\ensuremath{\operatorname{d}\!{#1}}}


% file-specific preamble -------------------------------------------------------
\tikzset{% 
    vtx/.style={inner sep=2pt,circle,fill=black}
}


% constants --------------------------------------------------------------------
\newcommand{\subject}{REPLACE}
\newcommand{\semester}{REPLACE}


% formatting -------------------------------------------------------------------
% Fonts
\usepackage{kpfonts}
\usepackage{dsfont}

% Adjust numbering
\numberwithin{equation}{section}
\counterwithin{figure}{section}
\counterwithout{section}{chapter}
\counterwithin*{chapter}{part}

% Footnote
\setfootins{0.5cm}{0.5cm} % footer space above
\renewcommand*{\thefootnote}{\fnsymbol{footnote}} % footnote symbol

% Table of Contents
\renewcommand{\thechapter}{\Roman{chapter}}
\renewcommand*{\cftchaptername}{Chapter } % Place 'Chapter' before roman
\setlength\cftchapternumwidth{4em} % Add space before chapter name
\cftpagenumbersoff{chapter} % Turn off page numbers for chapter
\maxtocdepth{section} % table of contents up to section

% Section / Subsection headers
\setsecnumdepth{section} % numbering up to and including "section"
\newcommand*{\shortcenter}[1]{%
    \sethangfrom{\noindent ##1}%
    \Large\boldmath\scshape\bfseries
    \centering
\parbox{5in}{\centering #1}\par}
\setsecheadstyle{\shortcenter}
\setsubsecheadstyle{\large\scshape\boldmath\bfseries\raggedright}

% Chapter Headers
\chapterstyle{verville}

% Page Headers / Footers
\copypagestyle{myruled}{ruled} % Draw formatting from echisting 'ruled' style
\makeoddhead{myruled}{}{}{\scshape\subject}
\makeevenfoot{myruled}{}{\thepage}{}
\makeoddfoot{myruled}{}{\thepage}{}
\pagestyle{myruled}
\setfootins{0.5cm}{0.5cm}
\renewcommand*{\thefootnote}{\fnsymbol{footnote}}

% Titlepage
\title{\subject}
\author{Alex Rutar\thanks{\itshape arutar@uwaterloo.ca}\\ University of Waterloo}
\date{\semester\thanks{Last updated: \today}}

\begin{document}
\pagenumbering{gobble}
\hypersetup{pageanchor=false}
\maketitle
\newpage
\frontmatter
\hypersetup{pageanchor=true}
\tableofcontents*
\newpage
\mainmatter


% main document ----------------------------------------------------------------
\chapter{Graph Colourings}
\section{List Colourings}
Recall that a colouring of a graph $G$ is an assignment to each $v\in V(G)$ an element $c(v)$ of some set $C$ called ``colors'' such that if $v$ and $v'$ are neighbours, then $c(v)\neq c(v')$.
Then the \textbf{chromatic number} $\chi(G)$ is the smallest cardinality $|C|$ such that there exists a colouring of $G$ from $C$.

There are some basic upper bounds on the chromatic number of a graph:
\begin{enumerate}[nl]
    \item $\chi(G)\leq|V(G)|$, by colouring every vertex distinctly
    \item $\chi(G)\leq\Delta(G)+1$, by randomly colouring the graph based on colours not used on the neighbours
\end{enumerate}
Note that these upper bounds are in fact tight; for example, the complete graph is tight for both, and an odd cycle is tight for (2).

There are some graphs for which the chromatic number is not known: consider the graph given by $V(G)=\R^2$ where vertices are adjacent if they have euclidean distance 1.
This graph is not $3-$colorable, by taking for example the subgraph
\begin{center}
    \begin{tikzpicture}[scale=3]
        \begin{scope}[rotate=16.78]
            \node[vtx] (a1) at (0,0) {};
            \node[vtx] (a2) at (-.577350269,-1) {};
            \node[vtx] (a3) at (.577350269,-1) {};
            \node[vtx] (a4) at (0,-2) {};
        \end{scope}
        \begin{scope}[rotate=-16.78]
            \node[vtx] (b1) at (0,0) {};
            \node[vtx] (b2) at (-.577350269,-1) {};
            \node[vtx] (b3) at (.577350269,-1) {};
            \node[vtx] (b4) at (0,-2) {};
        \end{scope}
        \draw (a1) -- (a2) -- (a4) -- (a3) -- (a1);
        \draw (a2) -- (a3);
        \draw (b1) -- (b2) -- (b4) -- (b3) -- (b1);
        \draw (b2) -- (b3);
        \draw (a4) -- (b4);
    \end{tikzpicture}
\end{center}
Recently there was a construction showing that the graph is not $4-$colourable, and there is an easy upper bound of $7$, so that $5\leq \chi(G)\leq 7$.

We also define the notion of a list colouring:
\begin{definition}
    A list assignment is an assignment of a set $L(v)$ of colors to each vertex $v$.
    Then a graph is $k-$list-colorable if you can always colour $V(G)$ whenever every vertex has a list of size at least $k$.
\end{definition}
Note that $\chi(G)\leq\chi_\ell(G)$ since asssigning an identical list of size $k$ is a valid list assignmnet and yields a standard coloring.
In many cases list colorings can be hard to determine, but in some cases the exact value is known.
Consider the complete bipartite graph $K_{k,q}$ where $q\geq k$.
We then have the following classification:
\begin{proposition}
    $\chi_\ell(K_{k,q})\leq k$ if and only if $q<k^k$, and $\chi_\ell(K_{k,q})=k+1$ if and only if $q\geq k^k$.
\end{proposition}
\begin{proof}
    Note that $\chi_\ell(K_{k,q})\leq k+1$ always works by taking arbitrary colors on the $k-$side, and on the $q-$side, since the lists have size $k$, there is always a distinct color.

    Now $q<k^k$.
    Try to color the $k$ vertices such that two vertices have the same color.
    If this works, then for every list of size $k$ on the $q-$side, there are only $k-1$ disallowed colours, so we may choose a valid color from the corresponding list.
    Otherwise, every vertex on the $k-$side has a distinct color; this is forced precisely when all the lists are disjoint.
    But then since $q<k^k$, there must be some selection of colors from the lists on the $k-$side such that the set of colors is distinct from every list on the $q-$side, and we may choose colors from the $q-$side without issue.

    Otherwise if $q\geq k^k$, consider lists given by disjoint sets on the $k-$side, and then for every possible assignment of colors on the $k-$side, give a corresponding list for some vertex of the $q-$side that contains a list with those colors.
    Since $q\geq k^k$, we will exhaust all possibilities, so there is no valid coloring from those lists.
\end{proof}
Recall that a planar graph $G$ is one for which there exists an embedding of $G$ into the plane such that each edge is a disjoint curve.
Note that it suffices to consider edges which are polygonal curves, which consist of a finite number of straight line segments; in fact we can also do it with straight line segments (requiring that the graph is simple).
\begin{theorem}[Thomassen]
    If $G$ is planar, then $\ch(G)\leq 5$.
\end{theorem}
In fact, we prove a stronger statement.
We call an ``almost-triangulation'' a planar drawing in which every face except possibly the infinite face is a triangle.
We prove this: let $w$ be a given almost-triangulation with lists of available colour $L(v)$ assigned to every vertex $v$ such that
\begin{enumerate}[nl]
    \item $|L(v)|=5$ for all vertices that are not on the infinite face,
    \item two neighbouring vertices of the infinite face, $a$ and $b$ are colored distinctly,
    \item and all other vertices of the infinite face have lists of 3 colours.
\end{enumerate}
Then this almost-triangulation has a proper list colouring with respect to the given lists.

This implies the theorem since any planar drawing can be made an almost-triangulation by adding edges, and 5-element lists can be reduced to lists of the size above.
\begin{proof}
    We consider two cases in an induction proof.
    \begin{enumerate}[nolistsep]
        \item There is a ``long diagonal'' connecting two of the vertices of the infinite face (that is not an edge of the infinite face).
        \item There is no long diagonal.
    \end{enumerate}
    The induction is on the number of vertices.
    When $n=1,2$ it is trivial, and when $n=3$ it is a 3-cycle and it is certainly fine.

    Now for the induction step, we have the two cases.
    \begin{enumerate}
        \item Cut the graph along the long diagonal to get $G_1,G_2$.
            Without loss of generality, $G_1$ is exactly as described in the statement, so it can be properly list coloured from the given lists.
            Then give the endpoints of the copied long diagonal in $G_2$ so that the endpoint colours are fixed; and by induction, colour it as well.
            Since the endpoints have the same colouring, we can put the two coloured graphs back together to obtain a proper list colouring of $G$.
        \item Let $u\in V(G)$ be the neighbour of $a$ on the infinite face different from $b$.
            Consider the neighbourhood of $u$, $N(u)=\{a,w,v_1,v_2,\ldots,v_k\}$ where $w$ is on the infinite face different from $a$.
            We have $|L(w)|=3$ and $|L(v_i)|=5$ for all $i=1,\ldots,k$ since there is no long diagonal.
            Choose two different colours $\gamma$ and $\Delta$ in $L(u)\setminus\{\alpha\}$; they certainly exist since $|L(u)|=3$.
            Delete $\gamma$ and $\delta$ from all the lists of vertices in $\{v_1,\ldots,v_k\}$, and then by induction we can list colour $G\setminus\{u\}$ from the modified lists.
            This can be extended to a list colouring of $G$ since $u$ shares no colour in its list with any $\{v_1,\ldots,v_k\}$, and at least one of $\delta$ or $\gamma$ will not be used in $w$.
    \end{enumerate}
\end{proof}

$n-$connected means if you remove any $n$ vertices, the graph remains connected

Take $K_4$, and have lists with colours $1,2,3,4$ (or any graph which is uniquely $4-$colorable).
Inscribe a triangle in each face with lists $\{1,2,4,5\}$, $\{1,3,4,5\}$, $\{2,3,4,5\}$.
Always align so that the degree 3 vertex is adjacent to the $1,2$ and $1,3$.
\end{document}


